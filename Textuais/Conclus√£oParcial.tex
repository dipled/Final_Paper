\chapter{Conclusões Parciais}\label{chap:conclusao}

O estudo de lógicas paraconsistentes mostra-se relevante para o desenvolvimento de \textit{softwares} capazes de lidar com informações contraditórias. Dentro desta família de lógicas, os sistemas de inconsistência formal destacam-se no contexto de bases de dados {---} sobretudo bases evolucionárias {---} já que internalizam o conceito de contraditoriedade dentro da sua linguagem. A \lfium{} é uma lógica de inconsistência formal com propriedades que facilitam o desenvolvimento de sistemas de gerenciamento de bancos de dados, por exemplo, mesmo na presença de inconsistência na base.

Assistentes de provas são ferramentas de \textit{software} que permitem ao usuário provar teoremas sobre objetos expressos dentro de si, sem que a verificação destas provas dependa de um julgamento humano para garantir sua validade. O Coq é um assistente de provas, bem como uma linguagem de programação, robusto e com um núcleo axiomático enxuto, que possui aplicações em diferentes áreas da matemática, como lógica, linguagens formais, linguística computacional e desenvolvimento de programas seguros~\cite{coqart}.

O presente trabalho define a linguagem, sintaxe e semântica da \lfium{}, além de revisar e desenvolver manualmente metateoremas que evidenciam características deste sistema, como a correção, completude e o metateorema da dedução, servindo como base para o prosseguimento do TCC2, no qual propõe-se implementar uma biblioteca da lógica \lfium{} em Coq e provar metapropriedades desta lógica dentro do assistente.

Sendo assim, no que segue, é apresentada uma lista de itens que propõem-se serem explorados no TCC2, juntamente com um cronograma para a execução de cada item.

\begin{enumerate}
    \item Definir a linguagem da \lfium{} na biblioteca;
    \item Implementar a sintaxe (cálculo de Hilbert) da \lfium{};
    \item Implementar os sistemas semânticos (matricial e bivaloração) da \lfium{};
    \item Desenvolver metateoremas da \lfium{} na biblioteca.
\end{enumerate}

  \setcounter{table}{1}

  \begin{table}[htbp]
    \centering
    \begin{tabular}{|c|c|c|c|c|c|c|c|c|}
      \hline
      \multirow{2}{*}{\textbf{\small{Item}}} & \textbf{\small{2024/1}} & \multicolumn{6}{c|}{\textbf{\small{2024/2}}} \\
      \cline{2-8}
      & \textbf{Dez} & \textbf{Jan} & \textbf{Fev} & \textbf{Mar} & \textbf{Abr} & \textbf{Maio} & \textbf{Jun} \\
      \hline
      \textbf{\small{1}}  & \cellcolor{gray} & \cellcolor{gray} &  &  &  &  & \\
      \hline
      \textbf{\small{2}}  &  & \cellcolor{gray} & \cellcolor{gray} &  &  &  & \\
      \hline
      \textbf{\small{3}}  &  & \cellcolor{gray} & \cellcolor{gray} & \cellcolor{gray} &  &  & \\
      \hline
      \textbf{\small{4}}  &  &  & \cellcolor{gray} & \cellcolor{gray} & \cellcolor{gray} & \cellcolor{gray} & \cellcolor{gray}\\
      \hline
    \end{tabular}
    \caption{Cronograma Proposto para o TCC2}
  \end{table}
