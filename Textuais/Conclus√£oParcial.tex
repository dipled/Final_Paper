\chapter{Conclusões Parciais}\label{chap:conclusao}

O estudo de lógicas paraconsistentes mostra-se relevante para o desenvolvimento de \textit{softwares} capazes de lidar com informações contraditórias. Dentro desta família de lógicas, os sistemas de inconsistência formal destacam-se no contexto de bases de dado {---} sobretudo bases evolucionárias {---} já que internalizam o conceito de contraditoriedade dentro da sua linguagem.

A \lfium{} é uma lógica de inconsistência formal com propriedades que facilitam o desenvolvimento de sistemas de gerenciamento de bancos de dados. Com isso, o presente trabalho define a linguagem, sintaxe e semântica da \lfium{}, além de revisar e desenvolver manualmente metateoremas que evidenciam características deste sistema, como a correção, completude e o metateorema da dedução.

Sendo assim, no que segue, é apresentada uma lista de itens que propõem-se serem explorados no TCC2, juntamente com um cronograma para a execução de cada item. Os objetivos explorados no TCC2 serão todos desenvolvidos no assistente de provas Coq.

\begin{enumerate}
    \item Definir a linguagem da \lfium{};
    \item Implementar a sintaxe (cálculo de Hilbert) da \lfium{};
    \item Implementar os sistemas semânticos (matricial e bivaloração) da \lfium{};
    \item Desenvolver metateoremas da \lfium{}.
\end{enumerate}


\migs{Não entendi pq os meses contam como duas colunas nessa tabela. Ademais, eu acho \textbf{\textit{extremamente}} otimista você achar que consegue fazer 3/4 do que propôs para o TCC2 antes do semestre começar. Você está comprimindo demais no começo do semestre e tens que levar em conta que você pode trabalhar nisso em dezembro/24 ainda. Por fim, seria interessante colocar uma célula tipo ``Objetivos'' em cima da coluna onde você está enumerando seus objetivos, fica menos feio do que só uma coluna de números.}

\begin{center}
        
    \newcolumntype{C}{>{\centering\arraybackslash}p{1.8em}}
    \newcolumntype{f}{>{\centering\arraybackslash}p{5em}}
    

    \begin{table}[h]
        \begin{tabular}{|f|C|C|C|C|C|C|C|C|C|C|}
            \cline{2-11}
            \multicolumn{1}{c|}{\multirow{1}{*}{}} & \multicolumn{2}{c|}{Janeiro} & \multicolumn{2}{c|}{Fevereiro} & \multicolumn{2}{c|}{Março} & \multicolumn{2}{c|}{Abril} & \multicolumn{2}{c|}{Maio} \\ \cline{1-11}
            1 & \cellcolor{gray} & \cellcolor{gray} & \cellcolor{gray} &~&~&~&~&~&~&~\\ \hline
            2 & \cellcolor{gray}& \cellcolor{gray} & \cellcolor{gray}&~&~&~&~&~&~&~\\ \hline
            3 & \cellcolor{gray}& \cellcolor{gray} & \cellcolor{gray}& \cellcolor{gray} & \cellcolor{gray} &~&~&~&~&~\\ \hline
            4 &~&~&~& \cellcolor{gray} & \cellcolor{gray} & \cellcolor{gray} & \cellcolor{gray} & \cellcolor{gray} & \cellcolor{gray} & \cellcolor{gray} \\ \hline
        \end{tabular}
    \caption{Cronograma proposto para o TCC2.}
\end{table}

\end{center}

\helena{ARRUMAR REFERENCIAS}