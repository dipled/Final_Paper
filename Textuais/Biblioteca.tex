\chapter{Biblioteca}\label{cap:biblioteca}

    \migs{Seguestão: renomear esse capítulo para algo tipo ``Implementação em Coq'' ou algo assim, ``Biblioteca'' fica muito genérico.}
Neste capítulo será \migscortar{feita}{ descrita} a implementação da biblioteca da lógica de inconsistência formal \lfium{} no assistente de provas Coq, bem como o desenvolvimento de alguns metateoremas apresentados na Seção~\ref{sec:metateoremas} dentro da biblioteca. A implementação será análoga àquela feita por~\citeshort{silveira2020implementacao}, que implementou uma biblioteca de lógica modal. Antes de tratar especificamente da implementação, o Coq será brevemente apresentado e caracterizado.

\section{Assistente de provas Coq}\label{sec:coq}
    Os assistentes de provas são ferramentas de \textit{software} que auxiliam o usuário no desenvolvimento de teoremas, permitindo que provas sejam verificadas na medida em que são escritas~\cite{geuvers2009proof}, conferindo a estes programas uma importância significativa na verificação e especificação formal de \textit{software}. Atualmente, existem diversos assistentes de provas como: Agda, Isabelle, Coq, Lean, HOL, Idris e Twelf. Cada um destes tem suas particularidades e diferenças em relação ao formalismo matemático utilizado como base.

    O Coq é um assistente de provas baseado no Cálculo de Construções Indutivas (CCI) que possui aplicações em diferentes áreas da matemática e da computação como (mas não limitado a) lógica, linguagens formais, linguística computacional e desenvolvimento de programas seguros~\cite{coqart}. Sob a ótica da Correspondência de Curry{-}Howard, o Coq é tanto uma linguagem de programação funcional quanto uma linguagem de prova, podendo ser dividido em quatro partes~\cite{silva2019certificaccao}:
    
    \begin{itemize}
        \item A linguagem de programação e especificação \textit{Gallina}, que goza da propriedade da normalização forte\footnote{Um \migscortar{objeto}{ termo-$\lambda$} é fortemente normalizável caso toda sequência de reescrita acabe numa forma normal (num termo irredutível). Um sistema no qual todos os \migscortar{objeto}{ termos-$\lambda$} são fortemente normalizáveis possui a propriedade da normalização forte.~\cite{nipkow2006rewriting}}, a qual garante que todo programa termina.
        \item A linguagem de comandos \textit{Vernacular}, que permite interagir com o assistente.
        \item O conjunto de táticas (\textit{tactics}) utilizadas para manipular elementos durante o desenvolvimento de uma prova.
        \item A linguagem $\pazocal{L}$tac, utilizada para implementar novas táticas e automatizar provas.
    \end{itemize}

    \migscortar{No restante do trabalho é presumido que o leitor tem um conhecimento básico acerca do o funcionamento do assistente de provas Coq no desenvolvimento de provas e verificação de programas. Se este não for o caso, uma consulta ao livro Logical Foundations~\cite{Pierce2017Logical} e ao trabalho de~\citeshort{silveira2020implementacao} é recomendada, a fim de conhecer o Coq e tomar nota das diferentes funcionalidades presentes no assistente.}{ No restante desse trabalho, conceitos básicos sobre o funcionamento do assistente de provas Coq e sobre seu uso no desenvolvimento de provas e verificação de provas não serão apresentados em grandes detalhes. O leitor interessado em tais assuntos é recomendado a consultar os trabalhos de~\citeshort{Pierce2017Logical} e~\citeshort{silveira2020implementacao}.}
