\chapter{Implementação em Rocq}\label{cap:implementacao}

Neste capítulo será descrita a implementação da biblioteca da lógica de inconsistência formal \lfium{} no assistente de provas Rocq (anteriormente conhecido como Coq), bem como o desenvolvimento de alguns metateoremas apresentados na Seção~\ref{sec:metateoremas} dentro da biblioteca. A implementação será análoga àquela feita por~\citeshort{silveira2020implementacao}, que implementou uma biblioteca de lógica modal. Antes de tratar especificamente da implementação, o Rocq será brevemente apresentado e caracterizado.

\section{Assistente de provas Rocq}\label{sec:rocq}
    Os assistentes de provas são ferramentas de \textit{software} que auxiliam o usuário no desenvolvimento de teoremas, permitindo que provas sejam verificadas na medida em que são escritas~\cite{geuvers2009proof}, conferindo a estes programas uma importância significativa na verificação e especificação formal de \textit{software}. Atualmente, existem diversos assistentes de provas como: Agda, Isabelle, Rocq, Lean, Idris e Twelf. Cada um destes tem suas particularidades e diferenças em relação ao formalismo matemático utilizado como base.

    O Rocq é um assistente de provas baseado no Cálculo de Construções Indutivas (CCI) que possui aplicações em diferentes áreas da matemática e da computação como (mas não limitado a) lógica, linguagens formais, linguística computacional e desenvolvimento de programas seguros~\cite{coqart}. Sob a ótica da Correspondência de Curry{-}Howard, o Rocq é tanto uma linguagem de programação funcional quanto uma linguagem de prova, podendo ser dividido em quatro partes~\cite{silva2019certificaccao}:
    
    \begin{itemize}
        \item A linguagem de programação e especificação \textit{Gallina}, que goza da propriedade da normalização forte\footnote{Um termo-$\lambda$ é fortemente normalizável caso toda sequência de reescrita acabe numa forma normal (num termo irredutível). Um sistema no qual todos os termos-$\lambda$ são fortemente normalizáveis possui a propriedade da normalização forte~\cite{nipkow2006rewriting}.}, a qual garante que todo programa termina.
        \item A linguagem de comandos \textit{Vernacular}, que permite interação com o assistente.
        \item O conjunto de táticas (\textit{tactics}) utilizadas para manipular elementos durante o desenvolvimento de uma prova.
        \item A linguagem $\pazocal{L}$tac, utilizada para implementar novas táticas e automatizar provas.
    \end{itemize}

    No restante desse trabalho, conceitos básicos sobre o funcionamento do assistente de provas Rocq e sobre seu uso no desenvolvimento e verificação de provas não serão apresentados em grandes detalhes. Ao leitor interessado em tais assuntos, é recomendada uma consulta aos trabalhos de~\citeshort{Pierce2017Logical} e~\citeshort{silveira2020implementacao}.

\section{Biblioteca em Rocq}\label{sec:biblioteca}
    Nesta seção será apresentada a biblioteca para a lógica \lfium{}, com os principais artefatos semânticos e sintáticos apresentados ao longo do trabalho, bem como provas para os metateoremas que foram provados manualmente. A biblioteca foi desenvolvida utilizando a versão 9.0.0 do Rocq e está disponível em \url{https://github.com/dipled/LFI1_Library}. Cada subseção representa um arquivo da biblioteca e as partes mais importantes de cada um destes será comentada ao longo do texto.

    \subsection{Utils.v}\label{sec:utils}

      Este arquivo contém algumas notações, definições e propriedades úteis que foram estabelecidas para facilitar a implementação.

      \begin{lstlisting}[name=Utils, frame=single, language=coq]
Require Import Arith Constructive_sets.
$\qquad\vdots$
Notation " a $\in$ A " := (In A a) (at level 10).
Notation " a $\not \in$ A " := (~In A a) (at level 10).
Notation " B $\cup$ C " := (Union B C) (at level 48, left associativity).
Notation " [ a ] " := (Singleton a) (at level 0, right associativity).
Notation " A $\subseteq$ B " := (Included A B) (at level 71). 
Notation " $\varnothing$ "     := (Empty_set).
      \end{lstlisting}

      As bibliotecas \texttt{Arith} e \texttt{Constructive\_sets} fornecem alguns teoremas e definições fundamentais para trabalhar com conjuntos em Rocq, na forma de estruturas chamadas de \texttt{Ensembles}, que são mapeamentos de um tipo qualquer \texttt{U} para \texttt{Prop}. A fim de facilitar a leitura do código, foram estabelecidas notações para conceitos de pertinência, união e contingência de conjuntos e conjuntos vazio e unitário.

      \begin{lstlisting}[name=Utils, frame=single, language=coq]
Theorem iff_neg : forall A B : Prop, (A <-> B) -> (~A <-> ~B).
$\qquad\vdots$
Theorem contra : forall A B : Prop, (A -> B) -> (~B -> ~A).
$\qquad\vdots$
Ltac destruct_conjunction H :=
match type of H with
| _ /\ _ => 
  let L := fresh "L" in
  let R := fresh "R" in
  destruct H as [L R]; destruct_conjunction L; destruct_conjunction R
| _ => idtac
end.
      \end{lstlisting}

      A tática \texttt{destruct\_conjunction}, implementada em $\pazocal{L}$tac, é uma tática que recebe como argumento uma hipótese \texttt{H} e caso esta hipótese seja uma conjunção, ela é destruída em duas novas hipóteses \texttt{L} e \texttt{R} e a tática é repetida recursivamente para cada uma destas hipóteses.

      \subsection{Language.v}\label{sec:language}

      Este arquivo contém a definição da linguagem da \lfium{}, como apresentada na Seção~\ref{sec:linguagem}.

      \begin{lstlisting}[name=Language, frame=single, language=coq]
Definition Atom := nat.

Inductive Formula : Set :=
  | Lit    : Atom -> Formula
  | Neg    : Formula -> Formula
  | And    : Formula -> Formula -> Formula
  | Or     : Formula -> Formula -> Formula
  | Imp    : Formula -> Formula -> Formula
  | Consistency   : Formula -> Formula.

Notation " x $\to$ y " := 
(Imp x y) (at level 8, right associativity).
$\qquad\vdots$
Notation " $\#$ x " :=
(Lit x) (at level 2, no associativity, x constr at level 1, format "$\#$ x").
      \end{lstlisting}
      
      Note que \texttt{Lit} é construído a partir de um \texttt{Atom}, que por sua vez é só um \textit{alias} para o tipo \texttt{nat} (um número natural), portanto o tipo indutivo \texttt{Formula} é enumerável.

    \subsection{Deduction.v}\label{sec:syntax}

      Este arquivo contém a definição da relação de consequência sintática da \lfium{}, através do cáclulo de Hilbert definido na Seção~\ref{sec:axiomatizacao}.

      \begin{lstlisting}[name=Syntax, frame=single, language=coq]
Inductive Ax : Set :=
  | Ax1      : Formula -> Formula -> Ax
  | Ax2      : Formula -> Formula -> Formula -> Ax
$\qquad\vdots$
  | bc1      : Formula -> Formula -> Ax
  | cf       : Formula -> Ax
  | ce       : Formula -> Ax
  | ci       : Formula -> Ax
$\qquad\vdots$
Definition instantiate (a : Ax) : Formula :=
  match a with 
  | Ax1 $\alpha$ $\beta$      => $\alpha$ $\to$ ($\beta$ $\to$ $\alpha$)
  | Ax2 $\alpha$ $\beta$ $\gamma$    => ($\alpha$ $\to$ ($\beta$ $\to$ $\gamma$)) $\to$ (($\alpha$ $\to$ $\beta$) $\to$ ($\alpha$ $\to$ $\gamma$))
$\qquad\vdots$
  | bc1 $\alpha$ $\beta$      => $\circ$$\alpha$ $\to$ ($\alpha$ $\to$ ($\neg$$\alpha$ $\to$ $\beta$))
  | cf $\alpha$         => $\neg$$\neg$$\alpha$ $\to$ $\alpha$
  | ce $\alpha$         => $\alpha$ $\to$ $\neg$$\neg$$\alpha$
  | ci $\alpha$         => $\neg$$\circ$$\alpha$ $\to$ ($\alpha$ $\land$ $\neg$ $\alpha$)
$\qquad\vdots$
  end.
      \end{lstlisting}

      Primeiro, o tipo \texttt{Ax} foi definido como um tipo indutivo (também chamado de tipo algébrico), com construtores que recebem como argumentos fórmulas correspondentes ao número de fórmulas dos axiomas apresentados na Definição~\ref{def:hilbert_lfi1}. 
      
      Depois, a função \texttt{instantiate} é definida. Ela recebe como argumento um axioma (que por sua vez foi construído a partir de fórmulas) e retorna uma fórmula correspondente àquela representada pelo axioma, por exemplo, o axioma~\ref{ax:axcf} representa a eliminação da dupla negação, portanto, a função \texttt{instantiate} receberia um dado \texttt{cf $\alpha$} do tipo \texttt{Ax}, onde $\alpha$ é uma fórmula qualquer da linguagem, e retornaria um dado $\neg\neg\alpha \to \alpha$ do tipo \texttt{Formula}.

      \begin{lstlisting}[name=Syntax, frame=single, language=coq]
Inductive deduction : Ensemble Formula -> Formula -> Prop :=
  | Premisse : forall ($\Gamma$ : Ensemble Formula) ($\phi$ : Formula), $\phi$ $\in$ $\Gamma$ -> deduction $\Gamma$ $\phi$
  | AxiomInstance : forall ($\Gamma$ : Ensemble Formula) (a : Ax), deduction $\Gamma$ (instantiate a)
  | MP : forall ($\Gamma$ : Ensemble Formula) ($\phi$ $\psi$ : Formula), (deduction $\Gamma$ ($\phi$ $\to$ $\psi$)) -> 
    (deduction $\Gamma$ $\phi$) -> deduction $\Gamma$ $\psi$.

Notation " $\Gamma$ $\vdash$ $\phi$ " := (deduction $\Gamma$ $\phi$) (at level 50, no associativity).
      \end{lstlisting}

      O tipo \texttt{deduction} é definido indutivamente a partir de um conjunto de fórmulas (premissas) e uma fórmula (conclusão) e retornando um dado do tipo \texttt{Prop} referente a possibilidade de construir uma derivação da conclusão a partir do conjunto de premissas, aplicando uma sequência de construtores, que representam os três casos de derivação descritos na Definição~\ref{def:derivacao}. O construtor \texttt{Premisse} afirma que se uma fórmula qualquer pertence ao conjunto de premissas, então ela é derivável a partir deste conjunto. O construtor \texttt{AxiomInstance} afirma que um axioma é sempre derivável a partir de qualquer conjunto de premissas. O construtor \texttt{MP} afirma que, dadas duas fórmulas $\phi$ e $\psi$ quaisquer, se for possível derivar $\phi \to \psi$ e $\phi$ a partir de um conjunto de fórmulas, então podemos derivar $\psi$ a partir deste mesmo conjunto. Por último, uma notação com o operador infixo $\vdash$ é estabelecida para facilitar a leitura de derivações.
    
    \subsection{Semantics.v}

      Este arquivo contém a definição dos dois sistemas semânticos apresentados na Seção~\ref{sec:semantica}. O primeiro sistema definido foi a semântica matricial, conforme as Definições~\ref{def:mat} e~\ref{def:conmat}.
      \begin{lstlisting}[name=Semantics, frame=single, language=coq]
Inductive MatrixDomain : Set :=
  | One
  | Half
  | Zero.

Definition designatedValue (a : MatrixDomain) : Prop :=
  match a with
  | Zero => False
  | _ => True
  end.

Definition andM (a b : MatrixDomain) : MatrixDomain :=
  match a, b with
  | Zero, _  => Zero
  | _, Zero  => Zero
  | One, One => One
  | _, _     => Half
  end.
$\qquad\vdots$
Notation " x $\land_{m}$ y " := 
(andM x y) (at level 20, left associativity).
$\qquad\vdots$
        
      \end{lstlisting}
      Note que \texttt{designatedValue} é um predicado sobre o domínio da matriz lógica, e retorna um \texttt{Prop} referente à pertinência de um dado valor ao conjunto $\{1,\meio{}\}$.

      Após definir a matriz lógica, é preciso estabelecer a noção de valorações sobre essa matriz, conforme apresentada na Definição~\ref{def:matlog}, que define valoração como um homomorfismo entre linguagem e a álgebra da matriz, preservando a estrutura dos seus operadores.
    
      \begin{lstlisting}[name=Semantics, frame=single, language=coq]
Definition preserveAnd (v : Formula -> MatrixDomain) : Prop := 
forall $\phi$ $\psi$: Formula, (v ($\phi$ $\land$ $\psi$)) = (v $\phi$) $\land_{m}$ (v $\psi$).
$\qquad\vdots$
Definition valuation (v : Formula -> MatrixDomain) : Prop :=
preserveOr v /\ preserveTo v /\ preserveAnd v /\ preserveNeg v /\ preserveCirc v.

Definition matrixEntails ($\Gamma$ : Ensemble Formula) ($\phi$ : Formula) := 
forall v : (Formula -> MatrixDomain),
valuation v -> 
(forall ($\psi$: Formula), 
  $\psi$ $\in$ $\Gamma$ -> designatedValue (v $\psi$)) -> 
    designatedValue (v $\phi$).

Notation " $\Gamma$ $\vDash$m $\phi$ " := (matrixEntails $\Gamma$ $\phi$) (at level 50, no associativity).
      \end{lstlisting}

      A definição \texttt{valuation} representa um predicado sobre qualquer mapeamento de \texttt{Formula} para \texttt{MatrixDomain}, que retorna \texttt{True} caso este mapeamento seja um homomorfismo sobre a linguagem e o domínio da matriz e retorna \texttt{False} caso contrário. A relação de consequência semântica é definida em \texttt{matrixEntails} (conforme a Definição~\ref{def:conmat}), que diz que, dado um conjunto de fórmulas (premissas) e uma fórmula (conclusão), esta fórmula será uma consequência semântica do conjunto de premissas caso toda valoração que satisfaz todas as fórmulas deste conjunto também satisfizer a conclusão.

      Depois disso, a semântica de bivalorações foi implementada, com base nas Definições~\ref{def:valoracoes} e~\ref{def:conval}.

      \begin{lstlisting}[name=Semantics, frame=single, language=coq]
Inductive BivaluationDomain : Set :=
  | Bot
  | Top.

Notation " $\bot$ " := Bot.
Notation " $\top$ " := Top.

Definition vAnd (v : Formula -> BivaluationDomain) : Prop :=
  forall $\phi$ $\psi$ : Formula, (v ($\phi$ $\land$ $\psi$)) = $\top$ <-> (v $\phi$ = $\top$) /\ (v $\psi$ = $\top$).
$\qquad\vdots$
Definition vNeg (v : Formula -> BivaluationDomain) : Prop :=
  forall $\phi$ : Formula, (v $\neg$$\phi$) = $\bot$ -> v $\phi$ = $\top$.

Definition vCon (v : Formula -> BivaluationDomain) : Prop :=
  forall $\phi$ : Formula, (v $\circ$$\phi$) = $\top$ -> (v $\phi$ = $\bot$) \/ (v $\neg$$\phi$ = $\bot$).

Definition vCi (v : Formula -> BivaluationDomain) : Prop :=
  forall $\phi$ : Formula, (v $\neg$$\circ$$\phi$) = $\top$ -> (v $\phi$ = $\top$) /\ (v $\neg$$\phi$ = $\top$).
$\qquad\vdots$
Definition vCip (v : Formula -> BivaluationDomain) : Prop :=
  forall $\phi$ $\psi$ : Formula, (v $\neg$($\phi$ → $\psi$)) = $\top$ <-> (v $\phi$ = $\top$) /\ (v $\neg$$\psi$ = $\top$).
Definition bivaluation (v : Formula -> BivaluationDomain) : Prop :=
  vAnd v /\ vOr v /\ vImp v /\ vNeg v /\ vCon v /\ vCi v /\
  vDne v /\ vDmAND v /\ vDmOR v /\ vCip v.
              \end{lstlisting}

              A bivaloração é definida sobre um tipo indutivo com construtores $\top$ e $\bot$, representando respectivamente os valores para verdade e falsidade do domínio. As cláusulas \texttt{vAnd} até \texttt{vCip} definem as condições que um mapeamento entre \texttt{Formula} e \texttt{BivaluationDomain} precisam respeitar para serem consideradas bivalorações, caracterizadas pelo predicado \texttt{bivaluation}. Estes mapeamentos apresentam um comportamento não-determinístico em algumas cláusulas, como na cláusula \texttt{vNeg}.

              Após definir estas cláusulas, algumas propriedades úteis sobre a semântica de bivalorações (como o terceiro excluído em \texttt{bivaluation\_lem}) são provadas.

              \begin{lstlisting}[name=Semantics, frame=single, language=coq]
Lemma bivaluation_lem : forall (v : Formula -> BivaluationDomain) ($\phi$ : Formula),
(v $\phi$ = $\top$) \/ (v $\phi$ = $\bot$).
Proof.$\;\ldots\;$Qed.

Lemma bivaluation_dec1 : forall (v : Formula -> BivaluationDomain) ($\phi$ : Formula),
v $\phi$ = $\top$ <-> ~ v $\phi$ = $\bot$.
Proof.$\;\ldots\;$Qed.

Lemma bivaluation_dec2 : forall (v : Formula -> BivaluationDomain) ($\phi$ : Formula),
v $\phi$ = $\bot$ <-> ~ v $\phi$ = $\top$.
Proof.$\;\ldots\;$Qed.

Definition vAndf (v : Formula -> BivaluationDomain) : Prop :=
forall $\phi$ $\psi$ : Formula, (v ($\phi$ $\land$ $\psi$)) = $\bot$ <-> (v $\phi$ = $\bot$) \/ (v $\psi$ = $\bot$).
$\qquad\vdots$
Definition vNegf (v : Formula -> BivaluationDomain) : Prop :=
forall $\phi$ : Formula, (v $\phi$) = $\bot$ -> (v $\neg$$\phi$) = $\top$.

Definition vConf (v : Formula -> BivaluationDomain) : Prop :=
forall $\phi$ : Formula, (v $\phi$ = $\top$) /\ (v $\neg$$\phi$ = $\top$) -> (v $\circ$$\phi$) = $\bot$.

Definition vCif (v : Formula -> BivaluationDomain) : Prop :=
forall $\phi$ : Formula, (v $\phi$ = $\bot$) \/ (v $\neg$$\phi$ = $\bot$) -> (v $\neg$$\circ$$\phi$) = $\bot$.
$\qquad\vdots$
Lemma bivaluation_additional : 
forall (v : Formula -> BivaluationDomain),
bivaluation v ->
vAndf v /\ vOrf v /\ vImpf v /\ vNegf v /\ vConf v /\ vCif v /\
vDnef v /\ vDmANDf v /\ vDmORf v /\ vCipf v.
Proof.$\;\ldots\;$Qed.
            \end{lstlisting}
            As provas de \texttt{bivaluation\_lem}, \texttt{bivaluation\_dec1} e \texttt{bivaluation\_dec2} seguem por análise de caso em \texttt{v($\phi$)}. O lema \texttt{bivaluation\_additional} demonstra cláusulas úteis na prova da completude em relação a semântica de bivalorações, que são derivadas das cláusulas estabelecidas originalmente. Nas provas manuais estas cláusulas foram consideradas triviais e, portanto, não foram demonstradas. Este lema segue por análise de caso em \texttt{v($\phi$)} e \texttt{v($\psi$)}.

          \subsection{Deduction\_metatheorem.v}

            Este arquivo contém a prova do metateorema da dedução dentro da biblioteca, análoga à prova manual do Teorema~\ref{teo:deducao}. Antes de provar este teorema, alguns lemas e proposições auxiliares foram demonstrados.
          
          \begin{lstlisting}[name=Deduction, frame=single, language=coq]
Proposition lfi1_reflexivity : 
forall ($\Gamma$ : Ensemble Formula) ($\phi$ : Formula),
  $\phi$ $\in$ $\Gamma$ -> $\Gamma$ $\vdash$ $\phi$.
Proof.$\;\ldots\;$Qed.

Proposition lfi1_monotonicity :
forall ($\Gamma$ $\Delta$ : Ensemble Formula) ($\phi$ : Formula),
  $\Delta$ $\vdash$ $\phi$ /\ $\Delta$ $\subseteq$ $\Gamma$ -> $\Gamma$ $\vdash$ $\phi$.
Proof.$\;\ldots\;$Qed.

Proposition lfi1_cut :
forall ($\Gamma$ $\Delta$ : Ensemble Formula) ($\phi$ : Formula),
  $\Delta$ $\vdash$ $\phi$ /\ (forall ($\delta$ : Formula), $\delta$ $\in$ $\Delta$ -> $\Gamma$ $\vdash$ $\delta$) -> $\Gamma$ $\vdash$ $\phi$.
Proof.$\;\ldots\;$Qed.

Lemma id : forall ($\Gamma$ : Ensemble Formula) ($\phi$ : Formula), $\Gamma$ $\vdash$ $\phi$ $\to$ $\phi$.
Proof.$\;\ldots\;$Qed.

          \end{lstlisting}

          Em particular, a prova da monotonicidade em \texttt{lfi1\_monotonicity} é de suma importância no desenvolvimento das demonstrações das metapropriedades mais elaboradas. A prova desta propriedade se dá por indução estrutural na derivação $\Delta \vdash \phi$.

          Depois disso, o teorema da dedução é finalmente provado.

          \begin{lstlisting}[name=Deduction, frame=single, language=coq]
Theorem deduction_metatheorem : forall ($\Gamma$ : Ensemble Formula) ($\alpha$ $\beta$ : Formula), 
(($\Gamma$ $\cup$ [$\alpha$]) $\vdash$ $\beta$) <-> ($\Gamma$ $\vdash$ $\alpha$ $\to$ $\beta$).
Proof. 
$\quad\vdots$
Qed.

Corollary proof_by_cases : forall ($\Gamma$ : Ensemble Formula) ($\alpha$ $\beta$ $\phi$ : Formula), 
($\Gamma$ $\cup$ [$\alpha$] $\vdash$ $\phi$) /\ ($\Gamma$ $\cup$ [$\beta$] $\vdash$ $\phi$) -> ($\Gamma$ $\cup$ [$\alpha$ $\lor$ $\beta$] $\vdash$ $\phi$).
Proof.$\;\ldots\;$Qed.
  
Corollary proof_by_cases_neg : forall ($\Gamma$ : Ensemble Formula) ($\alpha$ $\phi$ : Formula), 
($\Gamma$ $\cup$ [$\alpha$] $\vdash$ $\phi$) /\ ($\Gamma$ $\cup$ [$\neg$$\alpha$] $\vdash$ $\phi$) -> ($\Gamma$ $\vdash$ $\phi$).
Proof.$\;\ldots\;$Qed.
          \end{lstlisting}

          A prova segue por indução estrutural na derivação $\Gamma \cup [\alpha] \vdash \beta$ na ida e na derivação $\Gamma \vdash \alpha \to \beta$ na volta. O teorema nos cede dois corolários (\texttt{proof\_by\_cases} e \texttt{proof\_by\_cases\_neg}) que facilitam a prova da completude em relação à semântica de bivalorações. Ambos corolários seguem por derivações no cálculo de Hilbert utilizando o recém-provado teorema da dedução.

        \subsection{Soundness.v}
        
        Este arquivo implementa as provas desenvolvidas ao longo da Seção~\ref{sec:cor}, necessárias para desenvolver o metateorema da correção.
        \begin{lstlisting}[name=Soundness, frame=single, language=coq]
Definition h_formula ($\alpha$ : Formula) (v : Formula -> BivaluationDomain) : MatrixDomain :=
  match (v $\alpha$), (v $\neg$$\alpha$) with
  | $\top$, $\bot$ => One
  | $\top$, $\top$ => Half
  | $\bot$, _ => Zero
  end.

Lemma h_valuation : forall (v : Formula -> BivaluationDomain),
bivaluation v -> valuation (fun x => h_formula x v).
Proof.$\;\ldots\;$Qed.
        \end{lstlisting}

        A primeira definição feita é a da função \texttt{h\_formula} que constrói uma valoração sobre as matrizes a partir de uma bivaloração recebida como argumento, similar a função apresentada no Lema~\ref{lem:matval}. Na prova manual deste lema, o fato desta função ser uma valoração sobre as matrizes {--} um homomorfismo entre a linguagem e as matrizes {--} foi tomado como trivial. Para o Rocq, entretando, este fato não é imediato e deve ser demonstrado, o que é feito pelo lema \texttt{h\_valuation}, que segue por análise de caso nos valores de v$(\alpha)$ e v$(\neg \alpha)$ da função \texttt{h\_formula}.

        Depois disso, a prova do Lema~\ref{lem:matval} é desenvolvida.

        \begin{lstlisting}[name=Soundness, frame=single, language=coq]
Lemma bivaluation_matrix_lemma : forall (v : Formula -> BivaluationDomain),
bivaluation v -> 
(exists (h : Formula -> MatrixDomain),
  (forall $\phi$ : Formula, v $\phi$ = $\top$ <-> designatedValue (h $\phi$)) /\ valuation h).
Proof. 
$\quad\vdots$
Qed.

Corollary matrix_bivaluation_imp : forall ($\Gamma$ : Ensemble Formula) ($\alpha$ : Formula), 
($\Gamma$ $\vDash$m $\alpha$) -> ($\Gamma$ $\vDash$ $\alpha$).
Proof.$\;\ldots\;$Qed.
        \end{lstlisting}

        O lema \texttt{bivaluation\_matrix\_lemma} segue por análise de caso em v$(\phi)$ e v$(\neg \phi)$, utilizando a valoração obtida ao aplicar a função \texttt{h\_formula} numa bivaloração $v$ qualquer e o resultado obtido no lema \texttt{h\_valuation}. Com este resultado, a prova da ida da equivalência entre os sistemas semânticos é desenvolvida em \texttt{matrix\_bivaluation\_imp}, utilizando as definições das relações de consequência semântica matricial e de bivalorações, como feito no Corolário~\ref{cor:matval}.

        Finalmente, o Teorema~\ref{teo:correcao_mat} e o Corolário~\ref{cor:correcao_val} são desenvolvidos.

        \begin{lstlisting}[name=Soundness, frame=single, language=coq]
Theorem soundness_matrix : forall ($\Gamma$ : Ensemble Formula) ($\alpha$ : Formula), 
($\Gamma$ $\vdash$ $\alpha$) -> ($\Gamma$ $\vDash$m $\alpha$).
Proof.$\;\ldots\;$Qed.

Corollary soundness_bivaluations : forall ($\Gamma$ : Ensemble Formula) ($\alpha$ : Formula), 
($\Gamma$ $\vdash$ $\alpha$) -> ($\Gamma$ $\vDash$ $\alpha$).
Proof.$\;\ldots\;$Qed.
        \end{lstlisting}

        Em \texttt{soundness\_matrix}, a correção em relação à semântica matricial é provada por indução na derivação $\Gamma \vdash \alpha$, mostrando a validade de todos os axiomas e da regra \textit{modus ponens}. Esta análise de caso é facilitada pelas táticas \texttt{repeat} e \texttt{try} aplicadas em todos os \textit{subgoals} de prova simultaneamente, evitando repetições desnecessárias de casos de prova análogos entre si.

        O corolário \texttt{soundness\_bivaluations} segue pela aplicação de \texttt{soundness\_matrix} na hipótese $\Gamma \vdash \alpha$, o que resulta em $\Gamma \vDash_\text{m}\alpha$. Depois disso, a aplicação de \texttt{matrix\_bivaluation\_imp} em $\Gamma \vDash_\text{m}\alpha$ nos dá $\Gamma \vDash \alpha$, completando a prova.

        Com estes resultados, a prova de que a lógica \lfium{} se trata, de fato, de uma \lfi{} forte (em relação a $\neg$ e $\circ$) é desenvolvida, analogamente ao que foi feito no Corolário~\ref{cor:lfi_forte}.

        \begin{lstlisting}[name=LFI, frame=single, language=coq]
(** LFI1 is an LFI w.r.t $\neg$ and $\circ$, i.e. 
1) exists ($\alpha$ $\beta$ : Formula), $\sim$($\alpha$, $\neg$$\alpha$ $\vdash$ $\beta$)
2) exists ($\alpha$ $\beta$ : Formula), $\sim$($\alpha$, $\circ$$\alpha$ $\vdash$ $\beta$) /\ $\sim$($\neg$$\alpha$, $\circ$$\alpha$ $\vdash$ $\beta$)
3) forall ($\alpha$ $\beta$ : Formula), $\circ$$\alpha$, $\alpha$, $\neg$$\alpha$ $\vdash$ $\beta$
*)
Fixpoint valuation_condition_1 (x : Formula) : MatrixDomain :=
match x with 
|$\#$0 => Half
| a $\land$ b => (valuation_condition_1 a) $\land_{m}$ (valuation_condition_1 b)
| a $\lor$ b => (valuation_condition_1 a) $\lor_{m}$ (valuation_condition_1 b)
| a $\to$ b => (valuation_condition_1 a) $\to_{m}$ (valuation_condition_1 b)
| $\neg$a => $\neg_{m}$(valuation_condition_1 a)
| $\circ$a => $\circ_{m}$(valuation_condition_1 a)
|_ => Zero 
end.

Fixpoint valuation_condition_2_l (x : Formula) : MatrixDomain :=
match x with 
|$\#$0 => One
| a $\land$ b => (valuation_condition_2_l a) $\land_{m}$ (valuation_condition_2_l b)
| a $\lor$ b => (valuation_condition_2_l a) $\lor_{m}$ (valuation_condition_2_l b)
| a $\to$ b => (valuation_condition_2_l a) $\to_{m}$ (valuation_condition_2_l b)
| $\neg$a => $\neg_{m}$(valuation_condition_2_l a)
| $\circ$a => $\circ_{m}$(valuation_condition_2_l a)
|_ => Zero 
end.

Fixpoint valuation_condition_2_r (x : Formula) : MatrixDomain :=
match x with 
|$\#$0 => Zero
|$\neg$$\#$0 => One
| a $\land$ b => (valuation_condition_2_r a) $\land_{m}$ (valuation_condition_2_r b)
| a $\lor$ b => (valuation_condition_2_r a) $\lor_{m}$ (valuation_condition_2_r b)
| a $\to$ b => (valuation_condition_2_r a) $\to_{m}$ (valuation_condition_2_r b)
| $\neg$a => $\neg_{m}$(valuation_condition_2_r a)
| $\circ$a => $\circ_{m}$(valuation_condition_2_r a)
|_ => Zero 
end.

Proposition lfi1_is_lfi :
(exists ($\alpha$ $\beta$ : Formula), ~([$\alpha$] $\cup$ [$\neg$$\alpha$] $\vdash$ $\beta$)) /\
(exists ($\alpha$ $\beta$ : Formula), ~([$\circ$$\alpha$] $\cup$ [$\alpha$] $\vdash$ $\beta$) /\ ~([$\circ$$\alpha$] $\cup$ [$\neg$$\alpha$] $\vdash$ $\beta$)) /\
(forall ($\alpha$ $\beta$ : Formula), ([$\circ$$\alpha$] $\cup$ [$\alpha$] $\cup$ [$\neg$$\alpha$]) $\vdash$ $\beta$).
Proof.$\;\ldots\;$Qed.
        \end{lstlisting}

        A prova das condições 1 e 2 dependem da definição de valorações específicas que servem de contraexemplo para a validade da derivação, utilizando a contrapositiva da recém-provada correção (\texttt{soundness\_matrix}). Estas valorações são definidas recursivamente em \texttt{valuation\_condition\_1}, \texttt{valuation\_condition\_2\_l} e \texttt{valuation\_condition\_2\_r}. A\\prova da condição 3 é feita por derivação no cálculo de Hilbert, utilizando do teorema da dedução (\texttt{deduction\_metatheorem}) para facilitar o desenvolvimento.
    
    \subsection{Cardinality.v}\label{sec:cardinality}

        Este arquivo implementa alguns conceitos relacionados à cardinalidade de tipos indutivos, utilizados para provar que um dado tipo possui o mesmo número de elementos que outro. O módulo \texttt{IndefiniteDescription} contém um axioma chamado \texttt{constructive\_indefinite\\\_description}, que torna possível a definição de funções de escolha, necessárias para a prova do lema de Lindenbaum (Lema~\ref{lem:lindenbaum}). Este módulo também importa arquivos que implementam o axioma da escolha. Já o módulo \texttt{Classical\_sets} contém o princípio do terceiro excluído como axioma.

        \begin{lstlisting}[name=LFI, frame=single, language=coq]
From Stdlib Require Import Classical_sets IndefiniteDescription.

Theorem strong_lem : forall P : Prop, {P} + {~P}.
Proof.$\;\ldots\;$Qed. 

Theorem contraposition : forall A B : Prop, (A -> B) <-> (~B -> ~A).
Proof.$\;\ldots\;$Qed.

Inductive image_set {A B : Type} (f : A -> B) (M: Ensemble A) : Ensemble B :=
image_intro : forall a, a $\in$ M -> (f a) $\in$ (image_set f M).

Definition function_injective {A B : Type} (f: A -> B): Prop :=
  forall a1 a2, f a1 = f a2 -> a1 = a2.

Definition function_surjective {A B : Type} (f: A -> B): Prop :=
  forall b, exists a, f a = b.
  
Definition inverse_function {A B : Type} (f : A -> B) (g : B -> A) : Prop :=
  (forall x : A, g (f x) = x) /\ (forall y : B, f (g y) = y).

Record injection (A B: Type): Type := Build_injection {
  inj_f :> A -> B;
  in_inj : function_injective inj_f
}.

Record surjection (A B: Type): Type := Build_surjection {
  sur_f :> A -> B;
  su_surj : function_surjective sur_f
}.
          \end{lstlisting}

          O teorema \texttt{strong\_lem} representa uma versão forte do princípio do terceiro excluído, provado utilizando a versão normal do terceiro excluído juntamente com o axioma \texttt{constructive\\\_indefinite\_description}, que possibilita a definição do conjunto de Lindenbaum.
          
          O teorema \texttt{contraposition} define a versão forte da prova por contraposição, permitindo provas de $A \to B$ a partir de provas do tipo $\sim B \to \sim A$.
          
          Os \texttt{Records} \texttt{injection} e \texttt{surjection} facilitam o uso de funções injetoras e sobrejetoras, pois elementos destes tipos carregam consigo predicados sobre suas funções, que caracterizam-nas de acordo com sua injetividade e sobrejetividade.

    
    \subsection{Completeness.v}\label{sec:completeness}
        
        Este arquivo implementa as provas desenvolvidas ao longo das Seções~\ref{sec:comp},~\ref{sec:equiv} e~\ref{sec:comp_mat}, referentes aos metateoremas da completude (tanto em relação às matrizes, quanto às bivalorações) e da equivalência entre os sistemas semânticos.

        \begin{lstlisting}[name=Completeness, frame=single, language=coq]
Proposition In_lem {U : Type} : forall (A : Ensemble U) (x : U),
  x $\in$ A \/ x $\not \in$ A.
Proof. intros. apply classic. Qed.

Definition maximal_nontrivial ($\Gamma$ : Ensemble Formula) ($\phi$ : Formula) : Prop :=
  ~ ($\Gamma$ $\vdash$ $\phi$) /\ (forall ($\psi$ : Formula), $\psi$ $\not \in$ $\Gamma$ -> ($\Gamma$ $\cup$ [$\psi$] $\vdash$ $\phi$)).

Definition closed_theory ($\Gamma$ : Ensemble Formula) : Prop := 
forall $\phi$ : Formula, $\Gamma$ $\vdash$ $\phi$ <-> $\phi$ $\in$ $\Gamma$.

Lemma maximal_nontrivial_is_closed : forall ($\Gamma$ : Ensemble Formula) ($\phi$ : Formula),
  maximal_nontrivial $\Gamma$ $\phi$ -> closed_theory $\Gamma$.
Proof.$\;\ldots\;$Qed.
        \end{lstlisting}

        O teorema \texttt{In\_lem} representa o princípio do terceiro excluído aplicado à pertinência de um dado elemento a um conjunto. Os predicados \texttt{maximal\_nontrivial} e \texttt{closed\_theory} implementam, respectivamente, as Definições~\ref{def:nao-trivial_maximal} e~\ref{def:fechada} sobre conjuntos de fórmulas bem formadas. O lema \texttt{maximal\_nontrivial\_is\_closed} prova que todo conjunto maximal não trivial de fórmulas é uma teoria fechada (Lema~\ref{lem:nao_trivial_maximal_fechado}). Isto é feito por análise de caso em $\gamma \in \Gamma$, onde $\gamma$ é uma fórmula qualquer.

        Depois destas definições e lemas, é preciso estabelecer uma bivaloração específica que permite o desenvolvimento da prova de completude.
        
        \begin{lstlisting}[name=Completeness, frame=single, language=coq]
Definition completeness_valuation ($\Gamma$ : Ensemble Formula) : 
Formula -> BivaluationDomain :=
fun x =>
  match (strong_lem (x $\in$ $\Gamma$)) with
  | left _ => $\top$
  | right _ => $\bot$
  end.

Lemma completeness_valuation_is_bivaluation : 
forall ($\Gamma$ : Ensemble Formula) ($\phi$ : Formula), 
  (maximal_nontrivial $\Gamma$ $\phi$) -> bivaluation (completeness_valuation $\Gamma$).
Proof.$\;\ldots\;$Qed.
        \end{lstlisting}

        A função \texttt{completeness\_valuation} é provada como sendo uma bivaloração pelo lema \texttt{completeness\_valuation\_is\_bivaluation}, por análise de caso, mostrando que a função respeita todas as cláusulas necessárias (Definição~\ref{def:valoracoes}), como feito no Lema~\ref{lem:valoracao}.
    
        Depois disso, uma seção para a prova do lema de Lindenbaum é criada. Esta seção contém duas variáveis genéricas $\Gamma$ e $\phi$ e uma hipótese de que $\Gamma \nvdash \phi$.

        \begin{lstlisting}[name=Completeness, frame=single, language=coq]
Section Lindenbaum.

Variable ($\Gamma$ : Ensemble Formula)
          ($\phi$ : Formula).
Hypothesis Gamma_does_not_derive_phi : ~$\Gamma$ $\vdash$ $\phi$.
Fixpoint $\Gamma_i$ 
  (i : nat) (f: nat -> Formula) : Ensemble Formula :=
  match i with
  | O   => $\Gamma$
  | S n => match (strong_lem ((($\Gamma_i$ n f) $\cup$ [f n]) $\vdash$ $\phi$)) with
            | left _  => ($\Gamma_i$ n f)
            | right _ => ($\Gamma_i$ n f) $\cup$ [f n]
            end
  end.
Definition $\Delta$
  (f: nat -> Formula) : Ensemble Formula :=
fun ($\psi$ : Formula) => exists n : nat, $\psi$ $\in$ ($\Gamma_i$ n f).

        \end{lstlisting}

        A função recursiva \texttt{$\Gamma_i$} representa a função de escolha $\Gamma_i$ definida no início do Lema~\ref{lem:lindenbaum} e define um conjunto (\texttt{Ensemble}) de fórmulas. A função \texttt{f} recebida como argumento representa a organização das fórmulas da linguagem numa sequência. Já a função \texttt{$\Delta$} representa a união generalizada de todos os conjuntos definidos por \texttt{$\Gamma_i$} ($\Delta = \bigcup_{i=0}^{\infty}\Gamma_i$). 

        Então, alguns fatos sobre estes conjuntos são provados.

        \begin{lstlisting}[name=Completeness, frame=single, language=coq]
Fact $\Gamma_i$_included_$\Delta$ : 
forall (i : nat) (f : nat -> Formula),
  ($\Gamma_i$ i f) $\subseteq$ ($\Delta$ f).
Proof.$\;\ldots\;$Qed.

Fact $\Gamma_i$_does_not_derive_$\phi$ :
forall (i : nat) (f : nat -> Formula),
  ~(($\Gamma_i$ i f) $\vdash$ $\phi$).
Proof.$\;\ldots\;$Qed.

Fact $\Gamma_i$_m_included_$\Gamma_i$_n : 
forall (f : nat -> Formula) (m : nat) (n : nat), 
m <= n -> ($\Gamma_i$ m f) $\subseteq$ ($\Gamma_i$ n f).
Proof.$\;\ldots\;$Qed.
        \end{lstlisting}

        Os fatos \texttt{$\Gamma_i$\_included\_$\Delta$} e \texttt{$\Gamma_i$\_does\_not\_derive\_$\phi$} dizem, respectivamente, que todo $\Gamma_i$ está contido em $\Delta$ e que nenhum $\Gamma_i$ deriva $\phi$. Ambos são provados por indução em \texttt{i}. 
        
        O fato \texttt{$\Gamma_i$\_m\_included\_$\Gamma_i$\_n}, que diz que $\Gamma_m \subseteq \Gamma_n$, sempre que $m \leq n$, é provado utilizando a definição da função \texttt{$\Gamma_i$}.

        \begin{lstlisting}[name=Completeness, frame=single, language=coq]
Fact $\Delta$_$\Gamma_i$_con :
  forall (f : nat -> Formula) ($\delta$ : Formula), 
($\Delta$ f) $\vdash$ $\delta$ -> (exists n : nat, ($\Gamma_i$ n f) $\vdash$ $\delta$).
Proof.$\;\ldots\;$Qed.

Fact $\Delta$_does_not_derive_$\phi$ : 
  forall (f : nat -> Formula), 
  ~ ($\Delta$ f) $\vdash$ $\phi$.
Proof.$\;\ldots\;$Qed.

Fact not_in_$\Delta$_$\Gamma_i$ : forall ($\psi$ : Formula) (f : nat -> Formula),
  $\psi$ $\not \in$ ($\Delta$ f) -> forall n : nat, $\psi$ $\not \in$ ($\Gamma_i$ n f).
Proof.$\;\ldots\;$Qed.
        \end{lstlisting}
        
        O fato \texttt{$\Delta$\_$\Gamma_i$\_con} diz que se uma dada fórmula é derivável a partir de $\Delta$, então ela é derivável a partir de algum $\Gamma_i$. Ele segue por indução na hipótese \texttt{$\Delta$ f $\vdash$ $\delta$}. 
        
        O fato \texttt{$\Delta$\_does\_not\_derive\_$\phi$} demonstra a primeira condição para que $\Delta$ seja um conjunto maximal não trivial em relação a $\phi$ (Definição~\ref{def:nao-trivial_maximal}). Ele é provado utilizando as provas de \texttt{$\Gamma_i$\_does\_not\_derive\_$\phi$} e \texttt{$\Delta$\_$\Gamma_i$\_con} para mostrar que, supondo \texttt{$\Delta$ f $\vdash$ $\phi$} chega-se numa contradição. 
        
        O fato \texttt{not\_in\_$\Delta$\_$\Gamma_i$} diz que se uma dada fórmula não está em $\Delta$, então ela não está em nenhum $\Gamma_i$. Ele é demonstrado através da definição de $\Delta$.
        
        \begin{lstlisting}[name=Completeness, frame=single, language=coq]
Fact not_in_$\Gamma_i$_derives_$\phi$ : forall (f : nat -> Formula) (i : nat),
  (f i) $\not \in$ ($\Gamma_i$ (S i) f) -> ($\Gamma_i$ i f) $\cup$ [f i] $\vdash$ $\phi$.
Proof.$\;\ldots\;$Qed.

Lemma $\Delta$_maximal_nontrivial : forall (f : surjection nat Formula),
  maximal_nontrivial ($\Delta$ f) $\phi$.
Proof.$\;\ldots\;$Qed.

End Lindenbaum.
        \end{lstlisting}

        O fato \texttt{not\_in\_$\Gamma_i$\_derives\_$\phi$} diz que, se uma fórmula $\phi_i$ não está em $\Gamma_i$, entao $\Gamma_i, \phi_i \vdash \phi$. Ele segue por análise de caso em \texttt{$\Gamma_i$\ i\ f $\cup$ [f (i)] $\vdash$ $\varphi$}, utilizando o princípio do terceiro excluído.
        
        O lema \texttt{$\Delta$\_maximal\_nontrivial} segue utilizando a definição de sobrejeção\footnote{visto que, neste caso, considera-se a função que representa a organização das fórmulas numa sequência como sendo uma sobrejeção}, a definição \texttt{maximal\_nontrivial} e os fatos \texttt{$\Delta$\_does\_not\_derive\_$\varphi$}, \texttt{not\_in\_$\Gamma_i$\_derives\_$\varphi$} e \texttt{lfi1\_monotonicity}.

        Com isto, a completude da lógica \lfium{} pode enfim ser provada.

        \begin{lstlisting}[name=Completeness, frame=single, language=coq]
Axiom surjection_nat_formula : surjection nat Formula.

Theorem completeness_bivaluations : forall ($\Gamma$ : Ensemble Formula) ($\alpha$ : Formula), 
($\Gamma$ $\vDash$ $\alpha$) -> ($\Gamma$ $\vdash$ $\alpha$).
Proof.$\;\ldots\;$Qed.
  
Corollary completeness_matrix : forall ($\Gamma$ : Ensemble Formula) ($\alpha$ : Formula), 
($\Gamma$ $\vDash$m $\alpha$) -> ($\Gamma$ $\vdash$ $\alpha$).
Proof.$\;\ldots\;$Qed.

Corollary matrix_bivaluations_eq : forall ($\Gamma$ : Ensemble Formula) ($\alpha$ : Formula),
($\Gamma$ $\vDash$m $\alpha$) <-> ($\Gamma$ $\vDash$ $\alpha$).
Proof.$\;\ldots\;$Qed.
        \end{lstlisting}

        A proposição \texttt{surjection\_nat\_formula} estabelece a existência de uma sobrejeção entre \texttt{nat} e \texttt{Formula}, um fato que pode ser provado utilizando uma numeração de G{\"o}del~\cite{sep-goedel-incompleteness}, mas que para os propósitos do presente trabalho foi tomado como um axioma.

        O teorema \texttt{completeness\_bivaluations} (relativo ao Teorema~\ref{teo:completude_val}) segue pela contrapositiva\footnote{Utilizando o teorema \texttt{contraposition} (implementado na Seção~\ref{sec:utils}), que só pôde ser demonstrado visto que estamos assumindo o princípio do terceiro excluído como válido neste módulo.} ($\Gamma \nvdash \alpha \Longrightarrow \Gamma \nvDash \alpha$) e utiliza os fatos provados no lema de Lindenbaum juntamente com a definição da bivaloração \texttt{completeness\_valuation}. Este teorema nos cede dois corolários \texttt{completeness\_matrix} e \texttt{matrix\_bivaluations\_eq}, que representam, respectivamente, a completude da \lfium{} em relação à semântica matricial (Corolário~\ref{cor:completude_mat}) e a equivalência entre os sistemas semânticos (Corolário~\ref{cor:matvaleq}). Estes corolários seguem aplicando os resultados obtidos em \texttt{completeness\_bivaluations}, \texttt{soundness\_matrix} e \texttt{matrix\_bivaluation\_imp}.