\chapter{Conclusão}\label{chap:conclusao}

O estudo de lógicas paraconsistentes mostra-se relevante para o desenvolvimento de
softwares capazes de lidar com informações contraditórias. Dentro desta família de lógicas,
os sistemas de inconsistência formal destacam-se no contexto de bases de dados {---} sobretudo
bases evolucionárias {---} já que internalizam o conceito de consistência dentro da sua
linguagem. A \textbf{LFI1} é uma lógica de inconsistência formal com propriedades que facilitam o
desenvolvimento de sistemas de gerenciamento de bancos de dados, por exemplo, mesmo na
presença de contradições na base~\cite{carnielli2000formal}.

Assistentes de provas são ferramentas de software que permitem ao usuário provar
teoremas sobre objetos matemáticos expressos dentro de si, sem que a verificação destas provas dependa
de um julgamento humano para garantir sua validade. O Rocq é um assistente de provas (bem
como uma linguagem de programação) robusto e com um núcleo axiomático enxuto, que possui
aplicações em diferentes áreas da matemática, como lógica, linguagens formais, linguística
computacional e desenvolvimento de programas seguros~\cite{coqart}.

O presente trabalho iniciou-se apresentando a linguagem, a axiomatização e a semântica para a \lfium{} e desenvolvendo metateoremas que evidenciam características deste sistema. Depois disso, uma biblioteca para a lógica apresentada foi implementada em Rocq (disponível em \url{https://github.com/dipled/LFI1_Library}), com todos os artefatos sintáticos e semânticos definidos utilizando tipos indutivos. Finalmente, os metateoremas desenvolvidos anteriormente foram provados e verificados formalmente dentro da biblioteca, satisfazendo os objetivos propostos na Seção~\ref{sec:objetivos_especificos}.

Um poster feito a partir dos resultados obtidos foi apresentado na XXI edição do Encontro Brasileiro de Lógica (EBL) realizado em Serra Negra {-} SP nos dias 12 a 16 de maio de 2025.

Para trabalhos futuros, propõe-se o desenvolvimento de táticas $\pazocal{L}$tac, para automatizar a prova de teoremas utilizando a semântica matricial (visto que este é um processo algorítmico), a implementação de um sistema de \textit{tableau} (análogo ao sistema descrito em~\cite{tableaulfi}) e, por fim, desenvolver a prova da existência de uma sobrejeção de \texttt{nat} para \texttt{Formula} (apresentada na Seção~\ref{sec:completeness}), que até o presente foi tomada como axioma dentro da biblioteca.