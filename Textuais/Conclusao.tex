\chapter{Conclusões}
	\label{cap:Conclusao}
	Combinações de lógicas é um tópico relativamente novo e complexo dentro da lógica, tanto do ponto de vista filosófico quanto matemático,
	porém muito já foi estudado e desenvolvido sobre, algo que foi evidenciado no Capítulo~\ref{cap:FusoesLogicasModais} e no
	Apêndice~\ref{app:ProvaTransferenciaCompletude} deste trabalho, tal apresentação entende-se por satisfazer o primeiro objetivo deste trabalho, especificamente
	``Estudar os principais conceitos de combinações de lógicas, em especial, a fusão''. Assistentes de provas são ferramentas também novas, que permitem expressar
	grande parte da matemática em suas linguagens e que possibilitam seus usuários provarem propriedades sobre objetos expressos dentro de si. Ademais, essas provas independem da
	verificação de um humano para garantir a confiabilidade, logo, assistentes de provas são excelentes ferramentas para provar propriedades de sistemas complexos,
	como sistemas de software.

	Portanto, temos que assistentes de provas podem ser usados para modelar e verificar a corretude de sistemas lógicos resultantes da combinação de outros sistemas lógicos
	mais simples, algo que foi evidenciado com os desenvolvimentos deste trabalho. Foi possível modelar e verificar a corretude do sistema \SisT, um sistema resultante da fusão de
	dois sistemas lógicos mais simples. Entende-se que esta implementação satisfaz o segundo objetivo deste trabalho, especificamente ``Realizar um estudo de caso de fusões de lógicas
	modais no Coq''. Esta implementação não foi trivial e a prova de corretude não se deu por meio do método de transferência, porém ela indica que, pelo menos
	para casos restritos, é possível provar em Coq a corretude de sistemas lógicos resultantes da fusão.

	A principal contribuição deste trabalho foi a modelagem de sistemas sintáticos de lógicas resultantes da fusão de forma paramétrica, sendo também demonstrado que a
	fusão de sistemas sintáticos preserva derivações. Com isso, é possível representar, sintaticamente, sistemas lógicos resultantes da fusão de dois outros sistemas lógicos arbitrários.
	Entende-se que este resultado satisfaz o terceiro objetivo deste trabalho, especificamente, ``Modelar, de forma paramétrica, sistemas de lógicas multimodais
	resultantes de fusão de lógicas modais em Coq''.

	Apesar disto, a modelagem é complexa e não foi possível representar explicitamente a combinação dos sistemas semânticos de lógicas monomodais, apenas foi possível representar a
	combinação de seus sistemas sintáticos. Além do trabalho de representar os componentes do sistema lógico, é também interessante provar que estes estão corretos com relação aos
	sistemas básicos que foram combinados, algo que, devido à impossibilidade de modelar os sistemas semânticos, não foi possível de ser feito.

	Propõe-se como trabalhos futuros alterações tanto na biblioteca modal base quanto na extensão desenvolvida nesse trabalho, modificando a definição de
	frame da biblioteca base para que este seja parametrizado por algum tipo (como descrito na Seção~\ref{sec:Dificuldades}), refletir essa alteração nos n-frames da extensão
	para poder descrever a fusão de frames, lidar com a seleção de uma relação da lista de relações de uma n-frame de forma que não seja necessário utilizar a função
	\texttt{nth} ou variantes, modelar a união de assinaturas de linguagens e, por fim, provar a transferência de corretude e completude pela fusão.

% \begin{enumerate}
% 	\item Estudar os principais conceitos de combinações de lógicas, em especial, a fusão;
% 	\item Realizar um estudo de caso de fusões de lógicas modais no Coq;
% 	\item Modelar sistemas de lógicas multimodais resultantes de fusão de lógicas modais em Coq;
% \end{enumerate}

	% A preservação de propriedades pela fusão ainda está sendo estudada, logo não foi possível incluir todas as provas relevantes no seu respectivo capítulo, nem modelar
	% isso no Coq. Os resultados obtidos sobre a modelagem são positivos e indicam que é possível modelar a fusão de lógicas modais de forma genérica dentro do Coq e, possivelmente, também
	% dentro de outros assistentes de provas semelhantes ao Coq.

	% No que seguem, é apresentada uma lista de itens que propõe-se serem terminados para o TCC2, juntamente com uma tabela indicando um cronograma para a realização de cada item.

	% \begin{enumerate}
	% 	\item Terminar de provar a preservação de propriedades de lógicas multimodais resultantes de fusão;
	% 	\item Modelar a preservação de propriedades para o sistema \SisT;
	% 	\item Modelar fusão de lógicas modais de forma paramétrica
	% 	\item Modelar em Coq a prova de preservação de propriedades das lógicas multimodais resultantes de fusão
	% \end{enumerate}

	% \begin{table}[htbp]
	% 	\centering
	% 	\begin{tabular}{|c|c|c|c|c|c|c|c|c|}
	% 		\hline
	% 		\multirow{2}{*}{\textbf{\small{Etapas}}} & \textbf{\small{2022/2}} & \multicolumn{6}{c|}{\textbf{\small{2023/1}}} \\
	% 		\cline{2-8}
	% 		& \textbf{Dez} & \textbf{Jan} & \textbf{Fev} & \textbf{Mar} & \textbf{Abr} & \textbf{Maio} & \textbf{Jun} \\
	% 		\hline
	% 		\textbf{\small{1}}  & \cellcolor{gray} & \cellcolor{gray} &  &  &  &  & \\
	% 		\hline
	% 		\textbf{\small{2}}  &  & \cellcolor{gray} & \cellcolor{gray} &  &  &  & \\
	% 		\hline
	% 		\textbf{\small{3}}  &  & \cellcolor{gray} & \cellcolor{gray} & \cellcolor{gray} &  &  & \\
	% 		\hline
	% 		\textbf{\small{4}}  &  &  & \cellcolor{gray} & \cellcolor{gray} & \cellcolor{gray} & \cellcolor{gray} & \\
	% 		\hline
	% 	\end{tabular}
	% 	\caption{Cronograma Proposto para o TCC2}
	% \end{table}