\chapter{Introdução}

\noindent\label{cap:Introducao}
As lógicas paraconsistentes são uma família de lógicas na qual a presença de contradições não implica trivialidade, ou seja, são sistemas lógicos que possuem uma negação que não respeita o Princípio da Explosão\footnote{Definido como $\alpha \rightarrow (\neg \alpha \rightarrow \beta)$.}~\cite{carnielli2007}. Tradicionalmente, em lógicas ortodóxas, qualquer teoria que seja inconsistente {-} e, portanto, não respeite o Princípio da não-contradição\footnote{Definido como $\neg (\alpha \land \neg \alpha)$.} {-} será uma teoria trivial (uma teoria que possui todas as sentenças). Deste modo, as lógicas paraconsistentes surgem como uma ferramenta que permite tratar contradições sem trivializar o sistema lógico~\cite{Carnielli_Coniglio_2016}.

De acordo com~\cite{sep-logic-paraconsistent}, as motivações para o estudo de lógicas paraconsistentes podem ser observadas em diversos campos do conhecimento. Nas ciências naturais, por exemplo, teorias inconsistentes e não-triviais são comuns, como é o caso da teoria do átomo de Bohr, que, segundo~\cite{Brown2015-BROCAP-9}, deve possuir uma inferência paraconsistente. No campo da linguística, inconsistências não-triviais também são possíveis, como a preservação da noção espacial da palavra ``Próximo'' mesmo tratando-se de objetos impossíveis~\cite{McGinnis2013-MCGTUA}. Ademais, no contexto da computação, uma aplicação da paraconsistênica é o uso de lógicas de inconsistência formal para a modelagem e o desenvolvimento de bancos de dados evolucionários~\cite{carnielli2000formal}.

As lógicas de inconsistência formal (\textbf{LFI}s), são lógicas paraconsistentes que introduzem os conceitos de consistência e inconsistência como formas de representar o excesso de informações (por exemplo, evidência de $\alpha$ e evidência de $\neg \alpha$), para resgatar a capacidade de se obter a trivialidade em alguns casos~\cite{carnielli2007}. Ao explicitamente representar a consistência dentro da sua linguagem, é possivel estudar teorias inconsistentes sem necessariamente assumir que elas são triviais, porém possibilitando a trivialidade em situações específicas. A ideia por trás das \textbf{LFI}s é que deve-se respeitar as noções da lógica clássica o máximo possível, desviando desta somente na presença de contradições. Isto significa que, na ausência de contradições, o Princípio da Explosão deve ser tomado como válido~\cite{sep-logic-paraconsistent}. Segundo~\cite{Carnielli_Coniglio_2016}, na lógica \textbf{LFI1}, os conceitos de inconsistência e consistência são introduzidos à linguagem por meio do operador $\bullet$ para a incositência ou $\circ$ para a consitência. A linguagem da \textbf{LFI1} pode ser definida com qualquer um destes operadores. Isto será indicado subscrevendo-se o operador utilizado, como $\text{\textbf{LFI1}}_{\circ}$ e $\text{\textbf{LFI1}}_{\bullet}$. Desta forma, como veremos ao longo do presente trabalho, e possível resgatar a trivialidade através do Princípio da Explosão Gentil, definido, no caso da $\text{\textbf{LFI1}}_{\circ}$, como $\circ \alpha \rightarrow (\alpha \rightarrow (\neg \alpha \rightarrow \beta))$\cite{carnielli2007}. Este princípio diz que a trivialidade é obtida a partir da contradição de uma informação consistente.

%os conceitos de inconsistência e consistência são introduzidos à linguagem por meio do operador $\bullet$ para a incositência, definido\footnote{Esta definição diz que uma informação inconsistente é equivalente a uma informação contraditória.} como $\bullet \alpha \eqdef \alpha \land \neg \alpha$, ou $\circ$ para a consitência, definido\footnote{Esta definição diz que uma informação consistente é equivalente a uma informação não-contraditória.} como $\circ \alpha \eqdef \neg(\alpha \land \neg \alpha)$. A linguagem da \textbf{LFI1} pode ser definida com qualquer um destes operadores. Isto será indicado subscrevendo-se o operador utilizado, como $\text{\textbf{LFI1}}_{\circ}$ e $\text{\textbf{LFI1}}_{\bullet}$.



    \section{Objetivo Geral}


    \section{Objetivos Específicos}



    \section{Trabalhos Relacionados}


    \section{Metodologia}
        

    \section{Estrutura do Trabalho}
       