\chapter{Introdução}\label{cap:Introducao}


As lógicas paraconsistentes são uma família de lógicas na qual a presença de contradições não implica trivialidade, ou seja, são sistemas lógicos que possuem uma negação que não respeita o princípio da explosão, definido como $\alpha \to (\neg \alpha \to \beta)$~\cite{carnielli2007}. Tradicionalmente, em lógicas ortodoxas, qualquer teoria que seja inconsistente {--} e, portanto, não respeite o Princípio da Não-Contradição, definido como $\neg (\alpha \land \neg \alpha)$ {--} será uma teoria trivial (uma teoria que contem todas as sentenças). Deste modo, as lógicas paraconsistentes surgem como uma ferramenta que permite tratar contradições sem trivializar o sistema lógico~\cite{Carnielli_Coniglio_2016}.

De acordo com~\citeshort{sep-logic-paraconsistent}, as motivações para o estudo de lógicas paraconsistentes podem ser observadas em diversos campos do conhecimento. Nas ciências naturais, por exemplo, teorias inconsistentes e não-triviais são comuns, como é o caso da teoria do átomo de Bohr, que, segundo~\citeshort{Brown2015-BROCAP-9}, deve possuir um mecanismo de inferência paraconsistente\footnote{De acordo com a teoria, um elétron orbita o núcelo do átomo sem radiar energia. Porém, de acordo com as equações de Maxwell, que compõem parte da teoria de Bohr, um elétron que está acelerando em órbita deve radiar energia. Estes fatos são inconsistentes entre si, entretanto, não é possível inferir \textit{tudo} sobre o comportamento dos elétrons a partir disso. Portanto, o mecanismo de inferência deve se tratar de um mecanismo paraconsistente.}. No campo da linguística, inconsistências não-triviais também são possíveis, como a preservação da noção espacial da palavra ``Próximo'' mesmo tratando-se de objetos impossíveis\footnote{Por exemplo, na sentença ``Adam está próximo de um cubo esférico'', a noção espacial entre Adam e um objeto impossível é preservada~\cite{McGinnis2013-MCGTUA}.} Ademais, no contexto da computação, uma aplicação da paraconsistência é o uso de lógicas de inconsistência formal para a modelagem e o desenvolvimento de bancos de dados evolucionários~\cite{carnielli2000formal}.

As lógicas de inconsistência formal (\lfis{}) são lógicas paraconsistentes que introduzem na sua linguagem os conceitos de consistência e inconsistência como formas de representar o excesso de informações (por exemplo, evidência para $\alpha$ e evidência para $\neg \alpha$), para resgatar a capacidade de se obter a trivialidade em alguns casos~\cite{carnielli2007}. Ao explicitamente representar a consistência dentro da sua linguagem, é possível estudar teorias inconsistentes sem necessariamente assumir que elas são triviais, porém possibilitando a trivialidade em situações específicas. A ideia por trás das \lfis{} é que deve-se respeitar as noções da lógica clássica o máximo possível, desviando desta somente na presença de contradições. Isto significa que, na ausência de contradições, o princípio da explosão deve ser tomado como válido~\cite{sep-logic-paraconsistent}. Segundo~\citeshort{Carnielli_Coniglio_2016,new_advances_lfi}, na lógica \lfium{}, uma lógica paraconsistente e trivalorada, os conceitos de inconsistência e consistência são introduzidos à linguagem por meio do operador $\bullet$ para a inconsistência ou $\circ$ para a consistência, sendo que qualquer um destes pode ser usado para definir a linguagem da \lfium{}. Desta forma, como veremos ao longo do presente trabalho, e possível resgatar a trivialidade através do princípio da explosão Gentil, definido, no caso da $\text{\lfium{}}$, como $\circ \alpha \to (\alpha \to (\neg \alpha \to \beta))$~\cite{carnielli2007}. Este princípio diz que a trivialidade é obtida a partir da contradição de uma informação consistente.


Um sistema lógico capaz de lidar com informações inconsistentes é de grande interesse no campo da computação, sobretudo no gerenciamento de bancos de dados~\cite{carnielli2000formal}. Um banco de dados pode ser definido como um conjunto estruturado de relações finitas que armazena informações. Estas informações precisam satisfazer condições conhecidas como restrições de integridade antes de serem inseridas no banco~\cite{Codd}. As restrições são definidas pelo projetista do banco de dados no momento da implementação e podem ser formalizadas como sentenças de primeira ordem fixas~\cite{carnielli2000formal}. Conforme o banco de dados evolui, é preciso atualizar as informações contidas para refletir a realidade, contudo, como informações contraditórias não são permitidas pelas restrições de integridade, isso torna o processo de atualização difícil e trabalhoso. Ademais, a existência de bancos de dados que possam alterar suas restrições de integridade com o passar do tempo (conhecidos como bancos de dados evolucionários) é outro ponto de interesse que pode ser explorado com o uso das \lfis{}.

Concomitante aos estudos das lógicas paraconsistentes, avanços nas áreas da computação e da matemática {--} como a definição de teoria de tipos por Russell~(\citeyear{russell1903principles,Russell1908-RUSMLA}), a formulação desta teoria com base na sintaxe do Cálculo-$\lambda$ por~\citeshort{church1940formulation} e o descobrimento da Correspondência de Curry-Howard por~\citeshort{curry1958combinatory,howard1980formulae} {--} possibilitaram o desenvolvimento de assistentes de provas~\cite{harrison2014history}. Assistentes de provas são ferramentas da área de verificação formal, que buscam garantir que um programa está correto de acordo com uma especificação formal. Isto é feito a partir de provas desenvolvidas utilizando métodos matemáticos para a correção de propriedades de um \textit{software}~\cite{Chlipala_2013}. Tradicionalmente, a verificação da validade de provas é feita manualmente por avaliadores, que seguem o raciocínio do autor e dão um veredito baseado no quão convincente a prova é. Os assistentes de provas surgem como alternativas à verificação manual, possibilitando ao matemático {--} ou programador {--} verificar provas na medida em que elas são desenvolvidas, tornando este processo mais fácil e confiável~\cite{paulinmohring:hal-01094195}.

Assistentes de provas como Coq, Lean e Isabelle permitem ao usuário definir e provar propriedades sobre objetos matemáticos com valor computacional~\cite{geuvers2009proof}. No presente trabalho será utilizado o Coq, este que utiliza o Cálculo de Construções Indutivas como formalismo para o desenvolvimento de provas~\cite{TEAM_2024}. O Coq ganhou notoriedade como ferramenta de verificação formal após seu uso na prova de correção de diversos teoremas e sistemas computacionais complexos, como a prova do teorema das quatro cores~\cite{geuvers2009proof}, a certificação de um compilador para a linguagem de programação C~\cite{leroy2021compcert} e a prova da correção do algoritmo união-busca~\cite{union-find}.

A proposta deste trabalho é desenvolver uma biblioteca da lógica de inconsistência formal \lfium{} em Coq, de maneira análoga como foi feito para a lógica modal por~\citeshort{silveira2020implementacao}. Após a implementação da biblioteca, propõe-se que sejam provados metateoremas relevantes para a \lfium{} utilizando o Coq.
%os conceitos de inconsistência e consistência são introduzidos à linguagem por meio do operador $\bullet$ para a incositência, definido\footnote{Esta definição diz que uma informação inconsistente é equivalente a uma informação contraditória.} como $\bullet \alpha \eqdef \alpha \land \neg \alpha$, ou $\circ$ para a consitência, definido\footnote{Esta definição diz que uma informação consistente é equivalente a uma informação não-contraditória.} como $\circ \alpha \eqdef \neg(\alpha \land \neg \alpha)$. A linguagem da \lfium{} pode ser definida com qualquer um destes operadores. Isto será indicado subscrevendo-se o operador utilizado, como $\text{\lfium{}}_{\circ}$ e $\text{\lfium{}}_{\bullet}$.



    \section{Objetivo Geral}
        O objetivo geral deste trabalho é implementar uma biblioteca da \lfium{} em Coq, assim como desenvolver provas da completude, da correção e do metateorema da dedução dentro da biblioteca.


    \section{Objetivos Específicos}
        \begin{itemize}
            \item Estudar conceitos relevantes sobre lógicas paraconsistentes, em especial a \lfium{};
            \item Estudar e revisar as provas manuais para completude, correção e metateorema da dedução da \lfium{};
            \item Realizar um levantamento do estado da arte do desenvolvimento de lógicas paraconsistentes em assistentes de provas;
            \item Desenvolver uma biblioteca da \lfium{} em Coq, baseada na semântica e sintaxe previamente definidas;
            \item Desenvolver e verificar formalmente as provas para completude, correção e metateorema da dedução em Coq.
        \end{itemize}


    \section{Trabalhos Relacionados}
        A partir de um levantamento acerca do estado da arte do desenvolvimento de lógicas não-clássicas em computação na literatura, foram encontrados alguns trabalhos relacionados.
    
        Em~\cite{Villadsen2017}, os autores implementam uma biblioteca de uma lógica paraconsistente utilizando assistente de provas Isabelle. A lógica em questão possui uma quantidade infinita contável de valores verdades não-clássicos, sendo uma generalização da lógica trivalorada proposta por {\L}ukasiewicz, como definida por~\citeshort{sep-lukasiewicz}. Além de implementar a biblioteca, são provados metateoremas sobre esta lógica, como o número mínimo de valores verdades a serem analisados para determinar o valor verdade de uma fórmula e os metateoremas da redução e da dedução. Em outro trabalho,~\citeshort{schlichtkrull:LIPIcs.TYPES.2018.5} prova alguns outros metateoremas para esta mesma lógica dentro do assistente Isabelle, expandindo ainda mais a biblioteca já implementada.

        
        \citeshort{pdatalog} especificam uma linguagem de consulta a banco de dados baseada na lógica de inconsistência formal \lfium{}. A linguagem em questão, chamada de P{-}Datalog, é uma extensão paraconsistente de outra linguagem chamada $\text{Datalog}^{\neg}$, esta, por sua vez, se trata de uma linguagem dedutiva\footnote{Uma linguagem dedutiva de consulta é capaz de concluir fatos adicionais a partir do que está armazenado num determinado banco de dados~\cite{datalog}.} de consulta baseada na lógica de primeira ordem clássica. No trabalho é apresentada uma semântica bem-fundada para P{-}Datalog, bem como um método de avaliação \textit{bottom{-}up}, baseado numa computação de ponto fixo alternante (descrito em~\citeshort{fixpoint}). 
        
        Em~\citeshort{paralog}, os autores descrevem uma extensão da linguagem de programação ParaLog, chamada de ParaLog\_\textit{e}, que se propõe a mesclar conceitos de programação lógica clássica com conceitos de inconsistência, a fim de ampliar o escopo de programação lógica em ambientes com informações contraditórias. A linguagem ParaLog\_\textit{e} utiliza como base a lógica evidencial\footnote{Uma revisão sobre esta lógica pode ser encontrada em~\citeshort{DECARVALHOJUNIOR2024107342}.}, uma lógica paraconsistente com uma quantidade infinita (não-contável) de valores verdades, membros de $\{x \in \mathbb{R} \; | \; 0 \leq x \leq 1\} \times \{x \in \mathbb{R} \; | \; 0 \leq x \leq 1\}$ \migs{Essa frase ficou meio estranha, mas não consigo pensar numa maneira de reescrever ela melhor agora.}. No trabalho, os autores definem a sintaxe e a semântica da linguagem e apresentam exemplos de programas escritos nela.
    
    \section{Metodologia}\label{sec:metodologia}
      O desenvolvimento do presente trabalhou se iniciou com uma pesquisa bibliográfica envolvendo os fundamentos teóricos acerca da lógica tratada e uma análise das provas dos metateoremas da \lfium{} existentes na literatura. Depois disso, foi feito um levantamento do estado da arte de implementações de lógicas paraconsistentes em assistentes de provas diversos, bem como aplicações da paraconsistência no contexto da computação.

     Com a finalização desta etapa, foi iniciada a escrita do texto e o desenvolvimento das provas manuais dos metateoremas da \lfium{} com base na literatura pesquisada.
        

    \section{Estrutura do Trabalho}\label{sec:estrutura}
        O trabalho está organizado como segue: no Capítulo~\ref{cap:LFIs}, as lógicas de inconsistência formal são definidas e caracterizadas em relação a suas propriedades e motivações para seu desenvolvimento. No Capítulo~\ref{cap:LFI1}, a lógica \lfium{} é apresentada e sua linguagem, sintaxe e semântica são definidas formalmente, além disso, alguns metateoremas relevantes são provados. No Capítulo~\ref{cap:implementacao}, há uma breve contextualização sobre o desenvolvimento de provas e verificação de \textit{software} em assistentes de provas, com foco no Coq. 
        
    


       