\chapter{Introdução} 
\label{cap:Introducao}
As lógicas paraconsistentes são uma família de lógicas na qual a presença de contradições não implica trivialidade, ou seja, são sistemas lógicos que possuem uma negação que não respeita o Princípio da Explosão\footnote{Definido como $\alpha \rightarrow (\neg \alpha \rightarrow \beta)$.}~\cite{carnielli2007}. Tradicionalmente, em lógicas ortodóxas, qualquer teoria que seja inconsistente {-} e, portanto, não respeite o Princípio da não-contradição\footnote{Definido como $\neg (\alpha \land \neg \alpha)$.} {-} será uma teoria trivial (uma teoria que possui todas as sentenças). Deste modo, as lógicas paraconsistentes surgem como uma ferramenta que permite tratar contradições sem trivializar o sistema lógico~\cite{Carnielli_Coniglio_2016}.

De acordo com~\cite{sep-logic-paraconsistent}, as motivações para o estudo de lógicas paraconsistentes podem ser observadas em diversos campos do conhecimento. Nas ciências naturais, por exemplo, teorias inconsistentes e não-triviais são comuns, como é o caso da teoria do átomo de Bohr, que, segundo~\cite{Brown2015-BROCAP-9}, deve possuir uma inferência paraconsistente. No campo da linguística, inconsistências não-triviais também são possíveis, como a preservação da noção espacial da palavra ``Próximo'' mesmo tratando-se de objetos impossíveis~\cite{McGinnis2013-MCGTUA}. Uma aplicação de lógicas paraconsistentes no contexto da ciência da computação é o uso de lógicas de inconsistência formal (\textbf{LFI}s) para a modelagem e o desenvolvimento de bancos de dados evolucionários~\cite{carnielli2000formal}.



    \section{Objetivo Geral}


    \section{Objetivos Específicos}



    \section{Trabalhos Relacionados}


    \section{Metodologia}
        

    \section{Estrutura do Trabalho}
       