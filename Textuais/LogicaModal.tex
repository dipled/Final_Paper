\chapter{Lógica Modal}
    \label{cap:LogicaModal}

    Lógica modal é a lógica que trata do conceito de \textit{modalidades}, estes são operadores lógicos que qualificam
    verdades sobre proposições, segundo~\citeshort{goldblatt1993mathematics}.
    Por exemplo, podemos afirmar que uma certa sentença \textit{deve} ser verdadeira, \textit{pode} ser verdadeira,
    \textit{sempre será} verdadeira, dentre muitos outros. De acordo com~\citeauthoronline{zalta1995basic} (\citeyear{zalta1995basic}) os operadores de necessidade e possibilidade são chamados de
    ``modais'' pois eles especificam o modo como as subsequentes proposições devem ser interpretadas.
    Já~\citeshort{blackburn2001modal} afirmam que, se perguntar para três lógicos o que é lógica modal, os três darão respostas diferentes.

    Segundo~\citeshort{gabbay2003many}, o início do estudo de lógicas modais se dá com os estudos de Lewis~\cite{lewis1918survey,lewis1959symbolic},
    na tentativa dos autores de definir o conceito de \textit{implicação estrita}, esta
    seria uma implicação cuja interpretação é mais próxima da noção de linguagem natural do que da noção matemática. A implicação matemática, ou booleana
    (chamadas de implicação material pelo autor), ``se \PHI então \PSI'' seria substituída por ``é necessário que, se \PHI então \PSI'', seu
    objetivo com isso era resolver certos paradoxos que a implicação material traz. Por exemplo, a frase ``Se a Lua é feita de queijo, então \(2 \times 2 = 4\)''
    é uma implicação material correta, visto que \(2 \times 2\) de fato é igual a 4, apesar da frase ``A Lua é feita de queijo'' ser absurda. Já a frase
    ``É necessário que, se a Lua é feita de queijo então \(2 \times 2 = 4\)'' não é uma implicação estrita correta, visto que é fato que \(2 \times 2 = 4\)
    apesar da Lua não ser feita de queijo\footnote{Podemos representar essa frase na notação que será apresentada à frente da seguinte forma:
    \(\Box(p \to q)\), onde \(p \mapsto \text{``A lua é feita de queijo''}, q \mapsto ``2 \times 2 = 4\text{''}\).}.

    Para estudar implicação estrita, Lewis definiu cinco sistemas axiomáticos S1,\ldots,S5 independentemente de outros matemáticos contemporâneos.
    Destes, os sistemas S4 e S5 tornaram-se amplamente conhecidos dentro da lógica modal, não por causa dos trabalhos de Lewis, mas sim pelos trabalhos
    de~\citeshort{godel1986interpretation} e~\citeshort{orlov1928calculus}, em que ambos tentavam interpretar a lógica intuicionista dentro da lógica clássica.
    O estudo da semântica da lógica modal se iniciou com Carnap~\cite{carnap1942introduction}, porém, de acordo com~\citeshort{zalta1995basic}, havia
    um erro na definição de Carnap, em específico, repetições de modalidades não afetavam a interpretação semântica de uma fórmula. Por exemplo, a fórmula
    ``Necessariamente necessariamente \PHI'' seria semanticamente equivalente a ``Necessariamente \PHI''.

    Foi apenas com Kripke~\cite{kripke1959completeness,kripke1963semantical} que se obteve uma formulação definitiva da semântica da lógica modal.
    A semântica definida (conhecida como semântica de Kripke, ou semântica de Mundos Possíveis) lida com mundos e relações de acessibilidade entre eles.
    Nessa definição semântica, a validade de fórmulas depende de mundos. Em específico, fórmulas que não envolvem modalidades dependem apenas de um
    mundo ``local'', já fórmulas que contêm modalidades dependem dos mundos acessíveis pelo mundo local.

    A lógica modal que contém apenas as modalidades duais de necessidade (representada por \BOX) e possibilidade (representada por \DIA)
    é chamada de lógica modal alética. Essas não são as únicas modalidades que podem ser expressas em lógicas modais, conceitos como tempo,
    moralidade e conhecimento também podem ser expressos em uma linguagem modal.
    Caso uma linguagem modal contenha várias modalidades não duais, ela é chamada de lógica multimodal. Por exemplo uma lógica alética
    com as modalidades \BOXi{1}, \BOXi{2} e \BOXi{3} e suas respectivas duais \DIAi{1}, \DIAi{2} e \DIAi{3} ou uma lógica que contém as
    modalidades \DIA alética e a modalidade \(\mathbf{O}\) de obrigatoriedade da lógica modal deôntica.

    Uma lógica pode ser entendida como uma tripla \(\mathcal{L} = \langle \mathsf{A}, \vDash_{\mathsf{A}}, \vdash_{\mathsf{A}} \rangle\)
    onde \(\mathsf{A}\) é a linguagem (conjunto de todas as fórmulas bem formadas) de \Mathcal{L}, \(\vDash_{\mathsf{A}}\) é uma relação de consequência
    semântica definida sobre \(\mathsf{A}\) e \(\vdash_{\mathsf{A}}\) é uma relação de consequência sintática definida sobre \(\mathsf{A}\).
    Podemos omitir o subscrito \(\mathsf{A}\) por simplicidade. Relações de consequência permitem estabelecer uma noção de consequência (ou derivação) entre conjuntos
    (possivelmente vazios) de fórmulas e fórmulas, com base em certas regras semânticas ou sintáticas. Está definição é semelhante à definição apresentada
    por~\citeshort{carnielli2008analysis}.

    Neste capítulo, será apresentada a lógica modal normal, de forma sucinta, focando na lógica monomodal alética, que será simplesmente chamada de lógica modal.
    Na Seção~\ref{sec:LM-Linguagem} será definida a linguagem da lógica modal, na Seção~\ref{sec:LM-Semantica} será definida a semântica de mundos possíveis e
    algumas propriedades desta serão descritas, na Seção~\ref{sec:LM-Axiomatizacao} será descrita uma axiomatização para a lógica modal, na
    Seção~\ref{sec:LM-Correspondencia} são apresentados outros sistemas de lógica modal, assim como o conceito de correspondência de frames, na
    Seção~\ref{sec:LM-MetaPropriedades} serão apresentadas algumas meta propriedades de lógicas modais, na Seção~\ref{sec:LM-Multimodais} será apresentado o conceito
    de lógicas multimodais, sua linguagem, semântica e axiomatização, na Seção~\ref{sec:LM-OutrasModais} serão descritas outras lógicas modais e, por fim, na
    Seção~\ref{sec:LM-LogicaComoConjunto} será apresentado o conceito de lógica modal visto como um conjunto de fórmulas.

    No que segue, usaremos letras gregas minúsculas \(\gamma, \phi, \psi, \dots\) para representar fórmulas bem formadas de uma linguagem e letras gregas maiúsculas
    \(\Gamma, \Delta, \Sigma, \dots\) para representar conjuntos de fórmulas.

    \section{Linguagem}
        \label{sec:LM-Linguagem}
        A linguagem \(\mathsf{LM}\) da lógica modal apresentada é baseada nas definições de~\citeshort{gabbay2003many} e~\citeshort{chellas1980modal},
        onde \(\mathbb{P} = \{p_0, p_1, \dots\}\) é um conjunto contável de átomos, \(\top\) é a constante lógica de verdade,
        \(\bot\) é a constante lógica de falsidade e a linguagem é definida sobre a assinatura (conjunto de conectivos)
        \(\mathsf{C} = \{\Box, \Diamond, \neg, \land, \lor, \to\}\).
        Temos então a seguinte definição:

        \begin{definicao}[Linguagem da Lógica Modal]
            \label{def:LinguagemModal}
            A \textit{linguagem da lógica modal} é definida indutivamente como o menor conjunto que satisfaz as seguintes regras:
            \begin{align*}
                & \top, \bot \in \mathsf{LM}  \\
                & \mathbb{P} \subseteq \mathsf{LM} \\
                & \text{Se } \phi \in \mathsf{LM} \text{, então } \circ \phi \in \mathsf{LM}, \text{ sendo } \circ \in \{\Box, \Diamond, \neg\} \\
                & \text{Se } \phi, \psi \in \mathsf{LM} \text{, então } \phi \circ \psi \in \mathsf{LM}, \text{ sendo } \circ \in \{\land, \lor, \to\} \tag*\qed
            \end{align*}
        \end{definicao}

        Podemos utilizar uma assinatura mais simples para a lógica, onde outros conectivos podem ser representados a partir da seguintes
        dualidades:

        \begin{definicao}[Dualidades entre Conectivos]
            \label{def:Dualidade}
            Os conectivos apresentados anteriormente respeitam as seguintes dualidades:\\

            \begin{tabular}[htpb]{>{$}l<{$} >{$}c<{$} >{$}l<{$} >{$}c<{$} l}
                \Diamond \phi   & \eqdef & \neg \Box \neg \phi & \qquad \qquad        & Dualidade \BOX e \DIA \\
                \phi \to \psi   & \eqdef & \neg \phi \lor \psi &                      & Dualidade \(\to\) e \(\lor\) \\
                \neg \phi       & \eqdef & \phi \to \bot       & \qquad \qquad \qquad & Dualidade \(\neg\) e \(\bot\) \\
                \phi \land \psi & \eqdef & \neg (\neg \phi \lor \neg \psi) &          & Dualidade \(\land\) e \(\lor\)
            \end{tabular}
            \qed
        \end{definicao}

        Segundo~\citeshort{chellas1980modal}, o conjunto de subfórmulas de uma dada fórmula \PHI é o conjunto de todas as fórmulas \PSI que compõe \PHI.
        Temos então a seguinte definição:

        \begin{definicao}[Subfórmulas]
            O conjunto de \textit{subfórmulas} de uma fórmula \PHI, denotado por \(\funcao{Sub}(\phi)\), é definido indutivamente por:
            \begin{align*}
                &\funcao{Sub}(p_i) = \{p_i\}, p_i \in \mathbb{P} \\
                &\funcao{Sub}(\top) = \{\top\} \\
                &\funcao{Sub}(\bot) = \{\bot\} \\
                &\funcao{Sub}(\circ \phi) = \{\circ \phi\} \cup \funcao{Sub}(\phi), \circ \in \{\Box, \Diamond, \neg\} \\
                &\funcao{Sub}(\phi \circ \psi) = \{\phi \circ \psi\} \cup \funcao{Sub}(\phi) \cup \funcao{Sub}(\psi), \circ \in \{\land, \lor, \to\} \tag*\qed
            \end{align*}
        \end{definicao}

    \section{Semântica}
        \label{sec:LM-Semantica}
        Semanticamente, a lógica modal pode ser apresentada de forma algébrica, como em~\citeshort{roggia2012fusion} e~\citeshort{blackburn2001modal}, ou
        de forma relacional pela Semântica de Mundos Possíveis, como em~\citeshort{chellas1980modal},~\citeshort{zalta1995basic} e~\citeshort{gabbay2003many}.
        No presente trabalho usaremos a semântica de mundos possíveis. De acordo com~\citeshort{dewind2001modal}, o conceito de mundos possíveis que Kripke
        definiu vêm de ideias de Leibniz, que distinguia verdades em nosso mundo e verdades em todos os mundos possíveis de serem criados. Nessa linha de pensamento,
        para alguma proposição ser dita necessariamente verdadeira ela deve ser verdadeira em \textit{todos} os mundos possíveis. %A noção
        % de Kripke é levemente diferente, os mundos de Kripke estão relacionados entre si, logo, uma proposição é dita necessariamente
        % verdadeira, em um dado mundo \textit{x}, se essa proposição é verdadeira em todos os mundos que estão relacionados com \textit{x}.

        Formalmente, sendo \Mathcal{W} um conjunto de mundos e \Mathcal{R} uma relação binária definida sobre \Mathcal{W} denominada relação de acessibilidade,
        isto é \(\mathcal{R} \subseteq \mathcal{W} \times \mathcal{W}\). Dados \(w_0, w_1 \in \mathcal{W}\),
        a expressão \(w_0 \mathcal{R} w_1\) é lida como ``\(w_0\) está relacionado com \(w_1\)''. Outras interpretações podem chamar \Mathcal{R}
        de relação de alternatividade, relatividade ou vizinhança. %O par \(\langle \mathcal{W}, \mathcal{R} \rangle\) formam uma estrutura chamada de frame.

        \begin{definicao}[Frames]
            \textit{Um frame é um par} \(\mathcal{F} = \langle \mathcal{W}, \mathcal{R} \rangle\), onde \(\mathcal{W} \neq \emptyset\) e
            \(\mathcal{R} \subseteq \mathcal{W} \times \mathcal{W}\). \qed
        \end{definicao}

        Para atribuir valores verdade a fórmulas em mundos, devemos definir outra estrutura chamada de modelo\footnote{Alguns autores
        chama essa estrutura de ``Modelos de Kripke''.}, que associa frames a funções de valoração.

        \sloppy
        \begin{definicao}[Modelos]
            \textit{Modelos são pares de frames e funções de valoração} da forma
            \({\mathcal{M} = \langle \mathcal{F}, \mathcal{V} \rangle}\), onde \Mathcal{F} é um frame e \Mathcal{V} é uma função total binária
            definida por \(\mathcal{V}: \mathbb{P} \to 2^{\mathcal{W}}\), sendo \(2^{\mathcal{W}}\) o conjunto
            das partes de \Mathcal{W}. Caso não seja necessário explicitar o frame de um modelo, podemos apresentar modelos como
            \(\mathcal{M} = \langle \mathcal{W}, \mathcal{R}, \mathcal{V} \rangle\). \qed
        \end{definicao}

        A função \Mathcal{V} descreve o conjunto de mundos onde um dado átomo é interpretado como verdadeiro. Por exemplo,
        sendo \(\mathcal{W} = \{w_0, \dots, w_{10}\}\), \(\mathcal{V}(p_0) = \{w_0, w_1, w_{10}\}\) indica que o átomo \(p_0\) é verdadeiro
        nos mundos \(w_0, w_1 \text{ e } w_{10}\), já \(\mathcal{V}(p_1) = \emptyset\) indica que o átomo \(p_1\) não é verdadeiro em qualquer mundo.

        \begin{exemplo}[Frames, Modelos e Classes]
            \label{exe:FramesModelosClasses}
            Sendo \(\mathcal{W} = \langle w_0, w_1 \rangle\) um conjunto de mundos e \(\mathcal{R} = \{ \langle w_0, w_0 \rangle, \langle w_0, w_1 \rangle\}\)
            uma relação de acessibilidade definida sobre \Mathcal{W}. Podemos definir um frame
            \(\mathcal{F} = \langle \mathcal{W}, \mathcal{R} \rangle\) e uma função de valoração \Mathcal{V} como
            \(\mathcal{V}(p_0) = \{w_0, w_1\}, \mathcal{V}(p_1) = \{w_1\}, \mathcal{V}(p_i) = \emptyset, i \geq 3 \), logo, podemos
            construir um modelo \(\mathcal{M} = \langle \mathcal{F}, \mathcal{V} \rangle\).

            Podemos definir uma classe de frames \MathfrakI{F}{2} como a classe de todos os frames \(\langle \mathcal{W}, \mathcal{R} \rangle\) onde
            \(|\mathcal{W}| = 2\), logo \(\mathcal{F} \in \mathfrak{F}^2\). Também podemos definir uma classe de modelos \MathfrakI{M}{2} como
            a classe de todos os modelos \(\langle \mathcal{F}', \mathcal{V}' \rangle\) onde \(\mathcal{F}' \in \mathfrak{F}^2\), logo
            \(\mathcal{M} \in \mathfrak{M}^2\). \qed
        \end{exemplo}

        Podemos então definir valoração de fórmulas em mundos de um dado modelo, como apresentado em~\citeshort{chellas1980modal}.

        \begin{definicao}[Valoração de Fórmulas]
            Sendo \(\mathcal{M} = \langle \mathcal{W}, \mathcal{R}, \mathcal{V} \rangle\) e \(w_0 \in \mathcal{W}\),
            denotaremos por \(\mathcal{M}, w_0 \Vdash \phi\) ou \(w_0 \Vdash_\mathcal{M} \phi\) o fato que a fórmula \PHI é válida
            no mundo \(w_0\) do modelo \Mathcal{M} e definiremos por:
            \begingroup
            \allowdisplaybreaks
            \begin{align*}
                \mathcal{M}, w_0 & \Vdash \top  \\
                \mathcal{M}, w_0 & \nVdash \bot \\
                \mathcal{M}, w_0 & \Vdash p_i \text{ sse } w_0 \in \mathcal{V}(p_i), \text{ para } p_i \in \mathbb{P} \\
                \mathcal{M}, w_0 & \Vdash \neg \phi \text{ sse } \mathcal{M}, w_0 \nVdash \phi \\
                \mathcal{M}, w_0 & \Vdash \phi \land \psi \text{ sse } (\mathcal{M}, w_0 \Vdash \phi \text{ e } \mathcal{M}, w_0 \Vdash \psi) \\
                \mathcal{M}, w_0 & \Vdash \phi \lor \psi \text{ sse } (\mathcal{M}, w_0 \Vdash \phi \text{ ou } \mathcal{M}, w_0 \Vdash \psi) \\
                \mathcal{M}, w_0 & \Vdash \phi \to \psi \text{ sse } (\mathcal{M}, w_0 \nVdash \phi \text{ ou } \mathcal{M}, w_0 \Vdash \psi) \\
                \mathcal{M}, w_0 & \Vdash \Box \phi \text{ sse } \forall w_1 \in \mathcal{W}, (w_0 \mathcal{R} w_1 \to
                                 \mathcal{M}, w_1 \Vdash \phi ) \\
                \mathcal{M}, w_0 & \Vdash \Diamond \phi \text{ sse } \exists w_1 \in \mathcal{W}, (w_0 \mathcal{R} w_1 \land
                                 \mathcal{M}, w_1 \Vdash \phi ) \tag*\qed
            \end{align*}
            \endgroup
        \end{definicao}

        Podemos estender a noção de validade de fórmulas para (classes de) frames e modelos, como definido em~\citeshort{dewind2001modal}.

        \begin{definicao}[Validade em Modelo]
            \label{def:ValidadeModelo}
            Uma fórmula \PHI é dita \textit{válida em um modelo} \(\mathcal{M} = \langle \mathcal{W}, \mathcal{R}, \mathcal{V} \rangle\),
            denotado por \(\mathcal{M} \Vdash \phi\), se, e somente se, \(\forall w \in \mathcal{W}, \ \mathcal{M}, w \Vdash \phi\). \qed
        \end{definicao}

        \begin{definicao}[Validade em Classes de Modelos]
            \label{def:ValidadeClasseModelo}
            Uma fórmula \PHI é dita \textit{válida em uma classe de modelos} \(\mathfrak{M}\),
            denotado por \(\mathfrak{M} \Vdash \phi\), se, e somente se, \(\forall \mathcal{M} \in \mathfrak{M}, \ \mathcal{M} \Vdash \phi\). \qed
        \end{definicao}

        \begin{definicao}[Validade em Frame]
            \label{def:ValidadeFrame}
            Uma fórmula \PHI é dita \textit{válida em um frame} \(\mathcal{F} = \langle \mathcal{W}, \mathcal{R} \rangle\),
            denotado por \(\mathcal{F} \Vdash \phi\), se, e somente se, \(\forall \mathcal{M} = \langle \mathcal{F}, \mathcal{V} \rangle,
            \ \mathcal{M} \Vdash \phi\). \qed
        \end{definicao}

        \begin{definicao}[Validade em Classes de Frames]
            \label{def:ValidadeClasseFrame}
            Uma fórmula \PHI é dita \textit{válida em uma classe de frames} \(\mathfrak{F}\),
            denotado por \(\mathfrak{F} \Vdash \phi\), se, e somente se, \(\forall \mathcal{F} \in \mathfrak{F}, \ \mathcal{F} \Vdash \phi\). \qed
        \end{definicao}

        As definições de satisfazibilidade de fórmulas em estruturas são análogas as definições de validade, mas com alterações na quantificação, então, por exemplo,
        uma fórmula é dita satisfeita em um modelo se ela é verdadeira em algum mundo do modelo, já uma fórmula é dita satisfeita numa classe de modelos se é satisfeita
        em algum modelo da classe. Um conjunto de fórmulas é dito satisfeito/válido em alguma estrutura se toda fórmula deste conjunto é satisfeita/válida nesta estrutura.

        \begin{definicao}[Fórmulas Válidas]
            \label{def:FormulaValida}
            Uma \textit{fórmula \PHI é dita válida}, denotado por \(\Vdash \phi\), se, e somente se, \({\mathcal{F} \Vdash \phi}\) para qualquer frame \Mathcal{F}. \qed
        \end{definicao}

        % \begin{exemplo}[Valorações em Frames, Modelos e Classes]
        %     \label{exe:ValoracaoFrameModeloClasse}
        %    Para esse exemplo, usaremos as estruturas definidas no Exemplo~\ref{exe:FramesModelosClasses}. No modelo \Mathcal{M}, algumas valorações e validades
        %    corretas são: \(\mathcal{M}, w_0 \Vdash p_0 \lor p_3; \mathcal{M}, w_0 \Vdash \neg \Box p_3; \mathcal{M}, w_1 \Vdash \Box p_{10};
        %    \mathcal{M} \Vdash \Box p_0\).
        % \end{exemplo}

        Por fim, temos a consequência semântica, como descrito em~\citeshort{dewind2001modal}:

        \begin{definicao}[Relação de Consequência Semântica]
            Sendo \Mathcal{S} uma relação definida sobre \(2^{\mathsf{LM}} \times \mathsf{LM}\) e sendo \Mathfrak{M} uma classe de modelos,
            \Mathcal{S} será dita uma relação de consequência semântica onde, para todo \(\Gamma \subseteq \mathsf{LM}\) e \(\phi \in \mathsf{LM}\),
            \(\langle \Gamma, \phi \rangle \in \mathord{\mathcal{S}}\) se, e somente se,
            se para todo \(\gamma \in \Gamma, \mathcal{M} \Vdash \gamma\), então \(\mathcal{M} \Vdash \phi\), em qualquer modelo \(\mathcal{M} \in \mathfrak{M}\).
            Denotaremos consequências por \(\Gamma \vDash \phi\) ou \(\Gamma \vDash_{\mathfrak{M}} \phi\).
            Caso \(\Gamma = \emptyset\) escreveremos \(\vDash \phi\) ou \(\vDash_{\mathfrak{M}} \phi\) e \PHI será dita uma tautologia na classe \Mathfrak{M}. \qed
        \end{definicao}

        % No que segue, usaremos as notações \(\mathcal{S} \vDash \phi\) e \(\vDash_\mathcal{S} \phi\) para denotar validade/consequência semântica
        % em alguma estrutura \Mathcal{S}.
        % Por fim, temos o conceito de sistema semântico:
        % \begin{definicao}[Sistema Semântico]
        %     \label{def:SistemaSemantico}
        %     Um sistema semântico \(\vDash\) para uma lógica modal definida sobre a linguagem \textit{LM}, cuja classe de modelos é \Mathfrak{M},
        %     pode ser representado como a função \(\vDash_{\mathfrak{M}}\) de validade (valoração) em classes de modelos
        % \end{definicao}

    \section{Axiomatização}
        \label{sec:LM-Axiomatizacao}
        Existem, na literatura, diversos sistemas dedutivos para a lógica modal, dentre eles sistemas de dedução natural, como é o caso de~\citeshort{dewind2001modal},
        sistemas de \textit{tableaux}, como é o caso de~\citeshort{priest2008nonclassical} ou sistemas axiomáticos de Hilbert, como é o caso de
        ~\citeshort{chellas1980modal},~\citeshort{gabbay2003many} e~\citeshort{zalta1995basic}. O presente trabalho apresentará um sistema axiomático no modelo de Hilbert.

        De acordo com~\citeshort{gabbay2003many}, um sistema axiomático de Hilbert é composto de um conjunto possivelmente finito de fórmulas
        e de um conjunto finito de regras de inferência. Este conjunto de fórmulas é chamado de conjunto de axiomas da lógica em questão e
        a lógica é dita axiomatizável pelo conjunto.

        Uma derivação de uma fórmula \PHI em um sistema de Hilbert é uma sequência finita de fórmulas \(\psi_0,\dots,\psi_n,\phi\),
        onde toda fórmula é uma instância de um axioma ou o resultado da aplicação de regras de inferência em fórmulas anteriores
        na sequência. Instâncias de fórmulas são obtidas a partir da substituição de proposições em fórmulas, como descrito abaixo:

        \begin{definicao}[Substituição]
            \label{def:Substituicao}
            Sendo \PHI e \PSI fórmulas, a substituição em \PHI de \(p_i\) por \PSI, denotada por \(\phi\{p_i \mapsto \psi\}\) é definida como:
            \begin{align*}
                &p_i\{p_i \mapsto \psi\} = \psi \\
                &p_j\{p_i \mapsto \psi\} = p_j, \text{ caso } i \neq j \\
                &(\circ\phi)\{p_i \mapsto \psi\} = \circ(\phi\{p_i \mapsto \psi\}), \circ \in \{\Box\ \Diamond, \neg\} \\
                &(\phi_0 \circ \phi_1)\{p_i \mapsto \psi\} = (\phi_0\{p_i \mapsto \psi\} \circ \phi_1\{p_i \mapsto \psi\}), \circ \in \{\land, \lor, \to\} \tag*\qed
            \end{align*}
        \end{definicao}

        Temos o seguinte conjunto de regras de inferência:

        \begin{definicao}[Regras de Inferência]
            \label{def:RegrasInferencia}
            A lógica modal pode ser descrita pelas regras de Necessitação \((\funcao{Nec})\) e Modus Ponens \(\funcao{MP}\): \\

            \begin{minipage}{.45\textwidth}
                \begin{prooftree}
                    \AxiomC{$\Gamma \vdash \phi$}
                    \RightLabel{Nec}
                    \UnaryInfC{$\Gamma \vdash \Box \phi$}
                \end{prooftree}
            \end{minipage}%
            \begin{minipage}{.45\textwidth}
                \begin{prooftree}
                    \AxiomC{$\Gamma \vdash \phi \to \psi$}
                    \AxiomC{$\Gamma \vdash \phi$}
                    \RightLabel{MP}
                    \BinaryInfC{$\Gamma \vdash \psi$}
                \end{prooftree}
            \end{minipage}
            \qed
        \end{definicao}

        E o seguinte conjunto de axiomas, como apresentado em~\citeshort{silveira2020implementacao}. É de interesse ressaltar que este conjunto não é mínimo nem máximo,
        é possível apresentar uma axiomatização com menos axiomas, assim como com mais axiomas:

        \begin{definicao}[Axiomas da Lógica Modal]
            \label{def:AxiomaisModais}
            A lógica modal pode ser descrita pelo seguinte conjunto \LAMBDAlm de axiomas:
            \begingroup
            \allowdisplaybreaks
            \begin{align*}
                &p_0 \to (p_1 \to p_0) \tag{Ax1} \\
                &(p_0 \to (p_1 \to p_2)) \to ((p_0 \to p_1) \to (p_0 \to p_2)) \tag{Ax2} \\
                &(\neg p_1 \to \neg p_0) \to (p_0 \to p_1) \tag{Ax3} \\
                &p_0 \to (p_1 \to (p_0 \land p_1)) \tag{Ax4} \\
                &(p_0 \land p_1) \to p_0 \tag{Ax5} \\
                &(p_0 \land p_1) \to p_1 \tag{Ax6} \\
                &p_0 \to (p_0 \lor p_1) \tag{Ax7} \\
                &p_1 \to (p_0 \lor p_1) \tag{Ax8} \\
                &(p_0 \to p_2) \to ((p_1 \to p_2) \to (p_0 \lor p_1) \to p_2) \tag{Ax9} \\
                &\neg \neg p_0 \to p_0 \tag{Ax10} \\
                &\Box (p_0 \to p_1) \to (\Box p_0 \to \Box p_1) \tag{K} \label{Ax:K} \\
                &\Diamond (p_0 \lor p_1) \to (\Diamond p_0 \lor \Diamond p_1) \tag*{(Possibilidade) \ \qed} \label{Ax:Poss}
            \end{align*}
            \endgroup
        \end{definicao}
        Dados dois axiomas \PHI e \PSI, caso \(\phi \vdash \psi\) ou \(\psi \vdash \phi\), \PHI e \PSI são ditos dependentes, caso contrário são ditos independentes.
        Dois conjuntos de axiomas \(\Phi\) e \(\Psi\) são dito dependentes caso exista \(\phi_i \in \Phi\) e \(\psi_i \in \Psi\) onde \(\phi_i\) e \(\psi_i\) são dependentes,
        e são ditos independentes caso contrário. Dados conjuntos de axiomas \GAMMA e \SIGMA, escreveremos \(\Gamma \Oplus \Sigma\) para denotar o menor
        conjunto de axiomas independentes definido sobre \(\Gamma \cup \Sigma\), como apresentado em~\citeshort{gabbay2003many}. Caso \(\Sigma = \{\phi\}\), escreveremos \(\Gamma \Oplus \phi\).

        Com essas definições, podemos então apresentar o conceito de axiomatização para a lógica modal, como apresentado por~\citeshort{chellas1980modal}:

        \begin{definicao}[Axiomatização da Lógica Modal]
            Uma axiomatização é um par \(\langle\Sigma, \mathfrak{R}\rangle\), onde \SIGMA é um conjunto de axiomas e \Mathfrak{R} é um conjunto
            de regras de derivação. Caso \SIGMA seja finito, então a lógica axiomatizada por \SIGMA é dita \textit{finitamente axiomatizável}. \qed
        \end{definicao}

        Toda lógica monomodal normal pode ser axiomatizada pelo par \(\langle\Lambda_{\mathsf{LM}} \Oplus \Gamma, \{\funcao{Nec}, \funcao{MP}\}\rangle\),
        onde \GAMMA é um conjunto possivelmente vazio de axiomas adicionais. Caso \(\mathfrak{R} = \{\funcao{Nec}, \funcao{MP}\}\),
        podemos omitir \Mathfrak{R} da apresentação da axiomatização desta lógica.% Podemos então definir um sistema dedutivo para uma lógica modal:

        \begin{definicao}[Relação de Consequência Sintática]
            Sendo \Mathcal{S} uma relação definida sobre \(2^{\mathsf{LM}} \times \mathsf{LM}\) e sendo \(\langle \Sigma, \mathfrak{R}\rangle\) uma axiomatização,
            \Mathcal{S} será dita uma relação de consequência sintática onde, para todo \(\Gamma \subseteq \mathsf{LM}\) e \(\phi \in \mathsf{LM}\),
            \(\langle \Gamma, \phi \rangle \in \mathord{\mathcal{S}}\) se, e somente se, exista uma derivação de \PHI a partir das fórmulas
            do conjunto \(\Gamma \Oplus \Sigma\) usando as regras em \Mathfrak{R}.
            Denotaremos consequências por \(\Gamma \vdash \phi\). Caso \(\Gamma = \emptyset\) escrevemos \(\vdash \phi\) e
            \PHI será dito um teorema. \qed
        \end{definicao}

        % \begin{definicao}[Sistema Dedutivo]
        %     \label{def:SistemaSintatico}
        %     Um sistema dedutivo \(\vdash\) para uma lógica modal definida sobre a linguagem \textit{LM}, com um conjunto \SIGMA de axiomas adicionais,
        %     pode ser representado como uma axiomatização \(\vdash_{\langle\Lambda_{LM} \Oplus \Sigma\rangle}\), ou simplesmente
        %     \(\vdash_{\Sigma}\).
        % \end{definicao}

    \section{Sistemas de Lógica Modal e Correspondência de Frames}
        \label{sec:LM-Correspondencia}

        Como apresentado em~\citeshort{chellas1980modal} e~\citeshort{gabbay2003many}, a imposição de restrições sobre a relação de acessibilidade
        de um frame torna certos esquemas de fórmulas válidos nestes frames. Estes esquemas de fórmulas são ditos correspondentes
        aos frames que respeitam a restrição, por exemplo, o esquema \(\Box \phi \to \phi\) corresponde a frames cuja relação de acessibilidade
        respeita a propriedade de reflexividade, como apresentado na Tabela~\ref{tab:relacoesFrames}. O axioma \textrm{K}, apresentado
        na Seção~\ref{sec:LM-Axiomatizacao}, é correspondente a todos os frames. Se o frame de um modelo respeita uma dada restrição, então
        é dito que este modelo também respeita esta restrição, por exemplo, modelos de frames reflexivos são ditos (modelos) reflexivos.
        Algumas restrições notáveis e suas correspondentes fórmulas encontram-se na Tabela~\ref{tab:relacoesFrames}.
        \begin{table}[htpb]
            \centering
            \begin{tabular}{|l|l|l|c|}

                \hline Tipo de Relação & Condição no Frame & Esquema\\

                \hline Reflexiva & $\forall w \in \mathcal{W}, w\mathcal{R}w$  & \textit{T}: $\Box \phi \rightarrow \phi$\\

                \hline Transitiva & $\forall w_0,w_1,w_2 \in \mathcal{W},$ & \textit{4}: $\Box \phi \to \Box \Box \phi$\\
                    & \((w_0\mathcal{R}w_1 \land w_1\mathcal{R}w_2) \to w_0\mathcal{R}w_2\) &\\

                \hline Euclidiana & $\forall w_0,w_1,w_2 \in \mathcal{W}$, & \textit{5}: $\Diamond \phi \rightarrow \Box \Diamond \phi$\\
                    & $(w_0\mathcal{R}w_1 \land w_0\mathcal{R}w_2)$ $\to w_1\mathcal{R}w_2$ &\\

                \hline Simétrica & $\forall w_0,w_1 \in \mathcal{W}$, & \textit{B}: $\phi \to \Box \Diamond \phi$\\
                    &  $w_0\mathcal{R}w_1 \to w_1\mathcal{R}w_0$ &\\

                \hline Convergente & $\forall w_0,w_1,w_2, \exists w_3 \in \mathcal{W}$, & \textit{C}: $\Diamond \Box \phi \rightarrow \Box \Diamond \phi$\\
                    & $(w_0\mathcal{R}w_1 \land w_0\mathcal{R}w_2) \to (w_1\mathcal{R}w_3 \land w_2\mathcal{R}w_3)$ &\\

                \hline Serial & $\forall w_0, \exists w_1 \in \mathcal{W}, w_0\mathcal{R}w_1$ & \textit{D}: $\Box \phi \rightarrow \Diamond \phi$\\

                \hline Funcional & $\forall w_0,w_1,w_2 \in \mathcal{W}$, & \textit{CD}: $\Diamond \phi \rightarrow \Box \phi$\\
                    & $(w_0\mathcal{R}w_1 \land w_0\mathcal{R}w_2) \to w_1 = w_2$ &\\

                \hline Densa & $\forall w_0,w_1, \exists w_2 \in \mathcal{W}$, & \textit{C4}: $\Box \Box \phi \rightarrow \Box \phi$\\
                    & $w_0\mathcal{R}w_1 \to (w_0\mathcal{R}w_2 \land w_2\mathcal{R}w_1)$ &\\

                \hline

            \end{tabular}
            \caption{Relações e axiomas correspondentes}
            \small{Fonte: Adaptado de~\citeshort{silveira2020implementacao}.}
            \label{tab:relacoesFrames}
        \end{table}

        O sistema de lógica modal mais simples (também pode ser chamado de sistema básico da lógica modal) chama-se K, em homenagem à Saul Kripke.
        O sistema K é definido sobre todos os frames \(\langle \mathcal{W}, \mathcal{R} \rangle\), sem qualquer restrição sobre \Mathcal{R}.
        Sua relação de consequência semântica é descrita por uma relação \VDDASH definida sobre a classe
        \Mathfrak{M} de todos os modelos cuja relação de acessibilidade do frame não é restrita, e sua relação de
        consequência sintática é descrita por uma relação \VDASH definida sobre a axiomatização \(\langle\Lambda_{\mathsf{LM}}, \{\funcao{Nec}, \funcao{MP}\}\rangle\).
        Representaremos a lógica definida sobre o sistema K por \(\mathcal{L}_{\textbf{K}} = \langle \mathsf{LM}, \vDash_{\mathfrak{M}}, \vdash_{\textnormal{K}} \rangle \),
        ou simplesmente por \(\textbf{K} = \langle \mathsf{LM}, \vDash, \vdash \rangle\).

        A imposição de restrições sobre a relação de acessibilidade de frames dá surgimento a outros sistemas de lógica modal, distintos
        do sistema básico \textbf{K}. Sendo \(\mathcal{F}_x = \langle \mathcal{W}, \mathcal{R} \rangle\) um frame qualquer pertencente a uma classe de frames \Mathfraki{F}{x},
        onde a relação \Mathcal{R} de todo frame desta classe respeita alguma restrição \textit{X}. Como \Mathcal{R} é restrito, cada frame \Mathcali{F}{x} de \Mathfraki{F}{x}
        descreve uma classe de modelos \(\mathfrak{M}_x\) menor que a classe de todos os modelos sem restrição.
        Já uma axiomatização para \Mathcali{F}{x} será definida pelo par \(\langle\Lambda_{\mathsf{LM}} \Oplus \Xi, \{\funcao{Nec}, \funcao{MP}\}\rangle\),
        onde \(\Xi\) é o conjunto de axiomas correspondentes à \textit{X}. Portanto, a classe \Mathfraki{F}{x} gera o sistema
        \(\mathcal{L}_{\mathfrak{F}_x} = \langle \mathsf{LM}, \vDash_{\mathfrak{M}_x}, \vdash_\Xi \rangle\), ou simplesmente
        \({\mathbf{F}_x = \langle \mathsf{LM}, \vDash_{\mathbf{F}_x}, \vdash_{\mathbf{F}_x} \rangle}\).

        Sistemas de lógica modal geralmente são nomeados de acordo com os nomes de seus axiomas correspondentes (apresentados na Tabela~\ref{tab:relacoesFrames}), por
        exemplo, é chamado de sistema \textbf{KT} o sistema definido sobre a classe de modelos reflexivos\footnote{Lembre que o axioma \textit{K} é correspondente
        a todos os frames.}, já o sistema definido sobre a classe de modelos reflexivos e transitivos é chamado de \textbf{KT4}. Podemos omitir o \textbf{K} em nomes
        de sistemas. São exceções à essa regra os sistemas \textbf{S4} e \textbf{S5}, descritos no exemplo a seguir.

        \begin{exemplo}[Sistemas de Lógica Modal]
            \label{exe:SistemasModal}
            A partir do que foi apresentado na Tabela~\ref{tab:relacoesFrames}, podemos descrever outros sistemas lógicos, como apresentado abaixo
            (para evitar repetições, a linguagem \(\mathsf{LM}\) foi omitida das definições):

            \begin{itemize}
                \item \(\mathbf{T} = \langle \vDash_{\mathfrak{M}_{\mathcal{T}}}, \vdash_{\textit{T}} \rangle\), onde \(\mathfrak{M}_{\mathcal{T}}\) é a classe de
                modelos reflexivos;

                \item \(\mathbf{4} = \langle \vDash_{\mathfrak{M}_4}, \vdash_{\textit{4}} \rangle\), onde \(\mathfrak{M}_4\) é a classe de
                modelos transitivos;

                \item \(\mathbf{S4} = \langle \vDash_{\mathfrak{M}_{\mathcal{T}4}}, \vdash_{\textit{T}\Oplus\textit{4}} \rangle\), onde \(\mathfrak{M}_{\mathcal{T}4}\)
                é a classe de modelos reflexivos e transitivos;

                \item \(\mathbf{S5} = \langle \vDash_{\mathfrak{M}_{\mathcal{TB}4}}, \vdash_{\textit{T}\Oplus\textit{B}\Oplus\textit{4}} \rangle\),
                onde \(\mathfrak{M}_{\mathcal{TB}4}\) é a classe de modelos reflexivos, simétricos e transitivos. Alternativamente, podemos definir \textbf{S5} por:
                \begin{itemize}
                    \item \(\mathbf{S5} = \langle \vDash_{\mathfrak{M}_{\mathcal{TB}4}}, \vdash_{\mathbf{S4}\Oplus\textit{B}} \rangle\), onde
                    \(\mathbf{S4}\Oplus\textit{B}\) denota o menor conjunto de axiomas definido pelo conjunto de axiomas do sistema \textbf{S4} e o axioma \textit{B};

                    \item \(\mathbf{S5} = \langle \vDash_{\mathfrak{M}_{\mathcal{T}5}}, \vdash_{\textit{T}\Oplus 5} \rangle\), onde \(\mathfrak{M}_{\mathcal{T}5}\)
                    é a classe de modelos reflexivos e euclideanos. \qed
                \end{itemize}
            \end{itemize}
        \end{exemplo}
        Para denotar valoração de fórmulas ou validade em sistemas específicos de lógica modal usaremos a notação \(\mathcal{S}, w \Vdash_{\mathbf{X}} \phi\) ou
        \(w \Vdash_{\mathbf{X}, \mathcal{S}} \phi\), onde \Mathcal{S} é alguma estrutura (modelo, frame, classe de frames, classe de modelos)
        e \textbf{X} é um sistema. Omitiremos o \textbf{X} subscrito caso não seja necessário
        explicitar o sistema que está sendo discutido. Análogo para \VDDASH e \VDASH.

        Por fim, usaremos os termos ``sistema de lógica modal'', ``lógica modal'' e ``lógica'' como sinônimos no restante do texto.

    \section{Algumas Meta Propriedades}
        \label{sec:LM-MetaPropriedades}
        Sistemas de lógica modal são, assim como muitos outros sistemas lógicos, corretos e completos como foi demonstrado por
        ~\citeshort{silveira2020implementacao}, ~\citeshort{chellas1980modal} e~\citeshort{blackburn2001modal}, e também possuem outras propriedades interessantes,
        como decidibilidade e propriedade de modelos finitos, como demonstrado por~\citeshort{gabbay2003many}.
        Provar essas propriedades é uma tarefa grande, complexa e que se encontra fora do escopo do presente trabalho.
        Apesar disso, estas propriedades serão enunciadas, brevemente explicadas e provas delas na literatura serão indicadas.

        O problema de demonstrar corretude e completude de sistemas lógicos se resume a demonstrar que há uma correspondência entre suas relações
        de consequência sintática e semântica. Caso uma lógica seja correta então toda fórmula sintaticamente derivável também é semanticamente
        derivável, caso uma lógica seja completa toda fórmula semanticamente derivável também é sintaticamente derivável. Logo, as relações de
        consequência sintática e semântica de uma lógica correta e completa são igualmente expressivas.
        % Segundo~\cite{blackburn2001modal}, o problema de demonstrar corretude e completude de lógicas se resume a responder as seguintes perguntas:
        % ``Dada uma lógica especificada semanticamente, é possível dar uma especificação sintática para ela e, se sim, como?'' e ``Dada uma lógica
        % especificada sintaticamente, é possível dar uma especificação semântica para ela e, se sim, como?''.
        % O problema fica mais evidente se considerarmos os conjuntos \Mathcali{T}{1} e \Mathcali{T}{2} descritos abaixo.

        % O problema de correção e completude se torna mais evidente se considerarmos \(\mathcal{T}^{\mathbf{X}}_{1} = \{\phi \ | \ \vdash_{\mathbf{X}} \phi\}\)
        % como o conjunto de todos os teoremas de um sistema de lógica modal \textbf{X} e \(\mathcal{T}^{\mathbf{X}}_{2} = \{\phi \ | \ \mathfrak{M} \vDash_{\mathbf{X}} \phi\}\)
        % como sendo o conjunto de todas as fórmulas válidas na classe de modelos \Mathfrak{M} correspondente à \textbf{X}. Demonstrar a corretude de \textbf{X} se resume a demonstrar que
        % \(\mathcal{T}_1 \subseteq \mathcal{T}_2\), já demonstrar a completude fraca de \textbf{X} se resume a demonstrar que \(\mathcal{T}_2 \subseteq \mathcal{T}_1\).
        % Logo, se \textbf{X} for correta e completa, temos que \(\mathcal{T}_1 = \mathcal{T}_2\), isto é, os sistemas semânticos e sintáticos de \textbf{X} são igualmente
        % expressivos\footnote{Toda lógica fortemente completa também é fracamente completa, logo, basta provar que uma lógica é fracamente completa e correta para demonstrar
        % que seus sistemas sintáticos e semânticos são igualmente expressivos.}.

        As definições de completude e corretude abaixo foram retiradas de~\citeshort{blackburn2001modal}.

        \subsection{Corretude}
            \label{subsec:Corretude}

            \sloppy
            \begin{definicao}[Corretude]
                \label{def:Corretude}
                Uma lógica modal \Mathcal{L} é dita \textit{correta com respeito a} uma classe de frames (modelos) \Mathfrak{S}
                se, para qualquer fórmula \PHI e qualquer frame (modelo) \(\mathcal{S} \in \mathfrak{S}\), se \(\vdash_{\mathcal{L}} \phi\)
                então \(\mathcal{S} \vDash_{\mathcal{L}} \phi\). \qed
            \end{definicao}

            \begin{teorema}[Corretude da Lógica Modal \textbf{K}]
                \label{teo:CorretudeLogicaModal}
                A lógica modal \textbf{K} é correta para a classe de todos os frames.
            \end{teorema}

            \begin{proof}[Ideia da Prova do Teorema~\ref{teo:CorretudeLogicaModal}]
                Provas de corretude de lógicas modais se resumem a provar que os axiomas que axiomatizam uma lógica são corretos,
                isto é, são fórmulas válidas, e que as regras de derivação preservam validade, isto é, demonstrar que, ao aplicar uma
                regra de derivação em um conjunto de fórmulas válidas, teremos como resultado outra fórmula válida.

                Em particular, para provar a corretude da lógica modal \textbf{K} como apresentada neste trabalho,
                basta provar que os axiomas descritos na Seção~\ref{sec:LM-Axiomatizacao} são válidos e que as regras de derivação
                de Necessitação e Modus Ponens preservam validade.
                Como \textbf{K} é correta para a classe de todos os frames, qualquer outro sistema da lógica modal ``herda'' a corretude
                das regras de derivação e axiomas que axiomatizam \textbf{K}. Para provar a corretude de outros sistemas modais, basta provar
                que seu(s) axioma(s) correspondente(s) é(são) válido(s) na sua respectiva classe de frames.

                Provas de corretude podem ser encontradas em~\citeshort{silveira2020implementacao}, ~\citeshort{chellas1980modal} e~\citeshort{zalta1995basic}.
            \end{proof}

        \subsection{Completude}
            \label{subsec:Completude}

            \begin{definicao}[Completude]
                Sendo \Mathfrak{S} uma classe de frames (modelos), uma lógica \Mathcal{L} é dita \textit{fortemente completa} com respeito a \Mathfrak{S} se,
                para qualquer conjunto de fórmulas \GAMMA e qualquer fórmula \PHI, se \(\Gamma \vDash_{\mathfrak{S},\mathcal{L}} \phi\)
                então \(\Gamma \vdash_{\mathcal{L}} \phi\).

                Sendo \Mathfrak{S} uma classe de frames (modelos), uma lógica \Mathcal{L} é dita \textit{fracamente completa} com respeito à \Mathfrak{S} se,
                para qualquer fórmula \PHI, se \(\mathfrak{S} \vDash_{\mathcal{L}} \phi\) então \(\vdash_{\mathcal{L}} \phi\). \qed
            \end{definicao}

            \begin{teorema}[Completude da Lógica Modal \textbf{K}]
                \label{teo:CompletudeLogicaModal}
                A lógica modal \textbf{K} é fortemente completa para a classe de todos os frames.
            \end{teorema}

            Antes de apresentar a ideia da prova, é necessário definir o conceito de consistência de um conjunto de fórmulas:

            \sloppy
            \begin{definicao}[Consistência]
                Sendo \GAMMA um conjunto de fórmulas, \GAMMA é dito consistente se não existe fórmula \PHI tal que \({\phi \in \Gamma \text{ e } \neg \phi \in \Gamma}\). \qed
            \end{definicao}

            \begin{proof}[Ideia da Prova do Teorema~\ref{teo:CompletudeLogicaModal}]
                Ao contrário da corretude, provas de completude para lógica modal são consideravelmente complexas e, de acordo com~\citeshort{blackburn2001modal},
                são essencialmente teoremas sobre a existência de modelos, pois para demonstrar completude de uma dada lógica modal \Mathcal{L} com relação
                à uma dada classe de estruturas \Mathfrak{S}, é demonstrado que todo subconjunto consistente do conjunto de todos os teoremas de \Mathcal{L}
                é satisfazível em algum modelo.

                Então o problema se torna: ``como construir estes modelos?''. A resposta é: construindo modelos canônicos. Modelos canônicos são uma classe
                específica de modelos que estão intimamente relacionados com o conceito de conjuntos maximais consistentes de fórmulas, estes são os ``maiores''
                conjuntos de fórmulas consistentes. Este método de prova é bastante complexo e não há muito mais que possa ser explicado sobre sem antes abordar
                detalhes técnicos fora do escopo deste trabalho.~\citeshort{blackburn2001modal},~\citeshort{chellas1980modal} e~\citeshort{silveira2020implementacao} apresentam provas de
                completude.
            \end{proof}

    \section{Lógicas Multimodais}
        \label{sec:LM-Multimodais}
        Lógicas multimodais são uma extensão intuitiva do conceito de lógicas modais com apenas uma modalidade (ou um par de modalidades duais, como é o caso da lógica
        alética apresentada), que contém diversas modalidades distintas.
        %Por exemplo, podemos extender a lógica monomodal alética com \textit{n} operadores unários \BOXi{1},\ldots,\BOXi{n} e seus respectivos duais \DIAi{1},\ldots,\DIAi{n}.

        \subsection{Linguagem}
            \label{subsec:MultimodaisLinguagem}
            A linguagem \(\mathsf{LM}_{n}\) da lógica \textit{n}-modal é uma simples extensão da linguagem da lógica modal apresentada na Seção~\ref{sec:LM-Linguagem},
            onde, ao invés de termos apenas a modalidade \BOX e sua dual \DIA, temos \textit{n} modalidades \BOXi{1},\ldots,\BOXi{n} e suas duais \DIAi{1},\ldots,\DIAi{n}.
            O conceito de subfórmulas é estendido para lidar com as novas modalidades. Temos a seguinte definição de \textit{profundidade modal}, como
            apresentado por~\citeshort{gabbay2003many}, que será utilizado posteriormente na prova de transferência de completude pela fusão de lógicas modais:

            \sloppy
            \begin{definicao}[Profundidade Modal]
                A \textit{profundidade de uma fórmula} \PHI, denotada por \(\funcao{d}(\phi)\), é definida indutivamente por:
                \begin{align*}
                    &\funcao{d}(\phi) = \ 0, \text{ caso } \phi \in \mathbb{P} \cup \{\top,\bot\} \\
                    &\funcao{d}(\neg\phi) = \ \funcao{d}(\phi) \\
                    &\funcao{d}(\phi \circ \psi) = \ \funcao{max}(\funcao{d}(\phi), \funcao{d}(\psi)), \circ \in \{\land, \lor, \to\} \\
                    &\funcao{d}(\circ \phi) = \ \funcao{d}(\phi) + 1, \circ \in \{\Box_1,\dots,\Box_n,\Diamond_1,\dots,\Diamond_n\}
                \end{align*}
                Onde max é a função de máximo entre dois números. \qed
            \end{definicao}

        \subsection{Semântica e Axiomatização}
            \label{subsec:MultimodaisSemAx}
            Estender a semântica de mundos possíveis para lidar com diversas modalidades é também intuitivo, basta estender o conceito de frames para \textit{n-frames},
            como apresentado por~\citeshort{gabbay2003many}.

            \begin{definicao}[n-frames]
                \sloppy
                Um \textit{n-frame é uma n-upla} \({\mathcal{F}_n = \langle \mathcal{W}, \mathcal{R}_1,\dots,\mathcal{R}_n \rangle}\), onde \(\mathcal{W} \neq \emptyset\) e
                \(\mathcal{R}_i \subseteq \mathcal{W} \times \mathcal{W}, 1 \leq i \leq n\). \qed
            \end{definicao}

            \begin{definicao}[n-modelos]
                Um \textit{n-modelo é um par de n-frame e uma função de valoração} da forma
                \(\mathcal{M}_n = \langle \mathcal{F}_n, \mathcal{V} \rangle\), onde \Mathcal{V} é uma função total binária
                definida por \(\mathcal{V}: \mathbb{P} \to 2^{\mathcal{W}}\).
                Os n-modelos também podem ser apresentados como \({\mathcal{M}_n = \langle \mathcal{W}, \mathcal{R}_1,\dots,\mathcal{R}_n, \mathcal{V} \rangle}\). \qed
            \end{definicao}

            A definição de valoração de fórmulas em lógicas \textit{n}-modais é uma generalização do caso monomodal, onde os casos para \BOX e \DIA são substituídos por:
            \begin{align*}
                \mathcal{M}_n, w_0 & \Vdash \Box_i \phi \text{ sse } \forall w_1 \in \mathcal{W}, (w_0 \mathcal{R}_i w_1 \to
                                 \mathcal{M}_n, w_1 \Vdash \phi), 1 \leq i \leq n \\
                \mathcal{M}_n, w_0 & \Vdash \Diamond_i \phi \text{ sse } \exists w_1 \in \mathcal{W}, (w_0 \mathcal{R}_i w_1 \land
                                 \mathcal{M}_n, w_1 \Vdash \phi), 1 \leq i \leq n
            \end{align*}

            As definições de outros conceitos semânticos são estendidas da maneira intuitiva.

            \begin{definicao}[Axiomatização Multimodal]
                \label{def:AxiomatizacaoMultimodal}
                O conjunto de axiomas \(\Lambda_{\mathsf{LM}_n}\) de uma lógica \textit{n}-modal é uma generalização do conjunto \LAMBDAlm monomodal, onde o axioma
                \textit{K} é substituído por \textit{n} instâncias de:
                \[
                    \Box_i (p_0 \to p_1) \to (\Box_i p_0 \to \Box_i p_1), 1 \leq i \leq n \tag{K\textsubscript{i}} \label{Ax:Kn}
                \]
                e o conjunto de regras de derivação é uma generalização do caso monomodal, com \textit{n} instâncias da regra \(\funcao{Nec}\),
                ou seja, \(\mathfrak{R} = \{\funcao{Nec}_1, \dots, \funcao{Nec}_n, \funcao{MP}\}\). Logo, uma lógica n-modal pode ser axiomatizada pelo par
                \(\langle\Lambda_{\mathsf{LM}_n} \Oplus \Gamma, \{\funcao{Nec}_1, \dots, \funcao{Nec}_n, \funcao{MP}\}\rangle\), onde \GAMMA é um
                conjunto possivelmente vazio de axiomas adicionais. \qed
            \end{definicao}

            A mesma generalização pode ser aplicada para outros axiomas, como apresentado no exemplo abaixo.

            \begin{exemplo}[Esquemas Multimodais]
                Considerando os esquemas de fórmulas descritos na Tabela~\ref{tab:relacoesFrames}, temos as seguintes generalizações multimodais para alguns deles:
                \begin{align*}
                    \mathit{T}_i&: \ \Box_i \phi \rightarrow \phi \\
                    \mathit{4}_i&: \ \Box_i \phi \to \Box_i \Box_i \phi \\
                    \mathit{5}_i&: \ \Diamond_i \phi \rightarrow \Box_i \Diamond_i \phi \\
                    \mathit{B}_i&: \ \phi \to \Box_i \Diamond_i \phi \tag*\qed
                \end{align*}
            \end{exemplo}

            Relações de consequência sintática e semântica são estendidos de modo intuitivo.
            O sistema \textit{n}-modal básico \(\textbf{K}_n\) é definido sobre todos os n-frames sem restrições, de forma análoga ao sistema \textbf{K} monomodal.

            \begin{exemplo}[Sistemas Multimodais]
                \label{exe:SistemasMultiModal}
                Apresentaremos aqui as versões multimodais dos sistemas de lógica modal apresentados no Exemplo~\ref{exe:SistemasModal}.
                Para evitar repetições, a linguagem \(\mathsf{LM}_{n}\) foi omitida das definições:

                \begin{itemize}
                    \item \(\mathbf{T}_n = \langle \vDash_{\mathfrak{M}_{\mathcal{T}n}}, \vdash_{\mathit{T_i}} \rangle\), onde \(\mathfrak{M}_{\mathcal{T}n}\) é a classe de
                    n-modelos reflexivos;

                    \item \(\mathbf{4}_n = \langle \vDash_{\mathfrak{M}_{4n}}, \vdash_{\mathit{4_i}} \rangle\), onde \(\mathfrak{M}_{4n}\) é a classe dos
                    n-modelos transitivos;

                    \item \(\mathbf{S4}_n = \langle \vDash_{\mathfrak{M}_{\mathcal{T}4n}}, \vdash_{\mathit{T_i}\Oplus\mathit{4_i}} \rangle\), onde \(\mathfrak{M}_{\mathcal{T}4n}\)
                    é a classe dos n-modelos reflexivos e transitivos;

                    \item \(\mathbf{S5}_n = \langle \vDash_{\mathfrak{M}_{\mathcal{TB}4n}}, \vdash_{\mathit{T_i}\Oplus\mathit{B_i}\Oplus\mathit{4_i}} \rangle\),
                    onde \(\mathfrak{M}_{\mathcal{TB}4n}\) é a classe dos n-modelos reflexivos, simétricos e transitivos. Alternativamente, podemos definir \textbf{S5}\textsubscript{n} por:
                    \begin{itemize}
                        \item \(\mathbf{S5}_n = \langle \vDash_{\mathfrak{M}_{\mathcal{TB}4n}}, \vdash_{\mathbf{S4}_n\Oplus\mathit{B_i}} \rangle\), onde
                        \(\mathbf{S4}_n\Oplus\mathit{B_i}\) denota o menor conjunto de axiomas definido pelo conjunto de axiomas do sistema \textbf{S4}\textsubscript{n} e o
                        axioma \textit{B}\textsubscript{\textit{i}};

                        \item \(\mathbf{S5}_n = \langle \vDash_{\mathfrak{M}_{\mathcal{T}5n}}, \vdash_{\mathit{T_i}\Oplus 5_i} \rangle\), onde \(\mathfrak{M}_{\mathcal{T}5n}\)
                        é a classe dos n-modelos reflexivos e euclideanos. \qed
                    \end{itemize}
                \end{itemize}
            \end{exemplo}

            Um n-modelo \({\mathcal{M}_n = \langle \mathcal{W}, \mathcal{R}_1,\dots,\mathcal{R}_n, \mathcal{V} \rangle}\) é dito respeitar uma restrição se cada \Mathcali{R}{i}
            respeita a restrição, por exemplo, se toda \Mathcali{R}{i} é uma relação reflexiva, então o n-modelo é dito reflexivo. Alternativamente, um sistema descrito pela
            classe de n-modelos que respeitam uma dada propriedade \Mathcal{X} pode ser visto como o sistema onde cada modalidade \BOXi{i} se comporta como a modalidade \BOX do sistema monomodal
            que respeita a mesma propriedade \Mathcal{X}, como apresentado por~\citeshort{gabbay2003many}.

    \section{Outros Tipos de Lógicas Modais}
        \label{sec:LM-OutrasModais}
        Como afirmado no início deste capítulo, a lógica modal não lida apenas com conceito de \textit{necessidade} e \textit{possibilidade},
        de fato, linguagens modais são uma excelente ferramenta para raciocinar sobre estruturas relacionais, de acordo com~\citeshort{blackburn2001modal}.
        Estruturas relacionais estão presentes em diversas áreas do conhecimento, por exemplo, sistemas de transição para programas de computador,
        redes semânticas para representação de conhecimento ou estruturas linguísticas de atributo e valor, de acordo com~\citeshort{gabbay2003many}.

        Para raciocinar sobre estas e outras estruturas relacionais, é necessário especializar a lógica modal que foi apresentada até então, onde os conectivos
        \BOX e \DIA de necessidade e possibilidade são substituídos por outros conectivos modais que representam os conceitos desejados de modo mais preciso.
        Ademais, sistemas modais mais complexos podem ser construídos a partir da \textit{combinação} de sistemas modais mais simples~\cite{roggia2012fusion},
        algo que será visto mais à frente.

        Nesta seção, serão brevemente apresentadas as lógicas modais temporal, epistêmica e temporal epistêmica, com alguns exemplos de suas aplicações em computação.

        \subsection{Lógica Temporal}
            \label{subsec:LogicaTemporal}
            De acordo com~\citeshort{emerson1990temporal}, lógica temporal é um tipo especial de lógica modal que define um sistema para descrever e raciocinar sobre
            como a veracidade de afirmações mudam com o tempo. Já~\citeshort{gabbay2003many} define como uma aplicação natural e intuitiva da lógica modal. Mais ainda,
            \citeshort{emerson1990temporal} apresenta uma classificação de lógicas temporais, onde estas se dividem em: proposicional ou primeira ordem, global ou composicional,
            tempo ramificável ou linear, pontos de tempo ou intervalos de tempo, tempo discreto ou contínuo, passado ou futuro.

            \citeshort{gabbay2003many} apresenta cinco modalidades para lógica temporal, ``sempre no passado/futuro'', ``desde'', ``até'' e ``no próximo instante''.
            Já~\citeshort{emerson1990temporal} apresenta quatro modalidades para a lógica temporal linear, ``eventualmente'', ``sempre'', ``no próximo instante'' e ``até''.
            % representadas por \BOXi{P}, \BOXi{F}, \Mathcal{S}, \Mathcal{U} e \(\circ\), respectivamente, onde \BOXi{P} e \BOXi{F} tem as duais \DIAi{P} e \DIAi{F}, que
            % representam ``algum momento no passado/futuro'', respectivamente.

            % A frase ``sempre no passado/futuro \PHI'' (ou \(\Box_F/\Box_P \phi\) significa ``\PHI sempre foi/será verdade'',
            % ``\PHI desde \PSI'' (ou \(\phi \mathcal{S} \psi\) significa ``\PHI é verdade desde o momento em que \PSI é verdade'', a frase ``\PHI até \PSI''
            % (ou \(\phi \mathcal{U} \psi\) significa ``\PHI será verdade até \PSI for verdade'', por fim, a frase ``no próximo instante \PHI''
            % significa``no próximo instante de tempo, \PHI será verdadeiro''

            % Os dois últimos tem a mesma interpretação que a de~\cite{gabbay2003many}, já
            % ``eventualmente \PHI'' significa ``há algum instante de tempo futuro onde \PHI é verdadeiro'',
            % e ``sempre \PHI'' significa ``\PHI sempre foi e sempre será verdadeiro''.

            De acordo com~\citeshort{emerson1990temporal},~\citeshort{pnueli1977temporal} foi o primeiro a descrever a lógica temporal como formalismo para
            verificação de programas sequenciais e concorrentes. %, pois lógicas temporais são capazes de expressar fatos como programas (sequenciais ou paralelos) que operam continuamente.
            Extensões da lógica temporal também são muito usadas para formalizar e verificar programas, por exemplo~\citeshort{lamport1983specifying} descreve um formalismo
            baseado em lógica temporal para especificar programas concorrentes, este foi estendido em~\citeshort{lamport1994temporal} com a descrição de um sistema lógico, baseado em lógica temporal,
            chamado de TLA (\textit{Temporal Logic of Actions}\footnote{Lógica Temporal das Ações.}) para formalizar e raciocinar sobre programas concorrentes,
            por fim~\citeshort{lamport2002specifying} estende o sistema TLA gerando então o sistema \(\mathrm{TLA}^{+}\).

        \subsection{Lógica Epistêmica}
            \label{subsec:LogicaEpistemica}
            Segundo~\citeshort{fagin2004reasoning}, a lógica epistêmica é a lógica que estuda o conhecimento, de um ou mais agentes.
            \citeshort{gabbay2003many} afirma que a lógica epistêmica é relevante para diversas disciplinas, dentre elas: teoria dos jogos, inteligência artificial e sistemas
            multiagentes.

            \citeshort{gabbay2003many} apresenta uma formalização da lógica epistêmica como uma lógica multimodal com diversas modalidades \BOX, onde cada \BOXi{i} indica
            o conhecimento de um agente \textit{i} sobre uma dada fórmula ou proposição. Já~\citeshort{lescanne2004mechanizing} apresenta como uma lógica multimodal com
            três modalidades, \textit{K}\textsubscript{i}, \textit{E}\textsubscript{G} e \textit{C} que representam, respectivamente, conhecimento de um agente \textit{i},
            conhecimento de todos os agentes num grupo \textit{G} e conhecimento comum a todos os agentes.

            Por fim,~\citeshort{gabbay2003many} define que a lógica epistêmica pode ser interpretada em sistemas multimodais aléticos, caso o operador \BOX
            seja interpretado de forma epistêmica, onde cada sistema impõe restrições sobre o conhecimento de um agente. Por exemplo, em um interpretação epistêmica de
            \textbf{T}\textsubscript{n}, é imposta a restrição que tudo que um agente sabe é verdadeiro.

            \citeshort{gabbay2003many} descreve uma aplicação de lógica epistêmica no problema dos três homens sábios e os chapéus, apresentado como o problema pode ser modelado
            dentro de um sistema lógico epistêmico e descrevendo como o problema é usualmente resolvido.

        \subsection{Lógica Temporal Epistêmica}
            \label{subsec:LogicaTemporalEpistemica}
            Lógica Temporal Epistêmica é uma família de lógicas que foi construída com o intuito de formalizar o comportamento de agentes em sistemas
            multiagentes~\cite{gabbay2003many}.~\citeshort{fagin2004reasoning} afirma que a combinação de operadores epistêmicos e temporais numa única linguagem
            possibilita que afirmações sobre o desenvolvimento do conhecimento no sistema sejam feitas.

            A sua linguagem incorpora elementos da linguagem da lógica temporal e da lógica epistêmica para poder
            representar afirmações sobre o conhecimento de agentes com relação ao tempo, por exemplo ``O agente \textit{i} sempre soube \PHI''~\cite{fagin2004reasoning}.

            Em~\citeshort{fagin2004reasoning} é apresentada, como exemplo de uma lógica temporal epistêmica para sistemas multiagentes, a formalização de um sistema
            de transmissão de bits entre agentes.

    \section{Outra Forma de Definir Lógicas}
        \label{sec:LM-LogicaComoConjunto}
        A interpretação de lógicas (modais) como sistemas que contém uma linguagem e uma (ou mais) relação(ões) de consequência definida(s) sobre essa linguagem
        apresentada no começo deste capítulo não é a única definição possível do conceito de lógica. A forma utilizada por~\citeshort{blackburn2001modal},~\citeshort{gabbay2003many} e~\citeshort{fine1996transfer}
        por exemplo, é apresentando lógicas como conjuntos (usualmente infinitos) de fórmulas fechados para algum operador de consequência.
        Esta apresentação de lógica é útil para demonstrar meta propriedades de sistemas lógicos, como completude e corretude.

        Nesta interpretação, é definido um conjunto base de fórmulas, usualmente um conjunto de axiomas, como o apresentado na Definição~\ref{def:AxiomaisModais},
        ou o conjunto de todas as tautologias, e é definida uma operação de consequência (sintática ou semântica) sob a qual este conjunto deve ser fechado.
        O conceito de teoremas/tautologias/consequências é definido a partir do fato da fórmula pertencer ao conjunto, dependendo da relação sobre a qual o conjunto é construído.

        Diferentes autores apresentam definições levemente diferentes uma da outra, variando a assinatura utilizada ou o conjunto de axiomas base, porém todas
        seguem a forma geral descrita anteriormente. Por exemplo~\citeshort{blackburn2001modal} define lógica modal sobre a assinatura \(\{\neg, \Diamond, \lor\}\) utilizando
        o conjunto de todas as tautologias proposicionais juntamente com o axioma~\ref{Ax:K} e a fórmula \(\Diamond \phi \tofrom \neg \Box \neg \phi\),
        já~\citeshort{gabbay2003many} define lógica modal sobre a assinatura \(\{\neg, \Box, \Diamond, \land, \lor, \to\}\) utilizando um conjunto de axiomas
        semelhante ao apresentado na Definição~\ref{def:AxiomaisModais}.

        Para a definição abaixo, vamos considerar a linguagem \linguagem{} apresentada na Definição~\ref{def:LinguagemModal},
        o conjunto \LAMBDAlm apresentado na Definição~\ref{def:AxiomaisModais}, a regra de Substituição apresentada na Definição~\ref{def:Substituicao} e as
        regras de \textit{Modus Ponens} e Necessitação apresentadas na Definição~\ref{def:RegrasInferencia}.

        \begin{definicao}[Lógica Modal como Conjunto]
            \label{def:LogicaConjunto}
            Uma lógica modal definida sobre uma teoria \GAMMA é o menor conjunto \(\mathcal{L}\) tal que \(\Gamma \subseteq \mathcal{L}\), \(\Lambdalm \subseteq \mathcal{L}\)
            e \(\mathcal{L}\) é fechado para a regras de Substituição, \textit{Modus Ponens} e Necessitação. Se \(\phi \in \mathcal{L}\) então \PHI é dito um \textit{teorema}
            e escrevemos \(\vdash_{\mathcal{L}} \phi\), caso contrário escrevemos \(\nvdash_{\mathcal{L}} \phi\). \qed
        \end{definicao}

        Os sistemas modais apresentados na Seção~\ref{sec:LM-Correspondencia} podem também ser descritos com essa definição de lógica. O sistema \textbf{K}
        é definido a partir de \(\Gamma = \emptyset\), o Sistema \textbf{T} é definido a partir de \(\Gamma = \{\Box \phi \to \phi\}\), já o Sistema \textbf{S5}
        pode ser definido a partir de \(\Gamma = \{\Box \phi \rightarrow \phi, \Box \phi \to \Box \Box \phi, \phi \to \Box \Diamond \phi\}\) ou
        \(\Gamma = \{\Box \phi \rightarrow \phi, \Diamond \phi \rightarrow \Box \Diamond \phi\}\).

        Lógicas multimodais são definidas de forma análoga. Podemos usar a notação \(\vdash_{\mathcal{L}} \phi\) para indicar que \(\phi \in \mathcal{L}\),
        e \(\nvdash_{\mathcal{L}} \phi\) para indicar que \(\phi \notin \mathcal{L}\). Tendo apresentado o conceito de lógica como conjunto, podemos apresentar
        o conceito de dedutibilidade em lógicas vistas como conjuntos, como apresentado em~\citeshort{blackburn2001modal}:

        \begin{definicao}[Dedutibilidade]
            Sendo \GAMMA um conjunto de fórmulas, \PHI uma fórmula e \Mathcal{L} uma lógica, é dito que \textit{\PHI é dedutível em \Mathcal{L} a partir de \GAMMA} se
            \(\vdash_{\mathcal{L}} \phi\), ou existem fórmulas \(\gamma_0, \dots, \gamma_n \in \Gamma\) tal que:
            \[
                \vdash_{\mathcal{L}} (\gamma_0 \land \dots \land \gamma_n) \to \phi
            \]
            Caso \PHI seja dedutível em \Mathcal{L} a partir de \GAMMA, escrevemos \(\Gamma \vdash_{\mathcal{L}} \phi\), caso não seja, escrevemos
            \(\Gamma \nvdash_{\mathcal{L}} \phi\). \qed
        \end{definicao}

        Segundo~\citeshort{blackburn2001modal}, todas as lógicas são fechadas sob a operação de dedução: se \PHI é dedutível a partir das premissas
        \(\psi_0, \dots, \psi_n\), então \(\vdash \psi_0 \text{ e } \dots \text{ e } \vdash \psi_n\) implica que \(\vdash \phi\).

        Note que esta é uma apresentação sintática do conceito de lógica, porém ela não é muito diferente de uma apresentação semântica do conceito:
        considere \Mathfrak{F} uma classe de frames, a lógica definida por essa classe será dada pelo conjunto
        \(\mathcal{L}_{\mathfrak{F}} = \{\phi \ | \ \phi \in \mathsf{LM} \text{ e } \mathfrak{F} \Vdash \phi \}\).

        Não é difícil de observar que essa definição de lógicas modais é equivalente à definição apresentada antes neste capítulo.
        Considere, por exemplo, a lógica modal \textbf{K} apresentada na Seção~\ref{sec:LM-Correspondencia}, ela pode ser igualmente definida
        pelos conjuntos \(T_{\vdash} = \{\phi \ | \ \phi \in \mathsf{LM} \text{ e } \vdash_{\textnormal{K}} \phi \} \cup \{ (\gamma_0 \land \dots \land \gamma_n) \to \phi \ | \
        \phi, \gamma_0, \dots, \gamma_n \in \mathsf{LM} \text{ e } \gamma_0, \dots, \gamma_n \vdash_{\textnormal{K}} \phi \}\) e
        \(T_{\vDash} = \{\phi \ | \ \phi \in \mathsf{LM} \text{ e } \mathfrak{F} \Vdash \phi \} \cup \{ (\gamma_0 \land \dots \land \gamma_n) \to \phi \ | \
        \phi, \gamma_0, \dots, \gamma_n \in \mathsf{LM} \text{ e } \gamma_0, \dots, \gamma_n \vDash_{\mathfrak{M}} \phi \}\) (sendo \Mathfrak{F} a classe de frames irrestritos e
        \Mathfrak{M} a classe de modelos construídos com frames em \Mathfrak{F}), que representam, respectivamente, todas as fórmulas
        semântica e sintaticamente deriváveis no sistema \textbf{K}.

        Naturalmente, é possível expressar meta propriedades de lógicas apresentadas como conjuntos tal e qual é possível expressar para lógicas apresentadas como
        sistemas. As definições abaixo foram retiradas de~\citeshort{blackburn2001modal}.

        \begin{definicao}[Corretude como Conjunto]
            \label{def:CorretudeConjunto}
            Uma lógica \Mathcal{L} é dita \textit{correta com relação a uma classe de frames (modelos) \Mathfrak{S}} se, e somente se, se \(\phi \in \mathcal{L}\) então
            \(\mathfrak{S} \Vdash \phi\). Ou seja, \(\mathcal{L} \subseteq \mathcal{L}_{\mathfrak{S}}\) \qed
        \end{definicao}

        \begin{definicao}[Completude como Conjunto]
            \label{def:CompletudeConjunto}
            Sendo \Mathfrak{S} uma classe de frames (modelos), uma lógica \Mathcal{L} é dita \textit{fortemente completa} com respeito a \Mathfrak{S} se,
            para qualquer conjunto de fórmulas \GAMMA e qualquer fórmula \PHI, se \(\Gamma \vDash_{\mathfrak{S}} \phi\)
            então \(\Gamma \vdash \phi\).

            Sendo \Mathfrak{S} uma classe de frames (modelos), uma lógica \Mathcal{L} é dita \textit{fracamente completa} com respeito à \Mathfrak{S} se,
            para qualquer fórmula \PHI, se \(\mathfrak{S} \Vdash \phi\) então \(\vdash_{\mathcal{L}} \phi\). Ou seja, \(\mathcal{L}_{\mathfrak{S}} \subseteq \mathcal{L}\). \qed
        \end{definicao}

        % Podemos observar que na definição de lógicas como conjuntos, demonstrar corretude/completude de uma lógica com relação a alguma classe de estruturas
        % se resume a demonstrar que as fórmulas deriváveis nesta classe contém/estão contidas na própria lógica.
        Provar a completude e a corretude para lógicas vistas como conjuntos pode ser entendido como provar que as definições sintáticas e semânticas desta lógica são
        equivalentes, como afirmado por~\citeshort{blackburn2001modal}. Considerando \(\mathcal{L}\) uma lógica descrita sintaticamente, como apresentado anteriormente,
        e \(\mathcal{L}_{\mathfrak{F}}\) sua correspondente semântica com relação à alguma classe de frames \Mathfrak{F}. Caso \(\mathcal{L}\) seja correta para a classe \Mathfrak{F},
        então \(\mathcal{L} \subseteq \mathcal{L}_{\mathfrak{F}}\), caso \(\mathcal{L}\) seja completa, então \(\mathcal{L}_{\mathfrak{F}} \subseteq \mathcal{L}\),
        portanto, caso \(\mathcal{L}\) seja correta e completa temos que \(\mathcal{L} = \mathcal{L}_{\mathfrak{F}}\).

        Podemos então apresentar a seguinte definição, retirada de~\citeshort{blackburn2001modal}, cuja importância será vista a frente.

        \begin{definicao}[Consistência com Respeito a Conjunto]
            Sendo \GAMMA e \DDELTA conjuntos de fórmulas, \textit{\GAMMA é dito consistente com respeito a \DDELTA, ou \DDELTA-consistente}, se, e somente se,
            não existe \(\gamma\) tal que \(\gamma \in \Gamma \text{ e }\neg \gamma \in \Delta\).
            Uma \textit{fórmula \PHI é dita \DDELTA-consistente} se \(\{\phi\}\) é \DDELTA-consistente. \qed
        \end{definicao}

        Na definição anterior, caso \DDELTA seja uma lógica, \GAMMA será dito consistente com respeito a \DDELTA se \(\Gamma \nvdash_{\Delta} \bot\), pois
        \(\Gamma \vdash_{\Delta} \bot\) implica que existem fórmulas \(\gamma_0, \dots, \gamma_n \in \Gamma\) tal que
        \(\vdash_{\Delta} (\gamma_0 \land \dots \land \gamma_n) \to \bot\), ou seja, \(\vdash_{\Delta} \neg (\gamma_0 \land \dots \land \gamma_n) \lor \bot\).
        Isto é, existe pelo menos um \(\gamma_i\) que pertence à \GAMMA mas não pertence à \DDELTA ou \(\bot \in \Delta\).
        Caso \(\bot \in \Delta\), \DDELTA é dita uma lógica inconsistente e todo conjunto é inconsistente com respeito à \DDELTA.

        Por fim, temos o seguinte resultado auxiliar referente à completude, retirado de~\citeshort{blackburn2001modal}, cuja importância também será vista a frente:

        \begin{proposicao}
            \label{prop:CompletudeAlternativa}
            Uma lógica \Mathcal{L} é dita \textit{fortemente completa para uma classe de frames \Mathfrak{F}}, se, e somente se, todo conjunto de fórmulas
            \Mathcal{L}-consistente é satisfazível em algum frame de \Mathfrak{F}.
            Uma lógica \Mathcal{L} é dita \textit{fracamente completa para uma classe de frames \Mathfrak{F}}, se, e somente se, toda fórmula
            \Mathcal{L}-consistente é satisfazível em algum frame de \Mathfrak{F}.
        \end{proposicao}

        Satisfazibilidade em um frame pode ser entendida como satisfazibilidade em ao menos um modelo construído com aquele frame. Analogamente para classes de frames.

        \begin{proof}[Prova da Proposição~\ref{prop:CompletudeAlternativa}]
            A prova para completude fraca é um caso particular para o caso da completude forte, onde \(\Gamma = \emptyset\), logo, provaremos apenas o caso
            da completude forte. A prova é dividida em dois casos; no primeiro caso é provado que se uma lógica é fortemente completa
            então todo conjunto consistente é satisfazível, já no segundo caso é provado que se todo conjunto consistente é satisfazível então uma lógica
            é fortemente completa. Ambos os casos serão provados por contraposição.

            \begin{description}
                \item[Caso 1] Supondo algum conjunto de fórmulas \GAMMA e alguma fórmula \PHI tal que \(\Gamma \cup \{\phi\}\) é
                \Mathcal{L}-consistente e insatisfazível em \Mathfrak{F}.
                Como \(\Gamma \cup \{\phi\}\) é insatisfazível, temos, pela definição de \(\vDash\), que \(\Gamma \cup \{\phi\} \vDash_{\mathcal{M}} \bot\)\footnote{A definição de \(\vDash\)
                nos diz que ``Caso o antecedente é satisfazível em todos os mundos de algum modelo \Mathcal{M}, então o consequente também será'' Como \(\Gamma \cup \{\phi\}\) é insatisfazível em
                todos os mundos de qualquer modelo \Mathcal{M} construído com qualquer frame da classe \Mathfrak{F}, temos que qualquer fórmula é consequência semântica de \(\Gamma \cup \{\phi\}\).},
                em algum modelo \(\mathcal{M} = \langle \mathcal{F}, \mathcal{V} \rangle\) onde \(\mathcal{F} \in \mathfrak{F}\).
                Como \(\Gamma \cup \{\phi\}\) é \Mathcal{L}-consistente, temos que \(\Gamma \cup \{\phi\} \nvdash_{\mathcal{L}} \bot\).
                Portanto, \Mathcal{L} não é fortemente completo para \Mathfrak{F}.

                \item[Caso 2] Supondo que \Mathcal{L} não é fortemente completa com relação à \Mathfrak{F}. Logo, existe um conjunto
                de fórmulas \GAMMA e uma fórmula \PHI tal que \(\Gamma \vDash_{\mathfrak{F}} \phi\), porém \(\Gamma \nvdash_{\mathcal{L}} \phi\).
                Logo, o conjunto \(\Gamma \cup \{\neg \phi\}\) é \Mathcal{L}-consistente, porém não é satisfazível em qualquer frame da classe \Mathfrak{F}.\qedhere
            \end{description}
        \end{proof}

        Note que a Proposição~\ref{prop:CompletudeAlternativa} indica que há uma correspondência entre o conceito de completude com alguma classe de frames
        e satisfazibilidade de conjuntos \Mathcal{L}-consistente nesta classe.