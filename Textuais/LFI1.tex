\chapter{A Lógica de Inconsistência Formal LFI1}\label{cap:LFI1}

Com os avanços da internet no contexto do gerenciamento de bancos de dados, informações passaram a ser coletadas a partir de diferentes fontes que frequentemente se contradizem. Dada a existência das restrições de integridade {--} que impedem contradições {--} a atualização e manutenção de bancos de dados se torna um processo difícil~\cite{carnielli2000formal}. Portanto, uma lógica capaz de lidar com informações inconsistentes sem necessariamente sofrer com a trivialidade é de grande interesse. A lógica de inconsistência formal \lfium{} é capaz de lidar com contradições ao introduzir na sua assinatura o operador $\circ$ para representar a consistência, internalizando este conceito em sua linguagem. Uma informação é dita consistente caso ela e sua negação não sejam simultaneamente verdadeiras, ou seja, dada uma informação $\alpha$, sua consistência $\circ \alpha$ será equivalente a fórmula $\neg (\alpha \land \neg \alpha)$. Com a introdução deste novo operador, é possível lidar com a inconsistência de informações sem que trivialidade ocorra, já que {--} caso uma informação seja conhecidamente \textit{inconsistente}, ou seja, $\neg \circ \alpha$ {--} então ela se trata de uma contradição inofensiva, fruto do excesso de informações numa dada teoria. Com isso, na \lfium{}, o conjunto $\bigcirc(p)$ de fórmulas dependentes somente na variável $p$ (descrito na Definição~\ref{def:lfi}) assume forma $\{\circ p\}$.

No trabalho de~\citeshort{carnielli2000formal} a lógica \textbf{LFI1*} é definida como uma extensão de primeira ordem da lógica proposicional \lfium{}. A motivação para definir-se uma lógica de inconsistência formal de primeira ordem vem da natureza das informações contidas em bancos de dados, estas que podem ser compreendidas como sentenças de primeira ordem fixas~\cite{Codd}, entretanto, o presente trabalho trata somente da lógica proposicional \lfium{}. Ademais,~\citeshort{carnielli2000formal} tomam o operador de \textit{inconsistência} (denotado por $\bullet$) como primitivo. Isto foi feito pois o foco era explorar a \lfium{} como uma ferramenta para lidar com inconsistências em bancos de dados, portanto tomar a inconsistência como primitiva era de grande interesse. Entretanto, no presente trabalho, será utilizada a definição apresentada em~\citeshort{Carnielli_Coniglio_2016}, onde a linguagem é definida utilizando o operador $\circ$ como primitivo. Isto salienta algumas propriedades interessantes da negação $\neg$, como a presença das leis de De Morgan, axiomatizadas na Seção~\ref{sec:axiomatizacao}.

Este capítulo é dividido da seguinte forma: na Seção~\ref{sec:linguagem} é apresentada a linguagem da lógica proposicional \lfium{} bem como definições necessárias para desenvolver as provas de metateoremas. A Seção~\ref{sec:axiomatizacao} contém uma breve explicação sobre sistemas de prova sintáticos e um cálculo de Hilbert para a \lfium{} é definido. Na Seção~\ref{sec:semantica} a semântica da \lfium{} é definida a partir de matrizes lógicas e de uma semântica de valorações não determinística, assim como é provada a equivalência entre essas dois sistemas.

\section{Linguagem}\label{sec:linguagem}
    A lógica proposicional \lfium{} aqui apresentada é definida com base em~\citeshort{Carnielli_Coniglio_2016} sobre a linguagem $\ling{}$, que por sua vez é definida sobre um conjunto enumerável de átomos $\pazocal{P} = \{p_{n} \;|\; n \in \mathbb{N}\}$ e uma assinatura proposicional $\Sigma = \{\land^{2}, \lor^{2}, \to^{2}, \neg^{1}, \circ^{1}\}$. Como de costume, o conectivo $\land$ representa uma conjunção, $\lor$ representa uma disjunção, $\to$ representa uma implicação, $\neg$ representa uma negação e $\circ$ é o conectivo de consistência, definido de forma primitiva. No restante do texto a aridade destes conectivos será omitida. A linguagem $\ling{}$ da \lfium{} é definida da seguinte forma:

    \begin{definicao}[Linguagem da \lfium{}]
        A linguagem $\ling{}$ da \lfium{} é definida indutivamente como o menor conjunto a que respeita as seguintes regras:\label{def:ling}
        \begin{align*}
            & \text{1.~}\pazocal{P} \subseteq \ling{}                                                                                                                        \\
            & \text{2.~Se } \phi \in \ling{}, \text{então } \triangle  \phi \in \ling{}, \text{com } \triangle \in \{\neg, \circ\}                            \\
            & \text{3.~Se } \phi, \psi \in \ling{}, \text{então } \phi \otimes \psi \in \ling{}, \text{com } \otimes \in \{\land, \lor, \to\} \tag*\qed
        \end{align*}
    \end{definicao}

    A precedência dos conectivos é dada de maneira costumeira, com a adição do operador $\circ$ de consistência, seguindo a ordem (da maior precedência para a menor): $\circ$, $\neg$, $\land$, $\lor$, $\to$. Os conectivos binários $\land$ e $\lor$ são associativos à esquerda, ou seja, uma expressão do tipo $\alpha \land \beta \land \gamma$ é lida como $((\alpha \land \beta) \land \gamma)$, e o conectivo $\to$ é associativo à direita, ou seja, uma expressão do tipo $\alpha \to \beta \to \gamma$ é lida como $(\alpha \to (\beta \to \gamma))$.

    A linguagem da \lfium{} pode ser definida de maneira equivalente utilizando-se o operador de inconsistência (representado por $\bullet$), definido como $\bullet \alpha \eqdef \neg \circ \alpha$, como feito por~\citeshort{carnielli2000formal}. 

    % \begin{definicao}[Subfórmulas]
    %     \label{def:subf}
    %     O conjunto Sub$(\phi)$ de subfórmulas de uma fórmula $\phi$ é definido indutivamente da seguinte forma:
    %     \begin{align*}
    %          & \text{1.~Sub}(p_{i}) = \{p_{i}\}, \; p_{i} \in \pazocal{P}                                                                                                            \\
    %          & \text{2.~Sub}(\triangle \phi) = \{\triangle \phi\} \; \cup \;\text{Sub}(\phi), \; \triangle \in \{\neg, \circ\}                                                     \\
    %          & \text{3.~Sub}(\phi \otimes \psi) = \{\phi \otimes \psi\} \; \cup \;\text{Sub}(\phi) \; \cup \;\text{Sub}(\psi), \; \otimes \in \{\land, \lor, \to\} \tag*\qed
    %     \end{align*}
    % \end{definicao}


    Na Definição~\ref{def:complex} a função $C$ da complexidade de uma fórmula na lógica proposicional clássica foi recursivamente definida. É possível estendê-la para identificar a complexidade de uma fórmula na \lfium{} adicionando-se uma condição para o operador $\circ$:

    \begin{definicao}[Complexidade de uma fórmula na \lfium{}]
        Seja $\phi \in \ling{}$ uma fórmula bem formada, a complexidade $C(\phi)$ é dada adicionando-se a seguinte condição à Definição~\ref{def:complex}:\label{def:complex_lfi1}
        \begin{align*}
            & \text{Se } \phi = \circ \psi \text{, então } C(\phi) = C(\psi) + 2.\tag*\qed{}
        \end{align*}
        
    \end{definicao}

    Note que a complexidade de uma fórmula do tipo $\circ \alpha$ é estritamente maior que a complexidade de $\alpha$ e $\neg \alpha$. Isto se dá pois, como será evidenciado pela semântica de valorações na Definição~\ref{def:valoracoes}, existe uma dependência de $\circ \alpha$ em $\{\alpha, \neg \alpha\}$, como apresentada por~\citeshort{Carnielli_Coniglio_2016}.

\section{Axiomatização}\label{sec:axiomatizacao}

    A teoria das provas é uma das abordagens para o estudo das relações de consequência, onde a validade de uma inferência é atestada caso haja uma \textit{prova} das conclusões a partir das premissas. Uma prova consiste em uma sequência de passos bem definidos aplicados sobre conjuntos (ocasionalmente unitários) de fórmulas, com base nos princípios de um determinado sistema de provas. A teoria das provas é sintática\footnote{Vale notar que a separação \textit{prova {--} sintaxe {--} semântica {--} modelo} não é tão bem definida, algo que é explorado em~\citeshort{Prawitz2005-PRALCA-2}.} por natureza, ou seja, numa inferência $A \vdash B$, é relevante apenas a estrutura das fórmulas presentes em \textit{A} e \textit{B}, não sua interpretação ou valor-verdade. Essa estrutura é manipulada a fim de obter-se uma sequência de passos que {--} além de atestar sua validade {--} serve como argumento para tal~\cite{sep-logical-consequence}. Desta forma, pode-se definir um sistema de provas sintático para servir como relação de consequência para uma determinada lógica. 

    No contexto da \lfium{}, existem dois sistemas de prova sintáticos estabelecidos até o momento: um cálculo de Hilbert, descrito por~\citeshort{carnielli2000formal,Carnielli_Coniglio_2016} e um sistema de \textit{Tableau}, descrito por~\citeshort{tableaulfi}. No presente trabalho, foi escolhido o cálculo de Hilbert para definir a sintaxe da \lfium{}, dada a maior facilidade para desenvolver metateoremas em contraste ao sistema de \textit{Tableau}.

    O cálculo de Hilbert (também conhecido como sistema de Hilbert ou axiomatização de Hilbert) é um sistema composto por um conjunto de fórmulas, chamadas de \textit{axiomas} e um conjunto de \textit{regras de inferência}. Uma regra de inferência é formada por uma lista de fórmulas chamadas de premissas da regra e uma fórmula chamada de conclusão da regra~\cite{Restall1999-RESAIT-4}. Uma prova (também chamada de derivação ou dedução) do tipo $\Gamma \vdash \phi$ consiste em uma sequência finita de fórmulas \(\psi_0, \dots, \psi_n\), onde \(\psi_n = \phi\), e cada  $\psi_i\ (0 \leq i \leq n)$ é um axioma, um elemento do conjunto de premissas $\Gamma$ ou o resultado da aplicação de uma regra de inferência em fórmulas anteriores. Usualmente, o cálculo de Hilbert possui apenas uma regra de inferência, esta sendo o \textit{modus ponens}. Este também é o caso para o cálculo de Hilbert que será utilizado para a \lfium{}.
    
    O cálculo de Hilbert definido a seguir foi apresentado por~\citeshort{Carnielli_Coniglio_2016} como alternativa ao que havia sido definido em trabalhos anteriores~\cite{carnielli2000formal,carnielli2007}. Segundo os autores, esta definição evidencia algumas propriedades interessantes da negação $\neg$ como as leis de De Morgan.


    \begin{definicao}[Cálculo de Hilbert para \lfium{}]\label{def:hilbert_lfi1}
        A lógica \lfium{} é definida a partir da relação de consequência sintática $\conhil{}$ sobre a linguagem $\ling{}$ através do seguinte cálculo de Hilbert:

        \noindent\textbf{Axiomas}:
        \begin{align*}
            & \alpha \to (\beta \to \alpha)                                                     \tag{\textbf{Ax1}}            \label{ax:ax1}\\
            & (\alpha \to (\beta \to \gamma)) \to ((\alpha \to \beta) \to (\alpha \to \gamma )) \tag{\textbf{Ax2}}            \label{ax:ax2}\\
            & \alpha \to (\beta \to (\alpha \land \beta))                                       \tag{\textbf{Ax3}}            \label{ax:ax3}\\
            & (\alpha \land \beta) \to \alpha                                                   \tag{\textbf{Ax4}}            \label{ax:ax4}\\
            & (\alpha \land \beta) \to \beta                                                    \tag{\textbf{Ax5}}            \label{ax:ax5}\\
            & \alpha \to (\alpha \lor \beta)                                                    \tag{\textbf{Ax6}}            \label{ax:ax6}\\
            & \beta \to (\alpha \lor \beta)                                                     \tag{\textbf{Ax7}}            \label{ax:ax7}\\
            & (\alpha \to \gamma) \to ((\beta \to \gamma) \to ((\alpha \lor \beta) \to \gamma)) \tag{\textbf{Ax8}}            \label{ax:ax8}\\
            & (\alpha \to \beta) \lor \alpha                                                    \tag{\textbf{Ax9}}            \label{ax:ax9}\\
            & \alpha \lor \neg \alpha                                                           \tag{\textbf{Ax10}}           \label{ax:ax10}\\
            & \circ \alpha \to (\alpha \to (\neg \alpha \to \beta))                             \tag{\textbf{bc1}}            \label{ax:axbc1}\\
            & \neg \neg \alpha \to \alpha                                                       \tag{\textbf{cf}}             \label{ax:axcf}\\
            & \alpha \to \neg \neg \alpha                                                       \tag{\textbf{ce}}             \label{ax:axce}\\
            & \neg \circ \alpha \to (\alpha \land \neg \alpha)                                  \tag{\textbf{ci}}             \label{ax:axci}\\
            & \neg (\alpha \lor \beta) \to (\neg \alpha \land \neg \beta)                       \tag{\textbf{neg}$\lor_{1}$}  \label{ax:axneglor1}\\
            & (\neg \alpha \land \neg \beta) \to \neg (\alpha \lor \beta)                       \tag{\textbf{neg}$\lor_{2}$}  \label{ax:axneglor2}\\
            & \neg(\alpha \land \beta) \to (\neg \alpha \lor \neg \beta)                        \tag{\textbf{neg}$\land_{1}$} \label{ax:axnegland1}\\
            & (\neg \alpha \lor \neg \beta) \to \neg (\alpha \land \beta)                       \tag{\textbf{neg}$\land_{2}$} \label{ax:axnegland2}\\
            & \neg (\alpha \to \beta) \to(\alpha \land \neg \beta)                              \tag{\textbf{neg}$\to_{1}$}   \label{ax:axnegto1}\\
            & (\alpha \land \neg \beta) \to \neg(\alpha \to \beta)                              \tag{\textbf{neg}$\to_{2}$}   \label{ax:axnegto2}\\
    \end{align*}
        \\
        \noindent\textbf{Regra de inferência:}
        \begin{prooftree}
            \AxiomC{$\alpha, \alpha \to \beta$}
            \RightLabel{MP}
            \UnaryInfC{$\beta$}
        \end{prooftree}
        Desta forma, a lógica \lfium{} é definida como $\lfium{} = \langle \ling, \conhil \rangle.$\qed{}  
    \end{definicao}

    Os axiomas (\textbf{Ax1}) {--} (\textbf{Ax10}) e a regra de inferência \textbf{MP} (\textit{modus ponens}) são importados da lógica proposicional clássica. O axioma (\textbf{bc1}) é chamado de \textit{princípio da explosão gentil}. Os axiomas (\textbf{neg}$\lor_{1}$) {--} (\textbf{neg}$\to_{2}$) expressam as leis de De Morgan em relação a negação paraconsistente $\neg$. Os axiomas (\textbf{ce}) e (\textbf{cf}) expressam, respectivamente, a introdução e a eliminação da dupla negação. O axioma (\textbf{ci}) representa a inconsistência de uma informação.

    Com isso, uma derivação em \lfium{} pode ser definida:
    
    \begin{definicao}[Derivação em \lfium{}]
        Seja $\Gamma \cup \{\phi\} \subseteq \ling{}$ um conjunto de fórmulas, uma derivação de $\phi$ a partir de $\Gamma$ em \lfium{}, denotada como $\Gamma \conhil \phi$, é uma sequência finita de fórmulas \(\phi_0, \dots, \phi_n\) onde, para cada $1 \leq i \leq n$, alguma das seguintes condições é satisfeita:
        \begin{align*}
              \text{(i) } & \phi_{i} \text{ é um axioma;}\\
              \text{(ii) } & \phi_{i} \in \Gamma;\\
              \text{(iii) } & \text{existem } j,k < i \text{ de modo que } \phi_{i} \text{ é o resultado da aplicação de MP em } \phi_{j} \text{ e } \phi_{k}.\tag*\qed{}
        \end{align*}
    \end{definicao}

    Para ilustrar, provaremos um exemplo de derivação no cálculo de Hilbert apresentado:
    
    \begin{exemplo}\label{ex:1}
        A derivação {\normalfont{} $\circ \psi, \alpha \to (\psi \land \neg \psi), \neg \alpha \to (\psi \land \neg \psi) \conhil \phi$} é válida.
    \end{exemplo}

    \begin{proof}[Prova do Exemplo~\ref{ex:1}]
        A seguinte derivação completa a prova:
        \begin{align*}
            1.~& \circ \psi \tag{Premissa} \\
            2.~& \alpha \to (\psi \land \neg \psi) \tag{Premissa} \\
            3.~& \neg \alpha \to (\psi \land \neg \psi) \tag{Premissa} \\
            4.~& \alpha \lor \neg \alpha \tag{Ax10} \\
            5.~& (\alpha \to (\psi \land \neg \psi)) \to ((\neg \alpha \to (\psi \land \neg \psi)) \to ((\alpha \lor \neg \alpha) \to (\psi \land \neg \psi))) \tag{Ax8} \\
            6.~& (\neg \alpha \to (\psi \land \neg \psi)) \to ((\alpha \lor \neg \alpha) \to (\psi \land \neg \psi)) \tag{MP 2, 5}\\
            7.~& (\alpha \lor \neg \alpha) \to (\psi \land \neg \psi) \tag{MP 3, 6}\\
            8.~& \psi \land \neg \psi \tag{MP 4, 7} \\
            9.~& (\psi \land \neg \psi) \to \psi \tag{Ax4} \\
            10.~& (\psi \land \neg \psi) \to \neg \psi \tag{Ax5} \\
            11.~& \psi \tag{MP 8, 9}\\
            12.~& \neg \psi \tag{MP 8, 10}\\
            13.~& \circ \psi \to (\psi \to (\neg \psi \to \phi)) \tag{bc1}\\
            14.~& \psi \to (\neg \psi \to \phi) \tag{MP 1, 13}\\
            15.~& \neg \psi \to \phi \tag{MP 11, 14} \\
            16.~& \phi \tag{MP, 12, 15}
        \end{align*}
    \end{proof}

    \helena{enfiar algum texto aqui}.
\section{Semântica}\label{sec:semantica}
    De forma geral, a semântica é o estudo de como um sistema de símbolos (uma linguagem) internaliza informações, ou seja, é o estudo de como interpretar os símbolos de uma linguagem~\cite{brown2005encyclopedia}. Num sistema lógico, \textit{matrizes lógicas} são comumente usadas para estabelecer o comportamento esperado dos conectivos lógicos de sua assinatura. Outra forma de compreender a semântica de um sistema lógico é definir condições que caracterizam funções conhecidas como valorações, que define uma \textit{semântica de valorações}. Nesta seção, a semântica da \lfium{} será definida de duas formas distintas: a partir de uma \textit{matriz lógica} e a partir de uma \textit{semântica de valorações}. As definições e notações para os conceitos de \textit{álgebra} definidos aqui baseiam-se nos trabalhos de~\citeshort{Carnielli_Coniglio_2016},~\citeshort{Sikorski1966-SIKAOF} e~\citeshort{Rasiowa1963-RASTMO}.
    
    \subsection{Matriz Lógica e Valoração}
        Uma das formas de definir a semântica de uma lógica proposicional é definir uma \textit{matriz lógica} (também chamada de tabela-verdade) para os conectivos de sua assinatura proposicional. Para isso, é necessário definir o conceito de \textit{álgebra} para assinaturas proposicionais:

        \begin{definicao}[Álgebra para assinaturas proposicionais]\label{def:algebra}
            Uma álgebra para uma assinatura proposicional $\Theta$ é uma dupla $\pazocal{A} = \langle A, O \rangle$, onde $A$ é um conjunto não vazio (chamado de \textit{domínio} da álgebra) e $O$ é uma função de interpretação que associa cada conectivo n-ário $c \in \Theta$ à uma operação $c^{\pazocal{A}}\!\! : \; A^{n} \to A$ em $A$.\qed{}
        \end{definicao}

        Quando não for confuso, o mesmo símbolo será utilizado para representar um conectivo $c$ e sua interpretação $O(c) = c^{\pazocal{A}}$. Além disso o símbolo utilizado para se referir a uma álgebra $\langle A, O \rangle$ será simplesmente o símbolo para seu domínio $A$. Ademais, caso $\Theta$ seja finita, a função $O$ será substituída pela lista de conectivos de $\Theta$. Por exemplo, uma álgebra para a assinatura $\Sigma$ da \lfium{} é escrita como $\pazocal{A} = \langle A,\land, \lor, \to, \neg, \circ \rangle$.

        \begin{observacao}
            Uma linguagem definida sobre uma assinatura proposicional $\Theta = \{c_{1}, \ldots, c_{n}\}$ e um conjunto enumerável de átomos $\pazocal{P} = \{p_{n} \; | \; n \in \mathbb{N} \}$ pode ser compreendida como uma álgebra da forma $\langle \pazocal{L}_{\Theta}, c_{1}\ldots, c_{n} \rangle$, onde $\pazocal{L}_{\Theta}$ é um conjunto de fórmulas bem formadas a partir dos conectivos em $\Theta$ e dos átomos em $\pazocal{P}$. Denotamos esta álgebra simplesmente por $\pazocal{L}_{\Theta}$~\cite{Sikorski1966-SIKAOF,Wojcicki1984-WJCLOP}.
        \end{observacao}

        Duas álgebras $\langle A, o_1, \ldots, o_n \rangle$ e $\langle B, o'_1, \ldots, o'_n \rangle$ definidas sobre uma mesma assinatura proposicional são ditas \textit{similares} caso, para todo $1 \leq j \leq n$, as operações $o_j$ e $o'_j$ tenham mesma aridade. Uma função (mapeamento) entre duas estruturas algebraicas similares que preserva sua estrutura é chamada de \textit{homomorfismo}. Ou seja, dadas duas álgebras similares $\langle A, O \rangle$, $\langle B, O' \rangle$ definidas sobre uma assinatura proposicional $\Theta$, um mapeamento $h : A \to B$ será um \textit{homomorfismo} caso para todo conectivo $c \in \Theta$ de aridade $n$ e para todo $a_{0},\ldots,a_{n} \in A$ tem-se $h(c(a_{0},\ldots, a_{n})) = c(h(a_{0}),\ldots, h(a_{n}))$.

        \begin{definicao}[Matriz Lógica]
            Seja $\Theta$ uma assinatura proposicional. Uma \textit{matriz lógica} $\pazocal{M}$ definida sobre $\Theta$ é uma tripla $\pazocal{M} = \langle A, D, O \rangle$, tal que o par $\langle A, O \rangle$ é uma álgebra para $\Theta$ e $D$ é um subconjunto de $A$ cujos elementos são ditos \textit{designados} (estes são elementos de $A$ considerados como verdadeiros).\qed{}
        \end{definicao}

        Com isso, uma matriz lógica $\pazocal{M} = \langle A, D, O \rangle$ induz uma relação de consequência semântica para uma lógica tarskiana $\mathcal{L}$, definida sobre uma linguagem $\pazocal{L}_{\Theta}$, da seguinte forma: sendo $\Gamma \cup \{\alpha\} \in \pazocal{L}_{\Theta}$, tem-se $\Gamma \vDash_{\pazocal{M}} \alpha$ sse, para todo homomorfismo $h : \pazocal{L}_{\Theta} \to A$, se $h[\Gamma] \subseteq D$ então $h(\alpha) \in D$. Em particular, $\alpha$ é uma tautologia em $\mathcal{L}$ sse $h(\alpha) \in D$ para todo homomorfismo $h$. Perceba que o homomorfismo $h$ nada mais é do que uma função que associa fórmulas da linguagem $\pazocal{L}_{\Theta}$ a valores do domínio $A$ da matriz $\pazocal{M}$. Desta forma, $h$ é dito uma \textit{valoração} sobre $\pazocal{M}$.


       Uma lógica que apresenta três elementos no domínio de sua matriz lógica é dita \textit{trivalorada}. Algumas lógicas trivaloradas introduzem um terceiro valor além da verdade e falsidade para representar uma informação desconhecida, como é o caso da lógica de Kleene~\cite{manyvalued}. A \lfium{}, por sua vez, introduz o valor $\meio{}$ além dos valores clássicos $0$ e $1$ para representar uma informação inconsistente. Ou seja, caso $\alpha$ tenha o valor $\meio{}$, então $\neg \alpha$ também terá o valor $\meio{}$.

        \begin{definicao}[Matriz lógica da \lfium{}]
            A matriz lógica $\pazocal{M}_{\lfium{}} = \langle M, D, O \rangle$ com domínio $M = \{1, \meio{}, 0\}$ e um conjunto de valores designados $D = \{1, \meio{}\}$ é definida da seguinte forma:
            
            \setbool{@fleqn}{false}

            \vspace{\baselineskip} % MIGS: Adicionei esse espaçamento vertical adicional aqui pois o espaçamento no texto estava estranho e, eu acho, errado

            \noindent
            \begin{minipage}{0.3\textwidth}
                % Implication (→)
                \[
                    \begin{array}{c|ccc}%chktex 44 
                        \to & 1 & \meio{} & 0 \\
                        \hline%chktex 44 
                        1           & 1 & \meio{} & 0 \\
                        \meio{} & 1 & \meio{} & 0 \\
                        0           & 1 & 1           & 1 \\
                    \end{array}
                \]
            \end{minipage}
            \begin{minipage}{0.3\textwidth}
                % Conjunction (∧)
                \[
                    \begin{array}{c|ccc}%chktex 44 
                        \land       & 1           & \meio{} & 0 \\
                        \hline%chktex 44 
                        1           & 1           & \meio{} & 0 \\
                        \meio{} & \meio{} & \meio{} & 0 \\
                        0           & 0           & 0           & 0 \\
                    \end{array}
                \]
            \end{minipage}
            \begin{minipage}{0.3\textwidth}
                % Disjunction (∨)
                \[
                    \begin{array}{c|ccc}%chktex 44 
                        \lor        & 1 & \meio{} & 0           \\
                        \hline%chktex 44 
                        1           & 1 & 1           & 1           \\
                        \meio{} & 1 & \meio{} & \meio{} \\
                        0           & 1 & \meio{} & 0           \\
                    \end{array}
                \]
            \end{minipage}

            \vspace{0.5cm}

            \begin{minipage}{0.4\textwidth}
                % Negation (¬)
                \[
                    \begin{array}{c|c}%chktex 44 
                                    & \neg        \\
                        \hline%chktex 44 
                        1           & 0           \\
                        \meio{} & \meio{} \\
                        0           & 1           \\
                    \end{array}
                \]
            \end{minipage}
            \begin{minipage}{0.3\textwidth}
                \[
                    \begin{array}{c|c}%chktex 44 
                                    & \circ   \\
                        \hline%chktex 44 
                        1           & 1         \\
                        \meio{} & 0         \\
                        0           & 1         \\
                    \end{array}
                \]
            \end{minipage}

            \vspace{\baselineskip}
            \qed{}
        \end{definicao}

        A partir disso, definimos a relação de consequência semântica $\conmat$ da seguinte forma:

        \begin{definicao}[Relação de consequência semântica $\conmat$]
            Dado um conjunto de fórmulas $\Gamma \cup \{\phi\} \subseteq \ling{}$, tem-se $\Gamma \conmat \phi$ sse, para toda valoração $h : \ling{} \to M$ de $\pazocal{M}_{\lfium{}}$, se $h[\Gamma] \subseteq \{1, \meio{}\}$ então $h(\phi) \in \{1, \meio{}\}$.\qed{}
        \end{definicao}
        Para ilustrar, provaremos um exemplo de inferência na semântica matricial apresentada:

        \begin{exemplo}\label{ex:2}
            A inferência $ \conmat \circ \circ \alpha$ é válida.
        \end{exemplo}

        \begin{proof}[Prova do Exemplo~\ref{ex:2}]
            Vamos mostrar que, para toda valoração $h : \ling{} \to \{1,\meio{},0\}$ de $\pazocal{M}_{\lfium{}}$, $h(\circ \circ \alpha) \in \{1, \meio{}\}$. Para isso, construiremos uma tabela com todas as valorações possíveis para $\circ \circ \alpha$:
            \begin{center}
                \setbool{@fleqn}{false}
                \[
                    \begin{array}{|c|c|c|}%chktex 44 
                        \hline
                        \alpha      & \circ \alpha & \circ \circ \alpha   \\
                        \hline%chktex 44 
                        1           & 1            &    1\\
                        \meio{    } & 0            &    1\\
                        0           & 1            &    1\\
                        \hline
                    \end{array}
                \]
            \end{center}

            Como é possível observar, a coluna da fórmula $\circ \circ \alpha$ é composta somente por elementos pertencentes ao conjunto $\{1, \meio{}\}$, portanto, para toda valoração $h$ de $\pazocal{M}_{\lfium{}}$, $h(\circ \circ \alpha) \in \{1, \meio{}\}$. Logo $ \conmat \circ \circ \alpha$.
            
        \end{proof}

        
        Outra forma de definir a relação de consequência semântica de uma lógica é descrever uma semântica de valorações, esta que define um conjunto de cláusulas condicionais sobre funções~\cite{DaCosta1977-NEWASA-3}. Quando uma função respeita estas cláusulas, ela é chamada de valoração para a lógica.

        \begin{definicao} [Semântica de valorações para \textbf{$\text{LFI1}$}]\label{def:valoracoes}
            Uma função $v : \ling{} \to \{1, 0\}$ é uma valoração para a lógica \lfium{} caso ela satisfaça as seguintes cláusulas:
            \begin{align*}
                & v(\alpha \land \beta) = 1 \Longleftrightarrow v(\alpha) = 1 \text{ e } v(\beta) = 1\tag{\textbf{$vAnd$}}\\
                & v(\alpha \lor \beta) = 1 \Longleftrightarrow v(\alpha) = 1 \text{ ou } v(\beta) = 1\tag{\textbf{$vOr$}}\\
                & v(\alpha \to \beta) = 1 \Longleftrightarrow v(\alpha) = 0 \text{ ou } v(\beta) = 1\tag{\textbf{$vImp$}}\\
                & v(\neg \alpha) = 0 \Longrightarrow v(\alpha) = 1\tag{\textbf{$vNeg$}}\\
                & v(\circ \alpha) = 1 \Longrightarrow v(\alpha) = 0 \text{ ou } v(\neg \alpha) = 0\tag{\textbf{$vCon$}}\\
                & v(\neg \circ \alpha) = 1 \Longrightarrow v(\alpha) = 1 \text{ e } v(\neg \alpha) = 1\tag{\textbf{$vCi$}}\\
                & v(\neg \neg \alpha) = 1 \Longrightarrow v(\alpha) = 1\tag{\textbf{$vCf$}}\\
                & v(\alpha) = 1 \Longrightarrow v(\neg \neg \alpha) = 1\tag{\textbf{$vCe$}}\\
                & v(\neg (\alpha \land \beta)) = 1 \Longleftrightarrow v(\neg \alpha) = 1 \text{ ou } v(\neg \beta) = 1\tag{\textbf{$vDM_{\land}$}}\\
                & v(\neg (\alpha \lor \beta)) = 1 \Longleftrightarrow v(\neg \alpha) = 1 \text{ e } v(\neg \beta) = 1\tag{\textbf{$vDM_{\lor}$}}\\
                & v(\neg (\alpha \to \beta)) = 1 \Longleftrightarrow v(\alpha) = 1 \text{ e } v(\neg \beta) = 1\tag{\textbf{$vCip_{\to}$}}
            \end{align*}
            O conjunto de todas as valorações para a lógica \lfium{} será denotado por $V^{\lfium{}}$.\qed{}
        \end{definicao}

        Com isso, definimos a relação de consequência semântica $\conval$ da seguinte forma:

        \begin{definicao}[Relação de consequência semântica $\conval$]
            Dado um conjunto de fórmulas $\Gamma \cup \{\phi\} \subseteq \ling{}$, tem-se $\Gamma \conval \phi$ sse, para todo $v \in V^{\lfium{}}$, se $v[\Gamma] \subseteq \{1\}$ então $v(\phi) = 1$.\qed{}
        \end{definicao}


        \helena{ainda nao sei como explicar o nome dessas cláusulas tbh}

        Note que, por conta de $(vNeg)$ e $(vCon)$, o valor $v(\triangle \alpha)$ {--} para $\triangle \in \{\neg, \circ\}$ {--} não é determinado pelo valor $v(\alpha)$ da subfórmula $\alpha$. Ou seja, os conectivos $\neg$ e $\circ$ apresentam um comportamento não determinístico em relação a esta semântica de valorações. Por exemplo, caso $v(\alpha)$ seja $1$, pode-se ter $v(\neg \alpha)$ tanto $0$ como $1$ (mas não ambos). Isto pode ser observado na seguinte tabela de possíveis valorações para $\neg \phi$ e $\circ \phi$, dada uma fórmula $\phi$:

        \begin{table}[h]
            \setbool{@fleqn}{false}
            \[
                \begin{array}{|c|c|c|c}
                    \cline{1-3}
                    \phi & \neg\phi & \circ\phi & \\ \hline
                    0 & 1 & 1 & v_1 \\ \hline
                    \multirow{2}{*}{1} & 0 & 1 & v_2 \\ \cline{2-4}
                    & 1 & 0 & v_3 \\ \hline
                \end{array}
                \]
                \caption{Valorações possíveis para $\phi$, $\neg \phi$ e $\circ \phi$, considerando $(vNeg)$, $(vCon)$ e $(vCi)$.}
                \label{tab:negcirc}
        \end{table}

        Uma semântica de valorações que define funções que mapeiam para conjuntos com somente dois elementos é chamada de \textit{bivaloração}. Este tipo de semântica possui características interessantes que a distinguem da semântica matricial. A semântica de matrizes nos permite mostrar a validade de uma inferência com facilidade, contudo, alguns lógicos e filósofos demonstram descontentamento com a existência de múltiplos valores-verdade designados, argumentando que uma distinção deve ser feita entre ``valorações algébricas'' {--} correspondente às matrizes lógicas {--} e sua ``definição genuína'', em termos de bivalorações~\cite{Suszko1975-SUSROL}. Além disso, as bivalorações servem um papel importante na teoria dos modelos, como na comparação de sistemas lógicos e na associação de outros sistemas semânticos a uma lógica já estabelecida~\cite{bivalence}.

        
                
\section{Metateoremas}\label{sec:metateoremas}
    A metalógica é uma área da matemática que estuda sistemas lógicos, desenvolvendo metateoremas~\cite{Jacquette2002-JACACT-7}. Um metateorema é uma prova sobre propriedades de um sistema formal, sobretudo sobre suas relações de consequência, utilizando uma metalinguagem~\cite{Tarski1956-TARLSM, Rasiowa1963-RASTMO, Barile_2024}. Estas propriedades são chamadas de \textit{metapropriedades}. 
    
    Algumas metapropriedades importantes de serem provadas sobre um sistema lógico munido com uma relação de consequência sintática $\vdash$ e uma relação de consequência semântica $\vDash$ (como é o caso da \lfium{}) são a correção, que afirma que tudo que é derivável em $\vdash$ é válido em $\vDash$, e a completude, que afirma que tudo que é válido em $\vDash$ pode ser provado em $\vdash$. Outra metapropriedade relevante é a equivalência entre a semântica matricial e as bivalorações da \lfium{}, apresentados na Seção~\ref{sec:semantica}, já que ambos sistemas possuem suas particularidades e vantagens. Ademais, o metateorema da dedução (definido como $\Gamma, \alpha \vdash \beta \Longleftrightarrow \Gamma \vdash \alpha \to \beta$) é uma propriedade interessante que possibilita a obtenção de outros resultados pertinentes, como será evidenciado ao longo desta seção. Por fim, caracterizaremos a lógica \lfium{} como sendo uma lógica tarskiana, finitária e \lfi{} forte.
    
    Os lemas e teoremas desenvolvidos nesta seção seguem um caminho baseado no que foi apresentado por~\citeshort{Carnielli_Coniglio_2016} e por~\citeshort{DaCosta1977-NEWASA-3}, com modificações para se adequarem ao presente trabalho.


    \begin{lema}\label{lem:matval}
        Seja $v$ uma valoração em $V^{\lfium}$, então existe uma valoração $h$ sobre $\pazocal{M}_{\lfium}$ tal que, para todo $\phi \in \ling{}$, $v(\phi) = 1$ sse $h(\phi) \in \{1, \meio{}\}$.
    \end{lema}

    \begin{proof}[Prova do Lema~\ref{lem:matval}]
        Seja $v$ uma valoração em $V^{\lfium}$. Então, considere a valoração $h$ sobre $\pazocal{M}_{\lfium}$ tal que para toda variável proposicional $p$:
        \begin{center}
            \setbool{@fleqn}{false}
            \begin{equation*}
                h(p) =
                \begin{cases}
                  1 \text{ sse } v(p) = 1 \text{ e } v(\neg p) = 0 \\
                  \meio{} \text{ sse } v(p) = 1 \text{ e } v(\neg p) = 1 \\
                  0 \text{ sse } v(p) = 0
                \end{cases}
              \end{equation*}
        \end{center}

        Então, por indução na complexidade de uma fórmula $\phi \in \ling{}$, será provado o seguinte:
        \begin{center}
            \setbool{@fleqn}{false}
            \begin{equation*}
                h(\phi) =
                \begin{cases}
                  1 \text{ sse } v(\phi) = 1 \text{ e } v(\neg \phi) = 0 \\
                  \meio{} \text{ sse } v(\phi) = 1 \text{ e } v(\neg \phi) = 1 \\
                  0 \text{ sse } v(\phi) = 0
                \end{cases}
              \end{equation*}
        \end{center}

        \noindent \textbf{\textsc{Base.}} $C(\phi) = 1$.

        Como $C(\phi) = 1$, temos $\phi \in \pazocal{P}$ pela Definição~\ref{def:complex_lfi1}. Com isso, a própria definição de $h$ prova o caso \textbf{\textsc{Base}}.

        
        \noindent \textbf{\textsc{Passo.}} 

        \noindent \textbf{\textsc{Hipótese de indução (HI):}} A propriedade vale para toda fórmula $\phi$ tal que $C(\phi) = k$, com $k < n$.

        Vamos mostrar que a propriedade vale caso $C(\phi) = n$.

        \begin{provaporcasos}
            \casodeprova{} $\phi = \neg \alpha$.
                
                \begin{provaporsubcasos}
                    \subcasodeprova{$h(\phi) = 1$ sse $v(\phi) = 1$ e $v(\neg \phi) = 0$.}
                        
                        ($\Longrightarrow$) Supondo $h(\phi) = h(\neg \alpha) = \neg h(\alpha) = 1$, vamos provar $v(\phi) = 1$ e $v(\neg \phi) = 0$. 
                        
                        Pela matriz de $\neg$ temos $h(\alpha) = 0$. Então, por (HI), $v(\alpha) = 0$. 
                        
                        Por $(vNeg)$ temos $v(\neg \alpha) = v(\phi) = 1$. Enfim, por $(vCf)$, $v(\neg \phi) = v(\neg \neg \alpha) = v(\alpha) = 0$
                        
                        
                        ($\Longleftarrow$) Supondo $v(\phi) = v(\neg \alpha) = 1$ e $v(\neg \phi) = v(\neg \neg \alpha) = 0$, vamos provar $h(\phi) = h(\neg \alpha) = 1$.
                        
                        Por $(vCe)$ temos $v(\neg \neg \alpha) = v(\alpha) = 0$.
                        
                        Por (HI) temos $h(\alpha) = 0$. Então, $h(\phi) = h(\neg \alpha) = \neg h(\alpha) = 1$, pela matriz de $\neg$.
                    
                    \subcasodeprova{$h(\phi) = \meio{}$ sse $v(\phi) = 1$ e $v(\neg \phi) = 1$.}
                    
                        ($\Longrightarrow$) Supondo $h(\phi) = h(\neg \alpha) = \meio{}$, vamos provar $v(\phi) = 1$ e $v(\neg \phi) = 1$.
                        
                        Pela matriz de $\neg$ temos $h(\alpha) = \meio{}$. Então, por (HI), $v(\alpha) = 1$ e $v(\neg \alpha) = v(\phi) = 1$.
                        
                        Enfim, por $(vCe)$ temos $v(\neg \phi) = v(\neg \neg \alpha) = v(\alpha) = 1$.
                        
                        ($\Longleftarrow$) Supondo $v(\phi) = v(\neg \alpha) = 1$ e $v(\neg \phi) = v(\neg \neg \alpha)= 1$, vamos provar $h(\phi) = \meio{}$.

                        Por $(vCf)$ temos $v(\alpha) = v(\neg \neg \alpha) = 1$.
                        
                        Por (HI), temos $h(\alpha) = \meio{}$. Então, pela matriz de $\neg$, temos $h(\phi) = h(\neg \alpha) = \neg h(\alpha) = \meio{}$.
                    
                    \subcasodeprova{$h(\phi) = 0$ sse $v(\phi) = 0$.}
                        
                        ($\Longrightarrow$) Supondo $h(\phi) = h(\neg \alpha) = \neg h(\alpha) = 0$, vamos provar $v(\phi) = v(\neg \alpha) = 0$.
                        
                        Pela matriz de $\neg$ temos $h(\alpha) = 1$. Então, por (HI), $v(\alpha) = 1$ e $v(\phi) = v(\neg \alpha) = 0$.
                    
                        ($\Longleftarrow$) Supondo $v(\phi) = v(\neg \alpha) = 0$, vamos provar $h(\phi) = h(\neg \alpha) = 0$.
                    
                        Por $(vNeg)$ temos $v(\alpha) = 1$. Logo, por (HI), temos $h(\alpha) = 1$.
                        
                        Pela matriz de $\neg$ temos $h(\phi) = h(\neg \alpha) = \neg h(\alpha) = 0$.
                \end{provaporsubcasos}
                    
            \casodeprova{} $\phi = \circ \alpha$.
                    
                \begin{provaporsubcasos}
                    
                    \subcasodeprova{$h(\phi) = 1$ sse $v(\phi) = 1$ e $v(\neg \phi) = 0$.}

                        ($\Longrightarrow$) Supondo $h(\phi) = h(\circ \alpha) = \circ h(\alpha) = 1$, vamos provar $v(\phi) = 1$ e $v(\neg \phi) = 0$.
                        
                        Pela matriz de $\circ$, temos $h(\alpha) = 1$ ou $h(\alpha) = 0$.

                        \begin{provaporsubsubcasos}
                            \subsubcasodeprova{$h(\alpha) = 1$.} 
                                
                                Por (HI) temos $v(\alpha) = 1$ e $v(\neg \alpha) = 0$.
                            
                                Por $(vCi)$ temos $v(\neg \phi) = v(\neg \circ \alpha) = 0$. Então, por $(vNeg)$, temos $v(\phi) = v(\circ \alpha) = 1$.
                            
                            \subsubcasodeprova{$h(\alpha) = 0$.}
                                
                                Por (HI) temos $v(\alpha) = 0$.
                            
                                Por $(vCi)$, temos $v(\neg \phi) = v(\neg \circ \alpha) = 0$. Então, por $(vNeg)$ temos $v(\phi) = 1$.

                                \newcounter{buffer}
                                \setcounter{buffer}{\theSubSubCasos}
                        \end{provaporsubsubcasos}
                        
                        ($\Longleftarrow$) Supondo $v(\phi) = v(\circ \alpha) = 1$ e $v(\neg \phi) = v(\neg \circ \alpha) = 0$, vamos provar $h(\phi) = h(\circ \alpha) = 1$.
                        
                        Por $(vCon)$ temos $v(\alpha) = 0$ ou $v(\neg \alpha) = 0$.

                        \begin{provaporsubsubcasos}
                            \setcounter{SubSubCasos}{\thebuffer}

                            \subsubcasodeprova{$v(\alpha) = 0$.}

                                Por (HI) temos $h(\alpha) = 0$. Então, pela matriz de $\circ$, temos $h(\phi) = h(\circ \alpha) = \circ h(\alpha) = 1$.
                            
                            \subsubcasodeprova{$v(\neg \alpha) = 0$.} 

                                Por $(vNeg)$ temos $v(\alpha) = 1$. Então, por (HI), temos $h(\alpha) = 1$. 
                                
                                Finalmente, pela matriz de $\circ$, temos $h(\phi) = h(\circ \alpha) = \circ h(\alpha) = 1$.
                        \end{provaporsubsubcasos}
                        
                    \subcasodeprova{$h(\phi) = \meio{}$ sse $v(\phi) = 1$ e $v(\neg \phi) = 1$.}
                        
                        ($\Longrightarrow$) Pela matriz lógica de $\circ$, nunca temos $h(\phi) = h(\circ \alpha) = \meio{}$.
                        
                        ($\Longleftarrow$) Supondo $v(\phi) = 1$ e $v(\neg \phi) = 1$, então, por $(vCon)$ aplicado a $v(\phi) = v(\alpha) = 1$, temos $v(\alpha) = 0$ ou $v(\neg \alpha) = 0$. 
                        
                        Entretanto, por $(vCi)$ aplicado a $v(\neg \phi) = v(\neg \alpha) = 1$, temos $v(\alpha) = 1$ e $v(\neg \alpha) = 1$.

                        Portanto, nunca temos $v(\phi) = 1$ e $v(\neg \phi) = 1$.

                        
                    
                    \subcasodeprova{$h(\phi) = 0$ sse $v(\phi) = 0$.}

                        ($\Longrightarrow$) Supondo $h(\phi) = h(\circ \alpha) = \circ h(\alpha) = 0$, vamos provar $v(\phi) = v(\circ \alpha) = 0$.

                        Pela matriz de $\circ$, temos $h(\alpha) = \meio{}$. Então, por (HI), temos $v(\alpha) = 1$ e $v(\neg \alpha) = 1$. 
                        
                        Finalmente, por $(vCon)$, temos $v(\phi) = v(\circ \alpha) = 0$.
                    
                        ($\Longleftarrow$) Supondo $v(\phi) = v(\circ \alpha) = 0$, vamos provar $h(\phi) = h(\circ \alpha) = 0$.

                        Por $(vCe)$ temos $v(\neg \neg \circ \alpha) = 0$. Então, por $(vNeg)$, temos $v(\neg \circ \alpha) = 1$.

                        Por $(vCi)$ temos $v(\alpha) = 1$ e $v(\neg \alpha) = 1$. Logo, por (HI), temos $h(\alpha) = \meio{}$.

                        Finalmente, pela matriz de $\circ$, temos $h(\phi) = h(\circ \alpha) = \circ h(\alpha) = 0$.
                    
            \end{provaporsubcasos}
        
            \casodeprova{$\phi = \alpha \land \beta$.}

            \begin{provaporsubcasos}
                \subcasodeprova{$h(\phi) = h(\alpha \land \beta) = h(\alpha) \land h(\beta) = 1$ sse $v(\phi) = v(\alpha \land \beta) = 1$ e $v(\neg \phi) = v(\neg (\alpha \land \beta)) = 0$.}

                    Teremos $h(\phi) = h(\alpha \land \beta) = h(\alpha) \land h(\beta) = 1$

                    \qquad{}sse, pela matriz de $\land$, $h(\alpha) = 1$ e $h(\beta) = 1$

                    \qquad{}sse, por (HI), $v(\alpha) = 1$, $v(\neg \alpha) = 0$, $v(\beta) = 1$ e $v(\neg \beta) = 0$

                    \qquad{}sse, por $(vAnd)$, $v(\phi) = v(\alpha \land \beta) = 1$ e, por $(vDM_{\land})$, $v(\neg \phi) = v(\neg(\alpha \land \beta)) = 0$.
            
                \subcasodeprova{$h(\phi) = h(\alpha \land \beta) = h(\alpha) \land h(\beta) = \meio{}$ sse $v(\phi) = v(\alpha \land \beta) = 1$ e $v(\neg \phi) = v(\neg (\alpha \land \beta)) = 1$.}

                    Teremos $h(\phi) = h(\alpha \land \beta) = h(\alpha) \land h(\beta) = \meio{}$

                    \qquad{}sse, pela matriz de $\land$, temos $h(\alpha), h(\beta) \in \{1, \meio{}\}$ e temos $h(\alpha) = \meio{}$ ou $h(\beta) = \meio{}$
                    
                    \qquad{}sse, por (HI), temos $v(\alpha) = 1$ e $v(\beta) = 1$ e temos $v(\alpha) = v(\neg \alpha) = 1$ ou $v(\beta) = v(\neg \beta) = 1$

                    \qquad{}sse, pela distributividade da conjunção sobre a disjunção e pela idempotência da conjunção, temos $v(\alpha) = 1$, $v(\neg \alpha) = 1$ e $v(\beta) = 1$ ou temos $v(\alpha) = 1$, $v(\neg \beta) = 1$ e $v(\beta) = 1$

                    \qquad{}sse, pela distributividade da conjunção sobre a disjunção e pela idempotência da conjunção, temos $v(\alpha) = 1$ e $v(\beta) = 1$ e temos $v(\neg \alpha) = 1$ ou $v(\neg \beta) = 1$

                    \qquad{}sse, por $(vAnd)$ e $(vDM_{\land})$, temos $v(\phi) = v(\alpha \land \beta) = 1$ e $v(\neg \phi) = v(\neg (\alpha \land \beta)) = 1$.

                \subcasodeprova{$h(\phi) = h(\alpha \land \beta) = h(\alpha) \land h(\beta) = 0$ sse $v(\phi) = v(\alpha \land \beta) = 0$.}

                    Teremos $h(\phi) = h(\alpha \land \beta) = h(\alpha) \land h(\beta) = 0$

                    \qquad{}sse, pela matriz de $\land$, $h(\alpha) = 0$ ou $h(\beta) = 0$

                    \qquad{}sse, por (HI), $v(\alpha) = 0$ ou $v(\beta) = 0$

                    \qquad{}sse, por $(vAnd)$, $v(\phi) = v(\alpha \land \beta) = 0$.

            \end{provaporsubcasos}

            \casodeprova{$\phi = \alpha \lor \beta$.}

            \begin{provaporsubcasos}
                \subcasodeprova{$h(\phi) = h(\alpha \lor \beta) = h(\alpha) \lor h(\beta) = 1$ sse $v(\phi) = v(\alpha \lor \beta) = 1$ e $v(\neg \phi) = v(\neg (\alpha \lor \beta)) = 0$.}

                    Temos $h(\phi) = h(\alpha \lor \beta) = h(\alpha) \lor h(\beta) = 1$

                    \qquad{}sse, pela matriz de $\lor$, $h(\alpha) = 1$ ou $h(\beta) = 1$

                    \qquad{}sse, por (HI), $v(\alpha) = v(\beta) = 1$ e $v(\neg \alpha) = v(\neg \beta) = 0$

                    \qquad{}sse, por $(vOr)$, $v(\phi) = v(\alpha \lor \beta) = 1$ e, por $(vDM_{\lor})$, $v(\neg \phi) = v(\neg(\alpha \lor \beta)) = 0$.

                \subcasodeprova{$h(\phi) = h(\alpha \lor \beta) = h(\alpha) \lor h(\beta) = \meio{0}$ sse $v(\phi) = v(\alpha \lor \beta) = 1$ e $v(\neg \phi) = v(\neg (\alpha \lor \beta)) = 1$.}

                    Temos $h(\phi) = h(\alpha \lor \beta) = h(\alpha) \lor h(\beta) = \meio{}$

                    \qquad{}sse, pela matriz de $\lor$, temos $h(\alpha) = h(\beta) = \meio{}$ ou temos $h(\alpha) = \meio{}$ e $h(\beta) = 0$ ou temos $h(\alpha) = 0$ e $h(\beta) = \meio{}$.

                    \qquad{}sse, por (HI), temos $v(\alpha) = 1$, $v(\neg \alpha) = 1$, $v(\beta) = 1$ e $v(\neg \beta) = 1$ ou temos $v(\alpha) = 1$, $v(\neg \alpha) = 1$ e $v(\beta) = 0$ ou temos $v(\alpha) = 0$, $v(\beta) = 1$ e $v(\neg \beta) = 1$

                    \qquad{}sse, por $(vNeg)$, temos $v(\alpha) = 1$, $v(\neg \alpha) = 1$, $v(\beta) = 1$ e $v(\neg \beta) = 1$ ou temos $v(\alpha) = 1$, $v(\neg \alpha) = 1$, $v(\beta) = 0$ e $v(\neg \beta) = 1$ ou temos $v(\alpha) = 0$, $v(\neg \alpha) = 1$, $v(\beta) = 1$, $v(\neg \beta) = 1$

                    \qquad{}sse, pela distributividade da disjunção sobre a conjunção, temos $v(\neg \alpha) = 1$ e $v(\neg \beta) = 1$ e temos $v(\alpha) = 1$ e $v(\beta) = 1$ ou $v(\alpha) = 1$ e $v(\beta) = 0$ ou $v(\alpha) = 0$ e $v(\beta) = 1$.

                    \qquad{}sse, por $(vOr)$, temos $v(\neg \alpha) = 1$ e $v(\neg \beta) = 1$ e temos $v(\alpha \lor \beta) = 1$ ou $v(\alpha \lor \beta) = 1$ ou $v(\alpha \lor \beta) = 1$
                    
                    \qquad{}sse, por $(vDM_{\lor})$ e pela idempotência da disjunção, temos $v(\phi) = v(\alpha \lor \beta) = 1$ e $v(\neg \phi) = v(\neg(\alpha \lor \beta)) = 1$.

                \subcasodeprova{$h(\phi) = h(\alpha \lor \beta) = h(\alpha) \lor h(\beta) = 0$ sse $v(\phi) = v(\alpha \lor \beta) = 0$.}

                    Temos $h(\phi) = h(\alpha \lor \beta) = h(\alpha) \lor h(\beta) = 0$

                    \qquad{}sse, pela matriz de $\lor$, temos $h(\alpha) = 0$ e $h(\beta) = 0$

                    \qquad{}sse, por (HI), temos $v(\alpha) = 0$ e $v(\beta) = 0$

                    \qquad{}sse, por $(vOr)$, temos $v(\phi) = v(\alpha \lor \beta) = 0$.

            \end{provaporsubcasos}

            \casodeprova{$\phi = \alpha \to \beta$.}

            \begin{provaporsubcasos}
                \subcasodeprova{$h(\phi) = h(\alpha \to \beta) = h(\alpha) \to h(\beta) = 1$ sse $v(\phi) = v(\alpha \to \beta) = 1$ e $v(\neg \phi) = v(\neg (\alpha \to \beta)) = 0$.}

                    Temos $h(\phi) = h(\alpha \to \beta) = h(\alpha) \to h(\beta) = 1$

                    \qquad{}sse, pela matriz de $\to$, $h(\alpha) = 0$ ou $h(\beta) = 1$

                    \qquad{}sse, por (HI), temos $v(\alpha) = 0$ ou temos $v(\beta) = 1$ e $v(\neg \beta) = 0$

                    \qquad{}sse, por $(vImp)$ e $(vCip)$, temos $v(\alpha \to \beta) = 1$ e $v(\neg (\alpha \to \beta)) = 0$ ou temos $v(\alpha \to \beta) = 1$ e $v(\neg (\alpha \to \beta)) = 0$

                    \qquad{}sse, pela idempotência da disjunção, temos $v(\phi) = v(\alpha \to \beta) = 1$ e $v(\neg \phi) = v(\neg (\alpha \to \beta)) = 0$

                \subcasodeprova{$h(\phi) = h(\alpha \to \beta) = h(\alpha) \to h(\beta) = \meio{}$ sse $v(\phi) = v(\alpha \to \beta) = 1$ e $v(\neg \phi) = v(\neg (\alpha \to \beta)) = 1$.}

                    Temos $h(\phi) = h(\alpha \to \beta) = h(\alpha) \to h(\beta) = \meio{}$

                    \qquad{}sse, pela matriz de $\to$, temos $h(\alpha) = 1$ e $h(\beta) = \meio{}$ ou temos $h(\alpha) = \meio{}$ e $h(\beta) = \meio{}$

                    \qquad{}sse, por (HI), temos $v(\alpha) = v(\beta) = v(\neg \beta) = 1$ e $v(\neg \alpha) = 0$ ou temos $v(\alpha) = v(\neg \alpha) = v(\beta) = v(\neg \beta) = 1$

                    \qquad{}sse, por $(vImp)$ e $(vCip)$, temos $v(\alpha \to \beta) = v(\neg (\alpha \to \beta)) = 1$ e $v(\alpha \to \beta) = v(\neg (\alpha \to \beta)) = 1$

                    \qquad{}sse, pela idempotência da disjunção, temos $v(\phi) = v(\alpha \to \beta) = 1$ e $v(\neg \phi) = v(\neg (\alpha \to \beta)) = 1$.

                \subcasodeprova{$h(\phi) = h(\alpha \to \beta) = h(\alpha) \to h(\beta) = 0$ sse $v(\phi) = v(\alpha \to \beta) = 0$.}

                    Temos $h(\phi) = h(\alpha \to \beta) = h(\alpha) \to h(\beta) = 0$

                    \qquad{}sse, pela matriz de $\to$, temos $h(\alpha) = 1$ e $h(\beta) = 0$ ou temos $h(\alpha) = \meio{}$ e $h(\beta) = 0$

                    \qquad{}sse, por (HI), temos $v(\alpha) = 1$ e $v(\neg \alpha) = v(\beta) = 0$ ou temos $v(\alpha) = v(\neg \alpha) = 1$ e $v(\beta) = 0$

                    \qquad{}sse, por $(vImp)$, temos $v(\alpha \to \beta) = 0$ ou $v(\alpha \to \beta) = 0$

                    \qquad{}sse, pela idempotência da disjunção, temos $v(\alpha \to \beta) = 0$.
            \end{provaporsubcasos}
        \end{provaporcasos}
    \end{proof}

    \begin{corolario}\label{cor:matval}
        Para todo conjunto de fórmulas $\Gamma \cup \{\phi\} \subseteq \ling{}$:

        \centering
        $\Gamma \conmat \phi \Longrightarrow \Gamma \conval \phi$. 
    \end{corolario}

    \begin{proof}[Prova do Corolário~\ref{cor:matval}]
        Seja $\Gamma \cup \{\phi\} \subseteq \ling{}$ um conjunto de fórmulas e seja $v \in V^{\lfium{}}$ uma valoração. Vamos supor $\Gamma \conmat \phi$.

        Pelo Lema~\ref{lem:matval}, então existe uma valoração $h$ tal que, para todo $\psi \in \ling{}$, $v(\psi) = 1$ sse $h(\psi) \subseteq \{1, \meio{}\}$. Supondo $v[\Gamma] = 1$, então temos $h[\Gamma] \subseteq \ummeio{}$. Pela nossa suposição de $\Gamma \conmat \phi$, temos $h(\phi) \in \{1, \meio{}\}$. Novamente, pelo Lema~\ref{lem:matval}, temos $v(\phi) = 1$. Portanto, segue $\Gamma \conval \phi$.

    \end{proof}

    O metateorema da dedução é uma propriedade conveniente de ser provada, já que nos fornece corolários importantes a fim de desenvolver provas de outros metateoremas.
    
    \begin{teorema}[Metateorema da dedução para $\conhil$]\label{teo:deducao}
        Para todo conjunto de fórmulas $\Gamma \cup \{\alpha, \beta \} \subseteq \ling{}$:

        \centering
        {\normalfont{} $\Gamma, \alpha \conhil \beta \Longleftrightarrow \Gamma \conhil \alpha \to \beta$.}
    \end{teorema}

    O seguinte lema torna a prova do teorema mais imediata:
    \begin{lema}\label{lem:id}
        A derivação $\Gamma \conhil \alpha \to \alpha$ é válida para todo $\Gamma \cup \{\alpha\} \subseteq \ling{}$.
    \end{lema}
    
    \begin{proof}[Prova do Lema~\ref{lem:id}]
        A seguinte sequência de derivação demonstra o lema:
        
        \begin{align*}
            & \text{1. } (\alpha \to ((\alpha \to \alpha) \to \alpha)) \to ((\alpha \to (\alpha \to \alpha)) \to (\alpha \to \alpha))\tag{Ax2}\\
            & \text{2. } \alpha \to ((\alpha \to \alpha) \to \alpha)\tag{Ax1}\\
            & \text{3. } \alpha \to (\alpha \to \alpha)\tag{Ax1}\\
            & \text{4. } (\alpha \to (\alpha \to \alpha)) \to (\alpha \to \alpha)\tag{MP 1,2}\\
            & \text{5. } \alpha \to \alpha\tag{MP 3,4}
        \end{align*}
    \end{proof}

    \begin{proof}[Prova do Teorema~\ref{teo:deducao}] A prova será dividida em duas partes:\\
        ($\Longleftarrow$) Supondo $\Gamma \conhil \alpha \to \beta$, então existe uma sequência de derivação $\phi_{1} \ldots \phi_{n}$ onde $\phi_{n} = \alpha \to \beta$. 
        
        A seguinte sequência de derivação completa a prova de $\Gamma, \alpha \conhil \beta$:
        \begin{align*}
            \text{1}&.~ \; \ldots\\
            & \vdots \; ~\ddots\\
            \text{$n$}&.~ \alpha \to \beta\tag{Suposição}\\
            \text{$n + 1$}&.~ \alpha\tag{Premissa}\\
            \text{$n + 2$}&.~ \beta\tag{MP $n, n + 1$}
        \end{align*}

        \noindent  ($\Longrightarrow$) Supondo $\Gamma, \alpha \conhil \beta$, então existe uma sequência de derivação $\phi_{1} \ldots \phi_{n}$ onde $\phi_{n} = \beta$ a partir do conjunto de premissas $\Gamma \cup \{\alpha\}$. A prova de $\Gamma \conhil \alpha \to \beta$ é feita pela indução no tamanho $n$ da sequência de derivação:\\

        \noindent \textbf{\textsc{Base.}} $n = 1$.
        A sequência contem somente uma fórmula $\phi_{1} = \beta$. Portanto, existem duas possibilidades:
        \begin{enumerate}
            \item $\phi_{1}$ é um axioma.
            \item $\phi_{1} \in \Gamma \cup \{\alpha\}$.
        \end{enumerate}

        \begin{provaporcasos}
            \casodeprova{} $\phi_{1}$ é um axioma. 
            
                A seguinte derivação mostra $\Gamma \conhil \alpha \to \phi_{1}$:
                \begin{align*}
                    & \text{1. } \phi_{1} \tag{Axioma}\\
                    & \text{2. } \phi_{1} \to (\alpha \to \phi_{1}) \tag{Ax1}\\
                    & \text{3. } \alpha \to \phi_{1} \tag{MP 1,2}
                \end{align*}

                \casodeprova{} $\phi_{1} \in \Gamma \cup \{\alpha\}$. 
                
                Existem dois casos a serem considerados:

                \begin{enumerate}
                    \item[2.1] $\phi_{1} = \alpha$
                    \item[2.2] $\phi_{1} \in \Gamma$
                \end{enumerate}

                \begin{provaporsubcasos}
                    \subcasodeprova{} $\phi_{1} = \alpha$. 
                    
                        É necessário mostrar $\Gamma \conhil \alpha \to \alpha$, o que foi provado pelo Lema~\ref{lem:id}.

                    \subcasodeprova{} $\phi_{1} \in \Gamma$.
                    
                        Então $\Gamma \conhil \alpha \to \phi_{1}$ é provado pela seguinte sequência de derivações:
                        \begin{align*}
                            & \text{1. } \phi_{1} \tag{Premissa}\\
                            & \text{2. } \phi_{1} \to (\alpha \to \phi_{1}) \tag{Ax1}\\
                            & \text{3. } \alpha \to \phi_{1} \tag{MP 1, 2}
                        \end{align*}
                \end{provaporsubcasos}
                
            \end{provaporcasos}
            
            Portanto, $\Gamma \conhil \alpha \to \phi_{1}$ segue para o caso \textbf{\textsc{Base.}}.


        \noindent \textbf{\textsc{Passo.}}

        \noindent \textbf{\textsc{Hipótese de indução (HI):}} Para qualquer sequência de derivação $\Gamma, \alpha \conhil \beta$ de tamanho $i$, com $i < n$, tem-se $\Gamma \conhil \alpha \to \beta$. 

        É preciso mostrar que $\Gamma \conhil \alpha \to \beta$ segue caso a dedução $\Gamma, \alpha \conhil \beta$ seja de tamanho $n$. Então, vamos supor $\Gamma, \alpha \conhil \phi_{n}$ e mostrar $\Gamma \conhil \alpha \to \phi_{n}$.
        
        Ao analisar a obtenção de $\phi_{n}$ na sequência de derivação de $\Gamma, \alpha\conhil \phi_{n}$, existem três casos a se considerar:
        \begin{enumerate}
            \item $\phi_{n}$ é um axioma.
            \item $\phi_{n} \in \Gamma \cup \{\alpha\}$.
            \item $\phi_{n}$ é obtido por \textit{modus ponens} em duas fórmulas $\phi_{j}$ e $\phi_{k}$ com $j, k < n$.
        \end{enumerate}
        
         Os casos 1 e 2 são análogos aos casos provados na base.

         \noindent \textsc{Caso 3.} $\phi_{n}$ é obtido por \textit{modus ponens} em duas fórmulas $\phi_{j}$ e $\phi_{k}$ com $j, k < n$. 
         
         Então, $\phi_{k} = \phi_{j} \to \phi_{n}$ (ou $\phi_{j} = \phi_{k} \to \phi_{n}$, a prova para este caso é análoga). 
         
         Dada a nossa suposição de $\Gamma, \alpha \conhil \phi_{n}$, temos $\Gamma, \alpha \conhil \phi_{j}$ e $\Gamma, \alpha \conhil \phi_{j} \to \phi_{n}$ sequências de dedução anteriores à aplicação da regra do \textit{modus ponens} na linha $n$. 
         
         Então, pela (HI), temos $\Gamma \conhil \alpha \to \phi_{j}$ e $\Gamma \conhil \alpha \to (\phi_{j} \to \phi_{n})$. 
         
         A seguinte sequência de derivação mostra $\Gamma \conhil \alpha \to \phi_{n}$:
         \begin{align*}
             \text{1}&.~ \; \ldots\\
             &\vdots \; ~\ddots\\
             \text{$j$}&.~ \alpha \to \phi_{j} \tag{HI sobre $\phi_{j}$}\\
             &\vdots \; ~\ddots\\
             \text{$j + k$}&.~ \alpha \to (\phi_{j} \to \phi_{n}) \tag{HI sobre $\phi_{k}$}\\
             \text{$j + k + 1$}&.~ (\alpha \to (\phi_{j} \to \phi_{n})) \to ((\alpha \to \phi_{j}) \to (\alpha \to \phi_{n})) \tag{Ax2}\\
             \text{$j + k + 2$}&.~ (\alpha \to \phi_{j}) \to (\alpha \to \phi_{n}) \tag{MP $j + k$\text{,}$j + k + 1$}\\
              \text{$j + k + 3$}&.~ \alpha \to \phi_{n} \tag{MP $j$\text{,}$j + k + 2$}
         \end{align*}
         Portanto, temos $\Gamma \conhil \alpha \to \beta$ e a prova está finalizada.
         
    \end{proof}


    No desenvolvimento das provas que seguem a abreviação MTD representará o metateorema da dedução.
    
    Com a prova do metateorema da dedução para o cálculo de Hilbert da \lfium{}, os seguintes corolários são imediatos:

    \begin{corolario}\label{cor:intro_ou_esq}
        Para todo conjunto de fórmulas $\Gamma \cup \{\alpha, \beta, \phi\} \subseteq \ling{}$:

        \centering
        {\normalfont{}Se $\Gamma, \alpha \conhil \phi \text{ e }\Gamma, \beta \conhil \phi \text{ então } \Gamma, \alpha \lor \beta \conhil \phi$.}
    \end{corolario}

    \begin{proof}[Prova do Corolário~\ref{cor:intro_ou_esq}]
        Seja $\Gamma \cup \{\alpha, \beta, \phi\} \subseteq \ling{}$ um conjunto de fórmulas qualquer.
        
        Suponha $\Gamma, \alpha \conhil \phi \text{ e }\Gamma, \beta \conhil \phi$. 
        Então, por MTD (Teorema~\ref{teo:deducao}), temos $\Gamma \conhil \alpha \to \phi \text{ e }\Gamma \conhil \beta \to \phi$.

        A seguinte sequência de derivação mostra $\Gamma, \alpha \lor \beta \conhil \phi$:
        \begin{align*}
            1. ~& \alpha \to \phi \tag{MTD aplicado à suposição} \\
            2. ~& \beta \to \phi \tag{MTD aplicado à suposição} \\
            3. ~& \alpha \lor \beta \tag{Premissa} \\
            4. ~& (\alpha \to \phi) \to ((\beta \to \phi) \to ((\alpha \lor \beta) \to \phi)) \tag{Ax8} \\
            5. ~& (\beta \to \phi) \to ((\alpha \lor \beta) \to \phi) \tag{MP 1, 4}\\
            6. ~& (\alpha \lor \beta) \to \phi \tag{MP 2, 5} \\
            7. ~& \phi \tag{MP 3, 6}
        \end{align*}
    \end{proof}


    \begin{corolario}\label{cor:prova_por_casos}
        Para todo conjunto de fórmulas $\Gamma \cup \{\alpha, \beta, \phi\} \subseteq \ling{}$:

        \centering
        {\normalfont{}Se $\Gamma, \alpha \conhil \phi \text{ e }\Gamma, \neg\alpha \conhil \phi \text{ então } \Gamma \conhil \phi$.}
    \end{corolario}

    \begin{proof}[Prova do Corolário~\ref{cor:prova_por_casos}]
        Seja $\Gamma \cup \{\alpha, \beta, \phi\} \subseteq \ling{}$ um conjunto de fórmulas qualquer.

        Suponha $\Gamma, \alpha \conhil \phi \text{ e }\Gamma, \neg\alpha \conhil \phi$. Pelo Corolário~\ref{cor:intro_ou_esq}, temos $\Gamma, \alpha \lor \neg \alpha \conhil \phi$. Por MTD (Teorema~\ref{teo:deducao}), temos $\Gamma \conhil (\alpha \lor \neg \alpha) \to \phi$.

        A seguinte sequência de derivação mostra $\Gamma\conhil \phi$:
        \begin{align*}
            1. ~& (\alpha \lor \neg \alpha) \to \phi\tag{MTD aplicado à suposição} \\
            2. ~& \alpha \lor \neg \alpha \tag{Ax10} \\
            3. ~& \phi \tag{MP 1,2}
        \end{align*}
    \end{proof}

    A ideia da prova da correção para a \lfium{} é, como de costume, provar que todo axioma do cálculo de Hilbert $\conhil$ é válido na semântica matricial $\conmat$ e mostrar que a regra de inferência \textit{modus ponens} preserva sua validade. No desenvolvimento da prova, as igualdades do tipo $h(\triangle \phi) = \triangle h(\phi)$ e $h(\phi \otimes \psi) = h(\phi) \otimes h(\psi)$, onde $\triangle, \otimes$ são conectivos quaisquer, ficam implícitas.


    \begin{teorema}[Correção em relação a semântica matricial]\label{teo:correcao_mat}
        A lógica {\normalfont\lfium{}} é correta em relação a sua semântica matricial, ou seja, para todo conjunto de fórmulas $\Gamma \cup \{\alpha\} \subseteq \ling{}$:

        \centering
        {\normalfont{} $\Gamma \conhil \alpha \Longrightarrow \Gamma \conmat \alpha$.}
    \end{teorema}


    \begin{proof}[Prova do Teorema~\ref{teo:correcao_mat}]
        Seja $\Gamma \cup \{\alpha\} \subseteq{} \ling{}$ um conjunto de fórmulas.
        Supondo $\Gamma \conhil \alpha$, existe uma sequência de derivação $\phi_{1} \ldots \phi_{n}$ onde $\phi_{n} = \alpha$. A prova de $\Gamma \conmat \alpha$ é obtida por indução no tamanho $n$ da sequência de derivação:\\

        \noindent \textbf{\textsc{Base.}} $n = 1$. A sequência contém somente uma fórmula $\phi_{1} = \alpha$. Portanto, existem duas possibilidades:
        \begin{enumerate}
            \item $\phi_{1}$ é um axioma.
            \item $\phi_{1} \in \Gamma$.
        \end{enumerate}

        \begin{provaporcasos}
            
            \casodeprova{} $\phi_{1}$ é um axioma. Então vamos mostrar que para toda valoração $h : \ling{} \to M$ sobre $\mat{}$, se $h[\Gamma] \subseteq \ummeio{}$, então $h(\phi_{1}) \in \ummeio{}$. Como $\phi_{1}$ é um axioma, basta analisar todos os casos possíveis:

            \begin{provaporsubcasos}
                
                \subcasodeprova{} $\phi_{1} = \alpha \to (\beta \to \alpha)$
                    \begin{center}
                        \setbool{@fleqn}{false}
                        \[
                            \begin{array}{|c|c|c|}%chktex 44 
                                \hline
                                \alpha      & \beta & \alpha \to (\beta \to \alpha)  \\
                                \hline%chktex 44 
                                1            & 1            &    1\\
                                1            & \meio{}  &    1\\
                                1            & 0            &    1\\
                                \meio{}  & 1            &    \meio{}\\
                                \meio{}  & \meio{}  &    \meio{}\\
                                \meio{}  & 0            &    1\\
                                0            & 1            &    1\\
                                0            & \meio{}  &    1\\
                                0            & 0            &    1\\
                                \hline
                            \end{array}
                        \]
                    \end{center}

                \subcasodeprova{} $\phi_{1} = (\alpha \to (\beta \to \gamma)) \to ((\alpha \to \beta) \to (\alpha \to \gamma ))$.
                
                    \begin{center}
                        \setbool{@fleqn}{false}
                        \[
                            \begin{array}{|c|c|c|c|}%chktex 44 
                                \hline
                                \alpha      & \beta & \gamma & (\alpha \to (\beta \to \gamma)) \to ((\alpha \to \beta) \to (\alpha \to \gamma)) \\
                                \hline%chktex 44 
                                1           & 1           & 1           & 1 \\
                                1           & 1           & \meio{} & \meio{} \\
                                1           & 1           & 0           & 1 \\
                                1           & \meio{} & 1           & 1 \\
                                1           & \meio{} & \meio{} & \meio{} \\
                                1           & \meio{} & 0           & \meio{} \\
                                1           & 0           & 1           & 1 \\
                                1           & 0           & \meio{} & 1 \\
                                1           & 0           & 0           & 1 \\
                                \meio{} & 1           & 1           & 1 \\
                                \meio{} & 1           & \meio{} & \meio{} \\
                                \meio{} & 1           & 0           & \meio{} \\
                                \meio{} & \meio{} & 1           & 1 \\
                                \meio{} & \meio{} & \meio{} & \meio{} \\
                                \meio{} & \meio{} & 0           & \meio{} \\
                                \meio{} & 0           & 1           & 1 \\
                                \meio{} & 0           & \meio{} & \meio{} \\
                                \meio{} & 0           & 0           & \meio{} \\
                                0           & 1           & 1           & 1 \\
                                0           & 1           & \meio{} & 1 \\
                                0           & 1           & 0           & 1 \\
                                0           & \meio{} & 1           & 1 \\
                                0           & \meio{} & \meio{} & 1 \\
                                0           & \meio{} & 0           & 1 \\
                                0           & 0           & 1           & 1 \\
                                0           & 0           & \meio{} & 1 \\
                                0           & 0           & 0           & 1 \\
                                \hline
                            \end{array}
                        \]
                    \end{center}
                

                \subcasodeprova{} $\phi_{1} = \alpha \to (\beta \to (\alpha \land \beta))$. 

                \begin{center}
                    \setbool{@fleqn}{false}
                    \[
                        \begin{array}{|c|c|c|}%chktex 44 
                            \hline
                            \alpha      & \beta & \alpha \to (\beta \to (\alpha \land \beta)) \\
                            \hline%chktex 44 
                            1&1&               1\\ 
                            1&\meio{}&\meio{}\\
                            1&0&1\\
                            \meio{}&1&\meio{}\\
                            \meio{}&\meio{}&\meio{}\\
                            \meio{}&0&1\\
                            0&1&1\\
                            0&\meio{}&1\\
                            0&0&1\\
                            \hline
                        \end{array}
                    \]
                \end{center}

                   

                \subcasodeprova{} $\phi_{1} = (\alpha \land \beta) \to \alpha$. 


                \begin{center}
                    \setbool{@fleqn}{false}
                    \[
                        \begin{array}{|c|c|c|}%chktex 44 
                            \hline
                            \alpha      & \beta & (\alpha \land \beta) \to \alpha \\
                            \hline%chktex 44 
                            1 & 1 &                 1\\
                            1 & \meio{} &1\\
                            1 & 0 &1\\
                            \meio{} & 1 &\meio{}\\
                            \meio{} & \meio{} &\meio{}\\
                            \meio{} & 0 &1\\
                            0 & 1 &1\\
                            0 & \meio{} &1\\
                            0 & 0 &1\\
                            \hline
                        \end{array}
                    \]
                \end{center}
                

                \subcasodeprova{} $\phi_{1} = (\alpha \land \beta) \to \beta$.

                \begin{center}
                    \setbool{@fleqn}{false}
                    \[
                        \begin{array}{|c|c|c|}%chktex 44 
                            \hline
                            \alpha      & \beta & (\alpha \land \beta) \to \beta \\
                            \hline%chktex 44 
                            1 & 1 & 1\\
                            1 & \meio{} & \meio{}\\
                            1 & 0 & 1\\
                            \meio{} & 1 & 1\\
                            \meio{} & \meio{} & \meio{}\\
                            \meio{} & 0 & 1\\
                            0 & 1 & 1\\
                            0 & \meio{} & 1\\
                            0 & 0 & 1\\
                            \hline
                        \end{array}
                    \]
                \end{center}

                \subcasodeprova{} $\phi_{1} = \alpha \to (\alpha \lor \beta)$. 

                \begin{center}
                    \setbool{@fleqn}{false}
                    \[
                        \begin{array}{|c|c|c|}%chktex 44 
                            \hline
                            \alpha      & \beta & \alpha \to (\alpha \lor \beta) \\
                            \hline%chktex 44 
                            1 & 1 &1\\
                            1 & \meio{} &1\\
                            1 & 0 &1\\
                            \meio{} & 1 &1\\
                            \meio{} & \meio{} &\meio{}\\
                            \meio{} & 0 &\meio{}\\
                            0 & 1 &1\\
                            0 & \meio{} &1\\
                            0 & 0 &1\\
                            \hline
                        \end{array}
                    \]
                \end{center}
                
                \subcasodeprova{} $\phi_{1} = \beta \to (\alpha \lor \beta)$.

                \begin{center}
                    \setbool{@fleqn}{false}
                    \[
                        \begin{array}{|c|c|c|}%chktex 44 
                            \hline
                            \alpha      & \beta & \beta \to (\alpha \lor \beta) \\
                            \hline%chktex 44 
                            1 & 1 &1\\
                            1 & \meio{} &1\\
                            1 & 0 &1\\
                            \meio{} & 1 &1\\
                            \meio{} & \meio{} &\meio{}\\
                            \meio{} & 0 &1\\
                            0 & 1 &1\\
                            0 & \meio{} &\meio{}\\
                            0 & 0 &1\\
                            \hline
                        \end{array}
                    \]
                \end{center}

                \subcasodeprova{} $\phi_{1} = (\alpha \to \gamma) \to ((\beta \to \gamma) \to ((\alpha \lor \beta) \to \gamma))$. 

                \begin{center}
                    \setbool{@fleqn}{false}
                    \[
                        \begin{array}{|c|c|c|c|}%chktex 44 
                            \hline
                            \alpha      & \beta & \gamma & (\alpha \to \gamma) \to ((\beta \to \gamma) \to ((\alpha \lor \beta) \to \gamma)) \\
                            \hline%chktex 44 
                            1 & 1 & 1 &1\\
                            1 & 1 & \meio{} &\meio{}\\
                            1 & 1 & 0 &1\\
                            1 & \meio{} & 1 &1\\
                            1 & \meio{} & \meio{} &\meio{}\\
                            1 & \meio{} & 0 &1\\
                            1 & 0 & 1 &1\\
                            1 & 0 & \meio{} &\meio{}\\
                            1 & 0 & 0 &1\\
                            \meio{} & 1 & 1 &1\\
                            \meio{} & 1 & \meio{} &\meio{}\\
                            \meio{} & 1 & 0 &1\\
                            \meio{} & \meio{} & 1 &1\\
                            \meio{} & \meio{} & \meio{} &\meio{}\\
                            \meio{} & \meio{} & 0 &\meio{}\\
                            \meio{} & 0 & 1 &1\\
                            \meio{} & 0 & \meio{} &\meio{}\\
                            \meio{} & 0 & 0 &\meio{}\\
                            0 & 1 & 1 &1\\
                            0 & 1 & \meio{} &\meio{}\\
                            0 & 1 & 0 &1\\
                            0 & \meio{} & 1 &1\\
                            0 & \meio{} & \meio{} &\meio{}\\
                            0 & \meio{} & 0 &\meio{}\\
                            0 & 0 & 1 &1\\
                            0 & 0 & \meio{} &1\\
                            0 & 0 & 0 &1\\
                            \hline
                        \end{array}
                    \]
                \end{center}

                
                    
                \subcasodeprova{} $\phi_{1} = (\alpha \to \beta) \lor \alpha$. 

                \begin{center}
                    \setbool{@fleqn}{false}
                    \[
                        \begin{array}{|c|c|c|}%chktex 44 
                            \hline
                            \alpha      & \beta & (\alpha \to \beta) \lor \alpha \\
                            \hline%chktex 44 
                            1 & 1 & 1\\
                            1 & \meio{} & 1\\
                            1 & 0 & 1\\
                            \meio{} & 1 & 1\\
                            \meio{} & \meio{} & \meio{}\\
                            \meio{} & 0 & \meio{}\\
                            0 & 1 & 1\\
                            0 & \meio{} & 1\\
                            0 & 0 & 1\\
                            \hline
                        \end{array}
                    \]
                \end{center}
                    

                \subcasodeprova{} $\phi_{1} = \alpha \lor \neg \alpha$. 

                \begin{center}
                    \setbool{@fleqn}{false}
                    \[
                        \begin{array}{|c|c|c|}%chktex 44 
                            \hline
                            \alpha      & \neg \alpha & \alpha \lor \neg \alpha \\
                            \hline%chktex 44 
                            1 & 0 & 1\\
                            \meio{} & \meio{} & \meio{}\\
                            0 & 1 & 1\\
                            \hline
                        \end{array}
                    \]
                \end{center}
                
                 

                \subcasodeprova{} $\phi_{1} = \circ \alpha \to (\alpha \to (\neg \alpha \to \beta))$. 


                \begin{center}
                    \setbool{@fleqn}{false}
                    \[
                        \begin{array}{|c|c|c|c|c|}%chktex 44 
                            \hline
                            \alpha      & \neg \alpha &\circ \alpha & \beta & \circ \alpha \to (\alpha \to (\neg \alpha \to \beta)) \\
                            \hline%chktex 44 
                            1 & 0 & 1 & 1 & 1 \\
                            1 & 0 & 1 & \meio{} & 1 \\
                            1 & 0 & 1 & 0 & 1 \\
                            \meio{} & \meio{} & 0 & 1 & 1 \\
                            \meio{} & \meio{} & 0 & \meio{} & 1 \\
                            \meio{} & \meio{} & 0 & 0 & 1 \\
                            0 & 1 & 1 & 1 & 1 \\
                            0 & 1 & 1 & \meio{} & 1 \\
                            0 & 1 & 1 & 0 & 1 \\
                            \hline
                        \end{array}
                    \]
                \end{center}
                
                  
                \subcasodeprova{} $\phi_{1} = \neg \neg \alpha \to \alpha$. 

                \begin{center}
                    \setbool{@fleqn}{false}
                    \[
                        \begin{array}{|c|c|c|}%chktex 44 
                            \hline
                            \alpha      & \neg \neg \alpha & \neg \neg \alpha \to \alpha \\
                            \hline%chktex 44 
                            1 & 1 & 1\\
                            \meio{} & \meio{} & \meio{}\\
                            0 & 0 & 1\\
                            \hline
                        \end{array}
                    \]
                \end{center}
                    
                   
                \subcasodeprova{} $\phi_{1} = \alpha \to \neg \neg \alpha$. 
                
                \begin{center}
                    \setbool{@fleqn}{false}
                    \[
                        \begin{array}{|c|c|c|}%chktex 44 
                            \hline
                            \alpha      & \neg \neg \alpha &  \alpha \to \neg \neg \alpha\\
                            \hline%chktex 44 
                            1 & 1 & 1\\
                            \meio{} & \meio{} & \meio{}\\
                            0 & 0 & 1\\
                            \hline
                        \end{array}
                    \]
                \end{center}
                
                   
                \subcasodeprova{} $\phi_{1} = \neg \circ \alpha \to (\alpha \land \neg \alpha)$. 

                \begin{center}
                    \setbool{@fleqn}{false}
                    \[
                        \begin{array}{|c|c|c|c|}%chktex 44 
                            \hline
                            \alpha      & \neg \alpha & \neg \circ \alpha & \neg \circ \alpha \to (\alpha \land \neg \alpha)\\
                            \hline%chktex 44 
                            1 & 0 & 0 &1\\
                            \meio{} & \meio{} & 1&\meio{}\\
                            0 & 1 & 0&1\\
                            \hline
                        \end{array}
                    \]
                \end{center}
               

                \subcasodeprova{} $\phi_{1} = \neg (\alpha \lor \beta) \to (\neg \alpha \land \neg \beta)$. 

                \begin{center}
                    \setbool{@fleqn}{false}
                    \[
                        \begin{array}{|c|c|c|}%chktex 44 
                            \hline
                            \alpha      & \beta & \neg (\alpha \lor \beta) \to (\neg \alpha \land \neg \beta) \\
                            \hline%chktex 44 
                            1 & 1 & 1\\
                            1 & \meio{} & 1\\
                            1 & 0 & 1\\
                            \meio{} & 1 & 1\\
                            \meio{} & \meio{} &\meio{}\\ 
                            \meio{} & 0 & \meio{}\\
                            0 & 1 & 1\\
                            0 & \meio{} & \meio{}\\
                            0 & 0 & 1\\
                            \hline
                        \end{array}
                    \]
                \end{center}
                
                   

                \subcasodeprova{} $\phi_{1} = (\neg \alpha \land \neg \beta) \to \neg (\alpha \lor \beta)$.

                \begin{center}
                    \setbool{@fleqn}{false}
                    \[
                        \begin{array}{|c|c|c|}%chktex 44 
                            \hline
                            \alpha      & \beta & (\neg \alpha \land \neg \beta) \to \neg (\alpha \lor \beta) \\
                            \hline%chktex 44 
                            1 & 1 & 1\\
                            1 & \meio{} & 1\\
                            1 & 0 & 1\\
                            \meio{} & 1 & 1\\
                            \meio{} & \meio{} &\meio{}\\ 
                            \meio{} & 0 & \meio{}\\
                            0 & 1 & 1\\
                            0 & \meio{} & \meio{}\\
                            0 & 0 & 1\\
                            \hline
                        \end{array}
                    \]
                \end{center}

                \subcasodeprova{} $\phi_{1} = \neg(\alpha \land \beta) \to (\neg \alpha \lor \neg \beta)$. 

   
                \begin{center}
                    \setbool{@fleqn}{false}
                    \[
                        \begin{array}{|c|c|c|}%chktex 44 
                            \hline
                            \alpha      & \beta & \neg(\alpha \land \beta) \to (\neg \alpha \lor \neg \beta) \\
                            \hline%chktex 44 
                            1 & 1 & 1\\
                            1 & \meio{} & \meio{}\\
                            1 & 0 & 1\\
                            \meio{} & 1 & \meio{}\\
                            \meio{} & \meio{} &\meio{}\\ 
                            \meio{} & 0 & 1\\
                            0 & 1 & 1\\
                            0 & \meio{} & 1\\
                            0 & 0 & 1\\
                            \hline
                        \end{array}
                    \]
                \end{center}
                
               

                \subcasodeprova{} $\phi_{1} = (\neg \alpha \lor \neg \beta) \to \neg (\alpha \land \beta)$. 
   
                \begin{center}
                    \setbool{@fleqn}{false}
                    \[
                        \begin{array}{|c|c|c|}%chktex 44 
                            \hline
                            \alpha      & \beta & (\neg \alpha \lor \neg \beta) \to \neg (\alpha \land \beta) \\
                            \hline%chktex 44 
                            1 & 1 & 1\\
                            1 & \meio{} & \meio{}\\
                            1 & 0 & 1\\
                            \meio{} & 1 & \meio{}\\
                            \meio{} & \meio{} &\meio{}\\ 
                            \meio{} & 0 & 1\\
                            0 & 1 & 1\\
                            0 & \meio{} & 1\\
                            0 & 0 & 1\\
                            \hline
                        \end{array}
                    \]
                \end{center}

                \subcasodeprova{} $\phi_{1} = \neg (\alpha \to \beta) \to(\alpha \land \neg \beta)$. 

                \begin{center}
                    \setbool{@fleqn}{false}
                    \[
                        \begin{array}{|c|c|c|}%chktex 44 
                            \hline
                            \alpha      & \beta & \neg (\alpha \to \beta) \to(\alpha \land \neg \beta) \\
                            \hline%chktex 44 
                            1 & 1 & 1\\
                            1 & \meio{} & \meio{}\\
                            1 & 0 & 1\\
                            \meio{} & 1 & 1\\
                            \meio{} & \meio{} &\meio{}\\ 
                            \meio{} & 0 & \meio{}\\
                            0 & 1 & 1\\
                            0 & \meio{} & 1\\
                            0 & 0 & 1\\
                            \hline
                        \end{array}
                    \]
                \end{center}
               

                \subcasodeprova{} $\phi_{1} = (\alpha \land \neg \beta) \to \neg(\alpha \to \beta)$.

                \begin{center}
                    \setbool{@fleqn}{false}
                    \[
                        \begin{array}{|c|c|c|}%chktex 44 
                            \hline
                            \alpha      & \beta & (\alpha \land \neg \beta) \to \neg(\alpha \to \beta) \\
                            \hline%chktex 44 
                            1 & 1 & 1\\
                            1 & \meio{} & \meio{}\\
                            1 & 0 & 1\\
                            \meio{} & 1 & 1\\
                            \meio{} & \meio{} &\meio{}\\ 
                            \meio{} & 0 & \meio{}\\
                            0 & 1 & 1\\
                            0 & \meio{} & 1\\
                            0 & 0 & 1\\
                            \hline
                        \end{array}
                    \]
                \end{center}
                
            \end{provaporsubcasos}

            Com isso, o \textsc{Caso 1} está provado e $\Gamma \conmat \phi_{1}$ segue caso $\phi_{1}$ seja um axioma.

            \casodeprova{} $\phi_{1} \in \Gamma$. Logo, dada uma valoração $h : \ling{} \to M$ de $\mat{}$ se $h[\Gamma] \subseteq \{1, \meio{}\}$, temos $h(\phi_{1}) \in \{1, \meio{}\}$. Portanto, $\Gamma \conmat \phi_{1}$.

        \end{provaporcasos}

         \noindent \textbf{\textsc{Passo.}}
         
         \noindent \textbf{\textsc{Hipótese de indução (HI):}} Para qualquer sequência da derivação de $\Gamma \conhil \alpha$ de tamanho $k < n$, tem-se $\Gamma \conmat \alpha$. 
         
         Portanto, é preciso mostrar que $\Gamma \conmat \alpha$ segue caso a sequência de derivação de $\Gamma \conhil \alpha$ tenha tamanho $n$. Então, vamos supor $\Gamma \conhil \phi_{n}$.
         
         Ao analisar a obtenção de $\phi_{n}$ em $\Gamma \conhil \phi_{n}$, existem três casos a se considerar:
         
         \begin{enumerate}
            \item $\phi_{n}$ é um axioma.
            \item $\phi_{n} \in \Gamma$.
            \item $\phi_{n}$ é obtido por \textit{modus ponens} em duas fórmulas $\phi_{j}$ e $\phi_{k}$ com $j, k < n$. 
         \end{enumerate}
         
         Os casos 1 e 2 são análogos aos casos provados na base.
         
         \noindent \textsc{Caso 3.} $\phi_{n}$ é obtido por \textit{modus ponens} em duas fórmulas $\phi_{j}$ e $\phi_{k}$ com $j, k < n$. 
         
         Logo, $\phi_{k} = \phi_{j} \to \phi_{n}$ (ou $\phi_{j} = \phi_{k} \to \phi_{n}$, a prova para este caso é análoga). 
         
         Dada nossa suposição de $\Gamma \conhil \phi_{n}$, então $\Gamma \conhil \phi_{j}$ e $\Gamma \conhil \phi_{j} \to \phi_{n}$. 
         
         Pela (HI), temos $\Gamma \conmat \phi_{j}$ e $\Gamma \conmat \phi_{j} \to \phi_{n}$.

         Seja $h$ uma valoração qualquer sobre $\pazocal{M}_{\lfium{}}$. Então, vamos supor $h[\Gamma] \subseteq \{1, \meio{}\}$. 
         
         Temos $h(\phi_{j}) \in \{1, \meio{}\}$ e $h(\phi_{j} \to \phi_{n}) \in \{1,\meio{}\}$. 
         
         Pela matriz de $\to$, temos $h(\phi_{j}) = 0$ ou $h(\phi_{n}) \in \{1,\meio{}\}$. 
         
         Isto, unido ao fato de termos $h(\phi_{j}) \in \{1,\meio{}\}$, nos permite concluir $h(\phi_{n}) \in \{1,\meio{}\}$. 
         
         Portanto $\Gamma \conmat \phi_{n}$.

         \noindent Com isso, provamos $\Gamma \conmat \alpha$ e a prova está finalizada.

    \end{proof}

    Com a correção em relação a semântica matricial, o seguinte resultado é imediato:


    \begin{corolario}[Correção em relação a semântica de valorações]\label{cor:correcao_val}
        A lógica {\normalfont\lfium{}} é correta em relação a sua semântica de valorações, ou seja, para todo conjunto de fórmulas $\Gamma \cup \{\alpha\} \subseteq \ling{}$:

        \centering
        {\normalfont{} $\Gamma \conhil \alpha \Longrightarrow \Gamma \conval \alpha$.}
    \end{corolario}

    \begin{proof}[Prova do Corolário~\ref{cor:correcao_val}]
        Seja $\Gamma \cup \{\alpha\} \subseteq \ling{}$ um conjunto de fórmulas. Então, supondo $\Gamma \conhil \alpha$, temos, pelo Teorema~\ref{teo:correcao_mat}, $\Gamma \conmat \alpha$. Finalmente, pelo Corolário~\ref{cor:matval}, temos $\Gamma \conval \alpha$.
    \end{proof}

    Com o resultado anterior, a prova de que a lógica \lfium{} se trata de uma \lfi{} forte (ver Definição~\ref{def:lfi_forte_prop}) é imediata.

    \begin{corolario}\label{cor:lfi_forte}
        Seja $p$ uma variável proposicional qualquer e $\bigcirc(p) = \{\circ p\}$ um conjunto de fórmulas dependente somente de $p$. A lógica \lfium{} é uma \lfi{} forte em relação a $\neg$ e $\bigcirc(p)$.
    \end{corolario}

    \begin{proof}[Prova do Corolário~\ref{cor:lfi_forte}]
        Sejam $a, b \in \pazocal{P}$ variáveis proposicionais distintas e sejam $\phi, \psi \in \ling{}$ fórmulas quaisquer.

        \begin{adjustwidth}{1cm}{}

            \noindent\textbf{Prova da condição (i):} A prova das condições será feita com base nas valorações $v_1, v_2$ e $v_3$ presentes na Tabela~\ref{tab:negcirc}.
    
            \begin{adjustwidth}{1cm}{}

                \textbf{(i.a)} Vamos provar $a, \neg a \nconhil b$. Tomando a valoração $v_3$, com $v_3(b) = 0$, então temos $a, \neg a \nconval b$. Pela contraposta do Corolário~\ref{cor:correcao_val}, temos  $a, \neg a \nconhil b$.

                \noindent\textbf{(i.b)} Vamos provar $\circ a, a \nconhil b$. Tomando a valoração $v_2$, com $v_2(b) = 0$, então temos $a, \neg a \nconval b$. Pela contraposta do Corolário~\ref{cor:correcao_val}, temos $\circ a, a \nconhil b$.

                \noindent\textbf{(i.c)} Vamos provar $\circ a, \neg a \nconhil b$. Tomando a valoração $v_1$, com $v_1(b) = 0$, então temos $\circ a, \neg a \nconval b$. Pela contraposta do Corolário~\ref{cor:correcao_val}, temos $\circ a, \neg a \nconhil b$.
            \end{adjustwidth}
    
            \noindent\textbf{Prova da condição (ii):} A sequência de derivação abaixo mostra $\phi, \neg \phi, \circ \phi \conhil \psi$:
            \begin{align*}
                1.~ & \phi \tag{Premissa}       \\
                2.~ & \neg \phi \tag{Premissa}  \\
                3.~ & \circ \phi \tag{Premissa} \\
                4.~ & \circ \phi \to (\phi \to (\neg \phi \to \psi)) \tag{bc1} \\
                5.~ & \phi \to (\neg \phi \to \psi) \tag{MP 3, 4} \\
                6.~ & \neg \phi \to \psi \tag{MP 1, 5} \\
                7.~ & \psi \tag{MP 2, 6}
            \end{align*}

        \end{adjustwidth}


        Portanto, a \lfium{} se trata de uma \lfi{} forte.
    \end{proof}

    A prova da completude para a lógica \lfium{} depende das seguintes definições e lemas auxiliares para ser desenvolvida:

    \begin{proposicao}\label{prop:tarski}        
        A lógica $\lfium{} = \langle \ling, \conhil \rangle$ é tarskiana (Definição~\ref{def:tarski}).
    \end{proposicao}

    \begin{proof}[Prova da Proposição~\ref{prop:tarski}]
        Queremos mostrar que a \lfium{} satisfaz as seguintes condições para todo $\Gamma \cup \Delta \cup \{\phi\} \subseteq \ling{}$:
        \begin{itemize}
            \item [(i)] Se $\phi \in \Gamma$, então $\Gamma \conhil \phi$.
            \item [(ii)] Se $\Delta \conhil \phi$ e $\Delta \subseteq \Gamma$, então $\Gamma \conhil \phi$.
            \item [(iii)] Se $\Delta \conhil \phi$ e $\Gamma \conhil \delta$ para todo $\delta \in \Delta$, então $\Gamma \conhil \phi$.
        \end{itemize}
        \begin{adjustwidth}{1cm}{}
            \textbf{Prova do item (i)}. Vamos supor $\phi \in \Gamma$. Então, a seguinte derivação mostra $\Gamma \conhil \phi$:
            \begin{align*}
                1.~& \phi\tag{Premissa}
            \end{align*}

            \noindent{}\textbf{Prova do item (ii)}. Vamos supor $\Delta \conhil \phi$ e $\Delta \subseteq \Gamma$. Então, existe uma prova (sequência de fórmulas) de $\phi$ a partir do conjunto de fórmulas $\Delta$. A mesma sequência de fórmulas prova $\Gamma \conhil \phi$.
            
            \noindent{}\textbf{Prova do item (iii)}. Vamos supor $\Delta \conhil \phi$ e $\Gamma \conhil \delta$ para todo $\delta \in \Delta$. Então, existe uma prova $\phi_1, \ldots, \phi_n$ de $\phi_n = \phi$ a partir de $\Delta$. Vamos fazer uma indução no tamanho $n$ desta prova.
            
            \begin{adjustwidth}{1cm}{}
                
                \textbf{\textsc{Base.}} $n = 1$.
                Analisando a obtenção de $\phi$, existem dois casos:
                \begin{provaporcasos}
                \casodeprova{$\phi_1$ é um axioma}. 
                
                Neste caso, basta aplicar o mesmo axioma para mostrar $\Gamma \conhil \phi_1$.
                
                \casodeprova{$\phi_1 \in \Delta$}. 
                
                Neste caso, $\Gamma \conhil \phi_1$ por suposição.
            \end{provaporcasos}

            \noindent\textbf{\textsc{Passo.}} 
            
            \noindent \textbf{\textsc{Hipótese de indução (HI):}} A propriedade segue para todo $i < n$.

            Analisando a obtenção de $\phi_n$, existem três casos:
            \begin{provaporcasos}
                \casodeprova{$\phi_n$ é um axioma}. Análogo ao caso da base.
                
                \casodeprova{$\phi_n \in \Delta$}. Análogo ao caso da base.
                
                \casodeprova{$\phi_n$ é resultado de MP em duas fórmulas $\phi_j, \phi_k$ com $j, k < n$}.
                
                Então, $\phi_j = \phi_k \to \phi_n$ (ou $\phi_k = \phi_j \to \phi_n$, a prova para este caso é análoga). Além disso, temos $\Delta \conhil \phi_k$ e $\Delta \conhil \phi_k \to \phi_n$. Logo, por (HI), temos $\Gamma \conhil \phi_k$ e $\Gamma \conhil \phi_k \to \phi_n$. Então, basta derivar estas duas fórmulas e aplicar MP para obter $\phi_n$. Portanto, $\Gamma \conhil \phi_n$.
            \end{provaporcasos}
        \end{adjustwidth}
    \end{adjustwidth}

        Então, a \lfium{} é uma lógica tarskiana.
        
    \end{proof}

    
    \begin{proposicao}\label{prop:finit}        
        A lógica $\lfium{} = \langle \ling, \conhil \rangle$ é finitária (Definição ~\ref{def:padrao}).
    \end{proposicao}

    \begin{proof}[Prova da Proposição~\ref{prop:finit}]
        Seja $\Gamma \cup \{\phi\} \subseteq \ling{}$ um conjunto qualquer de fórmulas com $\Gamma \conhil \phi$. Então, existe uma sequência finita de derivação $\phi_1, \ldots, \phi_n$, a partir de $\Gamma$, onde $\phi_n = \phi$. 
        
        Vamos definir o conjunto $\Gamma_0$ como sendo o conjunto formado somente pelas premissas desta derivação ($\Gamma_0$ é finito). Como todo elemento de $\Gamma_0$ é uma premissa de $\Gamma$, é evidente que $\Gamma_0 \subseteq \Gamma$. Vamos provar $\Gamma_0 \conhil \phi$ por indução no tamanho $n$ da sequência de derivação.


        \begin{adjustwidth}{1cm}{}

            \textbf{\textsc{Base.}} $n = 1$.
            Analisando a obtenção de $\phi_1$ em $\Gamma \conhil \phi_1$, existem dois casos:
            \begin{provaporcasos}
                \casodeprova{$\phi_1$ é um axioma}. 
                
                    Neste caso, basta aplicar o mesmo axioma para mostrar $\Gamma_0 \conhil \phi_1$.
    
                \casodeprova{$\phi_1 \in \Gamma$}. 
                
                    Neste caso, $\Gamma_0 \conhil \phi_1$ pela construção de $\Gamma_0$.
            \end{provaporcasos}

            \noindent\textbf{\textsc{Passo.}} 
            
            \noindent \textbf{\textsc{Hipótese de indução (HI):}} A propriedade segue para todo $i < n$.

            Analisando a obtenção de $\phi_n$ em $\Gamma \conhil \phi_n$, existem três casos:
            \begin{provaporcasos}
                \casodeprova{$\phi_n$ é um axioma}. Análogo ao caso da base.
    
                \casodeprova{$\phi_n \in \Delta$}. Análogo ao caso da base.

                \casodeprova{$\phi_n$ é resultado de MP em duas fórmulas $\phi_j, \phi_k$ com $j, k < n$}.
                
                Então, $\phi_j = \phi_k \to \phi_n$ (ou $\phi_k = \phi_j \to \phi_n$, a prova para este caso é análoga). Além disso, temos $\Gamma \conhil \phi_k$ e $\Gamma \conhil \phi_k \to \phi_n$. Logo, por (HI), temos $\Gamma_0 \conhil \phi_k$ e $\Gamma_0 \conhil \phi_k \to \phi_n$. Então, basta derivar estas duas fórmulas e aplicar MP para obter $\phi_n$. Portanto, $\Gamma_0 \conhil \phi_n$.
            \end{provaporcasos}
        \end{adjustwidth}
        Então, a \lfium{} é finitária.

    \end{proof}



    \begin{definicao}[Conjunto de fórmulas maximal não-trivial]\label{def:nao-trivial_maximal}
        Seja $\mathcal{L}$ uma lógica tarskiana definida sobre uma linguagem $\pazocal{L}$ e sejam $\Gamma, \{\phi\}$ conjuntos de fórmulas de modo que $\Gamma \cup \{\phi\} \subseteq \pazocal{L}$. O conjunto $\Gamma$ é dito maximal não-trivial em relação a $\phi$ em $\mathcal{L}$ se $\Gamma \nVdash_{\mathcal{L}} \phi$ mas $\Gamma, \psi \Vdash_{\mathcal{L}} \phi$ para qualquer $\psi \notin \Gamma$.\qed{}
    \end{definicao}

    \begin{definicao}[Teoria fechada]\label{def:fechada}

        Seja $\mathcal{L}$ uma lógica tarskiana. Um conjunto de fórmulas $\Gamma$ é dito fechado em $\mathcal{L}$ (ou dito uma \textit{teoria fechada} em $\mathcal{L}$) se, para toda fórmula $\phi$, tem-se $\Gamma \Vdash_{\mathcal{L}} \phi$ sse $\phi \in \Gamma$.\qed{}
    \end{definicao}

    \begin{lema}\label{lem:nao_trivial_maximal_fechado}
        Todo conjunto de fórmulas maximal não-trivial em relação a $\phi$ em $\mathcal{L}$ é fechado em $\mathcal{L}$.
    \end{lema}

    \begin{proof}[Prova do Lema~\ref{lem:nao_trivial_maximal_fechado}]
        Seja $\Gamma$ uma conjunto de fórmulas maximal não-trivial em relação a $\phi$ em $\mathcal{L}$. 
        
        \noindent($\Longrightarrow$) Se $\psi \in \Gamma$, então $\Gamma \Vdash_{\mathcal{L}} \psi$, já que $\mathcal{L}$ é tarskiana. 
        
        \noindent($\Longleftarrow$) Se $\Gamma \Vdash_{\mathcal{L}} \psi$, então supondo $\psi \notin \Gamma$, temos $\Gamma, \psi \Vdash_{\mathcal{L}} \phi$, pela Definição~\ref{def:nao-trivial_maximal}. Pela propriedade do corte, segue que $\Gamma \Vdash_{\mathcal{L}} \phi$, o que contradiz o fato de $\Gamma$ ser maximal não-trivial em relação a $\phi$ em $\mathcal{L}$. Portanto, $\psi \in \Gamma$.
        
        Então $\Gamma$ é fechado em $\mathcal{L}$.
    \end{proof}





    Provaremos agora o lema de Lindenbaum, proposto originalmente por Adolf Lindenbaum (de acordo com~\citeshort{Tarski1956-TARLSM}) e adaptado por~\citeshort{los_lindenbaum,Wojcicki1984-WJCLOP}. A versão apresentada por~\citeshort{Carnielli_Coniglio_2016} utiliza o princípio da boa ordem e aplica uma recursão transfinita para que o lema siga verdadeiro mesmo quando se tratando de sistemas definidos sobre linguagens não enumeráveis. Para o propósito do presente trabalho, isso não se mostra necessário, tendo em vista que a linguagem da \lfium{} (Definição~\ref{def:ling}) é enumerável.

    \begin{lema}[Lindenbaum-{\L}os para \lfium{}]\label{lem:lindenbaum}
        Dada a lógica $\lfium{} = \langle \ling, \conhil \rangle$ então temos, para qualquer conjunto de fórmulas $\Gamma \cup \{\phi\} \subseteq \ling$, se $\Gamma \nconhil \phi$ então existe um conjunto de fórmulas $\Delta$, com $\Gamma \subseteq \Delta \subseteq \ling$, tal que $\Delta$ é um conjunto maximal não-trivial em relação a $\phi$ em \lfium{}.
    \end{lema}

    \begin{proof}[Prova do Lema~\ref{lem:lindenbaum}]\setbool{@fleqn}{false}
        Seja $\lfium{} = \langle \ling, \conhil \rangle$ a lógica \lfium{} e seja $\Gamma \cup \{\phi\} \subseteq \ling$ um conjunto de fórmulas com $\Gamma \nconhil \phi$. Arranje as fórmulas de $\ling$ numa sequência $\pazocal{C} = \phi_1 \ldots \phi_i \ldots$ e defina $\Gamma_i$ recursivamente da seguinte forma:
        \begin{align*}
                \text{(i)}~&\Gamma_0 = \Gamma\\
                \text{(ii)}~&\Gamma_i =
                \begin{cases}
                    \Gamma_{i - 1} &\text{se}~ \Gamma_{i - 1}, \phi_i \conhil \phi\\
                    \Gamma_{i - 1} \cup \{\phi_i\} &\text{se}~ \Gamma_{i - 1}, \phi_i \nconhil \phi
                \end{cases}
        \end{align*}

        Então, defina um conjunto de fórmulas $\Delta = \bigcup_{i=0}^{\infty}\Gamma_i$. Dividiremos a prova em algumas partes:
        
        \begin{adjustwidth}{1cm}{}
            \noindent(1) $\Gamma \subseteq \Delta \subseteq \ling$.

                Pela construção de $\Gamma_i$, temos $\Gamma_0 = \Gamma \subseteq \Gamma_i$ e $\Gamma_i \subseteq \ling$ para todo $i \in \mathbb{N}$. Como $\Delta$ é formado a partir da união de todos os conjuntos $\Gamma_i$, temos $\Gamma_0 \subseteq \Delta$ e $\Delta \subseteq \ling$.

            \noindent(2) $\Gamma_i \nconhil \phi$ para todo $i \geq 0$.

            Provaremos por indução em $i$:

            \begin{adjustwidth}{1cm}{}
                \noindent\textbf{\textsc{Base.}} $i = 0$.
                
                \begin{adjustwidth}{1cm}{}
                    
                    Temos $\Gamma_0 \nconhil \phi$ pela nossa hipótese de $\Gamma \nconhil \phi$.
                    
                \end{adjustwidth}
                
                \noindent\textbf{\textsc{Passo.}} 
                
                \noindent\textbf{Hipótese de indução (HI):} Para qualquer $k < i$, temos $\Gamma_k \nconhil \phi$.
                
                    Logo temos $\Gamma_{i-1} \nconhil \phi$ por (HI). Portanto, pela definição de $\Gamma_i$, temos $\Gamma_i \nconhil \phi$.
                    
            \end{adjustwidth}

            
            \noindent(3) Para todo $i \geq 0$, se $\phi_i \in \Delta$ então $\phi_i \in \Gamma_i$.



            \noindent(4) $\Delta \nconhil \phi$.

                Vamos supor $\Delta \conhil \phi$. 

            \noindent(5) $\Delta$ é maximal não-trivial em relação a $\phi$ em \lfium{}.

        \end{adjustwidth}
            
            
        \end{proof}
        
        

    \begin{teorema}[Completude]\label{teo:completude}
        A lógica {\normalfont\lfium{}} é completa em relação a sua semântica de valorações, ou seja, para todo conjunto de fórmulas $\Gamma \cup \{\alpha\} \subseteq \ling{}$:

        \centering
        {\normalfont{} $\Gamma \conval \alpha \Longrightarrow \Gamma \conhil \alpha$.}
    \end{teorema}

    \begin{proof}[Prova do Teorema~\ref{teo:completude}]
 

    \end{proof}


