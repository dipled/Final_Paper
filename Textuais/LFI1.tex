\chapter{LFI1}
\label{cap:LFI1}
BLABLABLABLABLA introduz um textinho % MIGS: Grande introdução 🔥

\section{Definições}
A lógica proposicional\footnote{Por mais que existam extensões de primeira ordem da \lfium{}, como a \textbf{LFI1*}, definida em~\citeshort{carnielli2000formal}, o presente trabalho trata somente do fragmento proposicional desta.} \lfium{} aqui apresentada é definida com base no livro de~\citeshort{Carnielli_Coniglio_2016} como $\mathcal{L} = \langle \pazocal{L}_{\Sigma}, \vdash \rangle$. A linguagem\footnote{A linguagem da \lfium{} pode ser definida de maneira equivalente utilizando-se o operador de consistência (representado por $\circ$), seguindo a definição $\circ \alpha \eqdef \neg \bullet \alpha$.} $\pazocal{L}_{\Sigma}$ da \lfium{} é definida sobre um conjunto enumerável de átomos $\pazocal{P} = \{p_{n} \;|\; n \in \omega\}$ e uma assinatura proposicional $\Sigma = \{\land^{2}, \lor^{2}, \rightarrow^{2}, \neg^{1}, \bullet^{1}\}$ da seguinte forma:

\begin{definicao}[Linguagem da \lfium{}]
    \label{def:ling}
    A linguagem $\pazocal{L}_{\Sigma}$ da \lfium{} é definida indutivamente como o menor conjunto a que respeita as seguintes regras:
    \begin{align*}
         & \text{1.~}\pazocal{P} \subseteq \pazocal{L}_{\Sigma}                                                                                                                        \\
         & \text{2.~Se } \phi \in \pazocal{L}_{\Sigma}, \text{então } \triangle  \phi \in \pazocal{L}_{\Sigma}, \text{com } \triangle \in \{\neg, \bullet\}                            \\
         & \text{3.~Se } \phi, \psi \in \pazocal{L}_{\Sigma}, \text{então } \phi \otimes \psi \in \pazocal{L}_{\Sigma}, \text{com } \otimes \in \{\land, \lor, \rightarrow\} \tag*\qed
    \end{align*}
\end{definicao}

\begin{definicao}[Subfórmulas]
    \label{def:subf}
    O conjunto Sub$(\phi)$ de subfórmulas de uma fórmula $\phi$ é definido indutivamente da seguinte forma:
    \begin{align*}
         & \text{1.~Sub}(p_{i}) = \{p_{i}\}, \; p_{i} \in \pazocal{P}                                                                                                            \\
         & \text{2.~Sub}(\triangle \phi) = \{\triangle \phi\} \; \cup \;\text{Sub}(\phi), \; \triangle \in \{\neg, \bullet\}                                                     \\
         & \text{3.~Sub}(\phi \otimes \psi) = \{\phi \otimes \psi\} \; \cup \;\text{Sub}(\phi) \; \cup \;\text{Sub}(\psi), \; \otimes \in \{\land, \lor, \rightarrow\} \tag*\qed
    \end{align*}
\end{definicao}

\begin{definicao}[Complexidade de fórmulas]
    \label{def:complex}
    Dada uma fórmula $\phi \in \pazocal{L}_{\Sigma}$, a complexidade $C(\phi)$ de de uma fórmula $\phi$ qualquer é definida recursivamente da seguinte forma:
    \begin{align*}
        & \text{1.~Se } \phi = p \text{, onde } p \in \pazocal{P} \text{, então } C(\phi) = 1;\\
        & \text{2.~Se } \phi = \neg \psi \text{, então } C(\phi) = C(\psi) + 1;\\
        & \text{3.~Se } \phi = \bullet \psi \text{, então } C(\phi) = C(\psi) + 2;\\
        & \text{4.~Se } \phi = \psi \otimes \gamma \text{, onde } \otimes \in \{\land, \lor, \rightarrow\} \text{, então } C(\phi) = C(\psi) + C(\gamma) + 1.\tag*\qed{}
    \end{align*}
\end{definicao}

\begin{definicao}[\lfium{}]
    \label{def:lfi1}
    A lógica \lfium{} é definida sobre a linguagem $\pazocal{L}_{\Sigma}$ através do seguinte cálculo de Hilbert:

    \noindent\textbf{Axiomas:}
    \begin{alignat}{2}
        \alpha \rightarrow (\beta \rightarrow \alpha)\\
        (\alpha \rightarrow (\beta \rightarrow \gamma)) \rightarrow ((\alpha \rightarrow \beta) \rightarrow (\alpha \rightarrow \gamma))
    \end{alignat}{2}
    
\end{definicao}