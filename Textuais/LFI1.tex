\chapter{LFI1}
\label{cap:LFI1}
BLABLABLABLABLA introduz um textinho

    
    \subsection{Linguagem}
    A lógica \textbf{LFI1}\footnote{Por mais que existam extensões de primeira ordem da \textbf{LFI1}, o presente trabalho trata somente do fragmento proposicional desta.} aqui apresentada é definida com base nos trabalhos de Carnielli, Coniglio e Marcos~\cite{carnielli2000formal,carnielli2007,Carnielli_Coniglio_2016} como uma estrutura da forma $\langle \mathcal{S}, \Vdash \rangle$, onde $\mathcal{S}$ é a sua linguagem (seu conjunto de fórmulas) e $\Vdash$ é uma relação de consequência de conclusão única definida como $\Vdash \;\subseteq \text{Pow}(\mathcal{S})\times \mathcal{S}$.
    A linguagem\footnote{A linguagem da \textbf{LFI1} pode ser definida de maneira equivalente utilizando-se tanto o operador de consistência (representado por $\circ$) quanto o operador de inconsistência (representado por $\bullet$), respeitando a equivalência $\circ \alpha \eqdef \neg \bullet \alpha$.} $\mathcal{S}$ da \textbf{LFI1} é definida sobre um conjunto enumerável de átomos $\mathcal{P} = \{p_{n} \:|\: n \in \omega\}$ e uma assinatura $\Sigma = \{\land, \lor, \rightarrow, \neg, \bullet\}$ da seguinte forma:

    \begin{definicao}[Linguagem da \textbf{LFI1}]
        \label{def:lang}
        A linguagem $\mathcal{S}$ da \textbf{LFI1} é definida indutivamente como o menor conjunto a partir das seguintes regras:
        \begin{align*}
            & \mathcal{P} \subseteq \mathcal{S}\\
            & \text{Se } \varphi \in \mathcal{S}, \text{então } \triangle  \varphi \in \mathcal{S}, \text{com } \triangle \in \{\neg, \bullet\}\\
            & \text{Se } \varphi, \psi \in \mathcal{S}, \text{então } \varphi \otimes \psi \in \mathcal{S}, \text{com } \otimes \in \{\land, \lor, \rightarrow\} \tag*\qed
        \end{align*}
    \end{definicao}