\chapter{LFI1}
\label{cap:LFI1}
BLABLABLABLABLA introduz um textinho % MIGS: Grande introdução 🔥

\section{Linguagem}
A lógica proposicional\footnote{Por mais que existam extensões de primeira ordem da \lfium{}, como a \textbf{LFI1*}, definida em~\citeshort{carnielli2000formal}, o presente trabalho trata somente do fragmento proposicional desta.} \lfium{} aqui apresentada é definida com base no livro de~\citeshort{Carnielli_Coniglio_2016} como $\mathcal{L} = \langle \pazocal{L}_{\Sigma}, \vdash \rangle$. A linguagem\footnote{A linguagem da \lfium{} pode ser definida de maneira equivalente utilizando-se o operador de consistência (representado por $\circ$), seguindo a definição $\circ \alpha \eqdef \neg \bullet \alpha$.} $\pazocal{L}_{\Sigma}$ da \lfium{} é definida sobre um conjunto enumerável de átomos $\pazocal{P} = \{p_{n} \;|\; n \in \omega\}$ e uma assinatura proposicional $\Sigma = \{\land^{2}, \lor^{2}, \rightarrow^{2}, \neg^{1}, \bullet^{1}\}$ da seguinte forma:

\begin{definicao}[Linguagem da \lfium{}]
    \label{def:lang}
    A linguagem $\pazocal{L}_{\Sigma}$ da \lfium{} é definida indutivamente como o menor conjunto a que respeita as seguintes regras:
    \begin{align*}
         & \pazocal{P} \subseteq \pazocal{L}_{\Sigma}\\
         & \text{Se } \varphi \in \pazocal{L}_{\Sigma}, \text{então } \triangle  \varphi \in \pazocal{L}_{\Sigma}, \text{com } \triangle \in \{\neg, \bullet\}                            \\
         & \text{Se } \varphi, \psi \in \pazocal{L}_{\Sigma}, \text{então } \varphi \otimes \psi \in \pazocal{L}_{\Sigma}, \text{com } \otimes \in \{\land, \lor, \rightarrow\} \tag*\qed
    \end{align*}
\end{definicao}

\begin{definicao}[Subfórmulas]
    \label{def:subf}
    O conjunto Sub$(\varphi)$ de subfórmulas de uma fórmula $\varphi$ é definido indutivamente por:
    \begin{align*}
         & \text{Sub}(p_{i}) = \{p_{i}\}, \; p_{i} \in \pazocal{P}                                                                                                                     \\
         & \text{Sub}(\triangle \varphi) = \{\triangle \varphi\} \; \cup \;\text{Sub}(\varphi), \; \triangle \in \{\neg, \bullet\}                                                     \\
         & \text{Sub}(\varphi \otimes \psi) = \{\varphi \otimes \psi\} \; \cup \;\text{Sub}(\varphi) \; \cup \;\text{Sub}(\psi), \; \otimes \in \{\land, \lor, \rightarrow\} \tag*\qed
    \end{align*}
\end{definicao}

