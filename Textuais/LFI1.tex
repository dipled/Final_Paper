\chapter{A Lógica de Inconsistência Formal LFI1}
\label{cap:LFI1}

\helena{coisas bancos de dados}. No trabalho de~\citeshort{carnielli2000formal} a lógica \textbf{LFI1*} é definida como uma extensão de primeira ordem da lógica proposicional \lfium{}. A motivação para definir-se uma lógica de inconsistência formal de primeira ordem vem da natureza das informações contidas em bancos de dados, estas que podem ser compreendidas como sentenças de primeira ordem fixas~\cite{Codd}. Entretanto, o presente trabalho trata somente da lógica proposicional \lfium{}. Nas seções que seguem, serão apresentados conceitos utilizados para definir e provar metapropriedades sobre a lógica \lfium{}.

\section{Linguagem}
A lógica proposicional \lfium{} aqui apresentada é definida com base em~\citeshort{Carnielli_Coniglio_2016} sobre a linguagem $\pazocal{L}_{\Sigma}$, que por sua vez é definida sobre um conjunto enumerável de átomos $\pazocal{P} = \{p_{n} \;|\; n \in \mathbb{N}\}$ e uma assinatura proposicional $\Sigma = \{\land^{2}, \lor^{2}, \rightarrow^{2}, \neg^{1}, \circ^{1}\}$. Como de costume, o conectivo $\land^{2}$ representa uma conjunção, $\lor^{2}$ representa uma disjunção, $\rightarrow^{2}$ representa uma implicação, $\neg^{1}$ representa uma negação e $\circ^{1}$ é o conectivo de consistência, definido de forma primitiva. No restante do texto as aridades destes conectivos será omitida. Portanto a linguagem $\pazocal{L}_{\Sigma}$ da \lfium{} é definida da seguinte forma:

\begin{definicao}[Linguagem da \lfium{}]
    \label{def:ling}
    A linguagem $\pazocal{L}_{\Sigma}$ da \lfium{} é definida indutivamente como o menor conjunto a que respeita as seguintes regras:
    \begin{align*}
         & \text{1.~}\pazocal{P} \subseteq \pazocal{L}_{\Sigma}                                                                                                                        \\
         & \text{2.~Se } \phi \in \pazocal{L}_{\Sigma}, \text{então } \triangle  \phi \in \pazocal{L}_{\Sigma}, \text{com } \triangle \in \{\neg, \circ\}                            \\
         & \text{3.~Se } \phi, \psi \in \pazocal{L}_{\Sigma}, \text{então } \phi \otimes \psi \in \pazocal{L}_{\Sigma}, \text{com } \otimes \in \{\land, \lor, \rightarrow\} \tag*\qed
    \end{align*}
\end{definicao}

A precedência dos conectivos também é dada de maneira costumeira, mas com a adição do operador $\circ$ de consistência, seguindo a ordem (da maior precedência para a menor): $\circ$, $\neg$, $\land$, $\lor$, $\rightarrow$. Os conectivos binários $\land$ e $\lor$ são associativos à esquerda, ou seja, uma expressão do tipo $\alpha \land \beta \land \gamma$ é lida como $(\alpha \land \beta) \land \gamma$, e o conectivo $\rightarrow$ é associativo à direita, ou seja, uma expressão do tipo $\alpha \rightarrow \beta \rightarrow \gamma$ é lida como $\alpha \rightarrow (\beta \rightarrow \gamma)$.

A linguagem da \lfium{} pode ser definida de maneira equivalente utilizando-se o operador de inconsistência (representado por $\bullet$), seguindo a definição $\bullet \alpha \eqdef \neg \circ \alpha$, como feito por~\citeshort{carnielli2000formal}. Naquele trabalho o foco era explorar a \lfium{} como uma ferramenta para lidar com inconsistências em bancos de dados, portanto tomar o operador $\bullet$ como primitivo era de grande interesse. Entretanto, no presente trabalho a escolha de seguir o que foi feito em~\cite{Carnielli_Coniglio_2016} (definir a semântica e a sintaxe em termos de $\circ$) foi tomada pois isto salienta algumas propriedades interessantes da negação $\neg$, como a presença das leis de De Morgan, axiomatizadas na Seção~\ref{sec:axiomatizacao}.

% \begin{definicao}[Subfórmulas]
%     \label{def:subf}
%     O conjunto Sub$(\phi)$ de subfórmulas de uma fórmula $\phi$ é definido indutivamente da seguinte forma:
%     \begin{align*}
%          & \text{1.~Sub}(p_{i}) = \{p_{i}\}, \; p_{i} \in \pazocal{P}                                                                                                            \\
%          & \text{2.~Sub}(\triangle \phi) = \{\triangle \phi\} \; \cup \;\text{Sub}(\phi), \; \triangle \in \{\neg, \circ\}                                                     \\
%          & \text{3.~Sub}(\phi \otimes \psi) = \{\phi \otimes \psi\} \; \cup \;\text{Sub}(\phi) \; \cup \;\text{Sub}(\psi), \; \otimes \in \{\land, \lor, \rightarrow\} \tag*\qed
%     \end{align*}
% \end{definicao}

No desenvolvimento de metateoremas sobre propriedades de uma determinada lógica, a indução na complexidade de uma fórmula é um método comum de prova. Para isso, define-se uma função recursiva $C(\phi) : \pazocal{L}_{\Sigma} \rightarrow \mathbb{N}$ que retorna, para uma dada fórmula, um número natural representando sua complexidade, baseada na quantidade de operadores e átomos.

\begin{definicao}[Complexidade de fórmulas]
    \label{def:complex}
    Dada uma fórmula $\phi \in \pazocal{L}_{\Sigma}$, a complexidade $C(\phi)$ é definida recursivamente da seguinte forma:
    \begin{align*}
         & \text{1.~Se } \phi = p \text{, onde } p \in \pazocal{P} \text{, então } C(\phi) = 1;                                                                           \\
         & \text{2.~Se } \phi = \neg \psi \text{, então } C(\phi) = C(\psi) + 1;                                                                                          \\
         & \text{3.~Se } \phi = \circ \psi \text{, então } C(\phi) = C(\psi) + 2;                                                                                       \\
         & \text{4.~Se } \phi = \psi \otimes \gamma \text{, onde } \otimes \in \{\land, \lor, \rightarrow\} \text{, então } C(\phi) = C(\psi) + C(\gamma) + 1.\tag*\qed{}
    \end{align*}
\end{definicao}
Note que a complexidade de uma fórmula do tipo $\circ \alpha$ é estritamente maior que a complexidade de $\alpha$ e $\neg \alpha$. Isto se dá pois, como evidenciado pela semântica de valorações na Definição~\ref{def:valoracoes}, exite uma dependência de $\circ \alpha$ em $\{\alpha, \neg \alpha\}$.

\section{Axiomatização}
\label{sec:axiomatizacao}
\helena{qq eu boto aqui ahuahuhuahua}

\migs{Eu sugiro uma breve explicação do que é um sistema sintático de Hilbert assim como uma breve explicação de que existem diversas axiomatizações (para a mesma assinatura) e uma explicação sobre pq vc escolheu a axiomatização que escolheu}

\begin{notacao}
    Utiliza-se $\alpha \leftrightarrow \beta$ para denotar uma bi-implicação, como forma de abreviar a fórmula $(\alpha \rightarrow \beta) \land (\beta \rightarrow \alpha)$. Além disso, utiliza-se $\bullet \alpha$ para denotar a inconsistência de uma fórmula $\alpha$, de modo a abreviar $\neg \circ \alpha$.
\end{notacao}

\begin{definicao}[\lfium{}]
    \label{def:lfi1}
    A lógica \lfium{} é definida sobre a linguagem $\pazocal{L}_{\Sigma}$ através do seguinte cálculo de Hilbert:

    \noindent\textbf{(Axiomas Livrão da paraconsistência VERSÃO DO CALCULO DE HILBERT $\text{LFI1}_{\circ}$)}:
    \begin{align*}
        & \alpha \rightarrow (\beta \rightarrow \alpha)\tag{\textbf{Ax1}}                                                                                     \\
        & (\alpha \rightarrow (\beta \rightarrow \gamma)) \rightarrow ((\alpha \rightarrow \beta) \rightarrow (\alpha \rightarrow \gamma ))\tag{\textbf{Ax2}} \\
        & \alpha \rightarrow (\beta \rightarrow (\alpha \land \beta))\tag{\textbf{Ax3}}                                                                       \\
        & (\alpha \land \beta) \rightarrow \alpha\tag{\textbf{Ax4}}                                                                                           \\
        & (\alpha \land \beta) \rightarrow \beta\tag{\textbf{Ax5}}                                                                                            \\
        & \alpha \rightarrow (\alpha \lor \beta)\tag{\textbf{Ax6}}                                                                                            \\
        & \beta \rightarrow (\alpha \lor \beta)\tag{\textbf{Ax7}}                                                                                             \\
        & (\alpha \rightarrow \gamma) \rightarrow ((\beta \rightarrow \gamma) \rightarrow ((\alpha \lor \beta) \rightarrow \gamma))\tag{\textbf{Ax8}}         \\
        & (\alpha \rightarrow \beta) \lor \alpha\tag{\textbf{Ax9}}                                                                                           \\
        & \alpha \lor \neg \alpha\tag{\textbf{Ax10}}                                                                                                          \\
        & \circ \alpha \rightarrow (\alpha \rightarrow (\neg \alpha \rightarrow \beta))\tag{\textbf{bc1}}                                                     \\
        & \neg \neg \alpha \rightarrow \alpha\tag{\textbf{cf}}
        \\
        & \alpha \rightarrow \neg \neg \alpha\tag{\textbf{ce}}
        \\
        & \neg \circ \alpha \rightarrow (\alpha \land \neg \alpha)\tag{\textbf{ci}}                                                                           \\
        & \neg (\alpha \lor \beta) \rightarrow (\neg \alpha \land \neg \beta)\tag{\textbf{neg}$\lor_{1}$}\\
        & (\neg \alpha \land \neg \beta) \rightarrow \neg (\alpha \lor \beta)\tag{\textbf{neg}$\lor_{2}$}\\
        & \neg(\alpha \land \beta) \rightarrow (\neg \alpha \lor \neg \beta)\tag{\textbf{neg}$\land_{1}$}\\
        & (\neg \alpha \lor \neg \beta) \rightarrow \neg (\alpha \land \beta)\tag{\textbf{neg}$\land_{2}$}\\
        & \neg (\alpha \rightarrow \beta) \rightarrow(\alpha \land \neg \beta)\tag{\textbf{neg}$\rightarrow_{1}$}\\
        & (\alpha \land \neg \beta) \rightarrow \neg(\alpha \rightarrow \beta)\tag{\textbf{neg}$\rightarrow_{2}$}
   \end{align*}
    \\
    \noindent\textbf{Regra de inferência:}
    \begin{prooftree}
        \AxiomC{$\alpha, \alpha \rightarrow \beta$}
        \RightLabel{MP}
        \UnaryInfC{$\beta$}
    \end{prooftree}
    \qed{}
\end{definicao}

\section{Semântica}
\label{sec:semantica}
\helena{qq eu boto aqui ahuahuhuahua}

\migs{Idem ao que eu falei acima}

Uma das formas de se definir a semântica de uma dada lógica $\pazocal{L}$ é definir uma \textit{matriz lógica} para seus conectivos. 
\begin{definicao}[Matriz Lógica]
    Uma matriz lógica é uma dupla $\pazocal{M} = \langle M, D \rangle$ onde $M$ é seu domínio e $D$ é seu conjunto de valores designados (valores considerados tautologias). \qed{}
\end{definicao}
Por exemplo, a matriz lógica $\pazocal{M}_{LPC} = \langle M, D \rangle$ da lógica proposicional clássica possui um domínio $M = \{1, 0\}$ e um conjunto de valores designados $D = \{1\}$.
\begin{definicao}[Lógica Trivalorada]
    Uma  lógica $\pazocal{L}$ é dita uma lógica trivalorada caso ela esteja associada a uma matriz $\pazocal{M} = \langle M, D \rangle$ de forma que seu domínio $M$ possua três elementos.
    \qed{}
\end{definicao}

\begin{definicao}[Matriz lógica da \lfium{}]
    A matriz lógica $\pazocal{M}_{\lfium{}} = \langle M, D \rangle$ com domínio $M = \{1, \meio{}, 0\}$ e um conjunto de valores designados $D = \{1, \meio{}\}$ é definida da seguinte forma:

    \noindent
    \begin{minipage}{0.3\textwidth}
        % Implication (→)
        \[
            \begin{array}{c|ccc}
                \rightarrow & 1 & \frac{1}{2} & 0 \\
                \hline
                1           & 1 & \frac{1}{2} & 0 \\
                \frac{1}{2} & 1 & \frac{1}{2} & 0 \\
                0           & 1 & 1           & 1 \\
            \end{array}
        \]
    \end{minipage}
    \begin{minipage}{0.3\textwidth}
        % Conjunction (∧)
        \[
            \begin{array}{c|ccc}
                \land       & 1           & \frac{1}{2} & 0 \\
                \hline
                1           & 1           & \frac{1}{2} & 0 \\
                \frac{1}{2} & \frac{1}{2} & \frac{1}{2} & 0 \\
                0           & 0           & 0           & 0 \\
            \end{array}
        \]
    \end{minipage}
    \begin{minipage}{0.3\textwidth}
        % Disjunction (∨)
        \[
            \begin{array}{c|ccc}
                \lor        & 1 & \frac{1}{2} & 0           \\
                \hline
                1           & 1 & 1           & 1           \\
                \frac{1}{2} & 1 & \frac{1}{2} & \frac{1}{2} \\
                0           & 1 & \frac{1}{2} & 0           \\
            \end{array}
        \]
    \end{minipage}

    \vspace{0.5cm}

    \begin{minipage}{0.5\textwidth}
        % Negation (¬)
        \[
            \begin{array}{c|c}
                            & \neg        \\
                \hline
                1           & 0           \\
                \frac{1}{2} & \frac{1}{2} \\
                0           & 1           \\
            \end{array}
        \]
    \end{minipage}
    \begin{minipage}{0.3\textwidth}
        \[
            \begin{array}{c|c}
                            & \circ   \\
                \hline
                1           & 1         \\
                \frac{1}{2} & 0         \\
                0           & 1         \\
            \end{array}
        \]
    \end{minipage}

    \noindent
    \qed{}
\end{definicao}

A lógica \lfium{} é, portanto, dita trivalorada.

\migs{Aqui acho interessante você explicar que essa é uma semântica não determinística e explicar pq você está incluindo ela, assim como eventualmente provar que ela é intertraduzível para a semântica de matrizes (isso deve ter no livro eu creio, se não fazemos juntos essa prova)}
\begin{definicao} [Semântica de valorações para \textbf{$\text{LFI1}_{\circ}$}]
    \label{def:valoracoes}
    A função $v : \pazocal{L}_{\Sigma} \rightarrow \{1, 0\}$ é uma valoração para $\text{LFI1}_{\circ}$ caso ela satisfaça as seguintes cláusulas:
    \begin{align*}
        & v(\alpha \land \beta) = 1 \Longleftrightarrow v(\alpha) = 1 \text{ e } v(\beta) = 1\tag{\textbf{$vAnd$}}\\
        & v(\alpha \lor \beta) = 1 \Longleftrightarrow v(\alpha) = 1 \text{ ou } v(\beta) = 1\tag{\textbf{$vOr$}}\\
        & v(\alpha \rightarrow \beta) = 1 \Longleftrightarrow v(\alpha) = 0 \text{ ou } v(\beta) = 1\tag{\textbf{$vImp$}}\\
        & v(\neg \alpha) = 0 \Longrightarrow v(\alpha) = 1\tag{\textbf{$vNeg$}}\\
        & v(\circ \alpha) = 1 \Longrightarrow v(\alpha) = 0 \text{ ou } v(\neg \alpha) = 0\tag{\textbf{$vCon$}}\\
        & v(\neg \circ \alpha) = 1 \Longrightarrow v(\alpha) = 1 \text{ e } v(\neg \alpha) = 1\tag{\textbf{$vCi$}}\\
        & v(\neg \neg \alpha) = 1 \Longrightarrow v(\alpha) = 1\tag{\textbf{$vCf$}}\\
        & v(\alpha) = 1 \Longrightarrow v(\neg \neg \alpha) = 1\tag{\textbf{$vCe$}}\\
        & v(\neg (\alpha \land \beta)) = 1 \Longleftrightarrow v(\neg \alpha) = 1 \text{ ou } v(\neg \beta) = 1\tag{\textbf{$vDM_{\land}$}}\\
        & v(\neg (\alpha \lor \beta)) = 1 \Longleftrightarrow v(\neg \alpha) = 1 \text{ e } v(\neg \beta) = 1\tag{\textbf{$vDM_{\lor}$}}\\
        & v(\neg (\alpha \rightarrow \beta)) = 1 \Longleftrightarrow v(\alpha) = 1 \text{ e } v(\neg \beta) = 1\tag{\textbf{$vCIp_{\rightarrow}$}}
    \end{align*}
    
\end{definicao}