\chapter{Exemplo Prova com Casos e Subcasos}
        \lipsum[1]
        \begin{provaporcasos}
        
            \casodeprova \lipsum[66]

            \begin{provaporsubcasos}
            
                \subcasodeprova \lipsum[75]

                \subcasodeprova \lipsum[75]

                \subcasodeprova \lipsum[75]

                \subcasodeprova \lipsum[75]

            \end{provaporsubcasos}

            \casodeprova \lipsum[66]

            \casodeprova \lipsum[66]

            \begin{provaporsubcasos}
                
                \subcasodeprova \lipsum[75]

                \begin{provaporsubsubcasos}
                    
                    \subsubcasodeprova \lipsum[75]

                \end{provaporsubsubcasos}

            \end{provaporsubcasos}
        \end{provaporcasos}


        \begin{proof}[Prova do Teorema~\ref{teo:correcao_val}]
            Supondo $\Gamma \conhil \alpha$, existe uma sequência de derivação $\phi_{1} \ldots \phi_{n}$ onde $\phi_{n} = \alpha$. A prova de $\Gamma \conval \alpha$ é obtida por indução no tamanho $n$ da sequência de derivação:\\
    
            \noindent \textbf{\textsc{Base}} $n = 1$. A sequência contém somente uma fórmula $\phi_{1} = \alpha$. Portanto, existem duas possibilidades:
            \begin{enumerate}
                \item $\phi_{1}$ é um axioma.
                \item $\phi_{1} \in \Gamma$.
            \end{enumerate}
    
            \begin{provaporcasos}
                
                \casodeprova{} $\phi_{1}$ é um axioma. Então basta mostrar que para todo $v \in V^{\lfium{}}$, se $v(\gamma) = 1$ para todo $\gamma \in \Gamma$, então $v(\phi_{1}) = 1$. Como $\phi_{1}$ é um axioma, basta analisar todos os casos possíveis:
    
                \begin{provaporsubcasos}
                    
                    \subcasodeprova{} $\phi_{1} = \alpha \to (\beta \to \alpha)$.
    
                        Supondo $v(\alpha \to (\beta \to \alpha)) = 0$, temos $v(\alpha) = 1$ e $v(\beta \to \alpha) = 0$ por $(vImp)$. 
                            
                        Logo, $v(\beta) = 1$ e $v(\alpha) = 0$ novamente por $(vImp)$. Isto resulta numa contradição. 
                        
                        Portanto $v(\alpha \to (\beta \to \alpha)) = 1$.
    
                    \subcasodeprova{} $\phi_{1} = (\alpha \to (\beta \to \gamma)) \to ((\alpha \to \beta) \to (\alpha \to \gamma ))$.
                    
                        Supondo $v((\alpha \to (\beta \to \gamma)) \to ((\alpha \to \beta) \to (\alpha \to \gamma))) = 0$, temos, por ($vImp$), $v(\alpha \to (\beta \to \gamma)) = 1 \text{ e } v((\alpha \to \beta) \to (\alpha \to \gamma)) = 0$.
    
                        Portanto, por $(vImp)$, temos $v(\alpha) = 0$ ou $v(\beta \to \gamma) = 1$, e $v(\alpha \to \beta) = 1$ e $v(\alpha \to \gamma) = 0$. 
                        
                        Por $(vImp)$ segue $v(\alpha) = 1$ e $v(\gamma) = 0$, logo, $v(\beta \to \gamma) = 1$ e, portanto, $v(\beta) = 0$ ou $v(\gamma) = 1$. 
                        
                        Porém, como $v(\alpha \to \beta) = 1$ e $v(\alpha) = 1$, então $v(\beta) = 1$, o que resulta numa contradição. Logo $v((\alpha \to (\beta \to \gamma)) \to ((\alpha \to \beta) \to (\alpha \to \gamma))) = 1$.
    
                    \subcasodeprova{} $\phi_{1} = \alpha \to (\beta \to (\alpha \land \beta))$. 
    
                        Supondo $v(\alpha \to (\beta \to (\alpha \land \beta))) = 0$, temos $v(\alpha) = 1$ e $v(\beta \to (\alpha \land \beta)) = 0$, por $(vImp)$, então $v(\beta) = 1$ e $v(\alpha \land \beta) = 0$ novamente por $(vImp)$. 
                        
                        Com isso, temos $v(\alpha) = 0$ ou $v(\beta) = 0$ por $(vAnd)$, mas isso resulta numa contradição, já que $v(\alpha) = 1$ e $v(\beta) = 1$. 
                        
                        Portanto $v(\alpha \to (\beta \to (\alpha \land \beta))) = 1$.
    
                    \subcasodeprova{} $\phi_{1} = (\alpha \land \beta) \to \alpha$. 
                    
                        Supondo $v((\alpha \land \beta) \to \alpha) = 0$. 
                    
                        Logo $v(\alpha \land \beta) = 1$ e $v(\alpha) = 0$ por $(vImp)$. 
                        
                        Então, $v(\alpha) = 1$ e $v(\beta) = 1$ por $(vAnd)$, o que resulta numa contradição. 
                        
                        Portanto $v((\alpha \land \beta) \to \alpha) = 1$.
    
                    \subcasodeprova{} $\phi_{1} = (\alpha \land \beta) \to \beta$. Como no caso anterior, \textit{mutatis mutandis}.
    
                    \subcasodeprova{} $\phi_{1} = \alpha \to (\alpha \lor \beta)$. 
                    
                        Supondo $v(\alpha \to (\alpha \lor \beta)) = 0$, temos $v(\alpha) = 1$ e $v(\alpha \lor \beta) = 0$, por $(vImp)$. 
                        
                        Então temos $v(\alpha) = 0$ e $v(\beta) = 0$ por $(vOr)$, o que resulta numa contradição. 
                        
                        Portanto $v(\alpha \to (\alpha \lor \beta)) = 1$.
    
                    \subcasodeprova{} $\phi_{1} = \beta \to (\alpha \lor \beta)$. Como no caso anterior, \textit{mutatis mutandis}.
    
                    \subcasodeprova{} $\phi_{1} = (\alpha \to \gamma) \to ((\beta \to \gamma) \to ((\alpha \lor \beta) \to \gamma))$. 
                    
                        Supondo $v((\alpha \to \gamma) \to ((\beta \to \gamma) \to ((\alpha \lor \beta) \to \gamma))) = 0$, temos, por $(vImp)$, $v((\alpha \to \gamma)) = 1$ e $v(((\beta \to \gamma) \to ((\alpha \lor \beta) \to \gamma))) = 0$. 
                        
                        Portanto, novamente por $(vImp)$, temos $v(\alpha) = 0$ ou $v(\gamma) = 1$. Além disso, temos $v(\beta \to \gamma) = 1$ e $v((\alpha \lor \beta) \to \gamma) = 0$. 
                        
                        Então, temos $v(\beta) = 0$ ou $v(\gamma) = 1$. Ademais, $v(\alpha \lor \beta) = 1$ e $v(\gamma) = 0$. 
                        
                        Finalmente, por $(vOr)$, temos $v(\alpha) = 1$ ou $v(\beta) = 1$. O fato de termos $v(\gamma) = 0$ nos permite concluir $v(\alpha) = 0$ e $v(\beta) = 0$, porém isto nos dá $v(\alpha) = 1$ (já que temos $v(\alpha) = 1$ ou $v(\beta) = 1$ e também $v(\beta) = 0$), o que resulta numa contradição, pois já temos $v(\alpha) = 0$. 
                        
                        Portanto, $v((\alpha \to \gamma) \to ((\beta \to \gamma) \to ((\alpha \lor \beta) \to \gamma))) = 1$.
    
                    \subcasodeprova{} $\phi_{1} = (\alpha \to \beta) \lor \alpha$. 
                        
                        Supondo $v((\alpha \to \beta) \lor \alpha) = 0$, então, por $(vOr)$, temos $v(\alpha \to \beta) = 0$ e $v(\alpha) = 0$. 
                        
                        Logo, por $(vImp)$, $v(\alpha) = 1$  e $v(\beta) = 0$, o que resulta numa contradição. 
                        
                        Portanto, $v((\alpha \to \beta) \lor \alpha) = 1$.
    
                    \subcasodeprova{} $\phi_{1} = \alpha \lor \neg \alpha$. 
                    
                        Supondo $v(\alpha \lor \neg \alpha) = 0$, temos por $(vOr)$, $v(\alpha) = 0$ e $v(\neg \alpha) = 0$. 
                        
                        Então, por $(vNeg)$, $v(\alpha) = 1$, o que resulta numa contradição. 
                        
                        Portanto, $v(\alpha \lor \neg \alpha) = 1$.
    
                    \subcasodeprova{} $\phi_{1} = \circ \alpha \to (\alpha \to (\neg \alpha \to \beta))$. 
                    
                        Supondo $v(\circ \alpha \to (\alpha \to (\neg \alpha \to \beta))) = 0$, então, por $(vImp)$, $v(\circ \alpha) = 1$ e $v(\alpha \to (\neg \alpha \to \beta)) = 0$. 
                        
                        Logo, por $(vCon)$, $v(\alpha) = 0$ ou $v(\neg \alpha) = 0$. Ademais, por $(vImp)$, $v(\alpha) = 1$ e $v(\neg \alpha \to \beta) = 0$. 
                        
                        Novamente por $(vImp)$, $v(\neg \alpha) = 1$ e $v(\beta) = 0$. Podemos concluir $v(\alpha) = 0$, já que temos $v(\neg \alpha) = 1$ e também temos $v(\alpha) = 0$ ou $v(\neg \alpha) = 0$. 
                        
                        Isto resulta numa contradição, pois temos $v(\alpha) = 1$ e $v(\alpha) = 0$. 
                        
                        Portanto, $v(\circ \alpha \to (\alpha \to (\neg \alpha \to \beta))) = 1$.
    
                    \subcasodeprova{} $\phi_{1} = \neg \neg \alpha \to \alpha$. 
                        
                        Supondo $v(\neg \neg \alpha \to \alpha) = 0$, então, por $(vImp)$, $v(\neg \neg \alpha) = 1$ e $v(\alpha) = 0$. 
                        
                        Por $(vDNE)$ temos $v(\alpha) = 1$, o que resulta numa contradição. 
                        
                        Portanto, $v(\neg \neg \alpha \to \alpha) = 1$.
    
                    \subcasodeprova{} $\phi_{1} = \alpha \to \neg \neg \alpha$. 
                    
                        Supondo $v(\alpha \to \neg \neg \alpha) = 0$, então, por $(vImp)$, $v(\alpha) = 1$ e $v(\neg \neg \alpha) = 0$. 
                        
                        Por $(vDNE)$ temos $v(\neg \neg \alpha) = 1$, o que resulta numa contradição. 
                        
                        Portanto, $v(\alpha \to \neg \neg \alpha) = 1$.
    
                    \subcasodeprova{} $\phi_{1} = \neg \circ \alpha \to (\alpha \land \neg \alpha)$. 
                    
                        Supondo $v(\neg \circ \alpha \to (\alpha \land \neg \alpha)) = 0$, temos, por $(vImp)$, $v(\neg \circ \alpha) = 1$ e $v(\alpha \land \neg \alpha) = 0$. 
                        
                        Por $(vCi)$ temos $v(\alpha) = 1$ e $v(\neg \alpha) = 1$. Ademais, por $(vAnd)$, temos $v(\alpha) = 0$ ou $v(\neg \alpha) = 0$. 
                        
                        Podemos concluir $v(\alpha) = 0$, já que temos $v(\alpha) = 0$ ou $v(\neg \alpha) = 0$ e também temos $v(\neg \alpha) = 1$. 
                        
                        Isto resulta numa contradição, pois temos $v(\alpha) = 1$ e  $v(\alpha) = 0$. 
                        
                        Portanto, $v(\neg \circ \alpha \to (\alpha \land \neg \alpha)) = 1$.
    
                    \subcasodeprova{} $\phi_{1} = \neg (\alpha \lor \beta) \to (\neg \alpha \land \neg \beta)$. 
                    
                        Supondo $v(\neg (\alpha \lor \beta) \to (\neg \alpha \land \neg \beta)) = 0$, temos, por $(vImp)$, $v(\neg (\alpha \lor \beta)) = 1$ e $v(\neg \alpha \land \neg \beta) = 0$. 
                        
                        Por $(vDM_{\lor})$ temos $v(\neg \alpha) = 1$ e $v(\neg \beta) = 1$. Ademais, por $(vAnd)$, temos $v(\neg \alpha) = 0$ ou $v(\neg \beta) = 0$. 
                        
                        Isto, unido ao fato de termos $v(\neg \beta) = 1$, nos permite concluir $v(\neg \alpha) = 0$, o que resulta numa contradição. 
                        
                        Portanto, $v(\neg (\alpha \lor \beta) \to (\neg \alpha \land \neg \beta)) = 1$.
    
    
                    \subcasodeprova{} $\phi_{1} = (\neg \alpha \land \neg \beta) \to \neg (\alpha \lor \beta)$. Como no caso anterior, \textit{mutatis mutandis}.
    
                    \subcasodeprova{} $\phi_{1} = \neg(\alpha \land \beta) \to (\neg \alpha \lor \neg \beta)$. 
                    
                    Supondo $v(\neg(\alpha \land \beta) \to (\neg \alpha \lor \neg \beta)) = 0$, temos, por $(vImp)$, $v(\neg(\alpha \land \beta)) = 1$ e $v(\neg \alpha \lor \neg \beta) = 0$. 
                    
                    Então, por $(vDM_{\land})$, temos $v(\neg \alpha) = 1$ ou $v(\neg \beta) = 1$. Além disso, por $(vOr)$, temos $v(\neg \alpha) = 0$ e $v(\neg \beta) = 0$. 
                    
                    Isto nos permite concluir $v(\neg \alpha) = 1$, pois temos $v(\neg \alpha) = 1$ ou $v(\neg \beta) = 1$ e também temos $v(\neg \beta) = 0$. 
                    
                    Isto resulta numa contradição, pois temos $v(\neg \alpha) = 1$ e $v(\neg \alpha) = 0$. 
                    
                    Portanto, $v(\neg(\alpha \land \beta) \to (\neg \alpha \lor \neg \beta)) = 1$.
    
                    \subcasodeprova{} $\phi_{1} = (\neg \alpha \lor \neg \beta) \to \neg (\alpha \land \beta)$. Como no caso anterior, \textit{mutatis mutandis}.
    
                    \subcasodeprova{} $\phi_{1} = \neg (\alpha \to \beta) \to(\alpha \land \neg \beta)$. 
                    
                    Supondo $v(\neg (\alpha \to \beta) \to (\alpha \land \neg \beta)) = 0$. Por $(vImp)$ temos $v(\neg (\alpha \to \beta)) = 1$ e $v(\alpha \land \neg \beta) = 0$. 
                    
                    Então, por $(vCip_{\to})$, temos $v(\alpha) = 1$ e $v(\neg \beta) = 1$. Ademais, por $(vAnd)$, temos $v(\alpha) = 0$ ou $v(\neg \beta) = 0$, o que unido ao fato de termos $v(\neg \beta) = 1$, nos permite concluir $v(\alpha) = 0$. 
                    
                    Isto resulta numa contradição, portanto $v(\neg (\alpha \to \beta) \to (\alpha \land \neg \beta)) = 1$.
    
                    \subcasodeprova{} $\phi_{1} = (\alpha \land \neg \beta) \to \neg(\alpha \to \beta)$. Como no caso anterior, \textit{mutatis mutandis}.
                    
                \end{provaporsubcasos}
    
                Com isso, o \textsc{Caso 1} está provado e $\Gamma \conval \phi_{1}$ segue caso $\phi_{1}$ seja um axioma.
    
                \casodeprova{} $\phi_{1} \in \Gamma$. Logo, se $v[\Gamma] \subseteq \{1\}$, temos $v(\phi_{1}) = 1$. Portanto, $\Gamma \conval \phi_{1}$.
    
            \end{provaporcasos}
    
             \noindent \textbf{\textsc{Passo}} Hipótese de indução (HI): Para qualquer sequência da derivação de $\Gamma \conhil \alpha$ de tamanho $k < n$, tem-se $\Gamma \conval \alpha$. 
             
             Portanto, é preciso mostrar que $\Gamma \conval \alpha$ segue caso a sequência de derivação de $\Gamma \conhil \alpha$ tenha tamanho $n$. Então, vamos supor $\Gamma \conhil \phi_{n}$.
             
             Ao analisar a obtenção de $\phi_{n}$ em $\Gamma \conhil \phi_{n}$, existem três casos a se considerar:
             
             \begin{enumerate}
                \item $\phi_{n}$ é um axioma.
                \item $\phi_{n} \in \Gamma$.
                \item $\phi_{n}$ é obtido por \textit{modus ponens} em duas fórmulas $\phi_{j}$ e $\phi_{k}$ com $j, k < n$. 
             \end{enumerate}
             
             Os casos 1 e 2 são análogos aos casos provados na base.
             
             \noindent \textsc{Caso 3.} $\phi_{n}$ é obtido por \textit{modus ponens} em duas fórmulas $\phi_{j}$ e $\phi_{k}$ com $j, k < n$. 
             
             Logo, $\phi_{k} = \phi_{j} \to \phi_{n}$ (ou $\phi{j} = \phi_{k} \to \phi_{n}$, a prova para este caso é análoga). 
             
             Dada nossa suposição de $\Gamma \conhil \phi_{n}$, então $\Gamma \conhil \phi_{j}$ e $\Gamma \conhil \phi_{j} \to \phi_{n}$. 
             
             Pela (HI), temos $\Gamma \conval \phi_{j}$ e $\Gamma \conval \phi_{j} \to \phi_{n}$. 
             
             Então, supondo $v[\Gamma] \subseteq \{1\}$, temos $v(\phi_{j}) = 1$ e $v(\phi_{j} \to \phi_{n}) = 1$. 
             
             Por $(vImp)$ temos $v(\phi_{j}) = 0$ ou $v(\phi_{n}) = 1$. 
             
             Isto, unido ao fato de termos $v(\phi_{j}) = 1$, nos permite concluir $v(\phi_{n}) = 1$. 
             
             Portanto $\Gamma \conval \phi_{n}$.
    
             \noindent Com isso, provamos $\Gamma \conval \alpha$ e a prova está finalizada.
    
        \end{proof}