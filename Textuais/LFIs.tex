\chapter{Lógicas de Inconsistência Formal}
\label{cap:LFIs}
No estudo de lógicas clássicas uma contradição é considerada inseparável da trivialidade, ou seja, se uma teoria possuir um subconjunto $\{\alpha,\neg \alpha\}$ de fórmulas, pode-se derivar qualquer sentença. Esta propriedade é chamada de \textit{explosividade}. Desta forma, as lógicas clássicas (e certas lógicas não-clássicas, como a lógica intuicionista), expressam sua \textit{explosividade} como representada pela seguinte equação:
\begin{center}
    Contradições = Trivialidade
\end{center}
As \textit{Lógicas de Inconsistência Formal} são lógicas paraconsistentes que se propõem a questionar a noção apresentada anteriormente sem abrir mão completamente da trivialidade. Isto é feito estabelecendo uma nova propriedade, chamada de \textit{explosividade gentil}, que resgata a trivialidade introduzindo o conceito de consistência na sua linguagem~\cite{carnielli2007}. A consistência é expressa na \textit{explosividade gentil} da seguinte forma:
\begin{center}
    Contradições + Consistência = Trivialidade
\end{center}
Definir uma lógica que consiga superar o tabu da \textit{explosividade} e, ao mesmo tempo, representar uma ferramenta legítima capaz de formalizar o raciocínio e separar inferências aceitáveis de inferências equivocadas é um dos objetivos do \textit{paraconsistentista}.\helena{Esse termo é do diabo mas o carnielli e o joão marcos usaram e eu achei até que bonitinho.} As \textit{Lógicas de Inconsistência Formal} cumprem este objetivo de maneira elegante, servindo um propósito importante no estudo de lógicas não-clássicas.\helena{eu boto aquele lerolero de banco de dados aqui tb?}.
Neste capítulo são apresentadas algumas definições necessárias para caracterizar as \textit{Lógicas de Inconsistência Formal}, baseadas em~\citeshort{Carnielli_Coniglio_2016} e em~\citeshort{carnielli2007}. Antes de definir as \lfis{} é preciso apresentar alguns conceitos básicos acerca de sistemas lógicos paraconsistentes. Nas definições que seguem, utiliza-se a seguinte representação: \migs{Isso deveria ir para a lista de símbolos}
\begin{itemize}
    \item Letras minúsculas do alfabeto latim $p, q, r, \ldots$ para representar fórmulas atômicas.
    \item Letras maiúsculas do alfabeto latim $A, B, C, \ldots$ para representar conjuntos quaisquer.
    \item Letra minúsculas do alfabeto grego $\alpha, \beta, \gamma, \ldots$ para representar fórmulas quaisquer.
    \item As letras $\Gamma, \Delta$ para representar teorias (conjunto de fórmulas).
    \item As letras $\Sigma, \Theta$ para representar a assinatura de uma linguagem.
    \item O operador $\vdash$ para representar uma relação de consequência sintática.
    \item O operador $\vDash$ para representar uma relação de consequência semântica.
    \item O operador $\Vdash$ para representar uma relação de consequência qualquer (sintática ou semântica).
\end{itemize}

Ademais, o presente trabalho segue o mesmo caminho de~\citeshort{Carnielli_Coniglio_2016}, baseando-se na teoria geral de relações de consequências para definir \textit{lógicas tarskianas} (lógicas com uma relação de consequência-T). Neste sentido, como a lógica \lfium{} se trata de uma \textit{lógica tarskiana}, o presente trabalho se restringe a trabalhar somente neste escopo.

\section{Paraconsistência}
\label{sec:paracons}

Uma lógica $\mathcal{L}$ será representada como uma dupla $\mathcal{L} = \langle \pazocal{L},\Vdash \rangle$, onde $\pazocal{L}$ é sua linguagem (seu conjunto de fórmulas) e $\Vdash$ é uma relação de consequência de conclusão única, definida como $\Vdash \;\subseteq \wp(\pazocal{L})\times\pazocal{L}$.

\begin{definicao}[Assinatura proposicional]
    \label{def:ass_prop}
    Uma assinatura proposicional $\Theta$ é um conjunto de conectivos lógicos com a informação acerca da aridade de cada um destes.\qed{}
\end{definicao}
Por exemplo, a assinatura proposicional para a lógica proposicional clássica pode ser definida como $\Theta_{LPC} = \{\land^{2}, \lor^{2}, \neg^{1}, \rightarrow^{2}\}$, onde o operador $\land^{2}$ representa uma conjunção, $\lor^{2}$ representa uma disjunção, $\neg^{1}$ representa uma negação e $\rightarrow^{2}$ representa uma implicação. No que segue do texto, as aridades de cada operador serão omitidas.

\begin{definicao}[Lógica proposicional]
    \label{def:proposicional}
    Um sistema lógico $\mathcal{L}$, definido sobre uma linguagem $\pazocal{L}$ é dito proposicional caso $\pazocal{L}$ seja definida a partir de um conjunto enumerável de átomos $\pazocal{P} = \{p_{i} \;| \; i \in \mathbb{N} \}$ e uma assinatura proposicional $\Theta$. Uma linguagem $\pazocal{L}$ definida sobre uma assinatura proposicional é chamada de linguagem proposicional.\qed{}
\end{definicao}

\begin{definicao}[Substituição]
    Uma substituição $\sigma$ de todas as ocorrências de uma variável $p_{i}$ por uma fórmula $\psi$ em uma fórmula $\phi$, é denotada por $\sigma(\phi) = \phi\{p_{i} \mapsto \psi\}$~\cite{dedo}. A substituição $\phi\{p_{i} \mapsto \psi\}$ é definida indutivamente como (considerando $\triangle$, $\otimes$ conectivos quaisquer de aridade 1 e 2 respectivamente):
    \begin{align*}
         & \text{1.~Se }\phi = p_{i} \text{ então, } \phi\{p_{i} \mapsto \psi\} = \psi;                                                                                             \\
         & \text{2.~Se }\phi = p_{j} \text{ e } j \neq i \text{ então, }\phi\{p_{i} \mapsto \psi\} = \phi;                                                                          \\
         & \text{3.~Se }\phi = \triangle \gamma \text{ então, } \phi\{p_{i} \mapsto \psi\} = \triangle(\gamma\{p_{i} \mapsto \psi\});                                                 \\
         & \text{4.~Se }\phi = \phi_{0} \otimes \phi_{1} \text{ então, } \phi\{p_{i} \mapsto \psi\} = \phi_{0}\{p_{i} \mapsto \psi\} \otimes \phi_{1}\{p_{i} \mapsto \psi\}.
    \end{align*}
    Ademais, a aplicação de $\sigma$ sobre uma teoria $\Gamma$ é definida como $\sigma[\Gamma] = \{\sigma(\phi) \; | \; \phi \in \Gamma\}$.\qed{}
\end{definicao}

\begin{notacao}
    Sejam $\Gamma, \Delta$ teorias e $\phi, \psi$ fórmulas, então $\Gamma, \Delta, \phi \Vdash \psi$ denota $\Gamma \cup \Delta \cup \{\phi\} \Vdash \psi$.
\end{notacao}

Tendo definido a noção de substituição para lógicas proposicionais, é possível descrever as lógicas estruturais como sendo lógicas que preservam uma dada derivação mesmo após aplicar uma substituição em suas premissas e conclusão:

\begin{definicao}[Lógica Estrutural]
    Uma lógica proposicional $\mathcal{L}$ definida sobre uma linguagem proposicional $\pazocal{L}_{\Theta}$ é dita \textit{estrutural} caso respeite a seguinte condição:
    \begin{align*}
         & \text{~~(v) Se } \Gamma \Vdash \alpha \text{ então } \sigma [\Gamma] \Vdash \sigma(\alpha) \text{, para toda substituição } \sigma \text{ de variável por fórmula.}
    \end{align*}
\end{definicao}

\begin{definicao}[Lógica Tarskiana]
    \label{def:tarski}
    Uma lógica $\mathcal{L}$, definida sobre uma linguagem $\pazocal{L}$ e munida com uma relação de consequência $\Vdash$ é dita \textit{Tarskiana} caso satisfaça as seguintes propriedades para todo $\Gamma \cup \Delta \cup \{\alpha\} \subseteq \pazocal{L}$:
    \begin{align}
         & \text{~~(i) Se } \alpha \in \Gamma \text{ então } \Gamma \Vdash \alpha;\tag{reflexividade}                                                                                       \\
         & \text{~(ii) Se } \Delta \Vdash \alpha \text{ e } \Delta \subseteq \Gamma \text{ então } \Gamma \Vdash \alpha;\tag{monotonicidade}                                                \\
         & \text{(iii) Se } \Delta \Vdash \alpha \text{ e } \Gamma \Vdash \delta \text{ para todo } \delta \in \Delta \text{ então } \Gamma \Vdash \alpha.\tag{\textit{cut} para conjuntos}
    \end{align}
    Uma lógica $\mathcal{L}$ é dita \textit{finitária} caso satisfaça o seguinte:
    \begin{align*}
         & \text{~(iv) Se } \Gamma \Vdash \alpha \text{ então existe conjunto finito } \Gamma_{0} \subseteq \Gamma \text{ tal que } \Gamma_{0} \Vdash \alpha.
    \end{align*}
    Por fim, uma lógica proposicional $\mathcal{L}$ é dita \textit{padrão} caso ela seja Tarskiana, finitária e estrutural.\qed{}
\end{definicao}


Com isto, é possível definir formalmente a \textit{paraconsistência} para lógicas Tarskianas.

\begin{definicao}[Lógica Tarskiana paraconsistente]
    \label{def:tarskiana_paracons}
    Uma lógica Tarskiana $\mathcal{L}$, definida sobre uma linguagem $\pazocal{L}$, é dita \textit{paraconsistente} se ela possuir uma negação $\neg$\footnote{Esta negação pode ser primitiva (pertencente à assinatura da linguagem) ou definida a partir de outras fórmulas.} tal que existem fórmulas $\alpha, \beta \in \pazocal{L}$ de modo que $\alpha, \neg \alpha \nVdash \beta$.\qed{}
\end{definicao}

Caso a linguagem de $\mathcal{L}$ possua uma implicação $\rightarrow$ que respeite o metateorema da dedução\footnote{Definido como $\Gamma, \alpha \Vdash \beta \Longleftrightarrow  \Gamma\Vdash \alpha \rightarrow \beta$.}, então $\mathcal{L}$ é paraconsistente somente se a fórmula $\alpha \rightarrow (\neg \alpha \rightarrow \beta)$ não for válida. Ou seja, o Princípio da Explosão é inválido (em relação a $\neg$), logo $\neg$ é uma negação \textit{não explosiva}.

\section{Inconsistência}

A motivação para o desenvolvimento das \lfis{} é possuir sistemas lógicos paraconsistentes nos quais é possível resgatar, de maneira \textit{controlada}, o Princípio da Explosão. Isto é feito definindo um conjunto $\bigcirc(p)$ de fórmulas dependentes somente de uma variável proposicional $p$. Caso uma lógica $\mathcal{L}$ seja explosiva ao unir-se um conjunto $\bigcirc(\alpha)$ {-} definido a partir de $\bigcirc(p)$ {-} com uma contradição $\{\alpha, \neg \alpha\}$, ou seja, se $\bigcirc(\alpha), \alpha, \neg \alpha \Vdash \beta$ para todo $\alpha$ e $\beta$ pertencentes à sua linguagem, e ainda $\bigcirc(\alpha), \alpha \nVdash \beta$ e $\bigcirc(\alpha), \neg \alpha \nVdash \beta$, então dizemos que $\mathcal{L}$ é \textit{gentilmente explosiva}.

\begin{notacao}
    Dado um átomo $p$, define-se $\bigcirc(p)$ como um conjunto não-vazio de fórmulas dependentes somente em $p$. Com base neste conjunto, define-se a notação $\bigcirc(\phi)$ para representar o conjunto obtido pela substituição de todas as ocorrências de $p$ por $\phi$ em todos os elementos de $\bigcirc(p)$, ou seja, para uma fórmula $\phi$ qualquer, $\bigcirc(\phi) = \{\psi[p \mapsto \phi] \; | \; \psi \in \bigcirc(p)\}$.
\end{notacao}



\begin{definicao}[Lógica de Inconsistência Formal]
    \label{def:lfi}
    Seja $\mathcal{L} = \langle \pazocal{L}_{\Theta}, \Vdash \rangle$ uma lógica padrão, de forma que sua assinatura proposicional $\Theta$ possua uma negação $\neg$. Seja $\bigcirc(p)$ um conjunto não-vazio de fórmulas dependentes somente na variável proposicional $p$. Então $\mathcal{L}$ será uma \textit{Lógica de Inconsistência Formal} (\lfi{}) (em relação a $\bigcirc(p)$ e $\neg$) caso ela respeite as seguintes condições:
    %(considerando $\bigcirc(\phi) = \{\psi(\phi) \; | \; \psi(p) \in \bigcirc(p)\}$): \helena{O conjunto $\bigcirc(\phi)$ é definido substituindo-se cada ocorrência $p$ por $\phi$ em cada elemento de $\bigcirc(\phi)$.}
    \begin{align*}
         & \text{~~(i) Existem } \gamma, \delta \in \pazocal{L}_{\Theta} \text{ de modo que } \gamma, \neg \gamma \nVdash \delta;               \\
         & \text{~(ii) Existem } \alpha, \beta \in \pazocal{L}_{\Theta} \text{ de modo que:}                                                    \\
         & \qquad \text{(ii.a)} \bigcirc(\alpha), \alpha \nVdash \beta;                                                                         \\
         & \qquad \text{(ii.a)} \bigcirc(\alpha), \neg \alpha \nVdash \beta;                                                                    \\
         & \text{(iii) Para todo } \phi, \psi \in \pazocal{L}_{\Theta} \text{ tem-se } \bigcirc(\phi), \phi, \neg \phi \Vdash \psi. \tag*\qed{}
    \end{align*}
\end{definicao}
\helena{Definir \lfi{} fraca e forte também?!?!?!?}
A condição (i) diz que toda \lfi{} é \textit{não-explosiva} (em relação a $\neg$) e a condição (iii) diz que toda \lfi{} é \textit{gentilmente explosiva} (em relação a $\bigcirc{p}$ e $\neg$).


