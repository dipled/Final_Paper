\chapter{Lógicas de Inconsistência Formal}\label{cap:LFIs}
No estudo de lógicas clássicas, uma contradição é considerada inseparável da trivialidade, ou seja, se uma teoria possuir um subconjunto $\{\alpha,\neg \alpha\}$ de fórmulas, pode-se derivar qualquer sentença. Esta propriedade é chamada de \textit{explosividade}. Desta forma, as lógicas clássicas (e certas lógicas não-clássicas, como a lógica intuicionista), expressam sua \textit{explosividade} como representada pela seguinte equação: \cortar{\migs{Esse parágrafo ficou muito bom}}
\begin{center}
    Contradições = Trivialidade
\end{center}
As \textit{lógicas de inconsistência formal} são lógicas paraconsistentes que se propõem a questionar a noção apresentada anteriormente sem abrir mão completamente da trivialidade. Isto é feito estabelecendo uma nova propriedade, chamada de \textit{explosividade gentil}, que resgata a trivialidade introduzindo o conceito de consistência na sua linguagem~\cite{carnielli2007}. A consistência é expressa na \textit{explosividade gentil} da seguinte forma:
\begin{center}
    Contradições + Consistência = Trivialidade
\end{center}
Definir uma lógica que consiga superar o tabu da \textit{explosividade} e, ao mesmo tempo, representar uma ferramenta legítima capaz de formalizar o raciocínio e separar inferências aceitáveis de inferências equivocadas é um dos objetivos dos lógicos que se descrevem como \textit{paraconsistentistas}. As lógicas de inconsistência formal cumprem este objetivo de maneira elegante, servindo um propósito importante no estudo de lógicas não-clássicas.

Neste capítulo, são apresentadas algumas definições necessárias para caracterizar as lógicas de inconsistência formal, baseadas em~\citeshort{Carnielli_Coniglio_2016} e em~\citeshort{carnielli2007}. Antes de definir as \lfis{} é preciso apresentar alguns conceitos básicos acerca de sistemas lógicos paraconsistentes.

Ademais, o presente trabalho segue o mesmo caminho de~\citeshort{Carnielli_Coniglio_2016}, baseando-se na teoria geral de relações de consequências para definir \textit{lógicas tarskianas}. Neste sentido, como a lógica \lfium{} se trata de uma \textit{lógica tarskiana}, o presente trabalho se restringe a trabalhar somente neste escopo. 

Nas Seções~\ref{sec:paracons} e~\ref{sec:incons}, serão apresentadas definições necessárias para identificar formalmente as lógicas paraconsistentes e lógicas de inconsistência formal. Estas definições se aplicam tanto a relações de consequência semântica quanto a relações de consequência sintática, denotadas genericamente pelo operador $\Vdash$. Este trabalho não se aprofunda em detalhes sobre conceitos envolvendo relações e operações de consequência, suas propriedades e as diferenças entre abordagens prova-teóricas modelo-teóricas. O leitor interessado em tais assuntos pode consultar os trabalhos de Wójcicki~(\citeyear{Wojcicki1984-WJCLOP,Wojcicki1988-WOJAAT,Wojcicki1988-WOJTOL}) e~\citeshort{analysis_and_synthesis_of_logic}.

\section{Paraconsistência}\label{sec:paracons}
    Um sistema lógico que se atreve a romper com o princípio da explosão {--} o qual afirma que a partir de uma teoria contraditória, qualquer conclusão segue {--} é dito paraconsistente. As justificativas para questionar tal princípio existem em diversos campos do conhecimento, como na linguística~\cite{McGinnis2013-MCGTUA}, na computação~\cite{carnielli2000formal} e até mesmo nas ciências naturais~\cite{Brown2015-BROCAP-9}. Outra justificativa para o desenvolvimento de sistemas lógicos paraconsistentes é um descontentamento com o caráter explosivo das lógicas ortodoxas. Por exemplo, uma discordância de que a partir da evidência de que ``Choveu na tarde de ontem.'' e da evidência de que ``Não choveu na tarde de ontem.'' pode-se concluir que ``Um triângulo tem quatro lados.''~\footnote{Esta forma de explosividade é um exemplo de uma contradição estudada pelas lógicas de relevância, que tratam da conexão entre as premissas e a conclusão de uma inferência~\cite{sep-logic-relevance}.}. Nesta seção, a paraconsistência será definida formalmente, partindo de definições básicas sobre lógica e classificando diferentes sistemas de acordo com propriedades acerca de sua \textit{relação de consequência}.

    Uma lógica $\mathcal{L}$ será representada como uma dupla $\mathcal{L} = \langle \pazocal{L},\Vdash \rangle$, onde $\pazocal{L}$ é sua linguagem (seu conjunto de fórmulas bem formadas) e $\Vdash$ é uma relação de consequência de conclusão única, definida como $\Vdash \;\subseteq \wp(\pazocal{L})\times\pazocal{L}$, sendo $\wp(\pazocal{L})$ o conjunto das partes de $\pazocal{L}$. Em uma consequência do tipo $\Gamma \Vdash \alpha$ (lida como ``$\alpha$ é uma consequência de $\Gamma$'') diz-se que o conjunto $\Gamma$ é o conjunto de premissas e $\alpha$ é a conclusão. A fim de facilitar a escrita e leitura, a seguinte notação será utilizada ao longo do texto:

    \begin{notacao}
        Sejam $\Gamma, \Delta$ conjuntos de fórmulas e $\phi, \psi$ fórmulas, então $\Gamma, \Delta, \phi \Vdash \psi$ denota $\Gamma \cup \Delta \cup \{\phi\} \Vdash \psi$.
    \end{notacao}

    \begin{definicao}[Assinatura proposicional]\label{def:ass_prop}
        Uma assinatura proposicional $\Theta$ é um conjunto de conectivos lógicos, cada um contendo a informação sobre sua aridade.\qed{}
    \end{definicao}
    Por exemplo, a assinatura proposicional para a lógica proposicional clássica pode ser definida como $\Theta_{LPC} = \{\land^{2}, \lor^{2}, \neg^{1}, \to^{2}\}$, onde o operador $\land^{2}$ representa uma conjunção, $\lor^{2}$ representa uma disjunção, $\neg^{1}$ representa uma negação e $\to^{2}$ representa uma implicação. No restante do texto, as aridades destes conectivos será omitida e a aridade de novos conectivos será apresentada somente na sua definição.

    Uma assinatura proposicional juntamente com um conjunto enumerável de átomos são base para a definição de uma linguagem proposicional, que por sua vez é utilizada para definir uma lógica proposicional, como é o caso da \lfium{}.

    \begin{definicao}[Lógica proposicional]\label{def:proposicional}
        Um sistema lógico $\mathcal{L}$, definido sobre uma linguagem $\pazocal{L}_{\Theta}$ é dito proposicional caso $\pazocal{L}_{\Theta}$ seja definida a partir de um conjunto enumerável de átomos $\pazocal{P} = \{p_{i} \;| \; i \in \mathbb{N} \}$ e uma assinatura proposicional $\Theta$. A linguagem $\pazocal{L}_{\Theta}$ é chamada de linguagem proposicional. Escreveremos $\pazocal{L}$ caso a assinatura da linguagem seja irrelevante ou possa ser inferida sem ambiguidade a partir do contexto. (Já que você já estava fazendo isso, adicionei um comentário na definição.)\qed{}
    \end{definicao}


    No desenvolvimento de metateoremas sobre propriedades de uma determinada lógica, a indução na complexidade de uma fórmula é um método comum de prova. Para isso, dada uma lógica $\mathcal{L}$ sobre uma linguagem $\pazocal{L}$, define-se uma função recursiva $C(\phi) : \pazocal{L} \to \mathbb{N}$ que retorna, para uma dada fórmula, um número natural representando sua complexidade, baseada na quantidade de operadores e átomos:

    \begin{definicao}[Complexidade de fórmulas para a lógica proposicional clássica]\label{def:complex}
        Dada uma fórmula $\phi \in \pazocal{L}_{LPC}$, a complexidade $C(\phi)$ é definida recursivamente da seguinte forma:
        \begin{align*}
            & \text{1.~Se } \phi = p \text{, onde } p \in \pazocal{P} \text{, então } C(\phi) = 1;\\
            & \text{2.~Se } \phi = \neg \psi \text{, então } C(\phi) = C(\psi) + 1;\\
            & \text{3.~Se } \phi = \psi \otimes \gamma \text{, onde } \otimes \in \{\land, \lor, \to\} \text{, então } C(\phi) = C(\psi) + C(\gamma) + 1.\tag*\qed{}
        \end{align*}
    \end{definicao}


    \begin{definicao}[Substituição]\label{def:substituicao}
        Uma substituição $\sigma$ de todas as ocorrências de uma variável $p_{i}$ por uma fórmula $\psi$ em uma fórmula $\phi$, é denotada por $\sigma(\phi) = \phi\{p_{i} \mapsto \psi\}$~\cite{dedo}. A substituição $\phi\{p_{i} \mapsto \psi\}$ é definida indutivamente como (considerando $\triangle$, $\otimes$ conectivos quaisquer de aridade 1 e 2 respectivamente):
        \begin{align*}
            & \text{1.~Se }\phi = p_{i} \text{ então, } \phi\{p_{i} \mapsto \psi\} = \psi;                                                                                             \\
            & \text{2.~Se }\phi = p_{j} \text{ e } j \neq i \text{ então, }\phi\{p_{i} \mapsto \psi\} = \phi;                                                                          \\
            & \text{3.~Se }\phi = \triangle \gamma \text{ então, } \phi\{p_{i} \mapsto \psi\} = \triangle(\gamma\{p_{i} \mapsto \psi\});                                                 \\
            & \text{4.~Se }\phi = \phi_{0} \otimes \phi_{1} \text{ então, } \phi\{p_{i} \mapsto \psi\} = \phi_{0}\{p_{i} \mapsto \psi\} \otimes \phi_{1}\{p_{i} \mapsto \psi\}.
        \end{align*}
        Uma fórmula $\alpha$ é dita \textit{instância de substituição} de uma fórmula $\beta$ caso exista uma substituição $\sigma$ tal que $\alpha = \sigma(\beta)$.\qed{}
    \end{definicao}

    \begin{notacao}
        Dada uma função $f : A \to B$ e um conjunto $A' \subseteq A$, $f[A']$ denota o conjunto $\{f(a) \; | \; a \in A'\}$.
    \end{notacao}


    \begin{definicao}[Lógica Tarskiana]\label{def:tarski}
        Uma lógica $\mathcal{L}$, definida sobre uma linguagem $\pazocal{L}$ e munida com uma relação de consequência $\Vdash$ é dita \textit{Tarskiana} caso satisfaça as seguintes propriedades para todo $\Gamma \cup \Delta \cup \{\alpha\} \subseteq \pazocal{L}$:
        \begin{align}
            \text{(i) } & \text{Se } \alpha \in \Gamma \text{ então } \Gamma \Vdash \alpha;\tag{reflexividade}                                                                                       \\
            \text{(ii) } & \text{Se } \Delta \Vdash \alpha \text{ e } \Delta \subseteq \Gamma \text{ então } \Gamma \Vdash \alpha;\tag{monotonicidade}                                                \\
            \text{(iii) } & \text{Se } \Delta \Vdash \alpha \text{ e } \Gamma \Vdash \delta \text{ para todo } \delta \in \Delta \text{ então } \Gamma \Vdash \alpha.\tag{corte}\\
            &\tag*\qed{}
        \end{align}
    \end{definicao}

    Com a noção de substituição para lógicas proposicionais apresentada na Definição~\ref{def:substituicao}, é possível definir lógica \textit{padrão} como sendo uma lógica tarskiana, \textit{finitária} e \textit{estrutural}:

    \begin{definicao}[Lógica padrão]\label{def:padrao}
        Uma lógica proposicional $\mathcal{L}$ definida sobre uma linguagem proposicional $\pazocal{L}$ é dita \textit{estrutural} caso respeite a seguinte condição para todo $\Gamma \cup \Delta \cup \{\alpha\} \subseteq \pazocal{L}$:
        \begin{align*}
            & \text{~(i) Se } \Gamma \Vdash \alpha \text{ então } \sigma [\Gamma] \Vdash \sigma(\alpha) \text{, para toda substituição } \sigma \text{ de variável por fórmula.}
        \end{align*}
        Uma lógica $\mathcal{L}$ é dita \textit{finitária} caso satisfaça a seguinte condição:
        \begin{align*}
            & \text{(ii) Se } \Gamma \Vdash \alpha \text{ então existe conjunto finito } \Gamma_{0} \subseteq \Gamma \text{ tal que } \Gamma_{0} \Vdash \alpha.
        \end{align*}
        Por fim, uma lógica proposicional $\mathcal{L}$ é dita \textit{padrão} caso ela seja Tarskiana, finitária e estrutural.\qed{}
    
    \end{definicao}

    Com isto, é possível definir formalmente o conceito de \textit{paraconsistência} para lógicas Tarskianas.

    \begin{definicao}[Lógica Tarskiana paraconsistente]\label{def:tarskiana_paracons}
        Uma lógica Tarskiana $\mathcal{L}$, definida sobre uma linguagem $\pazocal{L}$, é dita \textit{paraconsistente} se ela possuir uma negação\footnote{Esta negação pode ser primitiva (pertencente à assinatura da linguagem) ou definida a partir de outras fórmulas.} $\neg$ e existirem fórmulas $\alpha, \beta \in \pazocal{L}$ tal que $\alpha, \neg \alpha \nVdash \beta$.\qed{}
    \end{definicao}

    Caso a linguagem de $\mathcal{L}$ possua uma implicação $\to$ que respeite o metateorema da dedução\footnote{Definido como $\Gamma, \alpha \Vdash \beta \Longleftrightarrow  \Gamma\Vdash \alpha \to \beta$.}, então $\mathcal{L}$ é paraconsistente se e somente se a fórmula $\alpha \to (\neg \alpha \to \beta)$ não for válida. Ou seja, caso o princípio da explosão é inválido (em relação a $\neg$), o conectivo $\neg$ é considerado uma negação \textit{não explosiva}.

\section{Inconsistência Formal}\label{sec:incons}
    A motivação para o desenvolvimento das \lfis{} é a existência sistemas lógicos paraconsistentes nos quais é possível resgatar, de maneira \textit{controlada}, o princípio da explosão. Ao internalizar o conceito de consistência, as \lfis{} propõem a noção de que uma contradição que é reconhecidamente inconsistente numa dada teoria é inofensiva e é somente fruto do excesso de informação. O resgate \textit{controlado} da explosividade é feito definindo um conjunto $\bigcirc(p)$ de fórmulas dependentes somente de uma variável proposicional $p$. Caso uma lógica $\mathcal{L}$ seja explosiva ao unir-se um conjunto $\bigcirc(\alpha)$ com uma contradição $\{\alpha, \neg \alpha\}$ (ou seja, se  o seguinte for válido para todo $\alpha$ e $\beta$ pertencentes à sua linguagem: $\bigcirc(\alpha), \alpha, \neg \alpha \Vdash \beta$ e $\bigcirc(\alpha), \alpha \nVdash \beta$ e $\bigcirc(\alpha), \neg \alpha \nVdash \beta$) então dizemos que $\mathcal{L}$ é \textit{gentilmente explosiva}. Esta é uma das formas de definir as lógicas de inconsistência formal. Como será mostrado nesta seção, existem ao menos outras três definições diferentes para as \lfis{} e sua \textit{explosividade gentil}.

    \begin{notacao}
        Dado um átomo $p$, define-se $\bigcirc(p)$ como um conjunto não-vazio de fórmulas dependentes somente de $p$. Com base neste conjunto, define-se a notação $\bigcirc(\phi)$ para representar o conjunto obtido pela substituição de todas as ocorrências de $p$ por $\phi$ em todos os elementos de $\bigcirc(p)$, ou seja, para uma fórmula $\phi$ qualquer, $\bigcirc(\phi) = \{\psi\{p \mapsto \phi\} \; | \; \psi \in \bigcirc(p)\}$.
    \end{notacao}


    A definição a seguir foi proposta por~\citeshort{Carnielli_Coniglio_2016} e se encontra no meio de outras duas definições, uma mais generalista (apresentada na Definição~\ref{def:lfi_fraca}) e outra mais restrita (apresentada na Definição~\ref{def:lfi_forte}). Ela é utilizada para definir o que é informalmente descrito como \lfi{} no início desta seção.

    \begin{definicao}[Lógica de Inconsistência Formal]\label{def:lfi}
        Seja $\mathcal{L} = \langle \pazocal{L}_{\Theta}, \Vdash \rangle$ uma lógica padrão, de forma que sua assinatura proposicional $\Theta$ possua uma negação $\neg$. Seja $\bigcirc(p)$ um conjunto não-vazio de fórmulas dependentes somente da variável proposicional $p$. Então $\mathcal{L}$ será uma lógica de inconsistência formal (\lfi{}) (em relação a $\bigcirc(p)$ e $\neg$) caso respeite as seguintes condições:
        %(considerando $\bigcirc(\phi) = \{\psi(\phi) \; | \; \psi(p) \in \bigcirc(p)\}$): \helena{O conjunto $\bigcirc(\phi)$ é definido substituindo-se cada ocorrência $p$ por $\phi$ em cada elemento de $\bigcirc(\phi)$.}
        \begin{align*}
            \text{(i)} & \text{ Existem}~ \gamma, \delta \in \pazocal{L}_{\Theta} \text{, de modo que } \gamma, \neg \gamma \nVdash \delta;               \\
            \text{(ii)} & \text{ Existem}~ \alpha, \beta \in \pazocal{L}_{\Theta} \text{, de modo que:}                                                    \\
            & \qquad \text{(ii.a)}~ \bigcirc(\alpha), \alpha \nVdash \beta;                                                                         \\
            & \qquad \text{(ii.b)}~ \bigcirc(\alpha), \neg \alpha \nVdash \beta;                                                                    \\
            \text{(iii)} & \text{ Para todo}~ \phi, \psi \in \pazocal{L}_{\Theta} \text{, tem-se } \bigcirc(\phi), \phi, \neg \phi \Vdash \psi. \tag*\qed{}
        \end{align*}
    \end{definicao}

    A condição (i) diz que toda \lfi{} é \textit{não-explosiva} (em relação a $\neg$) e as condições (ii) e (iii) dizem que toda \lfi{} é \textit{gentilmente explosiva} (em relação a $\bigcirc(p)$ e $\neg$).

    Na literatura, existem outras duas definições para as lógicas de inconsistência formal, que relaxam a condição (ii) para obter uma definição mais uniforme.~\citeshort{Carnielli_Coniglio_2016}, definem \lfis{} \textit{fracas} da seguinte forma:


    \begin{definicao}[\lfi{} Fraca]\label{def:lfi_fraca}
        Seja $\mathcal{L} = \langle \pazocal{L}_{\Theta}, \Vdash \rangle$ uma lógica padrão, de forma que sua assinatura proposicional $\Theta$ possua uma negação $\neg$. Seja $\bigcirc(p)$ um conjunto não-vazio de fórmulas dependentes somente na variável proposicional $p$. Então $\mathcal{L}$ será uma \lfi{} \textit{fraca} (em relação a $\bigcirc(p)$ e $\neg$) caso ela respeite as seguintes condições:
        %(considerando $\bigcirc(\phi) = \{\psi(\phi) \; | \; \psi(p) \in \bigcirc(p)\}$): \helena{O conjunto $\bigcirc(\phi)$ é definido substituindo-se cada ocorrência $p$ por $\phi$ em cada elemento de $\bigcirc(\phi)$.}
        \begin{align*}
            \text{(i)} & \text{ Existem}~ \phi, \psi \in \pazocal{L}_{\Theta} \text{, de modo que } \phi, \neg \phi \nVdash \psi;\\
            \text{(ii)} & \text{ Existem}~ \phi, \psi \in \pazocal{L}_{\Theta} \text{, de modo que } \bigcirc(\phi), \phi \nVdash \psi;\\
            \text{(iii)} & \text{ Existem}~ \phi, \psi \in \pazocal{L}_{\Theta} \text{, de modo que } \bigcirc(\phi), \neg \phi \nVdash \psi;\\
            \text{(iv)} & \text{ Para todo}~ \phi, \psi \in \pazocal{L}_{\Theta} \text{, tem-se } \bigcirc(\phi), \phi, \neg \phi \Vdash \psi. \tag*\qed{}
        \end{align*}
    \end{definicao}

    Como é possível observar pelas duas definições acima, toda \lfi{} é uma \lfi{} fraca (já que a condição (ii) da Definição~\ref{def:lfi} satisfaz as condições (ii) e (iii) da Definição~\ref{def:lfi_fraca}), mas o inverso não é necessariamente verdade (observe que as váriaveis $\phi$ e $\psi$ das condições (ii) e (iii) não precisam ser iguais). Ademais, é possível estabelecer outra definição (também mais uniforme do que a Definição~\ref{def:lfi}) que introduz o conceito de \lfis{} \textit{fortes} como feito abaixo:

    \begin{definicao}[\lfi{} Forte]\label{def:lfi_forte}
        Seja $\mathcal{L} = \langle \pazocal{L}_{\Theta}, \Vdash \rangle$ uma lógica padrão, de forma que sua assinatura proposicional $\Theta$ possua uma negação $\neg$. Seja $\bigcirc(p)$ um conjunto não-vazio de fórmulas dependentes somente na variável proposicional $p$. Então $\mathcal{L}$ será uma \lfi{} \textit{forte} (em relação a $\bigcirc(p)$ e $\neg$) caso ela respeite as seguintes condições:
        %(considerando $\bigcirc(\phi) = \{\psi(\phi) \; | \; \psi(p) \in \bigcirc(p)\}$): \helena{O conjunto $\bigcirc(\phi)$ é definido substituindo-se cada ocorrência $p$ por $\phi$ em cada elemento de $\bigcirc(\phi)$.}
        \begin{align*}
            \text{(i)} & \text{ Existem}~ \alpha, \beta \in \pazocal{L}_{\Theta} \text{, de modo que:}\\
            & \qquad \text{(i.a)}~ \alpha, \neg \alpha \nVdash \beta;\\
            & \qquad \text{(i.b)}~ \bigcirc(\alpha), \alpha \nVdash \beta;\\
            & \qquad \text{(i.c)}~ \bigcirc(\alpha), \neg \alpha \nVdash \beta;\\
            \text{(ii)} & \text{ Para todo}~ \phi, \psi \in \pazocal{L}_{\Theta} \text{, tem-se } \bigcirc(\phi), \phi, \neg \phi \Vdash \psi. \tag*\qed{}
        \end{align*}
    \end{definicao}

    É imediato perceber que toda \lfi{} forte é uma \lfi{} (já que a condição (i) da Definição~\ref{def:lfi_forte} satisfaz as condições (i) e (ii) da Definição~\ref{def:lfi}), mas o inverso não é necessariamente verdade. Ademais, no escopo das lógicas proposicionais, é possível estabelecer uma forma mais simples de provar que uma dada lógica proposicional é uma \textbf{LFI} forte, tomando $\alpha$ e $\beta$ como dois átomos $p$ e $q$ quaisquer nas condições (i.a), (i.b) e (i.c) da definição acima.

    \begin{definicao}[\lfi{} forte para lógicas proposicionais]\label{def:lfi_forte_prop}
        Seja $\mathcal{L} = \langle \pazocal{L}_{\Theta}, \Vdash \rangle$ uma lógica padrão, de forma que sua assinatura proposicional $\Theta$ possua uma negação $\neg$ e sua linguagem $\pazocal{L}_{\Theta}$ seja definida sobre um conjunto enumerável de átomos $\pazocal{P} = \{p_{0},\ldots, p_{n}\}$. Seja $\bigcirc(p)$ um conjunto não-vazio de fórmulas dependentes somente na variável proposicional $p$. Então $\mathcal{L}$ será uma \lfi{} \textit{forte} (em relação a $\bigcirc(p)$ e $\neg$) caso ela respeite as seguintes condições:
        %(considerando $\bigcirc(\phi) = \{\psi(\phi) \; | \; \psi(p) \in \bigcirc(p)\}$): \helena{O conjunto $\bigcirc(\phi)$ é definido substituindo-se cada ocorrência $p$ por $\phi$ em cada elemento de $\bigcirc(\phi)$.}
        \begin{align*}
            & \text{~~(i) Existem } p, q \in \pazocal{P} \text{, de modo que:}\\
            & \qquad \text{(i.a) } p, \neg p \nVdash q;\\
            & \qquad \text{(i.b) } \bigcirc(p), p \nVdash q;\\
            & \qquad \text{(i.c) } \bigcirc(p), \neg p \nVdash q;\\
            & \text{~(ii) Para todo } \phi, \psi \in \pazocal{L}_{\Theta} \text{, tem-se } \bigcirc(\phi), \phi, \neg \phi \Vdash \psi. \tag*\qed{}
        \end{align*}
    \end{definicao}

    Esta definição será utilizada para provar que a lógica proposicional \lfium{} se trata de uma lógica de inconsistência formal forte. 