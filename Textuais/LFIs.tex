\chapter{Lógicas de Inconsistência Formal}
\label{cap:LFIs}
Neste capítulo são apresentadas algumas definições necessárias para caracterizar as lógicas de inconsistência formal, baseadas no trabalho de~\citeshort{Carnielli_Coniglio_2016}. Antes de definir as \textbf{LFI}s, entretanto, é preciso apresentar alguns conceitos acerca de sistemas lógicos paraconsistentes.

\section{Paraconsistência}
\label{sec:paracons}
Uma lógica $\mathcal{L}$ é definida como uma dupla $\mathcal{L} = \langle \pazocal{L},\vdash \rangle$, onde $\pazocal{L}$ é sua linguagem (seu conjunto de fórmulas) e $\vdash$ é uma relação de consequência de conclusão única, definida como $\vdash \;\subseteq \wp(\pazocal{L})\times\pazocal{L}$.

\begin{definicao}[Assinatura proposicional]
    \label{def:ass_prop}
    Uma assinatura proposicional $\Theta$ é um conjunto de conectivos lógicos com a informação acerca da aridade de cada um destes.
\end{definicao}
    Por exemplo, a assinatura proposicional para a lógica proposicional clássica pode ser definida como $\Theta_{LPC} = \{\land^{2}, \lor^{2}, \neg^{1}, \rightarrow^{2}\}$.

\begin{definicao}[Lógica proposicional]
    \label{def:proposicional}
    Um sistema lógico $\mathcal{L}$, definido sobre uma linguagem $\pazocal{L}$ é dito proposicional caso $\pazocal{L}$ seja definida a partir de um conjunto enumerável de átomos $\pazocal{P} = \{p_{i} \;| \; i \in \omega \}$ e uma assinatura proposicional $\Theta$. Uma linguagem $\pazocal{L}$ definida sobre uma assinatura proposicional é chamada de linguagem proposicional.\qed{}
\end{definicao}

\begin{definicao}[Lógica padrão]
    \label{def:padrao}
    Uma lógica $\mathcal{L}$, definida sobre uma linguagem $\pazocal{L}$ é dita \textit{Tarskiana} caso satisfaça as seguintes propriedades para todo $\Gamma \cup \Delta \cup \{\alpha\} \subseteq \pazocal{L}$:
    \begin{align*}
         & \text{~~(i) Se } \gamma \in \Gamma \text{ então } \Gamma \vdash \gamma;                                                                       \\
         & \text{~(ii) Se } \Delta \vdash \alpha \text{ e } \Delta \subseteq \Gamma \text{ então } \Gamma \vdash \alpha;                                 \\
         & \text{(iii) Se } \Delta \vdash \alpha \text{ e } \Gamma \vdash \delta \text{ para todo } \delta \in \Delta \text{ então } \Gamma \vdash \alpha.
    \end{align*}
    Uma lógica $\mathcal{L}$ é dita \textit{finitária} caso satisfaça o seguinte:
    \begin{align*}
         & \text{~(iv) Se } \Gamma \vdash \alpha \text{ então existe conjunto finito } \Gamma_{0} \subseteq \Gamma \text{ tal que } \Gamma_{0} \vdash \alpha.
    \end{align*}
    Uma lógica $\mathcal{L}$ definida sobre uma linguagem proposicional $\pazocal{L}$ é dita \textit{estrutural} caso respeite a seguinte condição:
    \begin{align*}
         & \text{~~(v) Se } \Gamma \vdash \alpha \text{ então } \sigma [\Gamma] \vdash \sigma(\alpha) \text{, para toda substituição } \sigma \text{ de fórmulas por variáveis.}\\
        & \text{\helena{Ou seriam variáveis por fórmulas????}}
    \end{align*}
    Por fim, uma lógica $\mathcal{L}$ é dita \textit{padrão} caso ela seja Tarskiana, finitária e estrutural.\qed
\end{definicao}
Com isto, é possível definir formalmente a \textit{paraconsistência} para lógicas Tarskianas.

\begin{definicao}[Lógica Tarskiana paraconsistente]
    \label{def:tarskiana_paracons}
    Uma lógica Tarskiana $\mathcal{L}$ é dita \textit{paraconsistente} se ela possuir uma negação $\neg$\footnote{Esta negação pode ser primitiva (pertencente à assinatura da linguagem) ou definida a partir de outros conectivos.} tal que $\alpha, \neg \alpha \nvdash \beta$ para \cortar{algumas} fórmulas $\alpha$ e $\beta$ da sua linguagem.\qed{}
\end{definicao}

Caso a linguagem de $\mathcal{L}$ possua uma implicação $\rightarrow$ que respeite o metateorema da dedução\footnote{Definido como $\Gamma, \alpha \vdash \beta \Longleftrightarrow \Gamma\vdash \alpha \rightarrow \beta$.}, então $\mathcal{L}$ é paraconsistente somente se a fórmula $\alpha \rightarrow (\neg \alpha \rightarrow \beta)$ não for válida. Ou seja, o Princípio da explosão, em relação a negação $\neg$, é inválido e, portanto, $\neg$ é uma negação não-explosiva.

\section{Inconsistência}
A motivação para o desenvolvimento das \textbf{LFI}s é possuir sistemas lógicos paraconsistentes nos quais é possível resgatar, de maneira \textit{controlada}, o Princípio da Explosão. Isto é feito definindo um conjunto $\bigcirc(p)$ de fórmulas dependentes somente de uma variável proposicional $p$. Caso uma lógica $\mathcal{L}$ seja explosiva ao unir-se um conjunto $\bigcirc(\alpha)$ com uma contradição $\{\alpha, \neg \alpha\}$, ou seja, se $\bigcirc(\alpha), \alpha, \neg \alpha \vdash \beta$ para todo $\alpha$ e $\beta$ pertencentes à sua linguagem, e existirem fórmulas $\phi$ e $\psi$ tal que $\bigcirc(\phi), \phi \nvdash \psi$ e $\bigcirc(\phi), \neg \phi \nvdash \psi$, então dizemos que $\mathcal{L}$ é \textit{gentilmente explosiva}. 

\begin{definicao}[Lógica de Inconsistência Formal]
    \label{def:lfi}
    Seja $\mathcal{L} = \langle \pazocal{L}_{\Sigma}, \vdash \rangle$ uma lógica padrão, de forma que sua assinatura proposicional $\Sigma$ possua uma negação $\neg$. Seja $\bigcirc(p)$ um conjunto não-vazio de fórmulas dependentes somente na variável proposicional $p$. Então $\mathcal{L}$ será uma \textit{Lógica de Inconsistência Formal} (\textbf{LFI}) (em relação a $\bigcirc(p)$ e $\neg$) caso ela respeite as seguintes condições (considerando $\bigcirc(\varphi) = \{\psi(\varphi) \; | \; \psi(p) \in \bigcirc(p)\}$) \helena{Parando pra pensar não entendi isso direito ainda.}



\end{definicao}


