
% ----------------------------------------------------------
% Apêndices
% ----------------------------------------------------------

% ---
% Inicia os apêndices
% ---
\begin{apendicesenv}

    \chapter{Prova Formal da Transferência de Completude pela Fusão de Lógicas Modais}
        \label{app:ProvaTransferenciaCompletude}

        \begin{proof}[Prova do Teorema~\ref{teo:TransCompletude}]
            Sendo \(\Gamma \subseteq \Theta\) um conjunto \Mathcali{L}{12}-consistente, devemos provar que \GAMMA é verdadeiro em algum mundo de um 12-modelo baseado
            em um 12-frame para \Mathcali{L}{12}. Como \GAMMA é \Mathcali{L}{12}-consistente, \GAMMA também é \Mathcali{L}{1}-consistente e \Mathcali{L}{2}-consistente. Ademais, como
            \(\Gamma \subseteq \Theta \subseteq \funcao{DC}_{\pi}(\Theta)\), existe um \PImodelo, baseado em um \PIframe para \Mathcali{L}{\pi}, onde \GAMMA é verdadeiro em
            algum mundo (devido à premissa de completude de \Mathcali{L}{1} e \Mathcali{L}{2}). Este \PImodelo será chamado de \textit{modelo inicial}, e será apresentado
            formalmente à frente.

            Inicialmente, devemos definir formalmente os conceitos de \PI-teorias, modelos etiquetados e elementos de modelos.
            \begin{definicao}[\PI-Teorias, \PIMODELOS Etiquetados e \PI-Elementos de Modelos]
                \label{def:Definicao1}
                \phantom{a}
                \begin{enumerate}[label=\textnormal{\ref{def:Definicao1}.\arabic*}]
                    \item Sendo \DDELTA um conjunto de fórmulas, a \textit{\PI-teoria de \DDELTA} é definida como, onde \(\mathcal{L}_{12}\) é uma lógica bimodal:\label{caso:Definicao1-1}
                    \[
                        \funcao{T}_{\pi}(\Delta) = \{\Box^{n}_{\pi} \delta \ | \ \delta \in \funcao{B}(\funcao{S}_{\pi}(\Delta)) \cap \mathcal{L}_{12} \text{ e }
                            \funcao{d}_\pi(\Box^{n}_{\pi} \delta) \leq \funcao{d}_\pi(\Delta) \}
                    \]

                    \item Um \textit{\PImodelo etiquetado} é uma tripla \(\langle \mathcal{M}, \mundobase, \Sigma(\mathcal{M}) \rangle\), onde \(\mathcal{M} = \langle \mathcal{W},
                    \mathcal{R}_{\pi}, \mathcal{V}_{\pi} \rangle\) e:
                    \begin{enumerate}[label=(\roman*)] \label{caso:Definicao1-2}
                        \item \Mathcal{M} é um \PImodelo para \(\mathcal{L}_{\pi}\);
                        \item \(\mundobase \in \mathcal{W}\);
                        \item \(\Sigma(\mathcal{M})\) é um conjunto de elementos (fórmulas interpretadas como átomos), fechado para a operação de sub-elementos sobre
                            elementos; \label{caso:Definicao1-2-3}
                        \item \(\funcao{T}_{\pi}(\Sigma(\mathcal{M}))\) é verdadeiro em \Mundobase em \Mathcal{M}. \label{caso:Definicao1-2-4}
                    \end{enumerate}

                    \item Dado um \PImodelo etiquetado \(\mathcal{E} = \langle \mathcal{M}, \mundobase, \Sigma(\mathcal{M}) \rangle\) onde
                    \(\mathcal{M} = \langle \mathcal{W}, \mathcal{R}_{\pi}, \mathcal{V}_{\pi} \rangle\), para cada mundo \(w \in \mathcal{W}\) \textit{o conjunto de elementos
                    \(\Sigma_{\mathcal{M}}(w)\) de w} é definido indutivamente como:
                        \begin{enumerate}[label=(\roman*)] \label{caso:Definicao1-3}
                            \item \(\Sigma_{\mathcal{M}}(\mundobase) = \Sigma(\mathcal{M})\); \label{caso:Definicao1-3-1}
                            \item Sendo \(w_i, w_j \in \mathcal{W}\), se \(w_i \mathcal{R}_{\pi} w_j\) e \(\Box_\pi \phi \in \Sigma_{\mathcal{M}}(w_i)\),
                                    então \(\funcao{SC}(\phi) \in \Sigma_{\mathcal{M}}(w_j)\). \label{caso:Definicao1-3-2} \qed
                        \end{enumerate}
                \end{enumerate}
            \end{definicao}

            O mundo \Mundobase é chamado de mundo base do modelo \Mathcal{M}, \(\Sigma(\mathcal{M})\) é chamado de conjunto de elementos do modelo \Mathcal{M} e
            \(\Sigma_{\mathcal{M}}(w)\) é o conjunto de elementos de \textit{w} em \Mathcal{M}.
            Sendo \(\mathcal{E} = \langle \mathcal{M}, \mundobase, \Sigma(\mathcal{M}) \rangle\) um modelo etiquetado, escreveremos \(w \in \mathcal{E}\)
            para indicar que \(w \in \mathcal{W}\), sendo \(\mathcal{M} = \langle \mathcal{W}, \mathcal{R}, \mathcal{V} \rangle\).
            % No restante da prova, usaremos o termo ``(\(\pi\)-) modelo'' para referir-se a modelos etiquetados, a não ser quando explicitado ao contrário.

            Com isso, podemos definir formalmente os conceitos de modelo e mundo iniciais, a partir de algum conjunto de fórmulas \GAMMA que deve ser satisfeito.

            \begin{definicao}[Modelo Inicial e Mundo Inicial]
                É chamado de \textit{modelo inicial} o \PImodelo etiquetado:
                \[
                    \modeloinicial = \langle \mathcal{I}, \mundoinicial, \funcao{SC}(\Gamma) \rangle
                \]
                onde o mundo base \Mundoinicial deste modelo é chamado de \textit{mundo inicial}. O modelo \(\mathcal{I}\) é o modelo que satisfaz
                \GAMMA e o mundo \Mundoinicial é um mundo onde \GAMMA é verdadeiro.
            \end{definicao}

            Podemos então provar o seguinte resultado auxiliar:

            \begin{lema}
                \label{teo:Lema1}
                Sendo \(\mathcal{E} = \langle \mathcal{M}, \mundobase, \Sigma(\mathcal{M}) \rangle\) um modelo etiquetado e \(w_i \in \mathcal{E}\), então:
                \begin{enumerate}[label=\textnormal{\ref{teo:Lema1}.\arabic*}]
                    \item \textnormal{Se \(\Sigma_{\mathcal{M}}(w_i) \neq \emptyset\) então \(\funcao{d}_{\pi}(\Sigma_{\mathcal{M}}(w_i)) = (\funcao{d}_{\pi}(\Sigma(\mathcal{M})) - \funcao{dist}(\mundobase, w_i))\);}\label{caso:Lema1-1}
                    \item \textnormal{Sendo \(w_i \neq \mundobase\), \(\Sigma_{\mathcal{M}}(w_i) = \emptyset\) sse \(\funcao{dist}(\mundobase, w_i) > \funcao{d}_{\pi}(\Sigma(\mathcal{M}))\).}\label{caso:Lema1-2} \qed
                \end{enumerate}
            \end{lema}

            \begin{proof}[Prova do Lema~\ref{teo:Lema1}]
                Temos dois casos, um para~\ref{caso:Lema1-1} outro para~\ref{caso:Lema1-2}:
                \begin{description}
                    \item[Caso~\ref{caso:Lema1-1}] Prova-se por indução em \(\funcao{dist}(\mundobase, w_j)\):
                        \begin{description}
                            \item[Base:] Pela definição de \(\funcao{dist}\), temos que \(\mundobase = w_j\), logo, \(\funcao{d}_{\pi}(\Sigma_{\mathcal{M}}(\mundobase)) = \funcao{d}_{\pi}(\Sigma(\mathcal{M}))\),
                            pela definição de \(\Sigma_{\mathcal{M}}\), temos \(\funcao{d}_{\pi}(\Sigma(\mathcal{M})) = \funcao{d}_{\pi}(\Sigma(\mathcal{M}))\).

                            \item[Hipotese:] Assumindo \(\forall w_k, \funcao{dist}(\mundobase, w_k) = k\),
                                temos que \(\funcao{d}_{\pi}(\Sigma_{\mathcal{M}}(w_k)) = \funcao{d}_{\pi}(\Sigma(\mathcal{M})) - \funcao{dist}(\mundobase, w_k)\).

                            \item[Passo:] Pelas definições de \(\Sigma_{\mathcal{M}}\) e \(\funcao{d}_{\pi}\), temos que \(\forall w_a, w_b,\) onde \(w_a \mathcal{R} w_b,
                                \funcao{d}_{\pi}(\Sigma_{\mathcal{M}}(w_a)) = \funcao{d}_{\pi}(\Sigma_{\mathcal{M}}(w_b)) + 1\).
                                Assumindo que \(\funcao{dist}(\mundobase, w_j) = k+1\), sabemos que existe algum \(w_x\) onde \(w_x \mathcal{R} w_j\) e \(\funcao{dist}(\mundobase, w_x) = k\).
                                Logo, pela hipótese podemos concluir que:
                                \[
                                    \funcao{d}_{\pi}(\Sigma(w_x)) = \funcao{d}_{\pi}(\Sigma(\mathcal{M})) - \funcao{dist}(\mundobase, w_x)
                                \]
                                Mais ainda, temos:
                                \begin{itemize}
                                    \item Pela definição de \(\Sigma_{\mathcal{M}}\): \(\Sigma(\mathcal{M}) = \Sigma_{\mathcal{M}}(\mundobase)\)
                                    \item Pelas conclusões anteriores: \(\funcao{d}_{\pi}(\Sigma_{\mathcal{M}}(w_x)) = \funcao{d}_{\pi}(\Sigma_{\mathcal{M}}(w_j)) + 1\)
                                \end{itemize}
                                Portanto temos:
                                \[
                                    \funcao{d}_{\pi}(\Sigma(w_j))+1 = \funcao{d}_{\pi}(\Sigma_{\mathcal{M}}(\mundobase)) - \funcao{dist}(\mundobase, w_x)
                                \]
                                Ou seja:
                                \[
                                    \funcao{d}_{\pi}(\Sigma(w_j)) = \funcao{d}_{\pi}(\Sigma_{\mathcal{M}}(\mundobase)) - (\funcao{dist}(\mundobase, w_x) + 1)
                                \]
                                Logo:
                                \[
                                    \funcao{d}_{\pi}(\Sigma(w_j)) = \funcao{d}_{\pi}(\Sigma_{\mathcal{M}}(\mundobase)) - \funcao{dist}(\mundobase, w_j)
                                \]
                        \end{description}

                    \item[Caso~\ref{caso:Lema1-2}] Consequência imediata do caso anterior e da definição de \(\Sigma_{\mathcal{M}}\). \qedhere
                \end{description}
            \end{proof}
            Este lema demonstra que, dado \(\funcao{d}_{\pi}(\Sigma(\mathcal{M}))\) finito, \(\funcao{d}_{\pi}(\Sigma_{\mathcal{M}}(w_j))\) irá diminuir com o aumento de \(\funcao{dist}(\mundobase, w_j)\) até
            \(\Sigma_{\mathcal{M}}(w_j)\) se tornar vazio, ou seja, o conjunto de elementos de um dado mundo em um modelo depende da distância do mundo até o mundo base do modelo.

            Com isso, podemos formalmente enunciar o conceito de ancoramento de modelos:

            \begin{definicao}[Ancoramento de Modelos]
                \label{def:Definicao2}
                Sendo \(\mathcal{E}_{0} = \langle \mathcal{M}, w_{\mathcal{M}}, \Sigma(\mathcal{M}) \rangle\) e \(\mathcal{E}_{1} = \langle \mathcal{N}, w_{\mathcal{N}}, \Sigma(\mathcal{N}) \rangle\)
                modelos etiquetados, onde \Mathcali{E}{0} tem tipo \PI, é dito que \textit{\Mathcali{E}{1} está ancorado em \Mathcali{E}{0}} se, e somente se:

                \begin{enumerate}[label=\textnormal{\ref{def:Definicao2}.\arabic*}]
                    \item \Mathcali{E}{1} tem tipo \OPI; \label{caso:Definicao2-1}

                    \item Sendo \(\mathcal{M} = \langle \mathcal{W}_{m}, \mathcal{R}_{m}, \mathcal{V}_{m} \rangle\) e
                    \(\mathcal{N} = \langle \mathcal{W}_{n}, \mathcal{R}_{n}, \mathcal{V}_{n} \rangle\) então,
                    \(\mathcal{W}_{m} \cap \mathcal{W}_{n} = \{w_{\mathcal{N}}\}\) e \(\mathcal{E}_{0} = \modeloinicial\) ou \(w_{\mathcal{N}}\)
                    não é o mundo base de \Mathcali{E}{0}; \label{caso:Definicao2-2}

                    \item \(\Sigma(\mathcal{N}) = \funcao{S}_{\pi}(\Sigma_{\mathcal{M}}(w_{\mathcal{N}}))\); \label{caso:Definicao2-3}

                    \item Para todo \(\phi \in \Sigma(\mathcal{N}): \mathcal{M}, w_{\mathcal{N}} \Vdash \phi\) sse \(\mathcal{N}, w_{\mathcal{N}} \Vdash \phi\) \label{caso:Definicao2-4} \qed
                \end{enumerate}
            \end{definicao}

            É fácil de observar que a definição de ancoramento é irreflexiva e antissimétrica, mais ainda, se \Mathcali{E}{1} está ancorado em \Mathcali{E}{0} e \textit{w} é o mundo base de
            \Mathcali{E}{1}, é dito que \textit{\Mathcali{E}{1} está ancorando em \Mathcali{E}{0} no mundo w}. Para cada \PImodelo \Mathcali{E}{0} e mundo \(w \in \mathcal{E}_{0}\),
            o conjunto \(\funcao{S}_{\pi}(\Sigma_{\mathcal{M}}(w))\) é chamado de conjunto de concordância de \textit{w} em \Mathcali{E}{0} pois, para cada \OPImodelo \Mathcali{E}{1} ancorado
            em \textit{w}, \Mathcali{E}{0} e \Mathcali{E}{1} devem concordar com a valoração deste conjunto em \textit{w}.

            Intuitivamente, podemos interpretar a operação de ancoramento de modelos como uma forma de ``alternar o contexto'' durante a valoração de uma fórmula que contenha múltiplas
            modalidades, pois permite que fórmulas com modalidades de um tipo sejam valoradas em um modelos de outro tipo.

            \begin{definicao}[Diagramas]
                \label{def:Diagramas}
                Para cada conjunto \Mathcali{L}{\pi}-consistente \DDELTA, \PImodelo \Mathcal{M} e mundo \textit{w} em \Mathcal{M}, o \textit{diagrama de \DDELTA em \Mathcal{M}
                no mundo w}, denotado por \(\funcao{DG}(\Delta)\) é dado por
                \[
                    \funcao{DG}(\Delta) = \{\delta \ | \ \delta \in \Delta \text{ e } \mathcal{M}, w \Vdash \delta\} \cup \{\neg \delta \ | \ \delta \in \Delta \text{ e } \mathcal{M}, w \nVdash \delta\}
                \]
                O \textit{diagrama de concordância \(\funcao{D}_{\mathcal{M}}(w)\) do mundo w em \Mathcal{M}} é definido como
                \(\funcao{D}_{\mathcal{M}}(w) = \funcao{DG}(\funcao{S}_{\pi}(\Sigma_{\mathcal{M}}(w)))\). \qed
            \end{definicao}

            É fácil observar que a condição~\ref{caso:Definicao2-4} é equivalente à impor a restrição que \(\funcao{D}_{\mathcal{M}}(w_{\mathcal{N}})\) seja verdadeiro
            no mundo \(w_{\mathcal{N}}\) no modelo \Mathcal{N}. Uma importante propriedade de \PImodelos etiquetados é a seguinte:

            \begin{lema}
                \label{teo:Lema2}
                Para todo \PImodelo etiquetado \Mathcal{E} e \(w \in \mathcal{E}\), \(\funcao{D}_{\mathcal{M}}(w)\) é \Mathcali{L}{12}-consistente. \qed
            \end{lema}

            \begin{proof}[Prova do Lema~\ref{teo:Lema2}]
                Inicialmente, devemos assumir que \(\funcao{D}_{\mathcal{M}}(w)\) é \Mathcali{L}{12}-inconsistente,
                para algum \(w \in \mathcal{E}\). Então, existe um conjunto finito e não vazio \(\Delta \subseteq \funcao{D}_{\mathcal{M}}(w)\) onde
                \(\neg \bigwedge \Delta \in \mathcal{L}_{12}\)\footnote{Onde \(\bigwedge \Delta = (\delta_1 \land \dots \land \delta_n)\).}, ou seja,
                existe algum \(\delta_i\) tal que \(\delta_i \in \Delta \text{ e } \neg\delta_i \in \mathcal{L}_{12}\).

                Assumindo que \(\funcao{dist}(w_{\mathcal{M}}, w) = k\). Sabemos que, pela definição de \(\funcao{D}_{\mathcal{M}}\), todo \(\delta \in \Delta\)
                é da forma \PHI ou \(\neg \phi\), sendo \(\phi \in \funcao{S}_{\pi}(\Sigma_{\mathcal{M}}(w))\).
                Portanto \(\phi \in \funcao{S}_{\pi}(\Sigma(\mathcal{M}))\) e \(\Delta \subseteq \funcao{B}(\funcao{S}_{\pi}(\Sigma(\mathcal{M})))\),
                mais ainda \(\funcao{d}_{\pi}(\phi) \leq \funcao{d}_{\pi}(\Sigma(\mathcal{M})) - k\) e, portanto, \(\funcao{d}_{\pi}(\Delta) \leq \funcao{d}_{\pi}(\Sigma(\mathcal{M})) - k\)
                (isso se dá pelo Lema~\ref{caso:Lema1-1}, no caso \(\Sigma_{\mathcal{M}} \neq \emptyset \text{, pois } \Delta \neq \emptyset\)).

                Temos então que \(\Box^{k}_{\pi} \neg \bigwedge \Delta \in \funcao{T}_{\pi}(\Sigma(\mathcal{M}))\) (pela Definição~\ref{caso:Definicao1-1}),
                onde \(\Box^{k}_{\pi} \neg \bigwedge \Delta\) é verdadeiro no mundo \Mundobase em \Mathcal{E}
                (pela Definição~\SubCaso{caso:Definicao1-2}{caso:Definicao1-2-4}). Isso implica que \(\neg \bigwedge \Delta\) é verdadeiro no mundo \textit{w}
                em \Mathcal{E}, contradizendo o fato que \(\bigwedge \Delta\) é também verdadeiro em \Mathcal{E} (pela Definição~\ref{def:Diagramas}).

                Portanto, \(\funcao{D}_{\mathcal{M}}(w)\) não pode ser \Mathcali{L}{12}-inconsistente.
            \end{proof}

            \begin{definicao}[Ancoramento Indireto]
                \label{def:AncoramenteoInidireto}
                Sendo \(\mathbf{E}\) um conjunto de \PImodelos e \OPImodelos etiquetados e sendo \(\mathcal{E}_{0}, \mathcal{E}_{1} \in \mathbf{E}\). A relação de ancoramento indireto,
                dita ``\textit{\Mathcali{E}{1} está indiretamente ancorado em \Mathcali{E}{0}}'', é definida indutivamente como:
                \begin{enumerate}[label=\textnormal{\ref{def:AncoramenteoInidireto}.\arabic*}]
                    \item Se \Mathcali{E}{1} está ancorado em \Mathcali{E}{0}, então \Mathcali{E}{1} está indiretamente ancorado em \Mathcali{E}{0};

                    \item Se, para algum \(\mathcal{E}_{2} \in \mathbf{E}\), onde \Mathcali{E}{2} está ancorado em \Mathcali{E}{0} e \Mathcali{E}{1} está indiretamente ancorado
                        em \Mathcali{E}{2}, então \Mathcali{E}{1} está indiretamente ancorado em \Mathcali{E}{0}. \qed
                \end{enumerate}
            \end{definicao}

            Com essa definição, podemos apresentar um dos conceitos mais importantes para a prova:

            \begin{definicao}[Brotos]
                \label{def:Definicao3}
                Um \textit{broto de \Modeloinicial} é qualquer conjunto \textbf{B} de \PImodelos e \OPImodelos etiquetados tal que:
                \begin{enumerate}[label=\textnormal{\ref{def:Definicao3}.\arabic*}]
                    \item \(\modeloinicial \in \mathbf{B}\) e \Modeloinicial não está ancorado em nada em \textbf{B}; \label{caso:Definicao3-1}

                    \item Para todo \PImodelo etiquetado \(\mathcal{E} \in \mathbf{B}, \text{ se } \mathcal{E} \neq \modeloinicial\) então \(\mathcal{E}\)
                        está indiretamente ancorado em \Modeloinicial; \label{caso:Definicao3-2}

                    \item Para cada \(\mathcal{E}_{0}, \mathcal{E}_{1} \in \mathbf{B}\), onde \(\mathcal{E}_{0} \neq \mathcal{E}_{1}\), \Mathcali{E}{0} e \Mathcali{E}{1} só compartilham
                        um mundo se um está ancorado no outro. \label{caso:Definicao3-3} \qed
                \end{enumerate}
            \end{definicao}

            Um broto pode ser entendido como um conjunto de modelos de tipos alternados, ancorados dois a dois, que inicia em \Modeloinicial e cujo intuito é
            permitir a valoração de fórmulas com múltiplas modalidades distintas. Podemos então demonstrar algumas propriedades sobre brotos.

            \begin{lema}
                \label{teo:Lema3}
                Sendo \textnormal{\textbf{B}} um broto de \Modeloinicial, então:
                \begin{enumerate}[label=\textnormal{\ref{teo:Lema3}.\arabic*}]
                    \item \textnormal{Para todo \(\mathcal{E}_{0} \in \mathbf{B}\) e \(w \in \mathcal{E}_{0}\), no máximo um \(\mathcal{E}_{1} \in \mathbf{B}\) está
                        ancorado em \Mathcal{E} no mundo \textit{w} \label{caso:Lema3-1};}

                    \item \textnormal{Para todo \(\mathcal{E}_{0} \in \mathbf{B}, \text{ se } \mathcal{E}_{0} \neq \modeloinicial\) então, \Mathcali{E}{0} está ancorado em exatamente um
                    \(\mathcal{E}_{1} \in \mathbf{B}\); \label{caso:Lema3-2}}

                    \item \textnormal{Todo \PImodelo etiquetado em \textbf{B} tem conjunto de mundos
                        disjuntos\footnote{Ou seja, apenas modelos etiquetados de tipos diferentes compartilham mundos.};}\label{caso:Lema3-3}

                    \item \textnormal{É chamada de \textit{cadeia de ancoramentos em \Mathcal{E} de comprimento n} a sequência finita \(\langle \mathcal{E}_{i} \ | \ i \leq n, n \geq 1 \rangle\)
                    de modelos etiquetados em \textbf{B} iniciando com \Modeloinicial e terminando com \Mathcal{E} onde, para cada \(1 < i \leq n\), \Mathcali{E}{i} está
                    ancorado em \Mathcali{E}{i-1}. Para todo \(\mathcal{E} \in \mathbf{B}\), existe exatamente uma cadeia de ancoramentos em \Mathcal{E} onde os elementos são
                    disjuntos dois a dois;}\label{caso:Lema3-4}

                    \item \textnormal{Se \(\mathcal{E}_{1} \in \mathbf{B}\) está ancorado em \(\mathcal{E}_{0} \in \mathbf{B}\) em \textit{w} e \(w \neq \mundoinicial\),
                    então \textit{w} não é o mundo base de \Mathcali{E}{0};}\label{caso:Lema3-5}

                    \item \textnormal{Para todo \PImodelo etiquetado \(\mathcal{E}_{0} = \langle \mathcal{M}, \mundobase, \Sigma(\mathcal{M}) \rangle \in \mathbf{B}\) e \(w \in \mathcal{E}_{0}\):}\label{caso:Lema3-6}
                    \begin{enumerate}[label=\textnormal{(\roman*)}]
                        \item \textnormal{Se \Mathcali{E}{0} está ancorado em \(\mathcal{E}_{1} = \langle \mathcal{N}, w_{\mathcal{N}}, \Sigma(\mathcal{N}) \rangle\)
                            então \(\Sigma(\mathcal{M}) \subseteq \Sigma(\mathcal{N})\);} \label{caso:Lema3-6-1}

                        \item \textnormal{\(\Sigma(\mathcal{M}) \subseteq \Sigma(\mathcal{\modeloinicial}) \subseteq \Theta\);} \label{caso:Lema3-6-2}

                        \item \textnormal{\(\funcao{D}_{\mathcal{M}}(w) \subseteq \Theta\);} \label{caso:Lema3-6-3}

                        \item \textnormal{\(\funcao{T}_{\pi}(\Sigma(\mathcal{M})) \subseteq \funcao{DC}_{\pi}(\Theta)\).} \label{caso:Lema3-6-4} \qed
                    \end{enumerate}
                \end{enumerate}
            \end{lema}

            \begin{proof}[Prova do Lema~\ref{teo:Lema3}]
                \phantom{a}
                \begin{description}
                    \item[Caso~\ref{caso:Lema3-1}] Pela Definição~\ref{caso:Definicao2-1}, sabemos que só podemos ancorar um modelo etiquetado de tipo \PI/\OPI em
                    um modelo etiquetado de tipo \OPI/\PI, pela Definição~\ref{caso:Definicao2-2} sabemos que estes dois modelos etiquetados compartilham apenas um mundo e,
                    pela Definição~\ref{caso:Definicao3-3} sabemos que dois modelos etiquetados distinto compartilham mundos apenas quando um está ancorado no outro.
                    Logo, concluímos que, para um mundo \(w \in \mathcal{E}_{0}\), há no máximo um modelo etiquetado \Mathcali{E}{1} em \textbf{B} ancorado neste mundo;

                    \item[Caso~\ref{caso:Lema3-2}] Pela Definição~\ref{caso:Definicao3-2} sabemos que todo \PImodelo etiquetado diferente de \Modeloinicial está indiretamente
                    ancorado em \(\modeloinicial\), esse fato, juntamente com as Definições~\ref{caso:Definicao3-3} e~\ref{caso:Definicao2-1}, nos diz que
                    todo modelo etiquetado em \textbf{B} diferente de \Modeloinicial está ancorado em exatamente um outro modelo etiquetado em \textbf{B};

                    \item[Caso~\ref{caso:Lema3-3}] Pelas Definições~\ref{caso:Definicao3-3} e~\ref{caso:Definicao2-1}, temos que todo par de modelos etiquetados de mesmo tipo
                    tem conjuntos de mundos disjuntos;

                    \item[Caso~\ref{caso:Lema3-4}] Pela definição indutiva de ancoramento indireto (Definição~\ref{def:AncoramenteoInidireto}) e pela
                    Definição~\ref{caso:Definicao3-2}, sabemos que essa cadeia de ancoramentos existe. Pelo Lema~\ref{caso:Lema3-2}, sabemos que todo modelo etiquetado diferente
                    de \Modeloinicial na cadeia tem exatamente um predecessor e, pela Definição~\ref{caso:Definicao3-1}, que nos diz que \Modeloinicial não está ancorado
                    em nenhum modelo etiquetado, sabemos que \Modeloinicial não tem predecessor na cadeia, portanto, pode haver apenas uma cadeia de ancoramentos em \Mathcal{E};

                    \item[Caso~\ref{caso:Lema3-5}] Decorre diretamente da Definição~\ref{caso:Definicao2-2};

                    \item[Caso~\textnormal{\SubCaso{caso:Lema3-6}{caso:Lema3-6-1}}] Decorre da Definição~\ref{caso:Definicao2-3}:
                    \(\Sigma(\mathcal{N}) = \funcao{S}_{\pi}(\Sigma_{\mathcal{M}}(w_{\mathcal{N}}))\) e do fato que \(\Sigma_{\mathcal{M}}(w) \subseteq \Sigma(\mathcal{M})\);

                    \item[Caso~\textnormal{\SubCaso{caso:Lema3-6}{caso:Lema3-6-2}}] Pelos Lemas~\ref{caso:Lema3-4} e~\SubCaso{caso:Lema3-6}{caso:Lema3-6-1} concluímos que
                    \(\Sigma(\mathcal{M}) \subseteq \Sigma(\modeloinicial)\), o que é suficiente para provar o caso, pois \(\Sigma(\modeloinicial) = \funcao{SC}(\Gamma) \subseteq \Theta\);

                    \item[Caso~\textnormal{\SubCaso{caso:Lema3-6}{caso:Lema3-6-3}}] Pelo Lema~\SubCaso{caso:Lema3-6}{caso:Lema3-6-2} e pelos fatos que
                    \(\funcao{D}_{\mathcal{M}}(w) \subseteq \funcao{B}(\Sigma_{\mathcal{M}}(w))\) e \(\Theta\) é fechado para \(\funcao{B}\), pela Definição~\ref{def:EspacoFormula};

                    \item[Caso~\textnormal{\SubCaso{caso:Lema3-6}{caso:Lema3-6-4}}] Pelo Lema~\SubCaso{caso:Lema3-6}{caso:Lema3-6-2} e o fato que \(\Theta\) é fechado
                    para \(\funcao{B}\), podemos concluir que \(\funcao{DC}_{\pi}(\funcao{B}(\Sigma({\mathcal{M}}))) \subseteq \funcao{DC}_{\pi}(\Theta)\) e, pela
                    Definição~\SubCaso{caso:Definicao1-2}{caso:Definicao1-2-4}, que nos diz que \(\funcao{T}_{\pi}(\Sigma(\mathcal{M}))\) é verdadeiro em \Mundobase,
                    temos que \(\funcao{T}_{\pi}(\Sigma(\mathcal{M})) \subseteq \funcao{DC}_{\pi}(\funcao{B}(\Sigma_{\mathcal{M}}))\). \qedhere

                \end{description}
            \end{proof}

            Tendo apresentado estas propriedades, é possível observar que um broto de \Modeloinicial pode ser visto como uma árvore, cuja raiz é \Modeloinicial e
            uma cadeia de ancoramentos para algum modelo etiquetado \Mathcal{E} é um ramo na árvore, partindo da raiz e indo até a folha \Mathcal{E}.

            \begin{definicao}[Função de Seleção de Modelos]
                \label{def:FuncaoSelecaoModelos}
                Uma \textit{função de seleção de modelo f} é uma função que associa, para cada \PI e para cada conjunto \(\mathcal{L}_{\pi}\)-consistente de fórmulas
                \(\Delta \subseteq \funcao{DC}_{\pi}(\Theta)\), um conjunto não vazio \(f_{\pi}(\Delta)\) de pares da forma \(\langle \mathcal{E}, w \rangle\) tal que \Mathcal{E} é
                um modelo etiquetado para \(\mathcal{L}_{\pi}\) que satisfaz \DDELTA em \textit{w}. Para uma dada função de seleção \textit{f}, definimos:
                \begin{itemize}
                    \item \(\mathbb{M}_{f} = \{ \mathcal{E} \ | \ \langle \mathcal{E}, w \rangle \in f_{\pi}(\Delta), \Delta \subseteq \mathsf{LM}_{12} \text{ e \DDELTA é }
                            \mathcal{L}_{\pi} \text{-consistente} \}\)

                    \item \(\mathbb{W}_{f} = \bigcup \{\mathcal{W} \ | \ \mathcal{W} \in \mathcal{M}, \mathcal{M} \in \mathcal{E}, \mathcal{E} \in \mathbb{M}_{f} \}\)

                    \item \(\aleph_{f} = max\{ \aleph_{0}, sup\left(\{|\mathcal{W}| \ | \ \mathcal{M} = \langle \mathcal{W}, \mathcal{R}, \mathcal{V} \rangle \in \mathbb{M}_{f}\}\right) \}\)
                \end{itemize}
                As funções de seleção de modelos analisadas respeitam as seguintes propriedades:
                \begin{enumerate}%[label=\textnormal{\ref{def:FuncaoSelecaoModelos}.\arabic*}]
                    \item \Mathbbi{W}{f} é um conjunto e \(|\mathbb{W}_{f}| > \aleph_{f}\);

                    \item Para cada conjunto \Mathcali{L}{12}-consistente \(\Delta \subseteq \funcao{DC}_{\pi}(\Theta)\), \(f_{\pi}(\Delta)\) é fechado para isomorfismo
                    em (ou seja, entre objetos de) \Mathbbi{W}{f}. \qed
                \end{enumerate}
            \end{definicao}

            Para uma dada função \textit{f}, iremos assumir que \(\langle \modeloinicial, \mundoinicial \rangle \in f_{\pi}(\Gamma \cup \funcao{T}_{\pi}(\Sigma(\modeloinicial)))\)
            e que os mundos de todos os brotos de \Modeloinicial estarão contidos no conjunto \Mathbbi{W}{f}.

            \begin{definicao}[Conjunto de Brotos]
                \label{def:ConjuntoBrotos}
                Um broto de \Modeloinicial é chamado de um \textit{f-broto} de \Modeloinicial se, para cada \PImodelo etiquetado no broto,
                \(\langle \mathcal{E}, \mundobase \rangle \in f_{\pi}(\funcao{DG}(\Sigma(\mathcal{M}) \cup \funcao{T}_{\pi}(\Sigma(\mathcal{M}))))\).

                Para cada \textit{f}, chamaremos de \textit{\(\mathfrak{B}_{f}\) o conjunto de todos os \textit{f}-brotos de \(\modeloinicial\)}. \qed
            \end{definicao}

            É importante ressaltar que \Mathfraki{B}{f} é um conjunto de conjuntos de modelos etiquetados, em específico, é um conjunto de conjuntos que satisfazem as
            Definição~\ref{def:Definicao3} e~\ref{def:ConjuntoBrotos}.

            \begin{lema}
                \label{teo:Lema4}
                Para cada \textit{f}, o conjunto \Mathfraki{B}{f} tem um elemento máximo. \qed
            \end{lema}

            \begin{proof}[Prova do Lema~\ref{teo:Lema4}]
                O Lema de Zorn~\cite{zorn1935remark} nos diz que, sendo \textit{X} um conjunto parcialmente ordenado (por alguma relação de ordenamento <), se todo
                subconjunto totalmente ordenado de \textit{X} tem um limite superior, então \textit{X} tem um elemento máximo (com relação a <).

                Considerando \Mathfraki{B}{f} parcialmente ordenado por \(\subseteq\) e sendo \textit{C} um subconjunto\footnote{Note que \textit{C} é um conjunto de conjuntos.}
                totalmente ordenado de \Mathfraki{B}{f}. O conjunto \(\bigcup C\) é um limite superior de \textit{C} pois \(S \subseteq C\), para todo broto \(S \in C\).
                Mais ainda, \(\bigcup C\) satisfaz todas as condições para ser um \textit{f}-broto (é fácil observar que a operação \(\cup\) preserva as
                propriedades~\ref{caso:Definicao3-1} --~\ref{caso:Definicao3-3} e~\ref{def:ConjuntoBrotos}).

                Portanto, \Mathfraki{B}{f} tem um elemento máximo.
            \end{proof}

            É importante esclarecer o que significa \Mathfraki{B}{f} ter um elemento máximo; como \Mathfraki{B}{f} é um conjunto de brotos e cada broto pode ser entendido
            como uma árvore cuja raiz é \Modeloinicial, o elemento máximo de \Mathfraki{B}{f} é a maior árvore no conjunto, ou seja,
            a maior sequência de modelos ancorados dois a dois partindo do modelo inicial. A importância de \Mathfraki{B}{f} ter um elemento máximo ficará clara a frente.

            O elemento máximo de \Mathfraki{B}{f} respeita a seguinte importante propriedade:

            \begin{lema}
                \label{teo:Lema5}
                Sendo \MathcalI{S}{+} um elemento máximo de \Mathfraki{B}{f}, então, para todo \(\mathcal{E}_{0} \neq \modeloinicial \in \mathcal{S}^{+}\) e mundo não base
                \(w \in \mathcal{E}_{0}\), então existe um \(\mathcal{E}_{1} \in \mathcal{S}^{+}\) ancorado em \Mathcali{E}{0} no mundo \textit{w}. \qed
            \end{lema}

            \begin{proof}[Prova do Lema~\ref{teo:Lema5}]
                Assumindo que \MathcalI{S}{+} é um elemento máximo de \Mathfraki{B}{f} e assumindo que \(\mathcal{E}_{0} \neq \modeloinicial \in \mathcal{S}^{+}\)
                é um \PImodelo etiquetado, \(w \in \mathcal{E}_{0}\) um mundo não base de \Mathcali{E}{0} e que não há qualquer
                \(\mathcal{E}_{1} \in \mathcal{S}^{+}\) ancorado em \Mathcali{E}{0} no mundo \textit{w}.

                Pelo Lema~\ref{teo:Lema2}, sabemos que o diagrama de concordância \(\funcao{D}_{\mathcal{M}}(w)\) de \textit{w} em \Mathcali{E}{0} é
                \Mathcali{L}{12}-consistente. Como \(\funcao{T}_{\mOPI}(\funcao{S}_{\pi}(\Sigma_{\mathcal{M}}(w)))\) é o um conjunto de teoremas de \Mathcali{L}{12},
                sabemos que \(\funcao{D}_{\mathcal{M}}(w) \cup \funcao{T}_{\mOPI}(\funcao{S}_{\pi}(\Sigma_{\mathcal{M}}(w)))\) é
                \Mathcali{L}{12}-consistente e, portanto, \Mathcali{L}{\mOPI}-consistente.

                Como \(\funcao{D}_{\mathcal{M}}(w) \cup \funcao{T}_{\mOPI}(\Sigma(\mathcal{M})) \subseteq \funcao{DC}_{\mOPI}(\Theta)\) (pelo Lema~\ref{caso:Lema3-6}) sabemos que
                \(\funcao{D}_{\mathcal{M}}(w) \cup \funcao{T}_{\mOPI}(\funcao{S}_{\pi}(\Sigma_{\mathcal{M}}(w)))\) é verdadeiro em algum mundo \(w_i\) de um \OPImodelo etiquetado \Mathcali{E}{1}
                tal que \(\langle \mathcal{E}_{1}, w_i \rangle \in f_{\pi}(\funcao{D}_{\mathcal{M}}(w) \cup \funcao{T}_{\mOPI}(\funcao{S}_{\pi}(\Sigma_{\mathcal{M}}(w))))\).
                Portanto, sendo \Mathcal{N} um modelo onde \(\mathcal{N} \in \mathcal{E}_{1}\) e \(\mathcal{W}_{\mathcal{N}} \in \mathcal{N}\), temos que
                \(\mathcal{W}_{\mathcal{N}} \subseteq \mathbb{W}_f\) e \(|\mathcal{W}_{\mathcal{N}}| \leq \aleph_{f}\).

                Como \textit{f} é fechado sob isomorfismo em \(\mathbb{W}_f\), podemos tomar \(w_i\) como sendo \textit{w}, declarar \textit{w} como o mundo
                base de \Mathcali{E}{1} e tomar \(\Sigma(\mathcal{N}) = \funcao{S}_{\pi}(\Sigma_{\mathcal{M}}(w))\)
                (portanto \(\funcao{D}_{\mathcal{M}(w)} = \funcao{DG}(\Sigma(\mathcal{N}))\)).

                Como \(\mathcal{S}^{+} \in \mathfrak{B}_{f}\), todo modelo etiquetado em \MathcalI{S}{+} tem, no máximo, cardinalidade\footnote{A cardinalidade de um modelo etiquetado
                se refere a cardinalidade do conjunto de mundos do modelo associado ao modelo etiquetado.} \(\aleph_{f}\).
                Pelo fato que \MathcalI{S}{+} tem uma estrutura de árvore, sabemos que \MathcalI{S}{+} contém, no máximo, \((\aleph_{f})^{n-1} = \aleph_{f}\)
                modelos etiquetados com cadeias de ancoramento de comprimento \textit{n}. Portanto, \MathcalI{S}{+} é um união contável de conjuntos de modelos etiquetados cuja
                cardinalidade é no máximo \(\aleph_{f}\), ou seja, \(|\mathcal{S}^{+}| \leq \aleph_{f}\).

                Portanto, sendo \(\mathcal{X} = \bigcup \{\mathcal{W} \ | \ \mathcal{W} \in \mathcal{M}, \mathcal{M} \in \mathcal{E}, \mathcal{E} \in \mathcal{S}^{+} \}\), temos
                \(|\mathcal{X}| \leq \aleph_{f} < |\mathbb{W}_{f}|\), adicionalmente \(|\mathcal{W}_{\mathcal{N}}| \leq \aleph_{f} < |\mathbb{W}_{f}|\).
                Como \(f_{\mOPI}(\funcao{DG}(\Sigma(\mathcal{N}) \cup \funcao{T}_{\mOPI}(\Sigma(\mathcal{N}))))\) é fechado sob isomorfismo em \(\mathbb{W}_{f}\),
                podemos assumir que todos os mundos não base de \Mathcali{E}{1} são elementos do conjunto \(\mathbb{W}_{f}-\mathcal{X}\).

                Assim, temos que \Mathcali{E}{1} satisfaz todas as condições para ser um \OPImodelo etiquetado ancorado em \Mathcali{E}{0} no mundo \textit{w}
                e que \Mathcali{E}{1} está indiretamente ancorado em \Modeloinicial (Definição~\ref{def:AncoramenteoInidireto}).
                Os mundos não base de \Mathcali{E}{1} são disjuntos do conjunto \(\mathcal{X}\) e \Mathcali{E}{1} não compartilha o mundo \textit{w}
                com qualquer outro modelo diferente de \Mathcali{E}{0} em \MathcalI{S}{+} (Definição~\ref{caso:Definicao3-1} e hipótese), portanto
                \(\mathcal{S}^{+} \cup \{\mathcal{E}_{1}\}\) é um \textit{f}-broto que contém \MathcalI{S}{+}, ou seja, \MathcalI{S}{+} não é máximo,
                o que é contraditório.
            \end{proof}

            Note que essa prova não afirma que \MathcalI{S}{+} será necessariamente infinito. Considere, por exemplo a cadeia de ancoramentos
            \(\modeloinicial, \dots, \mathcal{A}, \mathcal{B}\) e considere \(\mathcal{W}_{\mathcal{A}(\mathcal{B})}\) o conjunto de mundos do modelo
            associado ao modelo etiquetado \(\mathcal{A}(\mathcal{B})\), sendo \(\mathcal{W}_{\mathcal{A}} = \{w_{\mathcal{A}}, w_{\mathcal{B}}\}\),
            \(\mathcal{W}_{\mathcal{B}} = \{w_{\mathcal{B}}\}\) e \(\modeloinicial, \dots, \mathcal{A}, \mathcal{B} \in \mathcal{S}^{+}\). Portanto, existe um
            modelo etiquetado ancorado em um mundo não base de \Mathcal{A} onde não há nenhum outro modelo ancorado, pois seu único mundo é o mundo base.

            \begin{definicao}[Modelos Máximos]
                \label{def:Definicao4}
                Para cada elemento máximo \MathcalI{S}{+} em \Mathfraki{B}{f}, definimos o seu 12-modelo correspondente \(\mathcal{M}(\mathcal{S}^{+})\),
                sendo \(\mathcal{E} = \langle \mathcal{M}, \mundobase, \Sigma(\mathcal{M}) \rangle \in \mathcal{S}^{+}\) um modelo etiquetado qualquer, como:
                \begin{alignat*}{3}
                    &\mathcal{W}_{\mathcal{M}(\mathcal{S}^{+})}\ &=&\  \bigcup\{\mathcal{W} \ | \ \mathcal{W} \in \mathcal{M}\} \\
                    &\mathcal{R}_{\pi, \mathcal{M}(\mathcal{S}^{+})}\ &=&\  \bigcup\{\mathcal{R}_{\pi} \ | \ \mathcal{R}_{\pi} \in \mathcal{M}
                        \text{ e } \mathcal{M} \text{ é do tipo \PI}\} \\
                    &\mathcal{V}_{\mathcal{M}(\mathcal{S}^{+})}(p)\ &=&\  \bigcup\{\mathcal{V}_{\pi}(p) \ | \ \mathcal{V}_{\pi} \in \mathcal{M}\}
                \end{alignat*}
                Este modelo é chamado de \textit{modelo \Mathfraki{B}{f}-máximo} ou simplesmente \textit{modelo máximo}. \qed
            \end{definicao}

            Com essa definição, temos a seguinte propriedade importante:

            \begin{lema}
                \label{teo:Lema6}
                Sendo \(\mathcal{M}(\mathcal{S}^{+}) = \langle \mathcal{W}, \mathcal{R}_{1}, \mathcal{R}_2, \mathcal{V} \rangle\) um modelo máximo, então:
                \begin{enumerate}[label=\textnormal{\ref{teo:Lema6}.\arabic*}]
                    \item \textnormal{Todo \(w \in \mathcal{W}\) pertence exatamente a um modelo \(\mathcal{M}^{1}(w)\), este que pertence a um 1-modelo etiquetado
                    \(\mathcal{E}_{0} \in \elementomaximo\), e a um modelo \(\mathcal{M}^{2}(w)\), este que pertence a um 2-modelo etiquetado \(\mathcal{E}_{1} \in \elementomaximo\),
                    onde \(\mathcal{E}_{0}/\mathcal{E}_{1}\) está ancorado em \(\mathcal{E}_{1}/\mathcal{E}_{0}\) no mundo \textit{w};} \label{caso:Lema6-1}

                    \item \textnormal{O \PI-frame \(\langle \mathcal{W}, \mathcal{R}_{\pi} \rangle\) é a união disjunta de todos os \PI-frames
                    \(\langle \mathcal{W}, \mathcal{R}_{\pi} \rangle \in \mathcal{M}\) onde \(\mathcal{M} \in \mathcal{E} \text{ e } \mathcal{E} \in \elementomaximo\).} \label{caso:Lema6-2} \qed
                \end{enumerate}
            \end{lema}

            \begin{proof}[Prova do Lema~\ref{teo:Lema6}]
                \phantom{a}
                \begin{description}
                    \item[Caso~\ref{caso:Lema6-1}] Pela definição de \(\mathcal{M}(\elementomaximo)\), sabemos que \textit{w} está em algum \PImodelo \(\mathcal{M}_{\pi}(w)\).
                    Então temos dois casos para considerar:
                    \begin{enumerate}[label=(\roman*)]
                        \item \(w = \mundoinicial\) ou \(w \neq \mundoinicial\) e \textit{w} não é o mundo base de \(\mathcal{M}_{\pi}(w)\).\\
                            Nesse caso, temos que \textit{w} está em algum \OPImodelo \(\mathcal{M}_{\mOPI}(w)\) que está ancorado em \(\mathcal{M}_{\pi}(w)\)
                            no mundo \textit{w}, pelo Lema~\ref{teo:Lema5}.

                        \item \(w \neq \mundoinicial\) e \textit{w} é o mundo base de \(\mathcal{M}_{\pi}(w)\) (portanto \(\mathcal{M}_{\pi}(w) \neq \modeloinicial\)).\\
                            Nesse caso, \textit{w} está em algum \OPImodelo \(\mathcal{M}_{\mOPI}(w)\) tal que \(\mathcal{M}_{\pi}(w)\) está ancorado em \(\mathcal{M}_{\mOPI}(w)\)
                            no mundo \textit{w}, pelo Lema~\ref{caso:Lema3-2}.
                    \end{enumerate}
                    Em ambos os casos, temos que \textit{w} está em pelo menos um \(\mathcal{M}^{1}(w)\) e um \(\mathcal{M}^{2}(w)\), um ancorado no outro em \textit{w}.
                    A unicidade de \(\mathcal{M}^{1}(w)\) e \(\mathcal{M}^{2}(w)\) é consequência direta do Lema~\ref{caso:Lema3-3};

                    \item[Caso~\ref{caso:Lema6-2}] Pelo Lema~\ref{caso:Lema6-1}, sabemos que \(\mathcal{W} = \bigcup\{\mathcal{W} \ | \ \mathcal{M} \in \mathcal{E}, \mathcal{E} \in \mathcal{S}^{+}
                    \text{ e \Mathcal{M} é do tipo \PI}\}\) é valido para \(\pi \in \{1,2\}\), já o Lema~\ref{caso:Lema3-3} nos diz que os conjuntos de mundos de \PImodelos são mutuamente disjuntos.
                    \qedhere
                \end{description}
            \end{proof}

            Os principais resultados relacionados ao modelo \(\mathcal{M}(\elementomaximo)\) são o \textit{Lema de Concordância} e o \textit{Lema do Frame}.
            O Lema da Concordância nos diz que o modelo ``grande'' \(\mathcal{M}(\elementomaximo)\) concorda com todos os modelos ``pequenos'' \(\mathcal{M}_{\pi}(w)\) na valoração
            de todos os componentes booleanos de elementos em \(\Sigma_{\mathcal{M}_{\pi}(w)}(w)\) (lembre-se das Definições~\ref{def:FragmentosMonomodais} e~\ref{def:FuncoesElementos}).
            No que segue, escrevemos \(\mathcal{M}_{\pi}\) ao invés de \(\mathcal{M}_{\pi}(w)\) quando não for necessário indicar \textit{w}.

            \begin{lema}[Lema de Concondância]
                \label{teo:Lema7}
                Sendo \(\mathcal{M} = \langle \mathcal{W}, \mathcal{R}_{1}, \mathcal{R}_{2}, \mathcal{V} \rangle\) um modelo máximo. Então, para cada \(w \in \mathcal{W}\) e
                \(\phi \in \funcao{B}(\Sigma_{\mathcal{M}_{\pi}}(w))\), \PHI será verdadeiro em \textit{w} no \PImodelo
                \(\mathcal{M}_{\pi} = \langle \mathcal{W}_{\pi}, \mathcal{S}_{\pi}, \mathcal{V}_{\pi} \rangle\) se, e somente se, \PHI for verdadeiro em \textit{w} no 12-modelo \Mathcal{M}.
            \end{lema}

            \begin{proof}[Prova do Lema~\ref{teo:Lema7}]
                Provaremos por indução em \PHI:
                \begin{description}
                    \item[\textnormal{Caso Base \(\phi := p, p \in \mathbb{P}\):}] Dividimos essa prova em dois casos:
                    \begin{itemize}
                        \item[(\(\Rightarrow\))] Como \(\mathcal{M}_{\pi}, w \Vdash p\) temos que \(w \in \mathcal{V}_{\pi}(p)\), logo \(w \in \mathcal{V}(p)\)
                            pela Definição~\ref{def:Definicao4};

                        \item[(\(\Leftarrow\))] Como \(\mathcal{M}, w \Vdash p\) sabemos que \(w \in \mathcal{V}(p)\), logo temos que \(w \in \mathcal{V}_{\pi}^{1}(p)\)
                            para algum \(\mathcal{N} = \langle \mathcal{W}_{\pi}^{1}, \mathcal{S}_{\pi}^{1}, \mathcal{V}_{\pi}^{1} \rangle \in \elementomaximo\),
                            pela Definição~\ref{def:Definicao4}, onde \(\mathcal{N}\) deve ser o modelo \(\mathcal{M}^{1}(w)\) ou o modelo \(\mathcal{M}^{2}(w)\) no Lema~\ref{caso:Lema6-1}.
                            O Lema~\ref{caso:Lema6-1} também nos diz que um modelo deve estar ancorado no outro no mundo \textit{w} e, qualquer que seja o caso, temos que
                            \(p \in \Sigma_{\mathcal{M}^{1}}(w)\) e \(p \in \Sigma_{\mathcal{M}^{2}}(w)\), pela Definição~\ref{caso:Definicao2-3}.
                            A Definição~\ref{caso:Definicao2-4} nos diz que \(\mathcal{M}^{1}, w \Vdash p\) sse \(\mathcal{M}^{2}, w \Vdash p\), ou seja,
                            temos \(\mathcal{M}_{\pi}, w \Vdash p\).
                    \end{itemize}

                    \item[\textnormal{Caso \(\phi := \neg \psi\) e \(\phi := \psi \lor \gamma\):}] Decorrem diretamente da hipótese de indução.
                    % Temos a seguinte hipótese de indução:
                    %     \[
                    %         \forall w, \psi, \psi \in \funcao{B}(\Sigma_{\mathcal{M}_{\pi}}(w)) \land \mathcal{M}_{\pi}, w \Vdash \psi \Leftrightarrow \mathcal{M}, w \Vdash \psi
                    %     \]
                    %     Queremos provar \(\mathcal{M}_{\pi}, w \Vdash \neg \psi \Leftrightarrow \mathcal{M}, w \Vdash \neg \psi\), o que é logicamente equivalente
                    %     a hipótese de indução.

                    % \item[\textnormal{Caso \(\phi := \psi \lor \gamma\):}] Temos a seguinte hipótese de indução:
                    %     \begin{equation}
                    %         \forall w, \psi, \psi \in \funcao{B}(\Sigma_{\mathcal{M}_{\pi}}(w)) \land \mathcal{M}_{\pi}, w \Vdash \psi \Leftrightarrow \mathcal{M}, w \Vdash \psi
                    %     \end{equation}
                    %     Dividiremos a prova em dois casos:
                    %     \begin{itemize}
                    %         \item[(\(\Rightarrow\))] Como \(\mathcal{M}_{\pi}, w \Vdash \psi \lor \gamma\), temos dois casos disjuntos:

                    %         \begin{enumerate}[label=(\roman*)]
                    %             \item \(\mathcal{M}_{\pi}, w \Vdash \psi\), nesse caso, temos pela hipótese que \(\mathcal{M}, w \Vdash \psi\)

                    %             \item \(\mathcal{M}_{\pi}, w \Vdash \gamma\), nesse caso, temos pela hipótese que \(\mathcal{M}, w \Vdash \gamma\)
                    %         \end{enumerate}
                    %         Portanto, concluímos que \(\mathcal{M}, w \Vdash \psi \lor \gamma\).

                    %         \item[(\(\Leftarrow\))] Prova análoga ao caso anterior.
                    %     \end{itemize}

                    \item[\textnormal{Caso \(\phi := \Box_{\pi} \psi\):}] Temos a seguinte hipótese de indução:
                    \[
                        \forall w, \psi, \psi \in \funcao{B}(\Sigma_{\mathcal{M}_{\pi}}(w)) \land \mathcal{M}_{\pi}, w \Vdash \psi \Leftrightarrow \mathcal{M}, w \Vdash \psi
                    \]
                    Sabemos que \(\mathcal{M}_{\pi}, w \Vdash \Box_{\pi} \psi\) será válido sse \(\forall v, w \mathcal{S}_{\pi} v \to \mathcal{M}_{\pi}, v \Vdash \psi\) e,
                    pelo Lema~\ref{caso:Lema6-2}, sabemos que isso será válido sse \(\forall v, w \mathcal{R}_{\pi} v \to \mathcal{M}_{\pi}, v \Vdash \psi\)\footnote{Note que
                    \(\mathcal{R}_{\pi}\) se refere às relações \(\mathcal{R}_{1}\) e \(\mathcal{R}_{2}\) do modelo \(\mathcal{M}\).} for válido.
                    Pelo Lema~\ref{caso:Lema6-1}, sabemos que caso \(w \mathcal{R}_{\pi} v\) então \(\mathcal{M}_{\pi} = \mathcal{M}_{\pi}(w) = \mathcal{M}_{\pi}(v)\), ou seja,
                    \textit{w} e \textit{v} estão no mesmo \PImodelo.

                    Ademais, temos que \(\psi \in \funcao{B}(\Sigma_{\mathcal{M}_{\pi}}(v))\) pois \(\funcao{TC}(\psi) \subseteq \Sigma(\mathcal{M}_{\pi}(v))\)
                    (no caso onde \(w \neq v\), isso é devido a Definição~\SubCaso{caso:Definicao1-3}{caso:Definicao1-3-2}, já no caso onde \(w = v\), isso é devido as
                    Definições~\SubCaso{caso:Definicao1-2}{caso:Definicao1-2-3} e~\SubCaso{caso:Definicao1-3}{caso:Definicao1-3-1}).
                    Como \(\psi \in \funcao{B}(\Sigma_{\mathcal{M}_{\pi}}(v))\), podemos aplicar a hipótese de indução em \(\mathcal{M}_{\pi}, v \Vdash \psi\), portanto, temos:
                    \(\forall v, w \mathcal{R}_{\pi} v \to \mathcal{M}, v \Vdash \psi\), que será verdadeiro apenas se \(\mathcal{M}, w \Vdash \Box_{\pi} \psi\), portanto
                    provamos o caso.

                    \item[\textnormal{Caso \(\phi := \Box_{\mOPI} \psi\):}] Temos a seguinte hipótese de indução:
                    \[
                        \forall w, \psi, \psi \in \funcao{B}(\Sigma_{\mathcal{M}_{\pi}}(w)) \land \mathcal{M}_{\pi}, w \Vdash \psi \Leftrightarrow \mathcal{M}, w \Vdash \psi
                    \]
                    Vamos assumir que \(\Box_{\mOPI} \psi \in \Sigma_{\mathcal{M}_{\pi}}(w)\). Temos que ou \(\mathcal{M}_{\pi}\) está ancorado em
                    \(\mathcal{M}_{\mOPI}\) em \textit{w} ou vice versa. Em ambos os casos, o fato que \(\Box_{\mOPI} \psi \in \Sigma_{\mathcal{M}_{\mOPI}}(w)\) decorre da
                    Definição~\ref{caso:Definicao2-3} (no primeiro caso, pelo Lema~\SubCaso{caso:Lema3-6}{caso:Lema3-6-1}, no segundo caso pelo fato que \(\Box_{\mOPI} \psi\) é um \PI-elemento).
                    Pela Definição~\ref{caso:Definicao2-4} temos que \(\mathcal{M}_{\pi}, w \Vdash \Box_{\mOPI} \psi\) sse \(\mathcal{M}_{\mOPI}, w \Vdash \Box_{\mOPI} \psi\),
                    sendo que \(\mathcal{M}_{\mOPI}, w \Vdash \Box_{\mOPI} \psi\) será verdadeiro sse \(\mathcal{M}, w \Vdash \Box_{\mOPI} \psi\), o que pode ser provado de forma análoga ao
                    caso anterior, se substituirmos \(\Box_{\pi}\) por \(\Box_{\mOPI}\). \qedhere
                \end{description}
            \end{proof}

            \begin{definicao}[Condição de Frame]
                \label{def:CondicaoFrame}
                Uma função \textit{f satisfaz a condição de frame} se, para cada conjunto de fórmulas \Mathcali{L}{\pi}-consistente \(\Delta \subseteq \funcao{DC}_{\pi}(\Theta)\),
                os modelos em \(f_{\pi}(\Delta)\) são baseados em frames para \(\mathcal{L}_{\pi}\).\qed
            \end{definicao}

            Uma função \textit{f} que satisfaz a condição de frame existe quando \Mathcali{L}{1} e \Mathcali{L}{2} são completas com relação à \(\funcao{DC}_{\pi}(\Theta)\).
            A existência dessa função se dá pela completude de \Mathcali{L}{1} e \Mathcali{L}{2} - como \(\mathcal{L}_{\pi}\) é completa, temos que todo
            \(\Delta \subseteq \funcao{DC}_{\pi}(\Theta)\) é satisfazível em algum modelo baseado num frame para \(\mathcal{L}_{\pi}\), logo, pela definição de função de seleção de
            modelos, sabemos que todos os modelos em \(f_{\pi}(\Delta)\) são baseados em frames para \(\mathcal{L}_{\pi}\).

            \begin{lema}[Lema do Frame]
                \label{teo:Lema8}
                Sendo \(\mathcal{M} = \langle \mathcal{W}, \mathcal{R}_{1}, \mathcal{R}_{2}, \mathcal{V} \rangle\) um modelo \Mathfraki{B}{f}-máximo onde \textit{f} satisfaz
                a Definição~\ref{def:CondicaoFrame}. Então \(\langle \mathcal{W}, \mathcal{R}_{1}, \mathcal{R}_{2}\rangle\) é um 12-frame para \(\mathcal{L}_{12}\) \qed
            \end{lema}

            \begin{proof}[Prova do Lema~\ref{teo:Lema8}]
                Sabemos, pela definição de união disjunta e pela definição de frames, que conjuntos de frames são fechados para a operação de união
                disjunta\footnote{Por exemplo: \(\mathcal{F}_{1} = \langle \{w_{0}, w_{1}\}, \{\langle w_{0}, w_{1} \rangle, \langle w_{0}, w_{0} \rangle,
                \langle w_{1}, w_{1} \rangle\} \rangle\) e \(\mathcal{F}_{2} = \langle \{w_{2}, w_{3}\}, \{\langle w_{2}, w_{3} \rangle\} \rangle\), sua união disjunta será
                o frame \(\mathcal{F}_{12} = \langle \{w_{{0}_{1}}, w_{{1}_{1}}, w_{{2}_{2}}, w_{{3}_{2}}\}, \{\langle w_{{0}_{1}}, w_{{1}_{1}} \rangle, \langle w_{{0}_{1}}, w_{{0}_{1}} \rangle,
                \langle w_{{1}_{1}}, w_{{1}_{1}} \rangle, \langle w_{{2}_{2}}, w_{{3}_{2}} \rangle\} \rangle\).}.
                Com isso e o Lema~\ref{caso:Lema6-2}, temos que o par \(\langle \mathcal{W}, \mathcal{R}_{1} \rangle\) descreve um frame para a lógica \(\mathcal{L}_{1}\) e o par
                \(\langle \mathcal{W}, \mathcal{R}_{2} \rangle\) descreve um frame para a lógica \(\mathcal{L}_{2}\). Pela definição de 12-frames e pelo Teorema~\ref{teo:TransCorretude},
                concluímos a prova.
            \end{proof}

            Como \(\Gamma \subseteq \funcao{B}(\Sigma_{\modeloinicial}(\mundoinicial))\) é verdadeiro em \Mundoinicial no modelo \(\modeloinicial = \mathcal{M}^{1}(\mundoinicial)\), o
            Lema~\ref{teo:Lema7} nos diz que \GAMMA é verdadeiro em \Mundoinicial no 12-modelo \Mathcal{M}. Como \Mathcali{L}{1} e \Mathcali{L}{2} são completas, podemos assumir
            alguma função de seleção de modelos \textit{f} que satisfaz a Definição~\ref{def:CondicaoFrame}; portanto, o frame que define \(\mathcal{M}\) é um frame para
            \Mathcali{L}{12}, pelo Lema~\ref{teo:Lema8}.

            Portanto, com este método conseguimos construir um modelo que satisfaz todo subconjunto \(\mathcal{L}_{12}-\)consistente do espaço de fórmulas \(\Theta\).
            Assim, concluímos a prova do Teorema~\ref{teo:TransCompletude}.
        \end{proof}


\end{apendicesenv}
% ---