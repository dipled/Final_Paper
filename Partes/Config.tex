% ---- Arquivo com as configurações do PDF

% Alteração para fonte de capítulos:
\renewcommand{\ABNTEXchapterfont}{\normalfont}
\renewcommand{\ABNTEXchapterfont}{\bfseries}

% Definindo a cor azul em RGB:
\definecolor{blue}{RGB}{41,5,195}

% Informações do PDF:
\makeatletter
\hypersetup{
		pdftitle={\@title}, 
		pdfauthor={\@author},
    	pdfsubject={\imprimirpreambulo},
	    pdfcreator={Bruno Rafael dos Santos},
		pdfkeywords={Algoritmo \textit{RESSOL}}{Algoritmo de Tonelli-Shanks}{Lei de Reciprocidade Quadrática}{abntex2}{trabalho acadêmico}, 
		colorlinks=true,
    	linkcolor=blue,
    	citecolor=blue,
    	filecolor=magenta,
		urlcolor=blue,
		bookmarksdepth=4
}
\makeatother

% Posiciona figuras e tabelas no topo da página quando adicionadas sozinhas em uma página em branco (ver https://github.com/abntex/abntex2/issues/170):
\makeatletter
\setlength{\@fptop}{5pt} 
\makeatother

% Possibilita criação de Quadros e Lista de quadros (ver https://github.com/abntex/abntex2/issues/176):
\newcommand{\quadroname}{Quadro}
\newcommand{\listofquadrosname}{Lista de quadros}
\newfloat[chapter]{quadro}{loq}{\quadroname}
\newlistof{listofquadros}{loq}{\listofquadrosname}
\newlistentry{quadro}{loq}{0}

% Configurações para atender às regras da ABNT
\setfloatadjustment{quadro}{\centering}
\counterwithout{quadro}{chapter}
\renewcommand{\cftquadroname}{\quadroname\space} 
\renewcommand*{\cftquadroaftersnum}{\hfill--\hfill}

\setfloatlocations{quadro}{hbtp} % Ver https://github.com/abntex/abntex2/issues/176

% O tamanho do parágrafo é dado por:
\setlength{\parindent}{1.3cm}

% Controle do espaçamento entre um parágrafo e outro:
\setlength{\parskip}{0.2cm}  % tente também \onelineskip

% Configurações adicionadas por Bruno e Helena:
	% Fontes
	\DeclareMathAlphabet{\pazocal}{OMS}{zplm}{m}{n}
	\DeclareMathAlphabet{\mathscr}{OMS}{zplm}{m}{n}
% Definindo fonte caligráfica e negrita
	\DeclareMathAlphabet\mathbfcal{OMS}{cmsy}{b}{n}
	% Teoremas e etc:
		\newtheorem{proposicao}{Proposição}
		\newtheorem{lema}      {Lema}
		\newtheorem{corolario} {Corolário}
		\newtheorem{teorema}{Teorema}
		
		\theoremstyle{definition}
		\newtheorem{definição}{Definição}
		\newtheorem{definicao}{Definição}
		\newtheorem{notacao}{Notação}
		\newtheorem{exemplo}{Exemplo}
		
		\theoremstyle{remark}
		\newtheorem{observacao}{Observação}
	% Operadores e etc:
		\newcommand{\eqdef}{\mathrel{\overset{\makebox[0pt]{\mbox{\normalfont\tiny\sffamily def}}}{=}}}
		\renewcommand\qedsymbol{$\blacksquare$}
		\newcommand{\meio}{\frac{1}{2}}
		\newcommand{\conmat}{\vDash_{\pazocal{M}_{\lfium{}}}}
		\newcommand{\conval}{\vDash_{\lfium{}}}
		\newcommand{\lfium}{\textbf{LFI1}}
		\newcommand{\lfi}{\textbf{LFI}}
		\newcommand{\lfis}{\textbf{LFI}s}
		\newcommand{\conhil}{\vdash_{\lfium{}}}
		\newcommand{\citeshort}[1]{\citeauthoronline{#1} (\citeyear{#1})}
		\newcommand{\cortar}[1]{\textcolor{red}{\sout{#1}}}
		\newcommand{\ignore}[1]{\textcolor{blue}{\textbf{IGNOREM:} #1}}
		\newcommand{\helena}[1]{\textcolor{magenta}{\textbf{HELENA:} #1}}
		\newcommand{\migs}[1]{\textcolor{violet}{\textbf{MIGS:} #1}}
		\newcommand{\migscortar}[2]{\textcolor{violet}{\textbf{MIGS:} \sout{#1}{#2}}}
		\newcommand{\kaqui}[1]{\textcolor{teal}{\textbf{KAQUI:} #1}}

	% Highlight de código em coq:
		\definecolor{violet}{RGB}{80,5,100}
		\definecolor{teal}{RGB}{0,128,128}
		\definecolor{orange}{RGB}{255,128,13}
		\definecolor{darkgreen}{RGB}{0,100,0}
		\definecolor{darkred}{RGB}{100,0,0}
		\usepackage{listings, Estilos/coq, Estilos/coq-error}
	% Algoritmos:
		\usepackage[portuguese,linesnumbered,boxruled,noend]{algorithm2e}
		\usepackage{hyperref}
		\SetArgSty{textnormal}
		\SetNlSty{textbf}{}{:}
		\setlength{\algomargin}{2.5em}
		\SetKwInput{Entrada}{Entrada}
		\SetKwInput{Saida}{Sa\'{i}da}
		\SetKw{Retorna}{retorna}
		\SetKwFor{Enqto}{enquanto}{faça}{endw}
		\SetKwIF{Se}{eSe}{SeN}{se}{então}{e se}{senão}{fimse}