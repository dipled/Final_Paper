\chapter{Reciprocidade Quadrática}
\label{cap:reciprocidadequadratica}

Como objetivo secundário, será apresentado neste Capítulo a Lei de Reciprocidade Quadrática. Como motivação, esta lei permite tornar mais eficiente o algoritmo \textit{RESSOL} \cite{johndcookQuadraticReciprocity} e também possui outras aplicações como \textit{zero-knowledge proofs} \cite{Wright2016}. O conteúdo apresentado nesse Capítulo é baseado em \cite{book:2399854} e em \cite{youtuQuadraticReciprocity}.

O Teorema conhecido como Lei de Reciprocidade Quadrática possui a seguinte descrição:

\begin{teorema}[\textit{Reciprocidade Quadrática}] Sejam $p$ e $q$ primos ímpares (maiores que $2$) distintos, então:
\label{teorema:reciprocidadequad}
    \begin{enumerate}
        \item \label{item:recipquad1} $\left(\frac{p}{q} \right) \cdot \left(\frac{q}{p} \right) = (-1)^{\frac{p-1}{2} \cdot \frac{q-1}{2}}$

        \item \label{item:recipquad2} $\left(\frac{2}{p} \right) = (-1)^{\frac{p^2 - 1}{8}} = \begin{cases}
                1 \text{ se $p \equiv \pm 1 \pmod{8}$}
                \\
                -1 \text{ se $p \equiv \pm 3 \pmod{8}$}
                \end{cases}$
    \end{enumerate}
    
\end{teorema}
Para demostração deste teorema antes deve-se demonstrar um lema necessário para tal, que é o seguinte:

\begin{lema}[\textit{Gau$\ss$}] \label{lema:gauB}  Seja $p > 2$ um número primo e $a \in \mathbb{Z}$ um número coprimo de $p$, isto é, $\mdc(a, p) = 1$, sendo $s$ o número de elementos do conjunto
\begin{equation*}
    \left\{x \in \mathbb{Z} | x \in \left\{a, 2 \cdot a, 3 \cdot a, ..., \frac{p-1}{2} \cdot a\right\} \land x \bmod p > \frac{p-1}{2} \right\}
\end{equation*}
então
\begin{equation*}
    \left(\frac{a}{p}\right) = (-1)^s
\end{equation*}

\end{lema}

\noindent
\textit{Demonstração}: dado o seguinte conjunto $\left\{\pm 1, \pm 2, \pm 3, ..., \pm \frac{p-1}{2} \right\}$, observe que todos os elementos desse conjunto possuem \textit{inverso módulo $p$} de acordo com o Lema \ref{lema : mdcinv}, e, para todo $j \in \left[1, \frac{p-1}{2}\right]$ é possível escolher $\epsilon_j \in \{-1, 1\}$ e $m_j \in \left[1,2, ..., \frac{p-1}{2}\right]$ tal que $a \cdot j \equiv \epsilon_j \cdot m_j \pmod{p}$ (pois com tais valores de $\epsilon_j$ e $m_j$ pode-se obter um número equivalente em módulo $p$ a qualquer outro), e além disso, para $i, k \in \left[1, \frac{p-1}{2}\right]$, se $i \neq k$ então $m_i \neq m_k$, pois há uma combinação única $\epsilon_j \cdot m_j$ para cada resto possível. Tal afirmação, significa que, caso $m_i = m_k$ então:

    \begin{itemize}
        \item $a \cdot i \equiv a \cdot j \pmod{p}$ é o único caso possível, em que pelo Item \ref{item:propcong7-cancelamento}, $i \equiv j \pmod{p}$, e portanto $i = j$ (devido ao intervalo que os valores $i$ e $j$ pertencem);

        \item $a \cdot i \equiv -a \cdot j \pmod{p}$ é um caso impossível, pois pelo Item \ref{item:propcong7-cancelamento}, $i \equiv -j \pmod{p}$, mas isso é impossível dado o intervalo que esses valores $i$ e $j$ se encontram.
    \end{itemize}
\noindent
Com isso, tem-se que:
\begin{align*}
    (a \cdot 1) \cdot (a \cdot 2) \cdot ...  \cdot (a \cdot \frac{p-1}{2}) & \equiv \epsilon_1 \cdot \epsilon_2 \cdot \epsilon_3 \cdot ... \cdot \epsilon_{\frac{p-1}{2}} \cdot ... \cdot m_1 \cdot m_2 \cdot ... \cdot m_{\frac{p-1}{2}} \pmod{p}
    \\
    \Longleftrightarrow 
    a^{\frac{p-1}{2}} \cdot \left(\frac{p-1}{2} \right)! & \equiv \epsilon_1 \cdot \epsilon_2 \cdot \epsilon_3 \cdot ... \cdot \epsilon_{\frac{p-1}{2}} \cdot \left(\frac{p-1}{2} \right)! \pmod{p}
\end{align*}
como $\mdc(\left(\frac{p-1}{2} \right)!, p) = 1$, pelo Item \ref{item:propcong7-cancelamento} e pelo Teorema \ref{teorema:criteriodeeuler}:
\begin{align*}
    a^{\frac{p-1}{2}} \cdot \left(\frac{p-1}{2} \right)! & \equiv \epsilon_1 \cdot \epsilon_2 \cdot \epsilon_3 \cdot ... \cdot \epsilon_{\frac{p-1}{2}} \cdot \left(\frac{p-1}{2} \right)! \pmod{p}
    \\
    \Longleftrightarrow a^{\frac{p-1}{2}} & \equiv  \epsilon_1 \cdot \epsilon_2 \cdot \epsilon_3 \cdot ... \cdot \epsilon_{\frac{p-1}{2}} \pmod{p}
    \\
    \Longleftrightarrow \left(\frac{a}{p}\right) & \equiv  \epsilon_1 \cdot \epsilon_2 \cdot \epsilon_3 \cdot ... \cdot \epsilon_{\frac{p-1}{2}} \pmod{p}
\end{align*}
E como ambos os valores pertencem ao conjunto $\{-1, 1\}$, então:
\begin{align*}
    \left(\frac{a}{p}\right) & = \epsilon_1 \cdot \epsilon_2 \cdot \epsilon_3 \cdot ... \cdot \epsilon_{\frac{p-1}{2}}
\end{align*}
Dado que para todo $m_j$ tem-se que $m_j \cdot \epsilon_j \bmod{p} > \frac{p-1}{2}$ se e somente se $\epsilon_j < 0$, então:
\begin{align*}
    \left(\frac{a}{p}\right) = (-1)^s
\end{align*} \qed

\DeclarePairedDelimiter\ceil{\lceil}{\rceil}
\DeclarePairedDelimiter\floor{\lfloor}{\rfloor}

Outro teorema que deve-se provar é o seguinte:
\begin{teorema} \label{teorema:somatorio-grafico}Seja m$p$ e $q$ números primos ímpares, então:
    \begin{equation} \label{eq:pointsnum}
    \frac{p-1}{2} \cdot \frac{q-1}{2} = \sum_{i = 1}^{\frac{p-1}{2}} \floor*{\frac{q \cdot i}{p}} + \sum_{j = 1}^{\frac{q-1}{2}} \floor*{\frac{p \cdot j}{q}}
    \end{equation}
\end{teorema}

\noindent
\textit{Demonstração}: inicialmente define-se o seguinte conjunto:
\begin{align*}
    S = \left\{ (x, y) \in \mathbb{Z} \times \mathbb{Z} \;\bigg|\; 1 \leq x \leq \frac{p-1}{2} \land 1 \leq y \leq \frac{q-1}{2} \right\}
\end{align*}
Separando este conjunto em dois novos conjuntos $S_1$ e $S_2$ tal que:

\begin{align*}
    S_1 = \left\{ (x, y) \in \mathbb{Z} \times \mathbb{Z} \;\big|\; p \cdot y < q \cdot x \land (x,y) \in S\right\}
    \\
    S_2 = \left\{ (x, y) \in \mathbb{Z} \times \mathbb{Z} \;\big|\; p \cdot y > q \cdot x \land (x,y) \in S\right\}
\end{align*}
É fácil notar que estes conjuntos são disjuntos pois suas restrições são excludentes. Quanto a $S = S_1 \cup S_2$, note que, como $p \neq q$ não existem pontos em $S$ tais que $p \cdot y = q \cdot x$, pois caso existissem teriam-se números iguais com fatoração em primos diferentes (devido aos intervalos em que $x$ e $y$ estão), o que é impossível. Assim, como todo par $(x, y)$ satisfaz uma das restrições, então $S = S_1 \cup S_2$.

Agora, reescrevendo as restrições dos conjuntos, começando por $S_1$, tem-se:
\begin{align*}
    p \cdot y < q \cdot x \Longleftrightarrow y < \frac{q \cdot x}{p}
\end{align*}
Sabe-se que $y \in \mathbb{Z}$ e que $p \nmid q \cdot x$ (pois $x < p$ e $p \nmid q$ visto que $q$ é um primo diferente de $p$), então
\begin{align*}
    y < \frac{q \cdot x}{p} \Longleftrightarrow y \leq \floor*{\frac{q \cdot x}{p}} 
\end{align*}
Além disso, note que, como o valor máximo de $x$ é $\frac{p-1}{2}$ e $\frac{p-1}{p} < 1$ obtêm-se o seguinte:
\begin{align*}
    \frac{q \cdot x}{p} \leq \frac{q \cdot (p-1)}{2 \cdot p} < \frac{q}{2}
\end{align*}
portanto
\begin{align*}
    \floor*{\frac{q \cdot x}{p}} \leq \frac{q-1}{2} 
\end{align*}
Assim pode-se alterar a definição do conjunto $S_1$ para:
\begin{align*}
    S_1 = \left\{ (x, y) \in \mathbb{Z} \times \mathbb{Z} \;\big|\; 1 \leq x \leq \frac{p-1}{2} \land 1 \leq y \leq \floor*{\frac{q \cdot x}{p}} \right\}
\end{align*}
Quanto a restrição de $S_2$, de maneira semelhante, tem-se:
\begin{align*}
    p \cdot y > q \cdot x \Longleftrightarrow x < \frac{p \cdot y}{q}
\end{align*}
Sabe-se que $x \in \mathbb{Z}$ e que $q \nmid p \cdot y$ (pois $y < q$ e $q \nmid p$ visto que $p$ é um primo diferente de $q$), então
\begin{align*}
    x < \frac{p \cdot x}{q} \Longleftrightarrow x < \floor*{\frac{p \cdot y}{q}} 
\end{align*}
E como o valor máximo de $y$ é $\frac{q-1}{2}$ e $\frac{q-1}{q} < 1$, obtêm-se
\begin{align*}
    \frac{p \cdot y}{q} \leq \frac{p \cdot (q - 1)}{2 \cdot q} < \frac{p}{2}
\end{align*}
portanto
\begin{align*}
    \floor*{\frac{p \cdot y}{q}} \leq \frac{p-1}{2} 
\end{align*}
Assim pode-se alterar a definição do conjunto $S_2$ para:
\begin{align*}
    S_2 = \left\{ (x, y) \in \mathbb{Z} \times \mathbb{Z} \;\big|\; 1 \leq y \leq \frac{q-1}{2} \land 1 \leq x \leq \floor*{\frac{p \cdot y}{q}} \right\}
\end{align*}
Neste momento da demonstração, note que $|S| = \frac{p-1}{2} \cdot \frac{q-1}{2}$ (devido aos valores possíveis de escolha para $x$ e $y$ na montagem de um par). Para o conjunto $S_1$ note que o número de escolhas possíveis do valor de $y$ depende do valor de $x$, ou seja, para $x = i$ tem-se $\floor*{\frac{q \cdot i}{p}}$ pares possíveis, logo:
\begin{align*}
    |S_1| = \sum_{i = 1}^{\frac{p-1}{2}} \floor*{\frac{q \cdot i}{p}}
\end{align*}
De modo similar, para o conjunto $S_2$ observe que o número de escolhas possíveis do valor de $x$ depende do valor $y$, isto é, para $y = j$ tem-se $\floor*{\frac{p\cdot j}{q}}$ pares possíveis, logo:
\begin{align*}
    |S_2| = \sum_{j = 1}^{\frac{q-1}{2}} \floor*{\frac{p \cdot j}{q}}
\end{align*}
Como $|S| = |S_1| + |S_2|$, então:
\begin{equation*}
    \frac{p-1}{2} \cdot \frac{q-1}{2} = \sum_{i = 1}^{\frac{p-1}{2}} \floor*{\frac{q \cdot i}{p}} + \sum_{j = 1}^{\frac{q-1}{2}} \floor*{\frac{p \cdot j}{q}} 
\end{equation*} \qed

Por último, deve-se provar o seguinte teorema:

\begin{teorema} \label{teorema:t-igual} Seja $p$ um número primo ímpar, então, para $a \in \mathbb{Z}$, se $\mdc(a, 2 \cdot p) = 1$ ($a$ é ímpar) e sendo
    \begin{equation*}
        t = \sum_{i = 1}^{\frac{p-1}{2}} \floor*{\frac{a \cdot i}{p}}
    \end{equation*}
então
    \begin{equation*}
        \left(\frac{a}{p} \right) = (-1)^t
    \end{equation*}
\end{teorema}
\noindent
\textit{Demonstração}: dado o conjunto de restos $A = \{a \bmod{p}, 2 \cdot a \bmod{p}, ..., \frac{p-1}{2} \cdot a \bmod{p} \}$, define-se $R = \{r_1, r_2, ..., r_m\}$ tal que esse é o conjunto de restos em $A$ menores ou iguais a $\frac{p-1}{2}$ e $S =\{s_1, s_2, ..., s_n\}$ o conjunto de restos em $A$ maiores que $\frac{p-1}{2}$. Observe que para qualquer $i \in [1, \frac{p-1}{2}]$ existe $r \in R$ tal que
\begin{equation*}
    i \cdot a = \floor*{\frac{i \cdot a}{p}} \cdot p + r 
\end{equation*}
ou existe $s \in S$ tal que:
\begin{equation*}
    i \cdot a = \floor*{\frac{i \cdot a}{p}} \cdot p + s 
\end{equation*}
portanto
\begin{equation*}
    \sum_{i = 1}^{\frac{p-1}{2}} i \cdot a = p \cdot \sum_{i = 1}^{\frac{p-1}{2}} \floor*{\frac{i \cdot a}{p}} + \sum_{j = 1}^{m} r_j + \sum_{k = 1}^{n} s_k
\end{equation*}
manipulando essa equação, tem-se:
\begin{align}
    & \sum_{i = 1}^{\frac{p-1}{2}} i \cdot a = p \cdot \sum_{i = 1}^{\frac{p-1}{2}} \floor*{\frac{i \cdot a}{p}} + \sum_{j = 1}^{m} r_j + \sum_{k = 1}^{n} s_k
    \\ \label{eq:ultimapfvr}
    \Longleftrightarrow \;\;
    &
    a \cdot \sum_{i = 1}^{\frac{p-1}{2}} i = p \cdot \sum_{i = 1}^{\frac{p-1}{2}} \floor*{\frac{i \cdot a}{p}} + \sum_{j = 1}^{m} r_j + \sum_{k = 1}^{n} s_k
\end{align}
mas note que, de maneira similar ao que se teve na demonstração do Lema \ref{lema:gauB}, os conjuntos $R$ e $S$ não possuem valores repetidos, pois, para $i_1, i_2 \in [1, \frac{p-1}{2}]$, se $i_1 \cdot a \equiv i_2 \cdot a \pmod{p}$, então pelo Item \ref{item:propcong7-cancelamento} (como $\mdc(a, p) = 1$), tem-se que $i_1 \equiv i_2 \pmod{p}$, e portanto dado que $i_1, i_2 \in [1, \frac{p-1}{2}]$, se obtêm $i_1 = i_2$. Além disso, note que não é possível para $i_1, i_2 \in [1, \frac{p-1}{2}]$ que $i_1 \cdot a \equiv p - i_2 \cdot a \pmod{p}$, pois teria-se então $i_1 \cdot a \equiv - i_2 \cdot a \pmod{p}$ e por conseguinte (novamente pelo Item \ref{item:propcong7-cancelamento}) $i_1 \equiv -i_2 \pmod{p}$, o que é impossível dado que $i_1, i_2 \in [1, \frac{p-1}{2}]$. Portanto, sendo $S' = \{s \in S \;|\; p - s\}$, tem-se que:
\begin{equation*}
    \left\{1, 2, ..., \frac{p-1}{2}\right\} = R \cup S'
\end{equation*}
logo
\begin{equation} \label{eq:achoqehaultima}
    \sum_{i = 1}^{\frac{p-1}{2}} i = \sum_{k = 1}^{n} (p - s_k) + \sum_{j = 1}^{m} r_j
\end{equation}
e então multiplicando ambos os lados por $a$:
\begin{equation} \label{eq:lastonepls}
    a \cdot \sum_{i = 1}^{\frac{p-1}{2}} i = a \cdot \left( \sum_{k = 1}^{n} (p - s_k) + \sum_{j = 1}^{m} r_j \right)
\end{equation}
Realizando então a substituição $t = \sum_{i = 1}^{\frac{p-1}{2}} \floor*{\frac{i \cdot a}{p}}$ e \ref{eq:lastonepls} em \ref{eq:ultimapfvr}, obtêm-se:
\begin{equation*}
    a \cdot \left( \sum_{k = 1}^{n} (p - s_k) + \sum_{j = 1}^{m} r_j \right) = p \cdot t + \sum_{j = 1}^{m} r_j + \sum_{k = 1}^{n} s_k
\end{equation*}
e realizando manipulações:
\begin{align*}
    & a \cdot \left( \sum_{k = 1}^{n} (p - s_k) + \sum_{j = 1}^{m} r_j \right) = p \cdot t + \sum_{j = 1}^{m} r_j + \sum_{k = 1}^{n} s_k
    \\
    \Longleftrightarrow \; \;
    & a \cdot \left( n \cdot p - \sum_{k = 1}^{n} s_k + \sum_{j = 1}^{m} r_j \right) = p \cdot t + \sum_{j = 1}^{m} r_j + \sum_{k = 1}^{n} s_k
\end{align*}
somando $\sum_{k = 1}^{n} s_k$ e subtraindo $\sum_{j = 1}^{m} r_j$ de ambos os lados
\begin{align*}
    & a \cdot \left( n \cdot p - \sum_{k = 1}^{n} s_k + \sum_{j = 1}^{m} r_j \right) = p \cdot t + \sum_{j = 1}^{m} r_j + \sum_{k = 1}^{n} s_k
    \\
    \Longleftrightarrow \; \;
    & a \cdot n \cdot p + (a - 1) \cdot \left( - \sum_{k = 1}^{n} s_k + \sum_{j = 1}^{m} r_j \right) = p \cdot t + 2 \cdot \sum_{k = 1}^{n} s_k
    \\
    \Longleftrightarrow \; \;
    & n \cdot p + (a - 1) \cdot n \cdot p + (a - 1) \cdot \left( - \sum_{k = 1}^{n} s_k + \sum_{j = 1}^{m} r_j \right) = p \cdot t + 2 \cdot \sum_{k = 1}^{n} s_k
    \\
    \Longleftrightarrow \; \;
    & n \cdot p  + (a - 1) \cdot \left(n \cdot p - \sum_{k = 1}^{n} s_k + \sum_{j = 1}^{m} r_j \right) = p \cdot t + 2 \cdot \sum_{k = 1}^{n} s_k
    \\
    \Longleftrightarrow \; \;
    & n \cdot p  + (a - 1) \cdot \left(\sum_{k = 1}^{n} (p - s_k) + \sum_{j = 1}^{m} r_j \right) = p \cdot t + 2 \cdot \sum_{k = 1}^{n} s_k
\end{align*}
e pela Equação \ref{eq:achoqehaultima}, tem-se:
\begin{align*}
    & n \cdot p  + (a - 1) \cdot \left(\sum_{k = 1}^{n} (p - s_k) + \sum_{j = 1}^{m} r_j \right) = p \cdot t + 2 \cdot \sum_{k = 1}^{n} s_k
    \\
    \Longleftrightarrow \; \;
    & n \cdot p  + (a - 1) \cdot \sum_{i=1}^{\frac{p-1}{2}} i = p \cdot t + 2 \cdot \sum_{k = 1}^{n} s_k
\end{align*}
Então, pela fórmula da Soma de Gauss:
\begin{align*}
    & n \cdot p  + (a - 1) \cdot \sum_{i=1}^{\frac{p-1}{2}} i = p \cdot t + 2 \cdot \sum_{k = 1}^{n} s_k
    \\
    \Longleftrightarrow \; \;
    & n \cdot p  + (a - 1) \cdot \frac{1}{2} \cdot \frac{p-1}{2} \cdot \frac{p+1}{2} = p \cdot t + 2 \cdot \sum_{k = 1}^{n} s_k
    \\
    \Longleftrightarrow \; \;
    & n \cdot p  + (a - 1) \cdot \frac{p^2 - 1}{8} = p \cdot t + 2 \cdot \sum_{k = 1}^{n} s_k
    \\
    \Longleftrightarrow \; \;
    &  (a - 1) \cdot \frac{p^2 - 1}{8} = p \cdot (t - n) + 2 \cdot \sum_{k = 1}^{n} s_k
\end{align*}
Agora, trabalhando com congruência módulo $2$, tem-se:
\begin{align*}
    &  (a - 1) \cdot \frac{p^2 - 1}{8} = p \cdot (t - n) + 2 \cdot \sum_{k = 1}^{n} s_k
    \\
    \Longleftrightarrow \; \;
    &  (a - 1) \cdot \frac{p^2 - 1}{8} \equiv p \cdot (t - n) + 2 \cdot \sum_{k = 1}^{n} s_k \pmod{2}
\end{align*}
Dado que $2 \cdot \sum_{k = 1}^{n} s_k$ é par e $p \equiv 1 \pmod{2}$ (o que implica que $1 \cdot (t-n) \equiv p \cdot (t-n) \pmod{2}$, utilizando o Item \ref{item:propcong6-produto}), então:
\begin{align*}
    &  (a - 1) \cdot \frac{p^2 - 1}{8} \equiv p \cdot (t - n) + 2 \cdot \sum_{k = 1}^{n} s_k \pmod{2}
    \\
    \Longleftrightarrow \; \;
    &  (a - 1) \cdot \frac{p^2 - 1}{8} \equiv (t - n) \pmod{2}
\end{align*}
mas como $a$ é ímpar, chega-se em:
\begin{align*}
    0 \equiv (t - n) \pmod{2}
\end{align*}
portanto:
\begin{align*}
    n \equiv t \pmod{2}
\end{align*}
Por fim, observe que, aplicando o Lema \ref{lema:gauB} com $p$ e $a$, o valor $s$ deste teorema é igual ao valor $n$ (pois ambos são o número de restos em $A$ maiores que $\frac{p-1}{2}$), ou seja:
\begin{align*}
    \left(\frac{a}{p}\right) = (-1)^s = (-1)^n = (-1)^t 
\end{align*} \qed

Finalmente pode então se iniciar a demonstração do Teorema \ref{teorema:reciprocidadequad}:

\noindent
\textit{Demonstração}: inciando pela prova do Item \ref{item:recipquad1}, sendo:
\begin{equation*}
    t_1 = \sum_{j = 1}^{\frac{q-1}{2}} \floor*{\frac{p \cdot j}{q}}
\end{equation*}
e
\begin{equation*}
    t_2 = \sum_{i = 1}^{\frac{p-1}{2}} \floor*{\frac{q \cdot i}{p}}
\end{equation*} 
então usando o Teorema \ref{teorema:t-igual}:
\begin{align*}
    \left(\frac{p}{q} \right) \cdot \left(\frac{q}{p} \right) = (-1)^{t_1} \cdot (-1)^{t_2} = (-1)^{t_1 + t_2} 
\end{align*}
e pelo Teorema \ref{teorema:somatorio-grafico}:
\begin{align*}
    \left(\frac{p}{q} \right) \cdot \left(\frac{q}{p} \right) = (-1)^{\frac{p-1}{2} \cdot \frac{q-1}{2}}
\end{align*}
Portanto está provado o Item \ref{item:recipquad1}. Quanto ao Item \ref{item:recipquad2}, note que para um número primo $p$ ímpar, em relação ao módulo $4$ existem apenas as duas seguintes possibilidades:
\begin{enumerate}
    \item $p \equiv 1 \pmod{4}$
    \item $p \equiv 3 \pmod{4}$
\end{enumerate}
Se $p \equiv 1 \pmod{4}$ então existe $k \in \mathbb{Z}$ tal que $p = 4 \cdot k + 1$, logo, $\frac{p-1}{2} = 2 \cdot k$. Sendo assim, há, primeiramente, a possibilidade de que $p \equiv 1 \pmod{8}$, então, existe $j \in \mathbb{Z}$ tal que:
\begin{align*}
    p \equiv 1 \pmod{8}
    \Longleftrightarrow \; & p = 8 \cdot j + 1
    \\
    \Longleftrightarrow \; & 4 \cdot k + 1 = 8 \cdot j + 1
    \\
    \Longleftrightarrow \; & 4 \cdot k = 8 \cdot j
    \\
    \Longleftrightarrow \; & k = 2 \cdot j
\end{align*}
Portanto, como $k$ é par, $(-1)^k = 1$, e pelo Lema $\ref{lema:gauB}$ com $a = 2$, note que $k$ é igual ao valor $s$, pois para $1 \leq i \leq k = \frac{p-1}{4}$ tem-se $2 \leq 2\cdot i \leq \frac{p-1}{2}$ (isto é, $k = \frac{p-1}{4}$ restos menores ou iguais a $\frac{p-1}{2}$) e para $ k + 1 = \frac{p-1}{4} + 1 \leq i \leq \frac{p-1}{2} = 2 \cdot k$ tem-se $\frac{p-1}{2} - \left(\frac{p-1}{4} + 1\right) + 1 = \frac{p-1}{4} = 2 \cdot k - (k + 1) + 1 = k$, (ou seja, há $k = \frac{p-1}{4}$ restos maiores que $\frac{p-1}{2}$), portanto:
\begin{equation*}
    \left(\frac{2}{p}\right) = (-1)^{\frac{p-1}{4}} = (-1)^{2 \cdot j} = 1
    % = (-1)^{\frac{p^2 - 1}{8}}
\end{equation*}
Como $\frac{p-1}{4} = k = 2 \cdot j$ é par, então, $\frac{p-1}{4} \cdot \frac{p+1}{2} = \frac{p^2 - 1}{8}$ também é par, logo 
\begin{equation*}
    \left(\frac{2}{p}\right) = (-1)^{\frac{p-1}{4}} = (-1)^{2 \cdot j} = 1 = (-1)^{\frac{p^2 - 1}{8}}
\end{equation*}
Agora, caso $p \equiv 5 \pmod{8}$ (que é a outra única possibilidade no caso de $p \equiv 1 \pmod{4}$) tem-se que $p \equiv -3 \pmod{8}$, portanto existe $j \in \mathbb{Z}$ tal que:
\begin{align*}
    p \equiv -3 \pmod{8}
    \Longleftrightarrow \; & p = 8 \cdot j - 3
    \\
    \Longleftrightarrow \; & 4 \cdot k + 1 = 8 \cdot j - 3
    \\
    \Longleftrightarrow \; & 4 \cdot k = 8 \cdot j - 4
    \\
    \Longleftrightarrow \; & k = 2 \cdot j - 1
\end{align*}
Novamente, aplicando o Lema \ref{lema:gauB} para $a = 2$ tem-se $\left(\frac{2}{p} \right) = (-1)^s$, onde $s = \frac{p-1}{4}$, e como $\frac{p-1}{4}$ é ímpar (pois $\frac{p-1}{4} = k = 2 \cdot j - 1$):
\begin{align*}
    \left(\frac{2}{p} \right) = (-1)^{\frac{p-1}{4}} = -1
\end{align*}
E como $2 \cdot k + 1 = \frac{p-1}{2} + 1 = \frac{p+1}{2}$ é ímpar, então $\frac{p-1}{4} \cdot \frac{p+1}{2} = \frac{p^2 - 1}{8}$ também é ímpar (pois resulta da multiplicação de números ímpares), ou seja,
\begin{align*}
    \left(\frac{2}{p} \right) = (-1)^{\frac{p-1}{4}} = -1 = (-1)^{\frac{p^2-1}{8}}
\end{align*}
Logo, se concluí que:
\begin{equation} \label{eq:casesrecip1}
    \left(\frac{2}{p} \right) = (-1)^{\frac{p^2 - 1}{8}} = \begin{cases}
            1 \text{ se $p \equiv 1 \pmod{8}$}
            \\
            -1 \text{ se $p \equiv -3 \pmod{8}$}
            \end{cases}
\end{equation}
Por fim, para o caso de $p \equiv 3 \pmod{4}$, existe $k \in \mathbb{Z}$ tal que $p = 4 \cdot k + 3$ e por consequência, $\frac{p-1}{2} = 2 \cdot k + 1$. Assim, pelo Lema \ref{lema:gauB}, note antes que $\frac{p-1}{4}$ não é inteiro e o inteiro mais próximo (e maior) deste valor é $\frac{1}{2} \cdot (\frac{p-1}{2} + 1) = \frac{p+1}{4} = k + 1$, portanto para $k + 1 = \frac{p+1}{4} \leq i \leq \frac{p-1}{2} = 2 \cdot k + 1$ tem-se
$\frac{p-1}{2} < 2 \cdot i \leq p-1$, logo há 
$2 \cdot k + 1 - (k + 1) + 1 = k + 1 = \frac{p-1}{2} - \frac{p+1}{4} + 1 = \frac{p + 1}{4}$ restos maiores que $\frac{p-1}{2}$, então $s = \frac{p+1}{4}$.

Tratando então da possibilidade de que $p \equiv 3 \pmod{8}$, em que então existe  $j \in \mathbb{Z}$ tal que:
\begin{align*}
    p \equiv 3 \pmod{8}
    \Longleftrightarrow \; & p = 8 \cdot j + 3
    \\
    \Longleftrightarrow \; & 4 \cdot k + 3 = 8 \cdot j + 3
    \\
    \Longleftrightarrow \; & 4 \cdot k = 8 \cdot j 
    \\
    \Longleftrightarrow \; & k = 2 \cdot j
\end{align*}
Assim, $s = \frac{p+1}{4} = k + 1 = 2 \cdot j + 1 $, portanto $\frac{p+1}{4}$ é ímpar, e então:
\begin{equation*}
    \left(\frac{2}{p} \right) = (-1)^s = (-1)^{\frac{p+1}{4}} = -1
\end{equation*}
Como visto anteriormente $\frac{p-1}{2} = 2 \cdot k + 1$, logo $\frac{p-1}{2}$ é ímpar e então $\frac{p+1}{4} \cdot \frac{p-1}{2} = \frac{p^2-1}{8}$ é ímpar também (pois resulta do produto entre números ímpares), assim:
\begin{equation*}
    \left(\frac{2}{p} \right) = (-1)^s = (-1)^{\frac{p+1}{4}} = -1 = (-1)^{\frac{p^2-1}{8}}
\end{equation*}
Agora, se for o caso em que $p \equiv 7 \pmod{8}$ (pois $p \equiv 3 \pmod{4}$), então $p \equiv -1 \pmod{8}$, logo existe $j \in \mathbb{Z}$ tal que:
\begin{align*}
    p \equiv -1 \pmod{8}
    \Longleftrightarrow \; & p = 8 \cdot j + 1
    \\
    \Longleftrightarrow \; & 4 \cdot k + 3 = 8 \cdot j -1
    \\
    \Longleftrightarrow \; & 4 \cdot k = 8 \cdot j - 4
    \\
    \Longleftrightarrow \; & k = 2 \cdot j - 1
\end{align*}
Então, $s = \frac{p+1}{4} = k + 1 = 2 \cdot j$, portanto $\frac{p+1}{4}$ é par, logo:
\begin{equation*}
    \left(\frac{2}{p} \right) = (-1)^s = (-1)^{\frac{p+1}{4}} = 1
\end{equation*}
 Finalmente, $\frac{p+1}{4} \cdot \frac{p-1}{2} = \frac{p^2-1}{8}$ é também par (pois resulta do produto entre um número par e um número qualquer), assim:
\begin{equation*}
    \left(\frac{2}{p} \right) = (-1)^s = (-1)^{\frac{p+1}{4}} = 1 = (-1)^{\frac{p^2-1}{8}}
\end{equation*} 
Em que então se conclui que:
\begin{equation} \label{eq:casesrecip2}
    \left(\frac{2}{p} \right) = (-1)^{\frac{p^2 - 1}{8}} = \begin{cases}
            1 \text{ se $p \equiv -1 \pmod{8}$}
            \\
            -1 \text{ se $p \equiv 3 \pmod{8}$}
            \end{cases}
\end{equation}
Juntando \ref{eq:casesrecip1} e \ref{eq:casesrecip2}, tem-se:
 \begin{equation*}
    \left(\frac{2}{p} \right) = (-1)^{\frac{p^2 - 1}{8}} = \begin{cases}
            1 \text{ se $p \equiv \pm 1 \pmod{8}$}
            \\
            -1 \text{ se $p \equiv \pm 3 \pmod{8}$}
            \end{cases}
\end{equation*}
Portanto está provado o Item \ref{item:recipquad2}. \qed
