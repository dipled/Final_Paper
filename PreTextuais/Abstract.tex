% ---
% Abstract
% ---
% Na medida em que sistemas de computação modernos escalam, a existência de informações
% contraditórias torna-se inevitável. As lógicas ortodoxas não são capazes de tratar este tipo
% de informação sem que o princípio da explosão tome lugar. Com isso, o estabelecimento de
% sistemas paraconsistentes — sistemas nos quais a explosividade é cuidadosamente separada
% da contraditoriedade — é uma alternativa vantajosa quando comparados às lógicas ortodoxas.
% Neste contexto, as lógicas de inconsistência formal, sobretudo a LFI1, usufruem de propriedades
% interessantes que as garantem aplicações em diversos campos do conhecimento, como, por
% exemplo, no desenvolvimento de sistemas de gerenciamento de bancos de dados. Com isso, a
% prova de metateoremas sobre estas lógicas evidencia características das diferentes abordagens
% possíveis no estudo destes sistemas. Ademais, assistentes de provas como o Coq proporcionam
% aos teoremas neles desenvolvidos uma garantia de correção dificilmente encontrada em provas
% manuais. Este trabalho propõe explanar e definir a lógica de inconsistência formal LFI1, bem
% como desenvolver metateoremas para este sistema no assistente de provas Coq.
% resumo em inglês
\begin{resumo}[Abstract]
 \begin{otherlanguage*}{english}
    As modern computer systems scale, the existence of contradictory information becomes inevitable. Most orthodox logics are not able to cope with this kind of information without the principle of explosion taking place. Thus, establishing paraconsistent systems {--} systems in which explosiveness is carefully separated from contradictoriness {--} is an advantageous alternative compared to orthodox logics. From this perspective, the logics of formal inconsistency, specially \lfium{}, enjoy some interesting properties which allow them to be used in many areas, for example in the development of database management systems. In this light, proving metatheorems about these logics highlights characteristics of different approaches when studying these systems. Furthermore, proof assistants, such as Coq, guarantee the theorems proved inside them a degree of certainty about their correctness hardly ever found in manual proofs. The present work explores and defines the logic of formal inconsistency \lfium{}, as well as proves metatheorems for this system inside the Coq proof assistant.

   \textbf{Keywords}: Coq, paraconsistent logic, \lfium{}, logics of formal inconsistency, three-valued logic.
 \end{otherlanguage*}
\end{resumo}

