% ---
% Abstract
% ---

% resumo em inglês
\begin{resumo}[Abstract]
 \begin{otherlanguage*}{english}
	It is possible to model software components in logical systems, however complex software systems are not easily modelled in simple logical languages.
	To obtain a logical language capable of expressing complex properties, combining simpler logical systems is a possibility.
	One of the ways to combine logical systems is by means of fusions of modal logics. However, any kind of combination of logics is a complex task, that
	requires logical system to be treated as mathematical objects for they then be able to be combined. Modelling logical systems in proof assistants can
	make the combination process easier, since, in proof assistants, logical systems are already treated as mathematical objects. Moreover,
	proof assistants such as \textit{Coq} and \textit{Lean} have a large amount of automation tools that help in the process of developing proofs.
	As such, this work proposes to parametrically model multimodal logicas resulting from the fusion of logical systems, in the Coq proof assistant,
	and to prove the preservation of some properties of logics by the operation of fusion.
	After a review of the relevant literature, a case study of fusion of modal logics in Coq was modeled and the fusion of modal logics was  was
	parametrically implemented, where it was possible to parametrically model the fusion of syntactic systems of any modal logics.

   \textbf{Keywords}: Coq. Proof Assistants. Multimodal Logics. Fusion of Logics.
 \end{otherlanguage*}
\end{resumo}

