% ---
% Agradecimentos
% ---
\begin{agradecimentos}

    Sinto que poderia escrever dez vezes mais que escrevi aqui, mas
    espero que o pouco que escrevi seja o suficiente para expressar minha gratidão.
    %mas

    Primeiro devo agradecer aos muitos amigos que fiz durante a graduação, cuja amizade foi e ainda é muito importante para mim,
    %em especial agradeço
    %à Duda, à Ale, ao Dunzer, à Grasi, à Dan, à Elo, ao Pandini, ao Rodrigo, ao Gilson, ao Thiago, ao Kalyl, ao André, ao Jaasiel, ao Danilo, à Lais e ao Jota
    pela sua amizade e carinho imensuráveis, por sua companhia durante os dois anos de isolamento social, por todas as conversas, gargalhadas, jogatinas até tarde da noite,
    ajuda nas matérias, almoços juntos, conversas sem começo nem fim no centro acadêmico, noites bebendo, idas ao Kelson após os colóquios e por todos os inesquecíveis e
    incontáveis bons momentos que passei com vocês. Vocês tornaram meu tempo na Udesc melhor do que eu jamais poderia imaginar e me ajudaram quando mais precisava.

    Agradeço aos professores e organizadores do SPLogIC, por me escolherem para participar e por tudo que me ensinaram,
    e a todos os amigos que fiz durante o evento %, ao Robert, ao Bismarck, a ambos os Mayks, ao Filipe, à Paloma, à Tânia, à Jéssica e à Yasmim
    por toda sua amizade, por terem tornado aquelas duas semanas em Campinas ainda mais maravilhosas que já eram e por tudo que aprendi com vocês, durante e depois do evento.

    Agradeço muito aos meus amigos do grupo Função, por todas as conversas nos corredores e no grupo de Telegram, todos os colóquios, todos os cafés,
    todas as partidas de truco e poker, todas as idas ao Clover, todas as videochamadas durante o isolamento social que me trouxeram imensa alegria, todos os momentos
    no nosso apertado e bagunçado laboratório, todos os bons momentos que passei junto de vocês e tudo que eu aprendi com vocês. Vocês são incríveis, nunca teria imaginado
    que encontraria um grupo de amigos como vocês.

    Também devo agradecer a minha banca, Professores Cristiano e Kariston, pelo interesse no meu trabalho e no esforço para ler e avaliar ele,
    por tudo que aprendi com vocês e por todo o tempo que cada um dedicou a mim, tanto como parte deste trabalho quanto antes de chegar aqui.

    Agradeço ao meu coorientador, Professor Paulo, por todas as tardes e noites que me fez companhia no laboratório, por todo o tempo que dedicou para me ajudar com problemas
    da IC e do TCC e por tudo que me ensinou, do seu jeito bem peculiar.

    Devo agradecer à minha orientadora, Professora Karina, por me orientar e guiar de pouquinho em pouquinho até onde estou hoje,
    por todas as folhas de papel que rabiscou para me explicar o que quer que fosse que eu lhe perguntasse, por todo o imensurável tempo que dedicou a mim, por sempre arranjar
    um momento para me atender independentemente do quão atarefada estivesse ou do que quer que eu quisesse, por todos os cafés, por todos os almoços, todos os abraços,
    todo o seu carinho e afeto, por me perdoar após ter pronunciado errado seu sobrenome no vídeo do SIC, por ser uma grande inspiração para mim e por, para minha contínua surpresa,
    não ter enlouquecido com tudo que aconteceu nesses tantos anos que nos conhecemos. Obrigado por tudo. Se não fosse por você, eu nunca teria chegado até aqui.

    Por fim, devo agradecer a Udesc por tudo que aprendi e vivenciei aqui, apesar dos pesares, e ao Conselho Nacional de Desenvolvimento Científico e Tecnológico - CNPq
    pela bolsa de modalidade ITC-A de número 409707/2022-8 que apoiou este trabalho.
\end{agradecimentos}
% ---