% ---
% RESUMOS
% ---

% resumo em português
\setlength{\absparsep}{18pt} % ajusta o espaçamento dos parágrafos do resumo
\begin{resumo}
    É possível modelar componentes de sistemas de \textit{software} em sistemas lógicos,
    porém sistemas de \textit{software} complexos não são facilmente modelados em linguagens simples.
    Para obter uma linguagem lógica que seja capaz de expressar propriedades complexas pode-se combinar sistemas lógicos mais simples.
    Uma das formas de combinar lógicas é a fusão de lógicas modais. Porém, qualquer tipo de combinação de lógicas é uma tarefa complexa, que requer
    que sistemas lógicos sejam tratados como objetos matemáticos para então serem combinados entre si. Modelar sistemas lógicos
    em assistentes de provas pode facilitar o processo de combinação de lógicas pois, em assistentes de provas, sistemas lógicos já são tratados
    como objetos matemáticos. Mais ainda, assistentes como \textit{Coq} e \textit{Lean} possuem grande gama de ferramentas de automação
    que facilitam o processo de desenvolvimento de provas.
    Este trabalho propõe modelar de forma paramétrica lógicas multimodais resultantes da fusão de sistemas lógicos,
    no assistentes de provas Coq, e provar a preservação de algumas propriedades de lógicas pela operação de fusão.
    Após uma revisão da literatura relevante, foi modelado um estudo de caso de fusões de lógicas modais no Coq e foi implementado de forma
    paramétrica a fusão de lógicas modais, onde foi possível modelar, de forma paramétrica, a fusão de sistemas sintáticos de lógicas modais quaisquer.

 \textbf{Palavras-chave}: Coq. Assistentes de Provas. Lógicas Multimodais. Fusão de Lógicas.
\end{resumo}
