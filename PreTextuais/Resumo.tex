% ---
% RESUMOS
% ---

% resumo em português
\setlength{\absparsep}{18pt} % ajusta o espaçamento dos parágrafos do resumo
\begin{resumo}
    Na medida em que sistemas de computação modernos escalam, a existência de informações contraditórias torna-se inevitável. As lógicas ortodoxas não são capazes de tratar este tipo de informação sem que o princípio da explosão tome lugar. Com isso, o estabelecimento de sistemas paraconsistentes {---} sistemas nos quais a explosividade é cuidadosamente separada da contraditoriedade {---} é uma alternativa vantajosa quando comparados às lógicas ortodoxas. Neste contexto, as lógicas de inconsistência formal, sobretudo a \lfium{}, usufruem de propriedades interessantes que as garantem aplicações no desenvolvimento de sistemas de gerenciamento de bancos de dados. \cortar{Com isso, a prova de metateoremas sobre estas lógicas evidencia características das diferentes abordagens possíveis no estudo destes sistemas.} Ademais, assistentes de provas como o Coq proporcionam aos teoremas neles desenvolvidos uma garantia de correção dificilmente encontrada em provas manuais. Este trabalho propõe explanar e definir a lógica de inconsistência formal \lfium{}, bem como \migs{modelar(/implementar/codificar/ algum desses a gosto do freguês) e} desenvolver metateoremas para este sistema no assistente de provas Coq.

 \textbf{Palavras-chave}: Coq, Lógica paraconsistente, LFI1, Lógica de Inconsistência Formal, Lógica Trivalorada, Bancos de Dados.
\end{resumo}
