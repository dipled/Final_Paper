\documentclass[
    12pt,					% tamanho da fonte
    openright,				% capítulos começam em pág ímpar (insere página vazia caso preciso)
    oneside,				% para impressão em recto e verso (twoside). Oposto a (oneside)
    a4paper,				% tamanho do papel.
    chapter=TITLE,			% títulos de capítulos convertidos em letras maiúsculas
    section=TITLE,			% títulos de seções convertidos em letras maiúsculas
    sumario=abnt-6027-2012, %
    english,				% idioma adicional para hifenização
    brazil,					% o último idioma é o principal do documento
    fleqn,					% equações alinhadas a esquerda (UDESC/CCT)+
    ]{abntex2}

% ----------------------------------------------------------
% Pacotes básicos
% ----------------------------------------------------------
% \usepackage{Pacotes/abntex2}
\usepackage{amsmath, amssymb, amsfonts, amsthm, mathtools}
\usepackage{mathptmx} 							% Usa a fonte Times New Roman	 (UDESC/CCT)
\usepackage[T1]{fontenc}						% Seleção de códigos de fonte.
\usepackage[utf8]{inputenc}						% Codificação do documento (conversão automática dos acentos)
\usepackage{lastpage}							% Usado pela Ficha catalográfica
\usepackage{indentfirst}						% Indenta o primeiro parágrafo de cada seção.
\usepackage[dvipsnames,table]{xcolor}			% Controle das cores
\usepackage{graphicx}							% Inclusão de gráficos
\usepackage{microtype} 							% para melhorias de justificação
\usepackage[brazilian,hyperpageref]{backref}	% Paginas com as citações na bibl
\usepackage[alf,abnt-emphasize=bf,abnt-full-initials=yes]{abntex2cite}					% Citações padrão ABNT
\usepackage{adjustbox}							% Pacote de ajuste de boxes
\usepackage{subcaption}							% Inclusão de Subfiguras e sublegendas
\usepackage{enumitem}							% Personalização de listas
\usepackage[section]{placeins}					% Manter as figuras delimitadas na respectiva seção com a opção [section]
\usepackage{multirow}							% Multi colunas nas tabelas
\usepackage{array,tabularx} 					% Pacotes de tabelas
\usepackage{booktabs}							% Pacote de tabela profissional
\usepackage{rotating}							% Rotacionar figuras e tabelas
\usepackage{xfrac}								% Fazer frações n/d em linha
\usepackage{bm}									% Negrito em modo matemático
\usepackage{xstring}							% Manipulação de strings
\usepackage{chngcntr}							% Pacte usado para deixar numeração de equações sequencial (UDESC/CCT)
\usepackage{xspace}								% Espaçamento após comandos
\usepackage{bussproofs}							% Provas em estilo de árvore
\usepackage{hyperref}							% Suporte para hiperlinks
\usepackage{listings, lstCoq}					% Highlighting de código Coq
\usepackage{tikz}								% Imagens e Diagramas
\usepackage{ulem}								% Texto strikethrough
\usepackage{calrsfs}

\definecolor{darkgreen}{rgb}{0,0.6,0}
\definecolor{darkorange}{rgb}{0.8,0.4,0.2}
\definecolor{darkred}{rgb}{0.8,0,0}

\usetikzlibrary{arrows.meta,positioning,matrix}

\counterwithout{equation}{chapter}
% fonte: https://latex.org/forum/viewtopic.php?t=15392

% Comando para deixar numeração das equações contínua (1), (2), (3)... ao invés de organizar por capítulos (1.1)(1.2)... (2.1)(2.2)
%\renewcommand{\theequation}{\arabic{equation}}

%\numberwithin{equation}{section}


% Cabeçalho somente com numeração de página 10pt
\makepagestyle{PagNumReduzida}
\makeevenhead{PagNumReduzida}{\ABNTEXfontereduzida\thepage}{}{}
\makeoddhead{PagNumReduzida}{}{}{\ABNTEXfontereduzida\thepage}
%fonte: https://github.com/abntex/abntex2/wiki/HowToCustomizarCabecalhoRodape
%fonte: Manual memoir seção 7.3 pg. 111 pdf http://linorg.usp.br/CTAN/macros/latex/contrib/memoir/memman.pdf

% Personalização das opções das listas
\setlist[itemize]{leftmargin=\parindent}

% Citação online --- MODIFICAR ---
\newcommand{\citeshort}[1]{\citeauthoronline{#1} (\citeyear{#1})}

\newcommand{\me}{Elaborado pelo autor.}

% Configuração do pgfplots
% \pgfplotsset{compat=newest} %compat=1.14
% \pgfplotsset{plot coordinates/math parser=false}
% \newlength\figureheight
% \newlength\figurewidth

% Libraries do TiKz
% \usetikzlibrary{quotes,angles,arrows}
% \usetikzlibrary{through,calc,math}
% \usetikzlibrary{graphs,backgrounds,fit}
% \usetikzlibrary{shapes,positioning,patterns,shadows}
% \usetikzlibrary{decorations.pathreplacing}
% \usetikzlibrary{shapes.geometric}
% \usetikzlibrary{arrows.meta}
% \usetikzlibrary{external}

%\tikzexternalize[]
%\tikzexternalenable
%\tikzexternalize
%\tikzexternaldisable
%\tikzset{external/force remake}
%\tikzexternalize[shell escape=-enable-write18]

% Personalização das legendas
\usepackage[format = plain, %hang
			justification = centering,
			labelsep = endash,
			singlelinecheck = false,
			skip = 6pt,
			listformat = simple]{caption}

% Personalizações de tipo de colunas de tabelas
\newcolumntype{L}[1]{>{\raggedright\let\newline\\\arraybackslash\hspace{0pt}}m{#1}}
\newcolumntype{C}[1]{>{\centering\let\newline\\\arraybackslash\hspace{0pt}}m{#1}}
\newcolumntype{R}[1]{>{\raggedleft\let\newline\\\arraybackslash\hspace{0pt}}m{#1}}

% Personalizações de cores da UDESC
\definecolor{CapaAmareloUDESC}{RGB}{243,186,83}		% Especialização
\definecolor{CapaVerdeUDESC}{RGB}{0,112,52}			% Mestrado
\definecolor{CapaVermelhoUDESC}{RGB}{171,35,21}		% Doutorado
\definecolor{CapaAzulUDESC}{RGB}{38,54,118} 		% Pós-Doutorado

% CONFIGURAÇÕES DE PACOTES
% Configurações do pacote backref
% Usado sem a opção hyperpageref de backref
\renewcommand{\backrefpagesname}{Citado na(s) página(s):~}
% Texto padrão antes do número das páginas
\renewcommand{\backref}{}
% Define os textos da citação
\renewcommand*{\backrefalt}[4]{
	\ifcase #1 %
	Nenhuma citação no texto.%
	\or
	Citado na página #2.%
	\else
	Citado #1 vezes nas páginas #2.%
	\fi}%

% alterando o aspecto da cor azul
%\definecolor{blue}{RGB}{41,5,195}

% informações do PDF
\makeatletter
\hypersetup{
	%pagebackref=true,
	pdftitle={\@title},
	pdfauthor={\@author},
	pdfsubject={\imprimirpreambulo},
	pdfcreator={LaTeX with abnTeX2},
	pdfkeywords={abnt}{latex}{abntex}{abntex2}{trabalho academico},
	colorlinks=true,       		% false: boxed links; true: colored links
	linkcolor=blue,          	% color of internal links
	citecolor=blue,        	% color of links to bibliography
	filecolor=blue,      		% color of file links
	urlcolor=blue,			% color of external links
	bookmarksdepth=4
}
\makeatother


% \makeatletter
% \newcommand{\includetikz}[1]{%
% 	\tikzsetnextfilename{#1}%
% 	\input{#1.tex}%
% }
% \makeatother

% ---
% Possibilita criação de Quadros e Lista de quadros.
% Ver https://github.com/abntex/abntex2/issues/176
%
\newcommand{\quadroname}{Quadro}
\newcommand{\listofquadrosname}{Lista de quadros}

\newfloat[chapter]{quadro}{loq}{\quadroname}
\newlistof{listofquadros}{loq}{\listofquadrosname}
\newlistentry{quadro}{loq}{0}

% configurações para atender às regras da ABNT
\setfloatadjustment{quadro}{\centering}
\counterwithout{quadro}{chapter}
\renewcommand{\cftquadroname}{\quadroname\space}
\renewcommand*{\cftquadroaftersnum}{\hfill--\hfill}

\setfloatlocations{quadro}{hbtp} % Ver https://github.com/abntex/abntex2/issues/176
% ---


% Espaçamento depois do título
\setlength{\afterchapskip}{0.7\baselineskip}
% O tamanho do parágrafo é dado por:
\setlength{\parindent}{1.25cm}
% Controle do espaçamento entre um parágrafo e outro:
\setlength{\parskip}{0.0cm}  % tente também \onelineskip
%\SingleSpacing % Espaçamento simples
\OnehalfSpacing % Espaçamento 1,5 (UDESC/CCT)
%\DoubleSpacing	% Espaçamento duplo

% ---
% Margens - NBR 14724/2011 - 5.1 Formato
% ---
\setlrmarginsandblock{3cm}{2cm}{*}
\setulmarginsandblock{3cm}{2cm}{*}
\checkandfixthelayout[fixed]
% ---


% To use externalize consider
%https://tex.stackexchange.com/questions/182783/tikzexternalize-not-compatible-with-miktex-2-9-abntex2-package
%Lauro Cesar digged into the problem until he came with a solution for me to test. And it Works!
%
%According to this link:
%
%The package calc changed the commands \setcounter and friends to be fragile.
% So you have to make them robust. The example below uses etoolbox with \robustify:
%
\usepackage{etoolbox}
\robustify\setcounter
\robustify\addtocounter
\robustify\setlength
\robustify\addtolength


%% How to silence memoir class warning against the use of caption package?
%% https://tex.stackexchange.com/questions/391993/how-to-silence-memoir-class-warning-against-the-use-of-caption-package
%\usepackage{silence}
%\WarningFilter*{memoir}{You are using the caption package with the memoir class}
%\WarningFilter*{Class memoir Warning}{You are using the caption package with the memoir class}

% --------------------------------------------------------
% INICIO DAS CUSTOMIZACOES PARA A UDESC
% --------------------------------------------------------

% --------------------------------------------------------
% Fontes padroes de part, chapter, section, subsection e subsubsection
% --------------------------------------------------------
% --- Chapter ---
\renewcommand{\ABNTEXchapterfont}{\fontseries{b}} %\bfseries
\renewcommand{\ABNTEXchapterfontsize}{\normalsize}
% --- Part ---
\renewcommand{\ABNTEXpartfont}{\ABNTEXchapterfont}
\renewcommand{\ABNTEXpartfontsize}{\LARGE}
% --- Section ---
\renewcommand{\ABNTEXsectionfont}{\normalfont}
\renewcommand{\ABNTEXsectionfontsize}{\normalsize}
% --- SubSection ---
\renewcommand{\ABNTEXsubsectionfont}{\fontseries{b}} %\bfseries
\renewcommand{\ABNTEXsubsectionfontsize}{\normalsize}
% --- SubSubSection ---
\renewcommand{\ABNTEXsubsubsectionfont}{\itshape}
\renewcommand{\ABNTEXsubsubsectionfontsize}{\normalsize}

\renewcommand{\ABNTEXsubsubsubsectionfont}{\normalfont}
\renewcommand{\ABNTEXsubsubsubsectionfontsize}{\normalsize}
% ---

% --------------------------------------------------------
% Fontes das entradas do sumario
% --------------------------------------------------------

\renewcommand{\cftpartfont}{\ABNTEXpartfont\selectfont}
\renewcommand{\cftpartpagefont}{\normalsize\selectfont}

\renewcommand{\cftchapterfont}{\ABNTEXchapterfont\selectfont}
\renewcommand{\cftchapterpagefont}{\normalsize\selectfont}

\renewcommand{\cftsectionfont}{\ABNTEXsectionfont\selectfont}
\renewcommand{\cftsectionpagefont}{\normalsize\selectfont}

\renewcommand{\cftsubsectionfont}{\ABNTEXsubsectionfont\selectfont}
\renewcommand{\cftsubsectionpagefont}{\normalsize\selectfont}

\renewcommand{\cftsubsubsectionfont}{\normalfont\itshape\selectfont}
\renewcommand{\cftsubsubsectionpagefont}{\normalsize\selectfont}

\renewcommand{\cftparagraphfont}{\normalfont\selectfont}
\renewcommand{\cftparagraphpagefont}{\normalsize\selectfont}

% --------------------------------------------------------
% Usando os pacotes hyperref, uppercase...
% Para deixar a section do toc uppercase precisa de:
% --------------------------------------------------------
\usepackage{textcase}

\makeatletter

\let\oldcontentsline\contentsline
\def\contentsline#1#2{%
	\expandafter\ifx\csname l@#1\endcsname\l@section
	\expandafter\@firstoftwo
	\else
	\expandafter\@secondoftwo
	\fi
	{%
		\oldcontentsline{#1}{\MakeTextUppercase{#2}}%
	}{%
		\oldcontentsline{#1}{#2}%
	}%
}
\makeatother

% --------------------------------------------------------
% Renomenando as entradas de APÊNDICES E ANEXOS
% --------------------------------------------------------

\renewcommand{\apendicesname}{AP\^ENDICES}
\renewcommand{\anexosname}{ANEXOS}


% Manipulação de Strings
%\RequirePackage{xstring}

% Comando para inverter sobrenome e nome
\newcommand{\invertname}[1]{%
	\StrBehind{#1}{{}}, \StrBefore{#1}{{}}%
}%


% --------------------------------------------------------
% Alterando os estilos de Caption e Fonte
% --------------------------------------------------------
\makeatletter
% Define o comando \fonte que respeita as configurações de caption do memoir ou do caption
\renewcommand{\fonte}[2][\fontename]{%
	\M@gettitle{#2}%
	\memlegendinfo{#2}%
	\par
	\begingroup
	\@parboxrestore
	\if@minipage
	\@setminipage
	\fi
	\ABNTEXfontereduzida
	\configureseparator
	\captiondelim{\ABNTEXcaptionfontedelim}
	\@makecaption{#1}{\ignorespaces #2}\par
	\endgroup}


\captionstyle[\raggedright]{\raggedright}

\makeatother

\setlength{\cftbeforechapterskip}{0pt plus 0pt}
\renewcommand*{\insertchapterspace}{}

\newlength{\mylen}	% New length to use with spacing
\setlength{\mylen}{1pt}

\setlength{\cftbeforechapterskip}{\mylen}
\setlength{\cftbeforesectionskip}{\mylen}
\setlength{\cftbeforesubsectionskip}{\mylen}
\setlength{\cftbeforesubsubsectionskip}{\mylen}
\setlength{\cftbeforesubsubsubsectionskip}{\mylen}


% ---
% Ajuste das listas de abreviaturas e siglas ; e símbolos [Personalizada para UDESC com espaçamento 1,5]
% ---

% ---
% Redefinição da Lista de abreviaturas e siglas [Personalizada para UDESC com espaçamento 1,5]
\renewenvironment{siglas}{%
	\pretextualchapter{\listadesiglasname}
	\begin{symbols}
		\setlength{\itemsep}{0pt}	% Ajuste para Espaçamento 1,5 (UDESC/CCT)
	}{%
	\end{symbols}
	\cleardoublepage
}
% ---

% ---
% Redefinição da Lista de símbolos [Personalizada para UDESC com espaçamento 1,5]
\renewenvironment{simbolos}{%
	\pretextualchapter{\listadesimbolosname}
	\begin{symbols}
		\setlength{\itemsep}{0pt}	% Ajuste para Espaçamento 1,5 (UDESC/CCT)
	}{%
	\end{symbols}
	\cleardoublepage
}
% ---

% ---
% FIM DAS CUSTOMIZAÇÕES PARA A  Universidade do Estado de Santa Catarina - UDESC/CCT
% ---	% Inclui pacotes básicos

\graphicspath{{Texto/Figuras/}}
\renewcommand{\orientadorname}{Orientadora:}

% O pacote bm torna a fonte do mathcal feia, a linha abaixo desfaz isso
% (Se tirar o bm caga tudo, então ele fica)
\DeclareMathAlphabet{\mathcal}{OMS}{cmsy}{m}{n}
% Definindo fonte caligráfica e negrita
\DeclareMathAlphabet\mathbfcal{OMS}{cmsy}{b}{n}

% -----------------------------------------------------------------
% Informações de dados para CAPA e FOLHA DE ROSTO
% -----------------------------------------------------------------
\titulo{Implementação de uma biblioteca da Lógica de Inconsistência Formal LFI1 em Coq}%


\autor{Helena Vargas {}Tannuri}%
\orientador{Karina Girardi {}Roggia}%
\coorientador{Miguel Alfredo {}Nunes}%

\instituicao{Universidade do Estado de Santa Catarina, Centro de Ciências Tecnológicas, Bacharelado em Ciência da Computação}%

\tipotrabalho{Trabalho de Conclusão de Curso}

\preambulo{Trabalho de conclusão de curso submetido à Universidade do Estado de Santa Catarina
como parte dos requisitos para a obtenção do grau de Bacharel em Ciência da Computação}

\local{Joinville}%

\data{\the\year}%

\makeindex

% -----------------------------------------------------------------
% Início do documento
% -----------------------------------------------------------------
\begin{document}

\selectlanguage{brazil}

% Teoremas e Provas

% Estilo padrão, i.e., texto itálico
\newtheorem{teorema}   {Teorema}
\newtheorem{proposicao}{Proposição}
\newtheorem{lema}      {Lema}
\newtheorem{corolario} {Corolário}
\renewcommand{\proofname}{Prova}
\renewcommand\qedsymbol{$\blacksquare$}

% Estilo simples, i.e., texto normal
\theoremstyle{definition}
\newtheorem{definicao}{Definição}
\newtheorem{exemplo}  {Exemplo}

\AtEndEnvironment{teorema}  {\qed}%
\AtEndEnvironment{proposicao}  {\qed}%
% \AtEndEnvironment{lema}     {\qed}%
% \AtEndEnvironment{definicao}{\qed}%
% \AtEndEnvironment{exemplo}  {\qed}%

%% Grego


% Grego Minúsculo
\newcommand{\PHI}{\(\phi\)\xspace}
\newcommand{\PSI}{\(\psi\)\xspace}
\newcommand{\PI}{\(\pi\)\xspace}
\newcommand{\OPI}{\(\overline{\pi}\)\xspace}
\newcommand{\mOPI}{\overline{\pi}\xspace}
\newcommand{\ALPHA}{\(\alpha\)\xspace}
\newcommand{\BETA}{\(\beta\)\xspace}
\newcommand{\DELTA}{\(\delta\)\xspace}

% Grego Maiúsculo
\newcommand{\GAMMA}{\(\Gamma\)\xspace}
\newcommand{\DDELTA}{\(\Delta\)\xspace}
\newcommand{\SIGMA}{\(\Sigma\)\xspace}
\newcommand{\THETA}{\(\Theta\)\xspace}
\newcommand{\LAMBDA}{\(\Lambda\)\xspace}
\newcommand{\LAMBDAlm}{\(\Lambda_{\mathsf{LM}}\)\xspace}
\newcommand{\Lambdalm}{\Lambda_{\mathsf{LM}}}

% Modalidades e símbolos matemáticos

\newcommand{\VVDASH}{\(\Vdash\)\xspace}
\newcommand{\VDDASH}{\(\vDash\)\xspace}
\newcommand{\VDASH}{\(\vdash\)\xspace}

\newcommand{\ODOT}{\(\odot\)\xspace}
\newcommand{\OPLUS}{\(\oplus\)\xspace}
\newcommand{\OTIMES}{\(\otimes\)\xspace}
\newcommand{\OMINUS}{\(\ominus\)\xspace}

\newcommand{\tofrom}{\leftrightarrow}

\newcommand{\eqdef}{\mathrel{\overset{\makebox[0pt]{\mbox{\normalfont\tiny\sffamily def}}}{=}}}

% Fontes
\newcommand{\Mathcal}[1]{\(\mathcal{#1}\)\xspace}
\newcommand{\Mathcali}[2]{\(\mathcal{#1}_{#2}\)\xspace}
\newcommand{\MathcalI}[2]{\(\mathcal{#1}^{#2}\)\xspace}
\newcommand{\Mathcalii}[3]{\(\mathcal{#1}{#2}_{#3}\)\xspace}

\newcommand{\Mathfrak}[1]{\(\mathfrak{#1}\)\xspace}
\newcommand{\Mathfraki}[2]{\(\mathfrak{#1}_{#2}\)\xspace}
\newcommand{\MathfrakI}[2]{\(\mathfrak{#1}^{#2}\)\xspace}

\newcommand{\Mathbb}[1]{\(\mathbb{#1}\)\xspace}
\newcommand{\Mathbbi}[2]{\(\mathbb{#1}_{#2}\)\xspace}
\newcommand{\MathbbI}[2]{\(\mathbb{#1}{#2}\)\xspace}

% Comentários
\newcommand{\cortar}[1]{\textcolor{red}{\sout{#1}}}
\newcommand{\ignore}[1]{\textcolor{blue}{\textbf{IGNOREM:} #1}}
\newcommand{\helena}[1]{\textcolor{green}{\textbf{HELENA:} #1}}
\newcommand{\migs}[1]{\textcolor{violet}{\textbf{MIGS:} #1}}
\newcommand{\migscortar}[2]{\textcolor{violet}{\textbf{MIGS:} \sout{#1}{#2}}}
\newcommand{\kaqui}[1]{\textcolor{magenta}{\textbf{KAQUI:} #1}}

% Outros
\newcommand{\linguagem}[1]{\(\mathsf{LM}_{#1}\)\xspace}
\newcommand{\Linguagem}[1]{\mathsf{LM}_{#1}\xspace}
\newcommand{\funcao}[1]{\operatorname{#1}\xspace}
\newcommand{\inlinecoq}[1]{\lstinline[columns=fixed,language=coq]{#1}}

\newcommand{\Odot}  {\mathbin{\odot}}
\newcommand{\Oplus} {\mathbin{\oplus}}
\newcommand{\Otimes}{\mathbin{\otimes}}

% Abreviações
\newcommand{\Ltac}{\Mathcal{L}\unskip~tac}
\newcommand{\CalcLambda}{Cálculo-\(\lambda\)\xspace}
\newcommand{\CalcsLambda}{Cálculos-\(\lambda\)\xspace}
\newcommand{\SisT}{\(\textbf{KT} \Odot \textbf{K4}\)\xspace}

\newcommand{\CLST}{Cálculo-\(\lambda\) Simplesmente Tipado\xspace}
\newcommand{\TTML}{Teoria de Tipos de Martin-Löf\xspace}
\newcommand{\CCH}{Correspondência de Curry-Howard\xspace}

\newcommand{\PIMODELOS}{\PI-Modelos\xspace}
\newcommand{\PIMODELO} {\PI-Modelo\xspace}
\newcommand{\PImodelos}{\PI-modelos\xspace}
\newcommand{\PImodelo} {\PI-modelo\xspace}

\newcommand{\PIFRAMES}{\PI-Frames\xspace}
\newcommand{\PIFRAME} {\PI-Frame\xspace}
\newcommand{\PIframes}{\PI-frames\xspace}
\newcommand{\PIframe} {\PI-frame\xspace}

\newcommand{\OPIMODELOS}{\OPI-Modelos\xspace}
\newcommand{\OPIMODELO} {\OPI-Modelo\xspace}
\newcommand{\OPImodelos}{\OPI-modelos\xspace}
\newcommand{\OPImodelo} {\OPI-modelo\xspace}

\newcommand{\OPIFRAMES}{\OPI-Frames\xspace}
\newcommand{\OPIFRAME} {\OPI-Frame\xspace}
\newcommand{\OPIframes}{\OPI-frames\xspace}
\newcommand{\OPIframe} {\OPI-frame\xspace}

\newcommand{\Modeloinicial}{\(\mathbfcal{I}\)\xspace}
\newcommand{\modeloinicial}{\mathbfcal{I}\xspace}

\newcommand{\Mundoinicial}{\textbf{\textit{i}}\xspace}
\newcommand{\mundoinicial}{\textbf{\textit{i}}\xspace}

\newcommand{\Mundobase}{\textit{w}\textsubscript{\Mathcal{M}}\xspace}
\newcommand{\mundobase}{w_{\mathcal{M}}\xspace}

\newcommand{\ElementoMaximo}{\MathcalI{S}{+}\xspace}
\newcommand{\elementomaximo}{\mathcal{S}^{+}\xspace}

% Referência de Environments
\newcommand{\SubCaso}[2]{\ref{#1}.\ref{#2}}

% Citações
% MIGUEL: Esses comandos nunca funcionaram, eu deixei pois um dia pretendia arrumar, nunca arrumei lmao
% \newcommand{\cciteshort}[2]{\citeshort{#1} e~\citeshort{#2}}
% \newcommand{\ccciteshort}[3]{\citeshort{#1},~\citeshort{#2} e~\citeshort{#3}} 

\frenchspacing

% -----------------------------------------------------------------
% ELEMENTOS PRÉ-TEXTUAIS
% -----------------------------------------------------------------
\pretextual

% Você pode comentar os elementos que não deseja em seu trabalho;

% A capa pode ser escolhida dentro do arquivo Capa.tex (TCC, Master, Doc, ...)
% ---
% Capa
% ---


% --------------------------------------------------------
% Capa Padrão
% --------------------------------------------------------
\renewcommand{\imprimircapa}{%
	\begin{capa}%
		\center

		{\fontseries{b}\selectfont\MakeTextUppercase{UNIVERSIDADE DO ESTADO DE SANTA CATARINA -- UDESC}}

		{\fontseries{b}\selectfont\MakeTextUppercase{CENTRO DE CIÊNCIAS TECNOLÓGICAS -- CCT  }}

		{\fontseries{b}\selectfont\MakeTextUppercase{BACHARELADO EM CIÊNCIA DA COMPUTAÇÃO -- BCC  }}

		\vfill

		{\fontseries{b}\selectfont\MakeTextUppercase{\normalsize\imprimirautor}}

		\vfill
		\begin{center}
			{\fontseries{b}\selectfont\MakeTextUppercase{\imprimirtitulo}}
		\end{center}
		\vfill

		\vfill

		{\fontseries{b}\selectfont\MakeTextUppercase{\imprimirlocal}}
		\par
		{\fontseries{b}\selectfont \imprimirdata}
		\vspace*{1cm}
	\end{capa}
}

\imprimircapa					% Elemento Obrigatório
% ---
% Folha de rosto
% ---


\makeatletter

\renewcommand{\folhaderostocontent}{
	\begin{center}
		
		{\fontseries{b}\selectfont\MakeTextUppercase{\imprimirautor}}
		
		\vfill
		
		\begin{center}
			{\fontseries{b}\selectfont\MakeTextUppercase{\imprimirtitulo}}
		\end{center}
	
		\vspace*{1.5cm}

		\abntex@ifnotempty{\imprimirpreambulo}{%
			\hspace{.45\textwidth}
			{\begin{minipage}{.5\textwidth}
					\SingleSpacing
					\imprimirpreambulo\par
					\vspace*{4pt}
					{\imprimirorientadorRotulo~\imprimirorientador\par}
					\abntex@ifnotempty{\imprimircoorientador}{%
						{\imprimircoorientadorRotulo~\imprimircoorientador}%
					}%
			\end{minipage}}%
		}%
	
		
		\vfill
		
	{\fontseries{b}\selectfont\MakeTextUppercase{\imprimirlocal}}
	\par
	{\fontseries{b}\selectfont \imprimirdata}
	\vspace*{1cm}
	\end{center}
}


% (o * indica que haverá a ficha bibliográfica)
% ---
\imprimirfolhaderosto*
% ---


			% Elemento Obrigatório
% Caso não utilize a Ficha Catalográfica entre na folha de rosto e retire o * de dentro do arquivo Folha de Rosto
% 
% ---
% Inserir a ficha bibliografica
% ---

% Isto é um exemplo de Ficha Catalográfica, ou ``Dados internacionais de
% catalogação-na-publicação''. Você pode utilizar este modelo como referência. 
% Porém, provavelmente a biblioteca da sua universidade lhe fornecerá um PDF
% com a ficha catalográfica definitiva após a defesa do trabalho. Quando estiver
% com o documento, salve-o como PDF no diretório do seu projeto e substitua todo
% o conteúdo de implementação deste arquivo pelo comando abaixo:



% \begin{fichacatalografica}
%     \includepdf{fig_ficha_catalografica.pdf}
% \end{fichacatalografica}


%	\setlength{\parindent}{0cm}
%	\setlength{\parskip}{0pt}
\begin{fichacatalografica}
	%\sffamily
	%\rmfamily
	\ttfamily \hbadness=10000
	\vspace*{\fill}					% Posição vertical
	\begin{center}					% Minipage Centralizado
	Para gerar a ficha catalográfica de teses e \\ 
	dissertações acessar o link:  \\
	https://www.udesc.br/bu/manuais/ficha
	
	\vspace*{8pt}
	
%	\begin{minipage}[c]{8cm}
%	\centering \sffamily
%	 Ficha catalográfica elaborada pelo(a) autor(a), com auxílio do programa de geração automática da Biblioteca Setorial do CCT/UDESC
%	\end{minipage}
	\fbox{\begin{minipage}[c]{12.5cm}		% Largura
	\flushright
	{\begin{minipage}[c]{10.5cm}		% Largura
	\vspace{1.25cm}
	%\footnotesize
	\setlength{\parindent}{1.5em}
	\noindent \invertname{\imprimirautor} \par
	\imprimirtitulo{ }/{ }\imprimirautor. -- \imprimirlocal, \imprimirdata .\par
	\pageref{LastPage} p. : il. ; 30 cm.\par
	\vspace{1.5em}
	\imprimirorientadorRotulo~\imprimirorientador.\par
	\imprimircoorientadorRotulo~\imprimircoorientador.\par
	\imprimirtipotrabalho~--~\imprimirinstituicao, \imprimirlocal, \imprimirdata.\par
	\vspace{1.5em}
		1. Palavra-chave.
		2. Palavra-chave.
		3. Palavra-chave.
 		4. Palavra-chave.
		5. Palavra-chave.
		I. \invertname{\imprimirorientador}.
		II. \invertname{\imprimircoorientador}.
		III. \imprimirinstituicao.
		IV. Título. %
	\vspace{1.25cm}	%		
	\end{minipage}%
	}% 
	\end{minipage}}%
	
	\vspace*{0.5cm}
	
	\end{center}
\end{fichacatalografica}


%\begin{fichacatalografica}
%	\sffamily
%	\vspace*{\fill}					% Posição vertical
%	\begin{center}					% Minipage Centralizado
%	\fbox{\begin{minipage}[c][8cm]{13.5cm}		% Largura
%	\small
%	\imprimirautor
%	%Sobrenome, Nome do autor
%	
%	\hspace{0.5cm} \imprimirtitulo  / \imprimirautor. --
%	\imprimirlocal, \imprimirdata-
%	
%	\hspace{0.5cm} \pageref{LastPage} p. : il. (algumas color.) ; 30 cm.\\
%	
%	\hspace{0.5cm} \imprimirorientadorRotulo~\imprimirorientador\\
%	
%	\hspace{0.5cm}
%	\parbox[t]{\textwidth}{\imprimirtipotrabalho~--~\imprimirinstituicao,
%	\imprimirdata.}\\
%	
%	\hspace{0.5cm}
%		1. Palavra-chave1.
%		2. Palavra-chave2.
%		3. Palavra-chave3.
% 		4. Palavra-chave4.
%		5. Palavra-chave5.
%		I. Orientador.
%		II. Universidade xxx.
%		III. Faculdade de xxx.
%		IV. Título 			
%	\end{minipage}}
%	\end{center}
%\end{fichacatalografica}
% ---

	% Elemento Obrigatório (Verso da Folha)
% 
% ---
% Inserir errata
% ---
\begin{errata}
Elemento opcional. 

Exemplo:

\vspace{\onelineskip}

SOBRENOME, Prenome do Autor. Título de obra: subtítulo (se houver). Ano de depósito. Tipo do trabalho (grau e curso) - Vinculação acadêmica, local de apresentação/defesa, data.

\begin{table}[htb]
\center
\begin{tabular}{|p{2.4cm}|p{2cm}|p{3cm}|p{3cm}|}
  \hline
   \textbf{Folha} & \textbf{Linha}  & \textbf{Onde se lê}  & \textbf{Leia-se}  \\
    \hline
    1 & 10 & auto-conclavo & autoconclavo\\
   \hline
\end{tabular}
\end{table}

\end{errata}
% ---				% Elemento Opcional

% ---
% Inserir folha de aprovação
% ---

% Isto é um exemplo de Folha de aprovação, elemento obrigatório da NBR
% 14724/2011 (seção 4.2.1.3). Você pode utilizar este modelo até a aprovação
% do trabalho. Após isso, substitua todo o conteúdo deste arquivo por uma
% imagem da página assinada pela banca com o comando abaixo:
%
% \includepdf{folhadeaprovacao_final.pdf}
%
\begin{folhadeaprovacao}



    \begin{center}
        {\fontseries{b}\selectfont\MakeTextUppercase{\normalsize\imprimirautor}}
    \end{center}
    \vfill

    \vfill
    \begin{center}
        {\fontseries{b}\selectfont\MakeTextUppercase{\imprimirtitulo}}
    \end{center}
    \vfill


\abntex@ifnotempty{\imprimirpreambulo}{%
    \hspace{.45\textwidth}
    {\begin{minipage}{.5\textwidth}
            \SingleSpacing
            \imprimirpreambulo\par
            \vspace*{4pt}
            {\imprimirorientadorRotulo~\imprimirorientador\par}
            \abntex@ifnotempty{\imprimircoorientador}{%
                {\imprimircoorientadorRotulo~\imprimircoorientador}%
            }%
    \end{minipage}}%
}%


\vfill

    \begin{center}
        {\fontseries{b}\selectfont BANCA EXAMINADORA: }
        \vspace*{1.75cm}
    \end{center}

    {Orientadora:}

    \begin{center}
        \begin{minipage}{8.75cm}
            \begin{flushleft}
                \rule{8.75cm}{0.1mm}

                Dra. Karina Girardi Roggia \par
                UDESC
            \end{flushleft}
        \end{minipage}
    \end{center}

    \vspace*{\baselineskip}
    {Coorientador:}

    \begin{center}
        \begin{minipage}{8.75cm}
            \begin{flushleft}
                \rule{8.75cm}{0.1mm}

                Miguel Alfredo Nunes \par
                UNICAMP
            \end{flushleft}
        \end{minipage}
    \end{center}

    \vspace*{\baselineskip}
    {Membros:}

    \begin{center}
        \begin{minipage}{8.75cm}
            \begin{flushleft}
                % \vspace*{1.25cm}
                \rule{8.75cm}{0.1mm}

                Dr. Cristiano Damiani Vasconcellos \par
                UDESC

                \vspace*{1cm}
                \rule{8.75cm}{0.1mm}

                Me. Paulo Henrique Torrens \par
                University of Kent
            \end{flushleft}
        \end{minipage}
    \end{center}

    \vspace*{\fill}
    \begin{center}
    {\imprimirlocal, Junho de \imprimirdata}
    \end{center}
    \vspace*{0.25cm}
\end{folhadeaprovacao}
% ---




%\textbf{	{Orientador: \vspace{-16pt} }
%	\assinatura{\textbf{Prof. \imprimirorientador , Dr.} \\ Univ. XXX}
%	{Coorientador: \vspace{-16pt}}
%	\assinatura{\textbf{Prof. \imprimircoorientador , Dr.} \\ Univ. XXX}
%
%	{Membros: \vspace{-16pt} }
%
%	% --- Exemplo de assinaturas em sequência ---
%	\setlength{\ABNTEXsignwidth}{8.5cm}
%
%	\assinatura{\textbf{Prof. Professor, Dr.} \\ Univ. XXX}
%	\assinatura{\textbf{Prof. Professor, Dr.} \\ Univ. XXX}
%	\assinatura{\textbf{Prof. Professor, Dr.} \\ Univ. XXX}
%
%	% --- Exemplo de assinaturas lado a lado ---
%	\setlength{\ABNTEXsignwidth}{7.5cm}
    %
    %    \noindent\hfill\assinatura*{\textbf{Prof. Professor, Dr.} \\ Univ. XXX}%
    %    \hfill%
    %    \assinatura*{\textbf{Prof. Professor, Dr.} \\ Univ. XXX}%
    %    \hfill
    %
    %    \noindent\hfill\assinatura*{\textbf{Prof. Professor, Dr.} \\ Univ. XXX}%
    %    \hfill%
    %    \assinatura*{\textbf{Prof. Professor, Dr.} \\ Univ. XXX}%
    %    \hfill}		% Elemento Obrigatório
% % ---
% Dedicatória
% ---
\begin{dedicatoria}
   \vspace*{\fill}
%   \begin{flushright}
%   \noindent
%	Este trabalho é dedicado às crianças adultas que,\\
%	quando pequenas, sonharam em se tornar cientistas. 
%   \end{flushright}

{%
	\noindent\hspace{.5\textwidth}
	{\begin{minipage}{.5\textwidth}
			\begin{flushleft}
				Aos estudantes da Universidade do Estado de Santa Catarina, pela inspiração de sempre!
			\end{flushleft}
	\end{minipage}}%
\vspace*{3cm}
}%

\end{dedicatoria}
% ---
			% Elemento Opcional
% ---
% Agradecimentos
% ---
\begin{agradecimentos}



\end{agradecimentos}
% ---		% Elemento Opcional
% ---
% Epígrafe
% ---
\begin{epigrafe}
    \vspace*{\fill}
{%
    \noindent\hspace{.5\textwidth}
    {\begin{minipage}{.5\textwidth}
        \textit{``Different conclusions are reached when one fact is viewed from two separate points of view. When that happens, there is no immediate way to judge which point of view is the correct one. There is no way to conclude one’s own conclusion is the correct one. But for that exact reason, it is also premature to decide one’s own conclusion is wrong.''}\\(Senjougahara Hitagi {-} Bakemonogatari, [2009])
    \end{minipage}}%
    \vspace*{3cm}
}%
\end{epigrafe}
% ---				% Elemento Opcional
% ---
% RESUMOS
% ---

% resumo em português
\setlength{\absparsep}{18pt} % ajusta o espaçamento dos parágrafos do resumo
\begin{resumo}
    Na medida em que sistemas de computação modernos escalam, a existência de informações contraditórias torna-se inevitável. As lógicas ortodoxas não são capazes de tratar este tipo de informação sem que o princípio da explosão tome lugar. Com isso, sistemas paraconsistentes {---} sistemas nos quais a explosividade é cuidadosamente separada da contraditoriedade {---} são uma alternativa vantajosa quando comparados às lógicas ortodoxas. Neste contexto, as lógicas de inconsistência formal, sobretudo a \lfium{}, usufruem de propriedades interessantes que as garantem aplicações em diversos campos do conhecimento, como, por exemplo, no desenvolvimento de sistemas de gerenciamento de bancos de dados. Com isso, a prova de metateoremas sobre estas lógicas evidencia características das diferentes abordagens possíveis no estudo destes sistemas. Ademais, assistentes de provas como o Coq proporcionam aos teoremas neles desenvolvidos uma garantia de correção dificilmente encontrada em provas manuais. Este trabalho propõe explanar e definir a lógica de inconsistência formal \lfium{}, bem como desenvolver metateoremas para este sistema no assistente de provas Coq.

 \textbf{Palavras-chave}: Coq, lógica paraconsistente, \lfium{}, lógica de inconsistência formal, lógica trivalorada.
\end{resumo}
				% Elemento Obrigatório
% ---
% Abstract
% ---
% Na medida em que sistemas de computação modernos escalam, a existência de informações
% contraditórias torna-se inevitável. As lógicas ortodoxas não são capazes de tratar este tipo
% de informação sem que o princípio da explosão tome lugar. Com isso, o estabelecimento de
% sistemas paraconsistentes — sistemas nos quais a explosividade é cuidadosamente separada
% da contraditoriedade — é uma alternativa vantajosa quando comparados às lógicas ortodoxas.
% Neste contexto, as lógicas de inconsistência formal, sobretudo a LFI1, usufruem de propriedades
% interessantes que as garantem aplicações em diversos campos do conhecimento, como, por
% exemplo, no desenvolvimento de sistemas de gerenciamento de bancos de dados. Com isso, a
% prova de metateoremas sobre estas lógicas evidencia características das diferentes abordagens
% possíveis no estudo destes sistemas. Ademais, assistentes de provas como o Coq proporcionam
% aos teoremas neles desenvolvidos uma garantia de correção dificilmente encontrada em provas
% manuais. Este trabalho propõe explanar e definir a lógica de inconsistência formal LFI1, bem
% como desenvolver metateoremas para este sistema no assistente de provas Coq.
% resumo em inglês
\begin{resumo}[Abstract]
 \begin{otherlanguage*}{english}
    As modern computer systems scale, the existence of contradictory information becomes inevitable. Most orthodox logics are not able to cope with this kind of information without the principle of explosion taking place. Thus, establishing paraconsistent systems {--} systems in which explosiveness is carefully separated from contradictoriness {--} is an advantageous alternative compared to orthodox logics. From this perspective, the logics of formal inconsistency, specially \lfium{}, enjoy some interesting properties which allow them to be used in many areas, for example in the development of database management systems. In this light, proving metatheorems about these logics highlights characteristics of different approaches when studying these systems. Furthermore, proof assistants, such as Coq, guarantee the theorems proved inside them a degree of certainty about their correctness hardly ever found in manual proofs. The present work explores and defines the logic of formal inconsistency \lfium{}, as well as proves metatheorems for this system inside the Coq proof assistant.

   \textbf{Keywords}: Coq, paraconsistent logic, \lfium{}, logics of formal inconsistency, three-valued logic.
 \end{otherlanguage*}
\end{resumo}

				% Elemento Obrigatório

% ---
% inserir lista de ilustrações
% ---
% \pdfbookmark[0]{\listfigurename}{lof}
% \listoffigures*
% \cleardoublepage
% ---

% ---
% inserir lista de quadros
% ---
% \pdfbookmark[0]{\listofquadrosname}{loq}
% \listofquadros*
% \cleardoublepage
% ---


% ---
% inserir lista de tabelas
% ---
\pdfbookmark[0]{\listtablename}{lot}
\listoftables*
\cleardoublepage
% ---

% ---
% inserir lista de abreviaturas e siglas
% ---
\begin{siglas}
	\item[MP] \textit{Modus Ponens}
	\item[MTD] Metateorema da dedução
	\item[sse] se e somente se
	\item[CIC] \textit{Calculus of Inductive Constructions}
\end{siglas}
% % ---

% % ---
% % inserir lista de símbolos
% % ---
\begin{simbolos}
    \item [$p, q, r\ldots$] Variáveis atômicas.
    \item [$A, B, C, \ldots$] Conjuntos quaisquer.
    \item [$\alpha, \beta, \gamma, \ldots$] Fórmulas quaisquer.
    \item [$\Gamma, \Delta$] Conjuntos de fórmulas.
    \item [$\Sigma, \Theta$] Assinaturas de linguagens.
    \item[$\Vdash$] Relação de consequência qualquer (sintática ou semântica).
    \item[$\vdash$] Relação de consequência sintática.
    \item[$\vDash$] Relação de consequência semântica.
\end{simbolos}


% ---
				% Elemento Opcional
% ---
% inserir o sumario
% ---
\pdfbookmark[0]{\contentsname}{toc}
\tableofcontents*
\cleardoublepage
% ---
				% Elemento Obrigatório

% -----------------------------------------------------------------
% ELEMENTOS TEXTUAIS
% -----------------------------------------------------------------
\textual

\pagestyle{PagNumReduzida}						% Comando para cabeçalho somente com numeração de página 10pt
\aliaspagestyle{chapter}{PagNumReduzida}		% Deixar numeração da primeira página com tamanho igual ao resto da numeração
% ref.: https://groups.google.com/g/abntex2/c/CP7g8ZMgi-c/m/KjfEnn5b9a4J


% ---- Mantenha está estrutura, assim você deixa o trabalho mais organizado -------

\chapter{Introdução}

\noindent\label{cap:Introducao}
As lógicas paraconsistentes são uma família de lógicas na qual a presença de contradições não implica trivialidade, ou seja, são sistemas lógicos que possuem uma negação que não respeita o Princípio da Explosão\footnote{Definido como $\alpha \rightarrow (\neg \alpha \rightarrow \beta)$.}~\cite{carnielli2007}. Tradicionalmente, em lógicas ortodóxas, qualquer teoria que seja inconsistente {-} e, portanto, não respeite o Princípio da não-contradição\footnote{Definido como $\neg (\alpha \land \neg \alpha)$.} {-} será uma teoria trivial (uma teoria que possui todas as sentenças). Deste modo, as lógicas paraconsistentes surgem como uma ferramenta que permite tratar contradições sem trivializar o sistema lógico~\cite{Carnielli_Coniglio_2016}.

De acordo com~\cite{sep-logic-paraconsistent}, as motivações para o estudo de lógicas paraconsistentes podem ser observadas em diversos campos do conhecimento. Nas ciências naturais, por exemplo, teorias inconsistentes e não-triviais são comuns, como é o caso da teoria do átomo de Bohr, que, segundo~\cite{Brown2015-BROCAP-9}, deve possuir uma inferência paraconsistente. No campo da linguística, inconsistências não-triviais também são possíveis, como a preservação da noção espacial da palavra ``Próximo'' mesmo tratando-se de objetos impossíveis~\cite{McGinnis2013-MCGTUA}. No contexto da ciência da computação é o uso de lógicas de inconsistência formal para a modelagem e o desenvolvimento de bancos de dados evolucionários~\cite{carnielli2000formal}.

As lógicas de inconsistência formal (\textbf{LFI}s), são lógicas paraconsistentes que introduzem os conceitos de consistência e inconsistência como formas de representar o excesso de informações (por exemplo, evidência de $\alpha$ e evidência de $\neg \alpha$ sem evidência conclusiva para $\beta$), para resgatar a capacidade de se obter a trivialidade em alguns casos~\cite{carnielli2007}. Ao explicitamente representar a consistência dentro da sua linguagem, é possivel estudar teorias inconsistentes sem necessariamente assumir que elas são triviais. A ideia por trás das \textbf{LFI}s é que deve-se respeitar as noções da lógica clássica o máximo possível, desviando desta somente na presença de contradições. Isto significa que, na ausência de contradições, o Princípio da Explosão deve ser tomado como válido~\cite{sep-logic-paraconsistent}.



    \section{Objetivo Geral}


    \section{Objetivos Específicos}



    \section{Trabalhos Relacionados}


    \section{Metodologia}
        

    \section{Estrutura do Trabalho}
       

% -----------------------------------------------------------------
% ELEMENTOS PÓS-TEXTUAIS
% -----------------------------------------------------------------
\postextual

% Você pode comentar os elementos que não deseja em seu trabalho;

% Referências bibliográficas
\bibliography{Referencias}	% Elemento Obrigatório

%% ----------------------------------------------------------
% Glossário
% ----------------------------------------------------------

%Consulte o manual da classe abntex2 para orientações sobre o glossário.

%\glossary




% ----------------------------------------------------------
% Glossário (Formatado Manualmente)
% ----------------------------------------------------------

\chapter*{GLOSSÁRIO}
\addcontentsline{toc}{chapter}{GLOSSÁRIO}

{ \setlength{\parindent}{0pt} % ambiente sem indentação

\textbf{Ardósia}: Rocha metamórfica sílico-argilosa formada pela transformação da argila sob pressão e temperatura, endurecida em finas lamelas.

\textbf{Arenito}: rocha sedimentária de origem detrítica formada de grãos agregados por um cimento natural silicoso, calcário ou ferruginoso que comunica ao conjunto em geral qualidades de dureza e compactação.

\textbf{Feldspato}: grupo de silicatos de sódio, potássio, cálcio ou outros elementos que compreende dois subgrupos, os feldspatos alcalinos e os plagioclásios.






} % fim ambiente sem indentação


			% Elemento Opcional

% ----------------------------------------------------------
% Apêndices
% ----------------------------------------------------------

% ---
% Inicia os apêndices
% ---
\begin{apendicesenv}



\end{apendicesenv}
% ---			    % Elemento Opcional
%
% ----------------------------------------------------------
% Anexos
% ----------------------------------------------------------
%
% ---
% Inicia os anexos
% ---
\begin{anexosenv}

% Imprime uma página indicando o início dos anexos
%\partanexos

% ---
\chapter{TÍTULO}
% ---



\end{anexosenv}
				% Elemento Opcional
%
%%---------------------------------------------------------------------
%% INDICE REMISSIVO
%%---------------------------------------------------------------------

%\phantompart
%\printindex

%---------------------------------------------------------------------

%%---------------------------------------------------------------------
%% INDICE REMISSIVO (Formatado Manualmente)
%%---------------------------------------------------------------------

\chapter*{ÍNDICE}
\addcontentsline{toc}{chapter}{ÍNDICE}

{ \setlength{\parindent}{0pt}  % ambiente sem indentação
	
Andesito, 22, 50, 73

Argila, 52, 75, 121

Basalto, 25, 230, 235

	
	
	
	
} % fim ambiente sem indentação


		% Elemento Opcional

\end{document}

% -----------------------------------------------------------------
% Fim do Documento
% -----------------------------------------------------------------
