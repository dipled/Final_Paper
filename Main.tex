\documentclass[
    12pt,					% tamanho da fonte
    openright,				% capítulos começam em pág ímpar (insere página vazia caso preciso)
    oneside,				% para impressão em recto e verso (twoside). Oposto a (oneside)
    a4paper,				% tamanho do papel.
    chapter=TITLE,			% títulos de capítulos convertidos em letras maiúsculas
    section=TITLE,			% títulos de seções convertidos em letras maiúsculas
    english,				% idioma adicional para hifenização
    brazil,					% o último idioma é o principal do documento
    fleqn,					% equações alinhadas a esquerda (UDESC/CCT)+
    ]{abntex2}

\input{Pacotes}	% Inclui pacotes básicos
\allowdisplaybreaks{}
\graphicspath{{Texto/Figuras/}}
\renewcommand{\orientadorname}{Orientadora:}

% O pacote bm torna a fonte do mathcal feia, a linha abaixo desfaz isso
% (Se tirar o bm caga tudo, então ele fica)
%\DeclareMathAlphabet{\mathcal}{OMS}{cmsy}{m}{n}
\DeclareMathAlphabet{\pazocal}{OMS}{zplm}{m}{n}
\DeclareMathAlphabet{\mathscr}{OMS}{zplm}{m}{n}
% Definindo fonte caligráfica e negrita
\DeclareMathAlphabet\mathbfcal{OMS}{cmsy}{b}{n}

% \renewcommand{\labelenumi}{\roman{enumi}.}

% -----------------------------------------------------------------
% Informações de dados para CAPA e FOLHA DE ROSTO
% -----------------------------------------------------------------
\titulo{Implementação de uma biblioteca da Lógica de Inconsistência Formal LFI1 em Coq}%


\autor{Helena Vargas {}Tannuri}%
\orientador{Karina Girardi {}Roggia}%
\coorientador{Miguel Alfredo {}Nunes}%

\instituicao{Universidade do Estado de Santa Catarina, Centro de Ciências Tecnológicas, Bacharelado em Ciência da Computação}%

\tipotrabalho{Trabalho de Conclusão de Curso}

\preambulo{Trabalho de conclusão de curso submetido à Universidade do Estado de Santa Catarina
como parte dos requisitos para a obtenção do grau de Bacharel em Ciência da Computação}

\local{Joinville}%

\data{\the\year}%

\makeindex

% MIGUEL: Criei esse arquivo para melhor separar os comandos customizados que estavam largados pelo main
%           Seria uma boa ideia documentar o que cada um faz, mas isso é um problema para o futuro

% Teoremas e Provas

% Estilo padrão, i.e., texto itálico
\newtheorem{teorema}   {Teorema}
\newtheorem{proposicao}{Proposição}
\newtheorem{lema}      {Lema}
\newtheorem{corolario} {Corolário}
\renewcommand{\proofname}{Prova}
\renewcommand\qedsymbol{$\blacksquare$}

% Estilo simples, i.e., texto normal
\theoremstyle{definition}
\newtheorem{definicao}{Definição}
\newtheorem{notacao}{Notação}
\newtheorem{exemplo}{Exemplo}

\theoremstyle{remark}
\newtheorem{observacao}{Observação}

% \AtEndEnvironment{teorema}  {\qed}%
% \AtEndEnvironment{proposicao}  {\qed}%
% \AtEndEnvironment{lema}     {\qed}%
% \AtEndEnvironment{definicao}{\qed}%
% \AtEndEnvironment{exemplo}  {\qed}%

%% Grego

\renewcommand\phi{\varphi}

% Grego Minúsculo

% Grego Maiúsculo
\newcommand{\GAMMA}{\(\Gamma\)\xspace}
\newcommand{\DDELTA}{\(\Delta\)\xspace}
\newcommand{\SIGMA}{\(\Sigma\)\xspace}
\newcommand{\THETA}{\(\Theta\)\xspace}
\newcommand{\LAMBDA}{\(\Lambda\)\xspace}
\newcommand{\LAMBDAlm}{\(\Lambda_{\mathsf{LM}}\)\xspace}
\newcommand{\Lambdalm}{\Lambda_{\mathsf{LM}}}

% Modalidades e símbolos matemáticos

\newcommand{\meio}{\frac{1}{2}}

\newcommand{\VVDASH}{\(\Vdash\)\xspace}
\newcommand{\VDDASH}{\(\vDash\)\xspace}
\newcommand{\VDASH}{\(\vdash\)\xspace}

\newcommand{\ODOT}{\(\odot\)\xspace}
\newcommand{\OPLUS}{\(\oplus\)\xspace}
\newcommand{\OTIMES}{\(\otimes\)\xspace}
\newcommand{\OMINUS}{\(\ominus\)\xspace}

\newcommand{\tofrom}{\leftrightarrow}
\newcommand{\conmat}{\vDash_{\pazocal{M}_{\lfium{}}}}
\newcommand{\conval}{\vDash_{\lfium{}}}
\newcommand{\conhil}{\vdash_{\lfium{}}}

\newcommand{\eqdef}{\mathrel{\overset{\makebox[0pt]{\mbox{\normalfont\tiny\sffamily def}}}{=}}}

% Fontes
\newcommand{\Mathcal}[1]{\(\mathcal{#1}\)\xspace}
\newcommand{\Mathcali}[2]{\(\mathcal{#1}_{#2}\)\xspace}
\newcommand{\MathcalI}[2]{\(\mathcal{#1}^{#2}\)\xspace}
\newcommand{\Mathcalii}[3]{\(\mathcal{#1}{#2}_{#3}\)\xspace}

\newcommand{\Mathfrak}[1]{\(\mathfrak{#1}\)\xspace}
\newcommand{\Mathfraki}[2]{\(\mathfrak{#1}_{#2}\)\xspace}
\newcommand{\MathfrakI}[2]{\(\mathfrak{#1}^{#2}\)\xspace}

\newcommand{\Mathbb}[1]{\(\mathbb{#1}\)\xspace}
\newcommand{\Mathbbi}[2]{\(\mathbb{#1}_{#2}\)\xspace}
\newcommand{\MathbbI}[2]{\(\mathbb{#1}{#2}\)\xspace}

% Comentários
\newcommand{\cortar}[1]{\textcolor{red}{\sout{#1}}}
\newcommand{\ignore}[1]{\textcolor{blue}{\textbf{IGNOREM:} #1}}
\newcommand{\helena}[1]{\textcolor{magenta}{\textbf{HELENA:} #1}}
\newcommand{\migs}[1]{\textcolor{violet}{\textbf{MIGS:} #1}}
\newcommand{\migscortar}[2]{\textcolor{violet}{\textbf{MIGS:} \sout{#1}{#2}}}
\newcommand{\kaqui}[1]{\textcolor{teal}{\textbf{KAQUI:} #1}}

% Outros
\newcommand{\linguagem}[1]{\(\mathsf{LFI1}_{#1}\)\xspace}
\newcommand{\Linguagem}[1]{\mathsf{LFI1}_{#1}\xspace}
\newcommand{\funcao}[1]{\operatorname{#1}\xspace}
\newcommand{\inlinecoq}[1]{\lstinline[columns=fixed,language=coq]{#1}}

\newcommand{\Odot}  {\mathbin{\odot}}
\newcommand{\Oplus} {\mathbin{\oplus}}
\newcommand{\Otimes}{\mathbin{\otimes}}

% Abreviações
\newcommand{\lfium}{\textbf{LFI1}}
\newcommand{\lfi}{\textbf{LFI}}
\newcommand{\lfis}{\textbf{LFI}s}
\newcommand{\Ltac}{\Mathcal{L}\unskip~tac}
\newcommand{\CalcLambda}{Cálculo-\(\lambda\)\xspace}
\newcommand{\CalcsLambda}{Cálculos-\(\lambda\)\xspace}
\newcommand{\SisT}{\(\textbf{KT} \Odot \textbf{K4}\)\xspace}

\newcommand{\CLST}{Cálculo-\(\lambda\) Simplesmente Tipado\xspace}
\newcommand{\TTML}{Teoria de Tipos de Martin-Löf\xspace}
\newcommand{\CCH}{Correspondência de Curry-Howard\xspace}

\newcommand{\PIMODELOS}{\PI-Modelos\xspace}
\newcommand{\PIMODELO} {\PI-Modelo\xspace}
\newcommand{\PImodelos}{\PI-modelos\xspace}
\newcommand{\PImodelo} {\PI-modelo\xspace}

\newcommand{\PIFRAMES}{\PI-Frames\xspace}
\newcommand{\PIFRAME} {\PI-Frame\xspace}
\newcommand{\PIframes}{\PI-frames\xspace}
\newcommand{\PIframe} {\PI-frame\xspace}

\newcommand{\OPIMODELOS}{\OPI-Modelos\xspace}
\newcommand{\OPIMODELO} {\OPI-Modelo\xspace}
\newcommand{\OPImodelos}{\OPI-modelos\xspace}
\newcommand{\OPImodelo} {\OPI-modelo\xspace}

\newcommand{\OPIFRAMES}{\OPI-Frames\xspace}
\newcommand{\OPIFRAME} {\OPI-Frame\xspace}
\newcommand{\OPIframes}{\OPI-frames\xspace}
\newcommand{\OPIframe} {\OPI-frame\xspace}

\newcommand{\Modeloinicial}{\(\mathbfcal{I}\)\xspace}
\newcommand{\modeloinicial}{\mathbfcal{I}\xspace}

\newcommand{\Mundoinicial}{\textbf{\textit{i}}\xspace}
\newcommand{\mundoinicial}{\textbf{\textit{i}}\xspace}

\newcommand{\Mundobase}{\textit{w}\textsubscript{\Mathcal{M}}\xspace}
\newcommand{\mundobase}{w_{\mathcal{M}}\xspace}

\newcommand{\ElementoMaximo}{\MathcalI{S}{+}\xspace}
\newcommand{\elementomaximo}{\mathcal{S}^{+}\xspace}

% Referência de Environments
\newcommand{\SubCaso}[2]{\ref{#1}.\ref{#2}}

% Ambiente customizado para provas com casos e subcasos

\newcounter{Casos}    % Contador que enumera os casos da prova
\newcounter{SubCasos} % Contador que enumera o subcaso da prova
\newenvironment{provaporcasos} % Ambiente customizado para fazer provas por casos
    {
        % O que é feito quando este ambiente é aberto    
        \setcounter{Casos}{0} % resetando o counter por via das dúvidas
        \begin{description}[font=\mdseries\scshape] % abrindo o ambiente de description com a formatação bonita
    }   
    {
        % O que é feito quando esse ambiente é fechado
        \setcounter{Casos}{0} % resetando o counter por via das dúvidas
        \end{description} % fechando o ambiente
    }

% Comando para printar o caso com a numeração correta
\newcommand{\casodeprova}{\refstepcounter{Casos}\item[Caso \theCasos{}.]}

% Ambiente customizado para fazer provas por casos que contenham subcasos
% Só deve ser usado dentro de um ambiente provaporcasos
\newenvironment{provaporsubcasos}
    {
        % O que é feito quando este ambiente é aberto    
        \setcounter{SubCasos}{0} % resetando o counter por via das dúvidas
        \begin{description}[font=\mdseries\scshape] % abrindo o ambiente de description com a formatação bonita
    }   
    {
        % O que é feito quando esse ambiente é fechado
        \setcounter{SubCasos}{0} % resetando o counter por via das dúvidas
        \end{description} % fechando o ambiente
    }

% Comando para printar o subcaso com a numeração correta
\newcommand{\subcasodeprova}{\refstepcounter{SubCasos}\item[Subcaso \theCasos{}.\theSubCasos{}.]}

% Para testar, mantenha comentado se não estiver ativamente usando
\usepackage{lipsum}

% -----------------------------------------------------------------
% Início do documento
% -----------------------------------------------------------------
\begin{document}

\selectlanguage{brazil}

% Citações
% MIGUEL: Esses comandos nunca funcionaram, eu deixei pois um dia pretendia arrumar, nunca arrumei lmao
% \newcommand{\cciteshort}[2]{\citeshort{#1} e~\citeshort{#2}}
% \newcommand{\ccciteshort}[3]{\citeshort{#1},~\citeshort{#2} e~\citeshort{#3}} 

\frenchspacing

% -----------------------------------------------------------------
% ELEMENTOS PRÉ-TEXTUAIS
% -----------------------------------------------------------------
\pretextual{}

% Você pode comentar os elementos que não deseja em seu trabalho;

% A capa pode ser escolhida dentro do arquivo Capa.tex (TCC, Master, Doc, ...)
% Informações para CAPA e FOLHA DE ROSTO:
\titulo{Implementação de uma biblioteca da Lógica de Inconsistência Formal LFI1 em Coq}

\local{Brasil}
\data{\today}

\autor{Helena Vargas {}Tannuri}%
\orientador{Karina Girardi {}Roggia}%
\coorientador{Miguel Alfredo {}Nunes}%
\instituicao{
  Universidade do Estado de Santa Catarina -- UDESC
  % \par
  % Departamento de Ciência da Computação
  \par
  Bacharelado em Ciência da Computação
}

\tipotrabalho{Trabalho de Conclusão de Curso}
% O preambulo deve conter o tipo do trabalho, o objetivo, 
% o nome da instituição e a área de concentração 
\preambulo{Trabalho de Conclusão de Curso apresentado ao curso de Bacharelado em Ciência da Computação do Centro de Ciências Tecnológicas da Universidade do Estado de Santa Catarina, como requisito parcial para a obtenção do grau de Bacharel em Ciência da Computação.}

% \preambulo{Trabalho de Conclusão de Curso apresentado como requisito parcial para obtenção do título de Bacharelado em Ciência da Computação pelo Centro de Ciências Tecnológicas da Universidade do Estado de Santa Catarina, como requisito parcial para a obtenção do grau de Bacharel em Ciência da Computação.}					% Elemento Obrigatório
\include{PreTextuais/FolhadeRosto}			% Elemento Obrigatório
% Caso não utilize a Ficha Catalográfica entre na folha de rosto e retire o * de dentro do arquivo Folha de Rosto
% \include{PreTextuais/FichaCatalografica}	% Elemento Obrigatório (Verso da Folha)
% \include{PreTextuais/Errata}				% Elemento Opcional

% ---
% Inserir folha de aprovação
% ---

% Isto é um exemplo de Folha de aprovação, elemento obrigatório da NBR
% 14724/2011 (seção 4.2.1.3). Você pode utilizar este modelo até a aprovação
% do trabalho. Após isso, substitua todo o conteúdo deste arquivo por uma
% imagem da página assinada pela banca com o comando abaixo:
%
% \includepdf{folhadeaprovacao_final.pdf}
%
\begin{folhadeaprovacao}



    \begin{center}
        {\fontseries{b}\selectfont\MakeTextUppercase{\normalsize\imprimirautor}}
    \end{center}
    \vfill

    \vfill
    \begin{center}
        {\fontseries{b}\selectfont\MakeTextUppercase{\imprimirtitulo}}
    \end{center}
    \vfill


\abntex@ifnotempty{\imprimirpreambulo}{%
    \hspace{.45\textwidth}
    {\begin{minipage}{.5\textwidth}
            \SingleSpacing
            \imprimirpreambulo\par
            \vspace*{4pt}
            {\imprimirorientadorRotulo~\imprimirorientador\par}
            \abntex@ifnotempty{\imprimircoorientador}{%
                {\imprimircoorientadorRotulo~\imprimircoorientador}%
            }%
    \end{minipage}}%
}%


\vfill

    \begin{center}
        {\fontseries{b}\selectfont BANCA EXAMINADORA: }
        \vspace*{1.75cm}
    \end{center}

    {Orientadora:}

    \begin{center}
        \begin{minipage}{8.75cm}
            \begin{flushleft}
                \rule{8.75cm}{0.1mm}

                Dra. Karina Girardi Roggia \par
                UDESC
            \end{flushleft}
        \end{minipage}
    \end{center}

    \vspace*{\baselineskip}
    {Coorientador:}

    \begin{center}
        \begin{minipage}{8.75cm}
            \begin{flushleft}
                \rule{8.75cm}{0.1mm}

                Miguel Alfredo Nunes \par
                UNICAMP
            \end{flushleft}
        \end{minipage}
    \end{center}

    \vspace*{\baselineskip}
    {Membros:}

    \begin{center}
        \begin{minipage}{8.75cm}
            \begin{flushleft}
                % \vspace*{1.25cm}
                \rule{8.75cm}{0.1mm}

                Dr. Cristiano Damiani Vasconcellos \par
                UDESC

                \vspace*{1cm}
                \rule{8.75cm}{0.1mm}

                Me. Paulo Henrique Torrens \par
                University of Kent
            \end{flushleft}
        \end{minipage}
    \end{center}

    \vspace*{\fill}
    \begin{center}
    {\imprimirlocal, Junho de \imprimirdata}
    \end{center}
    \vspace*{0.25cm}
\end{folhadeaprovacao}
% ---




%\textbf{	{Orientador: \vspace{-16pt} }
%	\assinatura{\textbf{Prof. \imprimirorientador , Dr.} \\ Univ. XXX}
%	{Coorientador: \vspace{-16pt}}
%	\assinatura{\textbf{Prof. \imprimircoorientador , Dr.} \\ Univ. XXX}
%
%	{Membros: \vspace{-16pt} }
%
%	% --- Exemplo de assinaturas em sequência ---
%	\setlength{\ABNTEXsignwidth}{8.5cm}
%
%	\assinatura{\textbf{Prof. Professor, Dr.} \\ Univ. XXX}
%	\assinatura{\textbf{Prof. Professor, Dr.} \\ Univ. XXX}
%	\assinatura{\textbf{Prof. Professor, Dr.} \\ Univ. XXX}
%
%	% --- Exemplo de assinaturas lado a lado ---
%	\setlength{\ABNTEXsignwidth}{7.5cm}
    %
    %    \noindent\hfill\assinatura*{\textbf{Prof. Professor, Dr.} \\ Univ. XXX}%
    %    \hfill%
    %    \assinatura*{\textbf{Prof. Professor, Dr.} \\ Univ. XXX}%
    %    \hfill
    %
    %    \noindent\hfill\assinatura*{\textbf{Prof. Professor, Dr.} \\ Univ. XXX}%
    %    \hfill%
    %    \assinatura*{\textbf{Prof. Professor, Dr.} \\ Univ. XXX}%
    %    \hfill}		% Elemento Obrigatório
% \include{PreTextuais/Dedicatoria}			% Elemento Opcional
% ---
% Agradecimentos
% ---
\begin{agradecimentos}
Valeu chino e nameh tomara q vcs encontrem jesus cristo


\end{agradecimentos}
% ---		% Elemento Opcional
% ---
% Epígrafe
% ---
\begin{epigrafe}
    \vspace*{\fill}
{%
    \noindent\hspace{.5\textwidth}
    {\begin{minipage}{.5\textwidth}
        \textit{``Different conclusions are reached when one fact is viewed from two separate points of view. When that happens, there is no immediate way to judge which point of view is the correct one. There is no way to conclude one’s own conclusion is the correct one. But for that exact reason, it is also premature to decide one’s own conclusion is wrong.''}\\(Senjougahara Hitagi {---} Bakemonogatari, [2009])
    \end{minipage}}%
    \vspace*{3cm}
}%
\end{epigrafe}
% ---				% Elemento Opcional
% ---
% RESUMOS
% ---

% resumo em português
\setlength{\absparsep}{18pt} % ajusta o espaçamento dos parágrafos do resumo
\begin{resumo}
   
 \textbf{Palavras-chave}: Coq, Lógica paraconsistente, LFI1, Lógica de Inconsistência Formal, Lógica Trivalorada, Bancos de Dados.
\end{resumo}
				% Elemento Obrigatório
% ---
% Abstract
% ---

% resumo em inglês
\begin{resumo}[Abstract]
 \begin{otherlanguage*}{english}
	It is possible to model software components in logical systems, however complex software systems are not easily modelled in simple logical languages.
	To obtain a logical language capable of expressing complex properties, combining simpler logical systems is a possibility.
	One of the ways to combine logical systems is by means of fusions of modal logics. However, any kind of combination of logics is a complex task, that
	requires logical system to be treated as mathematical objects for they then be able to be combined. Modelling logical systems in proof assistants can
	make the combination process easier, since, in proof assistants, logical systems are already treated as mathematical objects. Moreover,
	proof assistants such as \textit{Coq} and \textit{Lean} have a large amount of automation tools that help in the process of developing proofs.
	As such, this work proposes to parametrically model multimodal logicas resulting from the fusion of logical systems, in the Coq proof assistant,
	and to prove the preservation of some properties of logics by the operation of fusion.
	After a review of the relevant literature, a case study of fusion of modal logics in Coq was modeled and the fusion of modal logics was  was
	parametrically implemented, where it was possible to parametrically model the fusion of syntactic systems of any modal logics.

   \textbf{Keywords}: Coq. Proof Assistants. Multimodal Logics. Fusion of Logics.
 \end{otherlanguage*}
\end{resumo}

				% Elemento Obrigatório

% ---
% inserir lista de ilustrações
% ---
% \pdfbookmark[0]{\listfigurename}{lof}
% \listoffigures*
% \cleardoublepage
% ---

% ---
% inserir lista de quadros
% ---
% \pdfbookmark[0]{\listofquadrosname}{loq}
% \listofquadros*
% \cleardoublepage
% ---


% ---
% inserir lista de tabelas
% ---
\pdfbookmark[0]{\listtablename}{lot}
\listoftables*
\cleardoublepage
% ---

% ---
% inserir lista de abreviaturas e siglas
% ---
\begin{siglas}
	\item[MP] \textit{Modus Ponens}
	\item[MTD] Metateorema da dedução
	\item[sse] se e somente se
	\item[CIC] \textit{Calculus of Inductive Constructions}
\end{siglas}
% % ---

% % ---
% % inserir lista de símbolos
% % ---


% \begin{simbolos}
%   \item[@] Arroba
%   \item[\%] Porcento
%   \item[$^\circ$C] Graus Celsius
%   \item[Ca] Cálcio
% \end{simbolos}

% ---
				% Elemento Opcional
\include{PreTextuais/Sumario}				% Elemento Obrigatório

% -----------------------------------------------------------------
% ELEMENTOS TEXTUAIS
% -----------------------------------------------------------------
\textual{}

\pagestyle{PagNumReduzida}						% Comando para cabeçalho somente com numeração de página 10pt
\aliaspagestyle{chapter}{PagNumReduzida}		% Deixar numeração da primeira página com tamanho igual ao resto da numeração
% ref.: https://groups.google.com/g/abntex2/c/CP7g8ZMgi-c/m/KjfEnn5b9a4J


% ---- Mantenha esta estrutura, assim você deixa o trabalho mais organizado -------

\chapter{Introdução} 
\label{cap:Introducao}


    \section{Objetivo Geral}


    \section{Objetivos Específicos}



    \section{Trabalhos Relacionados}


    \section{Metodologia}
        

    \section{Estrutura do Trabalho}
       
\chapter{Lógicas de Inconsistência Formal}
\label{cap:LFIs}
No estudo de lógicas clássicas uma contradição é considerada inseparável da trivialidade, ou seja, se uma teoria possuir um subconjunto $\{\alpha,\neg \alpha\}$ de fórmulas, pode-se derivar qualquer sentença. Esta propriedade é chamada de \textit{explosividade}. Desta forma, as lógicas clássicas (e certas lógicas não-clássicas, como a lógica intuicionista), expressam sua \textit{explosividade} como representada pela seguinte equação:
\begin{center}
    Contradições = Trivialidade
\end{center}
As \textit{Lógicas de Inconsistência Formal} são lógicas paraconsistentes que se propõem a questionar a noção apresentada anteriormente sem abrir mão completamente da trivialidade. Isto é feito estabelecendo uma nova propriedade, chamada de \textit{explosividade gentil}, que resgata a trivialidade introduzindo o conceito de consistência na sua linguagem~\cite{carnielli2007}. A consistência é expressa na \textit{explosividade gentil} da seguinte forma:
\begin{center}
    Contradições + Consistência = Trivialidade
\end{center}
Definir-se uma lógica que consiga superar o tabu da \textit{explosividade} e, ao mesmo tempo, representar uma ferramenta legítima capaz de formalizar o raciocínio e separar inferências aceitáveis de inferências equivocadas é o objetivo do \textit{paraconsistentista}.\helena{Esse termo é do diabo mas o carnielli usou e eu achei até que bonitinho.} As \textit{Lógicas de Inconsistência Formal} cumprem este objetivo de maneira elegante, servindo um propósito importante no estudo de lógicas não-clássicas.\helena{eu boto aquele lerolero de banco de dados aqui tb?}.
Neste capítulo são apresentadas algumas definições necessárias para caracterizar as \textit{Lógicas de Inconsistência Formal}, baseadas em~\citeshort{Carnielli_Coniglio_2016} e em~\citeshort{carnielli2007}. Antes de definir as \lfis{} é preciso apresentar alguns conceitos básicos acerca de sistemas lógicos paraconsistentes. Nas definições que seguem, utiliza-se a seguinte representação:
\begin{itemize}
    \item Letras minúsculas do alfabeto latim $p, q, r, \ldots$ para representar fórmulas atômicas.
    \item Letras maiúsculas do alfabeto latim $A, B, C, \ldots$ para representar conjuntos quaisquer.
    \item Letra minúsculas do alfabeto grego $\alpha, \beta, \gamma, \ldots$ para representar fórmulas quaisquer.
    \item As letras $\Gamma, \Delta$ para representar conjuntos de fórmulas.
    \item As letras $\Sigma, \Theta$ para representar a assinatura de uma linguagem.
    \item O operador $\vdash$ para representar uma relação de consequência sintática.
    \item O operador $\vDash$ para representar uma relação de consequência semântica.
\end{itemize}

Ademais, as definições que sucedem seguem o mesmo caminho de~\citeshort{Carnielli_Coniglio_2016}, baseando-se na teoria geral de relações de consequências para definir \textit{lógicas tarskianas} (lógicas com uma relação de consequência-T). Neste sentido, como a lógica \lfium{} se trata de uma \textit{lógica tarskiana}, o presente trabalho se restringe a trabalhar somente neste escopo.

\section{Paraconsistência}
\label{sec:paracons}

Uma lógica $\mathcal{L}$ será representada como uma dupla $\mathcal{L} = \langle \pazocal{L},\vdash \rangle$, onde $\pazocal{L}$ é sua linguagem (seu conjunto de fórmulas) e $\vdash$ é uma relação de consequência de conclusão única, definida como $\vdash \;\subseteq \wp(\pazocal{L})\times\pazocal{L}$.

\begin{definicao}[Assinatura proposicional]
    \label{def:ass_prop}
    Uma assinatura proposicional $\Theta$ é um conjunto de conectivos lógicos com a informação acerca da aridade de cada um destes.\qed{}
\end{definicao}
Por exemplo, a assinatura proposicional para a lógica proposicional clássica pode ser definida como $\Theta_{LPC} = \{\land^{2}, \lor^{2}, \neg^{1}, \rightarrow^{2}\}$.

\begin{definicao}[Lógica proposicional]
    \label{def:proposicional}
    Um sistema lógico $\mathcal{L}$, definido sobre uma linguagem $\pazocal{L}$ é dito proposicional caso $\pazocal{L}$ seja definida a partir de um conjunto enumerável de átomos $\pazocal{P} = \{p_{i} \;| \; i \in \mathbb{N} \}$ e uma assinatura proposicional $\Theta$. Uma linguagem $\pazocal{L}$ definida sobre uma assinatura proposicional é chamada de linguagem proposicional.\qed{}
\end{definicao}

\begin{definicao}[Substituição]
    Uma substituição $\sigma$ de todas as ocorrências de uma variável $p_{i}$ por uma fórmula $\psi$ em uma fórmula $\phi$, é denotada por $\sigma(\phi) = \phi[p_{i} \mapsto \psi]$~\cite{dedo}. A substituição $\phi[p_{i} \mapsto \psi]$ é definida indutivamente como (considerando $\triangle$, $\otimes$ conectivos quaisquer de aridade 1 e 2 respectivamente):
    \begin{align*}
         & \text{1.~Se }\phi = p_{i} \text{ então, } \phi[p_{i} \mapsto \psi] = \psi;                                                                                             \\
         & \text{2.~Se }\phi = p_{j} \text{ e } j \neq i \text{ então, }\phi[p_{i} \mapsto \psi] = \phi;                                                                          \\
         & \text{3.~Se }\phi = \triangle \gamma \text{ então, } \phi[p_{i} \mapsto \psi] = \triangle(\gamma[p_{i} \mapsto \psi]);                                                 \\
         & \text{4.~Se }\phi = \phi_{0} \otimes \phi_{1} \text{ então, } \phi[p_{i} \mapsto \psi] = \phi_{0}[p_{i} \mapsto \psi] \otimes \phi_{1}[p_{i} \mapsto \psi].\tag*\qed{}
    \end{align*}
\end{definicao}

\begin{notacao}
    Sejam $\Gamma, \Delta$ conjuntos de fórmulas e $\phi, \psi$ fórmulas quaisquer, então $\Gamma, \Delta, \phi \vdash \psi$ denota $\Gamma \cup \Delta \cup \{\phi\} \vdash \psi$.
\end{notacao}

\begin{definicao}[Lógica Tarskiana]
    \label{def:tarski}
    Uma lógica $\mathcal{L}$, definida sobre uma linguagem $\pazocal{L}$ e munida com uma relação de consequência $\vdash$ é dita \textit{Tarskiana} caso satisfaça as seguintes propriedades para todo $\Gamma \cup \Delta \cup \{\alpha\} \subseteq \pazocal{L}$:
    \begin{align}
         & \text{~~(i) Se } \alpha \in \Gamma \text{ então } \Gamma \vdash \alpha;\tag{reflexividade}                                                                                       \\
         & \text{~(ii) Se } \Delta \vdash \alpha \text{ e } \Delta \subseteq \Gamma \text{ então } \Gamma \vdash \alpha;\tag{monotonicidade}                                                \\
         & \text{(iii) Se } \Delta \vdash \alpha \text{ e } \Gamma \vdash \delta \text{ para todo } \delta \in \Delta \text{ então } \Gamma \vdash \alpha.\tag{\textit{cut} para conjuntos} \\\tag*\qed{}
    \end{align}
    Uma lógica $\mathcal{L}$ é dita \textit{finitária} caso satisfaça o seguinte:
    \begin{align*}
         & \text{~(iv) Se } \Gamma \vdash \alpha \text{ então existe conjunto finito } \Gamma_{0} \subseteq \Gamma \text{ tal que } \Gamma_{0} \vdash \alpha.
    \end{align*}
    Uma lógica proposicional $\mathcal{L}$ definida sobre uma linguagem proposicional $\pazocal{L}_{\Theta}$ é dita \textit{estrutural} caso respeite a seguinte condição:
    \begin{align*}
         & \text{~~(v) Se } \Gamma \vdash \alpha \text{ então } \sigma |\Gamma| \vdash \sigma(\alpha) \text{, para toda substituição } \sigma \text{ de variável por fórmula.}
    \end{align*}
    Por fim, uma lógica proposicional $\mathcal{L}$ é dita \textit{padrão} caso ela seja Tarskiana, finitária e estrutural.\qed{}
\end{definicao}


Com isto, é possível definir formalmente a \textit{paraconsistência} para lógicas Tarskianas.

\begin{definicao}[Lógica Tarskiana paraconsistente]
    \label{def:tarskiana_paracons}
    Uma lógica Tarskiana $\mathcal{L}$, definida sobre uma linguagem $\pazocal{L}$, é dita \textit{paraconsistente} se ela possuir uma negação $\neg$\footnote{Esta negação pode ser primitiva (pertencente à assinatura da linguagem) ou definida a partir de outras fórmulas.} tal que existem fórmulas $\alpha, \beta \in \pazocal{L}$ de modo que $\alpha, \neg \alpha \nvdash \beta$.\qed{}
\end{definicao}

Caso a linguagem de $\mathcal{L}$ possua uma implicação $\rightarrow$ que respeite o metateorema da dedução\footnote{Definido como $\Gamma, \alpha \vdash \beta \Longleftrightarrow  \Gamma\vdash \alpha \rightarrow \beta$.}, então $\mathcal{L}$ é paraconsistente somente se a fórmula $\alpha \rightarrow (\neg \alpha \rightarrow \beta)$ não for válida. Ou seja, o Princípio da Explosão é inválido (em relação a $\neg$), logo $\neg$ é uma negação \textit{não explosiva}.

\section{Inconsistência}

A motivação para o desenvolvimento das \lfis{} é possuir sistemas lógicos paraconsistentes nos quais é possível resgatar, de maneira \textit{controlada}, o Princípio da Explosão. Isto é feito definindo um conjunto $\bigcirc(p)$ de fórmulas dependentes somente de uma variável proposicional $p$. Caso uma lógica $\mathcal{L}$ seja explosiva ao unir-se um conjunto $\bigcirc(\alpha)$ {-} definido a partir de $\bigcirc(p)$ {-} com uma contradição $\{\alpha, \neg \alpha\}$, ou seja, se $\bigcirc(\alpha), \alpha, \neg \alpha \vdash \beta$ para todo $\alpha$ e $\beta$ pertencentes à sua linguagem, e ainda $\bigcirc(\alpha), \alpha \nvdash \beta$ e $\bigcirc(\alpha), \neg \alpha \nvdash \beta$, então dizemos que $\mathcal{L}$ é \textit{gentilmente explosiva}.

\begin{notacao}
    Dado um átomo $p$, define-se $\bigcirc(p)$ como um conjunto não-vazio de fórmulas dependentes somente em $p$. Com base neste conjunto, define-se a notação $\bigcirc(\phi)$ para representar o conjunto obtido pela substituição de todas as ocorrências de $p$ por $\phi$ em todos os elementos de $\bigcirc(p)$, ou seja, para uma fórmula $\phi$ qualquer, $\bigcirc(\phi) = \{\psi[p \mapsto \phi] \; | \; \psi \in \bigcirc(p)\}$.
\end{notacao}



\begin{definicao}[Lógica de Inconsistência Formal]
    \label{def:lfi}
    Seja $\mathcal{L} = \langle \pazocal{L}_{\Theta}, \vdash \rangle$ uma lógica padrão, de forma que sua assinatura proposicional $\Theta$ possua uma negação $\neg$. Seja $\bigcirc(p)$ um conjunto não-vazio de fórmulas dependentes somente na variável proposicional $p$. Então $\mathcal{L}$ será uma \textit{Lógica de Inconsistência Formal} (\lfi{}) (em relação a $\bigcirc(p)$ e $\neg$) caso ela respeite as seguintes condições:
    %(considerando $\bigcirc(\phi) = \{\psi(\phi) \; | \; \psi(p) \in \bigcirc(p)\}$): \helena{O conjunto $\bigcirc(\phi)$ é definido substituindo-se cada ocorrência $p$ por $\phi$ em cada elemento de $\bigcirc(\phi)$.}
    \begin{align*}
         & \text{~~(i) Existem } \gamma, \delta \in \pazocal{L}_{\Theta} \text{ de modo que } \gamma, \neg \gamma \nvdash \delta;               \\
         & \text{~(ii) Existem } \alpha, \beta \in \pazocal{L}_{\Theta} \text{ de modo que:}                                                    \\
         & \qquad \text{(ii.a)} \bigcirc(\alpha), \alpha \nvdash \beta;                                                                         \\
         & \qquad \text{(ii.a)} \bigcirc(\alpha), \neg \alpha \nvdash \beta;                                                                    \\
         & \text{(iii) Para todo } \phi, \psi \in \pazocal{L}_{\Theta} \text{ tem-se } \bigcirc(\phi), \phi, \neg \phi \vdash \psi. \tag*\qed{}
    \end{align*}
\end{definicao}
\helena{Definir \lfi{} fraca e forte também?!?!?!?}
A condição (i) diz que toda \lfi{} é \textit{não-explosiva} (em relação a $\neg$) e a condição (iii) diz que toda \lfi{} é \textit{gentilmente explosiva} (em relação a $\bigcirc{p}$ e $\neg$).



\chapter{LFI1}
\label{cap:LFI1}
BLABLABLABLABLA introduz um textinho

    
    \section{Linguagem}
    A lógica \textbf{LFI1}\footnote{Por mais que existam extensões de primeira ordem da \textbf{LFI1}, como a \textbf{LFI1*}, definida em~\cite{carnielli2000formal}, o presente trabalho trata somente do fragmento proposicional desta.} aqui apresentada é definida com base nos trabalhos de Carnielli, Coniglio, Marcos e Amo~\cite{carnielli2000formal, possible-translation, carnielli2007,Carnielli_Coniglio_2016} como uma estrutura da forma $\langle \mathcal{S}, \Vdash \rangle$, onde $\mathcal{S}$ é a sua linguagem (seu conjunto de fórmulas) e $\Vdash$ é uma relação de consequência de conclusão única definida como $\Vdash \;\subseteq \wp(\mathcal{S})\times\mathcal{S}$. A linguagem\footnote{A linguagem da \textbf{LFI1} pode ser definida de maneira equivalente utilizando-se o operador de consistência (representado por $\circ$), seguindo a definição $\circ \alpha \eqdef \neg \bullet \alpha$.} $\mathcal{S}$ da \textbf{LFI1} é definida sobre um conjunto enumerável de átomos $\mathcal{P} = \{p_{n} \;|\; n \in \omega\}$ e uma assinatura $\Sigma = \{\land, \lor, \rightarrow, \neg, \bullet\}$ da seguinte forma:

    \begin{definicao}[Linguagem da \textbf{LFI1}]
        \label{def:lang}
        A linguagem $\mathcal{S}$ da \textbf{LFI1} é definida indutivamente como o menor conjunto a partir das seguintes regras:
        \begin{align*}
            & \mathcal{P} \subseteq \mathcal{S}\\
            & \text{Se } \varphi \in \mathcal{S}, \text{então } \triangle  \varphi \in \mathcal{S}, \text{com } \triangle \in \{\neg, \bullet\}\\
            & \text{Se } \varphi, \psi \in \mathcal{S}, \text{então } \varphi \otimes \psi \in \mathcal{S}, \text{com } \otimes \in \{\land, \lor, \rightarrow\} \tag*\qed
        \end{align*}
    \end{definicao}

    \begin{definicao}[Subfórmulas]
        \label{def:subf}
        O conjunto Sub$(\varphi)$ de subfórmulas de uma fórmula $\varphi$ é definido indutivamente por:
        \begin{align*}
            & \text{Sub}(p_{i}) = \{p_{i}\}, \; p_{i} \in \mathcal{P}\\
            & \text{Sub}(\triangle \varphi) = \{\triangle \varphi\} \; \cup \;\text{Sub}(\varphi), \; \triangle \in \{\neg, \bullet\}\\
            & \text{Sub}(\varphi \otimes \psi) = \{\varphi \otimes \psi\} \; \cup \;\text{Sub}(\varphi) \; \cup \;\text{Sub}(\psi), \; \otimes \in \{\land, \lor, \rightarrow\} \tag*\qed
        \end{align*}
    \end{definicao}


\chapter{Biblioteca}\label{cap:biblioteca}

    \migs{Seguestão: renomear esse capítulo para algo tipo ``Implementação em Coq'' ou algo assim, ``Biblioteca'' fica muito genérico.}
Neste capítulo será \migscortar{feita}{ descrita} a implementação da biblioteca da lógica de inconsistência formal \lfium{} no assistente de provas Coq, bem como o desenvolvimento de alguns metateoremas apresentados na Seção~\ref{sec:metateoremas} dentro da biblioteca. A implementação será análoga àquela feita por~\citeshort{silveira2020implementacao}, que implementou uma biblioteca de lógica modal. Antes de tratar especificamente da implementação, o Coq será brevemente apresentado e caracterizado.

\section{Assistente de provas Coq}\label{sec:coq}
    Os assistentes de provas são ferramentas de \textit{software} que auxiliam o usuário no desenvolvimento de teoremas, permitindo que provas sejam verificadas na medida em que são escritas~\cite{geuvers2009proof}, conferindo a estes programas uma importância significativa na verificação e especificação formal de \textit{software}. Atualmente, existem diversos assistentes de provas como: Agda, Isabelle, Coq, Lean, HOL, Idris e Twelf. Cada um destes tem suas particularidades e diferenças em relação ao formalismo matemático utilizado como base.

    O Coq é um assistente de provas baseado no Cálculo de Construções Indutivas (CCI) que possui aplicações em diferentes áreas da matemática e da computação como (mas não limitado a) lógica, linguagens formais, linguística computacional e desenvolvimento de programas seguros~\cite{coqart}. Sob a ótica da Correspondência de Curry{-}Howard, o Coq é tanto uma linguagem de programação funcional quanto uma linguagem de prova, podendo ser dividido em quatro partes~\cite{silva2019certificaccao}:
    
    \begin{itemize}
        \item A linguagem de programação e especificação \textit{Gallina}, que goza da propriedade da normalização forte\footnote{Um \migscortar{objeto}{ termo-$\lambda$} é fortemente normalizável caso toda sequência de reescrita acabe numa forma normal (num termo irredutível). Um sistema no qual todos os \migscortar{objeto}{ termos-$\lambda$} são fortemente normalizáveis possui a propriedade da normalização forte.~\cite{nipkow2006rewriting}}, a qual garante que todo programa termina.
        \item A linguagem de comandos \textit{Vernacular}, que permite interagir com o assistente.
        \item O conjunto de táticas (\textit{tactics}) utilizadas para manipular elementos durante o desenvolvimento de uma prova.
        \item A linguagem $\pazocal{L}$tac, utilizada para implementar novas táticas e automatizar provas.
    \end{itemize}

    \migscortar{No restante do trabalho é presumido que o leitor tem um conhecimento básico acerca do o funcionamento do assistente de provas Coq no desenvolvimento de provas e verificação de programas. Se este não for o caso, uma consulta ao livro Logical Foundations~\cite{Pierce2017Logical} e ao trabalho de~\citeshort{silveira2020implementacao} é recomendada, a fim de conhecer o Coq e tomar nota das diferentes funcionalidades presentes no assistente.}{ No restante desse trabalho, conceitos básicos sobre o funcionamento do assistente de provas Coq e sobre seu uso no desenvolvimento de provas e verificação de provas não serão apresentados em grandes detalhes. O leitor interessado em tais assuntos é recomendado a consultar os trabalhos de~\citeshort{Pierce2017Logical} e~\citeshort{silveira2020implementacao}.}

\chapter{Conclusões Parciais}\label{chap:conclusao}

O estudo de lógicas paraconsistentes mostra-se relevante para o desenvolvimento de \textit{softwares} capazes de lidar com informações contraditórias. Dentro desta família de lógicas, os sistemas de inconsistência formal destacam-se no contexto de bases de dados {---} sobretudo bases evolucionárias {---} já que internalizam o conceito de contraditoriedade dentro da sua linguagem. A \lfium{} é uma lógica de inconsistência formal com propriedades que facilitam o desenvolvimento de sistemas de gerenciamento de bancos de dados, por exemplo, mesmo na presença de inconsistência na base.

Assistentes de provas são ferramentas de \textit{software} que permitem ao usuário provar teoremas sobre objetos expressos dentro de si, sem que a verificação destas provas dependa de um julgamento humano para garantir sua validade. O Coq é um assistente de provas, bem como uma linguagem de programação, robusto e com um núcleo axiomático enxuto, que possui aplicações em diferentes áreas da matemática, como lógica, linguagens formais, linguística computacional e desenvolvimento de programas seguros~\cite{coqart}.

O presente trabalho define a linguagem, sintaxe e semântica da \lfium{}, além de revisar e desenvolver manualmente metateoremas que evidenciam características deste sistema, como a correção, completude e o metateorema da dedução, servindo como base para o prosseguimento do TCC2, no qual propõe-se implementar uma biblioteca da lógica \lfium{} em Coq e provar metapropriedades desta lógica dentro do assistente.

Sendo assim, no que segue, é apresentada uma lista de itens que propõem-se serem explorados no TCC2, juntamente com um cronograma para a execução de cada item.

\begin{enumerate}
    \item Definir a linguagem da \lfium{} na biblioteca;
    \item Implementar a sintaxe (cálculo de Hilbert) da \lfium{};
    \item Implementar os sistemas semânticos (matricial e bivaloração) da \lfium{};
    \item Desenvolver metateoremas da \lfium{} na biblioteca.
\end{enumerate}

  \setcounter{table}{1}

  \begin{table}[htbp]
    \centering
    \begin{tabular}{|c|c|c|c|c|c|c|c|c|}
      \hline
      \multirow{2}{*}{\textbf{\small{Item}}} & \textbf{\small{2024/1}} & \multicolumn{6}{c|}{\textbf{\small{2024/2}}} \\
      \cline{2-8}
      & \textbf{Dez} & \textbf{Jan} & \textbf{Fev} & \textbf{Mar} & \textbf{Abr} & \textbf{Maio} & \textbf{Jun} \\
      \hline
      \textbf{\small{1}}  & \cellcolor{gray} & \cellcolor{gray} &  &  &  &  & \\
      \hline
      \textbf{\small{2}}  &  & \cellcolor{gray} & \cellcolor{gray} &  &  &  & \\
      \hline
      \textbf{\small{3}}  &  & \cellcolor{gray} & \cellcolor{gray} & \cellcolor{gray} &  &  & \\
      \hline
      \textbf{\small{4}}  &  &  & \cellcolor{gray} & \cellcolor{gray} & \cellcolor{gray} & \cellcolor{gray} & \cellcolor{gray}\\
      \hline
    \end{tabular}
    \caption{Cronograma Proposto para o TCC2}
  \end{table}

% \chapter{Exemplo Prova com Casos e Subcasos}
        \lipsum[1]
        \begin{provaporcasos}
        
            \casodeprova \lipsum[66]

            \begin{provaporsubcasos}
            
                \subcasodeprova \lipsum[75]

                \subcasodeprova \lipsum[75]

                \subcasodeprova \lipsum[75]

                \subcasodeprova \lipsum[75]

            \end{provaporsubcasos}

            \casodeprova \lipsum[66]

            \casodeprova \lipsum[66]

            \begin{provaporsubcasos}
                
                \subcasodeprova \lipsum[75]

                \begin{provaporsubsubcasos}
                    
                    \subsubcasodeprova \lipsum[75]

                \end{provaporsubsubcasos}

            \end{provaporsubcasos}
        \end{provaporcasos}


        \begin{proof}[Prova do Teorema~\ref{teo:correcao_val}]
            Supondo $\Gamma \conhil \alpha$, existe uma sequência de derivação $\phi_{1} \ldots \phi_{n}$ onde $\phi_{n} = \alpha$. A prova de $\Gamma \conval \alpha$ é obtida por indução no tamanho $n$ da sequência de derivação:\\
    
            \noindent \textbf{\textsc{Base}} $n = 1$. A sequência contém somente uma fórmula $\phi_{1} = \alpha$. Portanto, existem duas possibilidades:
            \begin{enumerate}
                \item $\phi_{1}$ é um axioma.
                \item $\phi_{1} \in \Gamma$.
            \end{enumerate}
    
            \begin{provaporcasos}
                
                \casodeprova{} $\phi_{1}$ é um axioma. Então basta mostrar que para todo $v \in V^{\lfium{}}$, se $v(\gamma) = 1$ para todo $\gamma \in \Gamma$, então $v(\phi_{1}) = 1$. Como $\phi_{1}$ é um axioma, basta analisar todos os casos possíveis:
    
                \begin{provaporsubcasos}
                    
                    \subcasodeprova{} $\phi_{1} = \alpha \to (\beta \to \alpha)$.
    
                        Supondo $v(\alpha \to (\beta \to \alpha)) = 0$, temos $v(\alpha) = 1$ e $v(\beta \to \alpha) = 0$ por $(vImp)$. 
                            
                        Logo, $v(\beta) = 1$ e $v(\alpha) = 0$ novamente por $(vImp)$. Isto resulta numa contradição. 
                        
                        Portanto $v(\alpha \to (\beta \to \alpha)) = 1$.
    
                    \subcasodeprova{} $\phi_{1} = (\alpha \to (\beta \to \gamma)) \to ((\alpha \to \beta) \to (\alpha \to \gamma ))$.
                    
                        Supondo $v((\alpha \to (\beta \to \gamma)) \to ((\alpha \to \beta) \to (\alpha \to \gamma))) = 0$, temos, por ($vImp$), $v(\alpha \to (\beta \to \gamma)) = 1 \text{ e } v((\alpha \to \beta) \to (\alpha \to \gamma)) = 0$.
    
                        Portanto, por $(vImp)$, temos $v(\alpha) = 0$ ou $v(\beta \to \gamma) = 1$, e $v(\alpha \to \beta) = 1$ e $v(\alpha \to \gamma) = 0$. 
                        
                        Por $(vImp)$ segue $v(\alpha) = 1$ e $v(\gamma) = 0$, logo, $v(\beta \to \gamma) = 1$ e, portanto, $v(\beta) = 0$ ou $v(\gamma) = 1$. 
                        
                        Porém, como $v(\alpha \to \beta) = 1$ e $v(\alpha) = 1$, então $v(\beta) = 1$, o que resulta numa contradição. Logo $v((\alpha \to (\beta \to \gamma)) \to ((\alpha \to \beta) \to (\alpha \to \gamma))) = 1$.
    
                    \subcasodeprova{} $\phi_{1} = \alpha \to (\beta \to (\alpha \land \beta))$. 
    
                        Supondo $v(\alpha \to (\beta \to (\alpha \land \beta))) = 0$, temos $v(\alpha) = 1$ e $v(\beta \to (\alpha \land \beta)) = 0$, por $(vImp)$, então $v(\beta) = 1$ e $v(\alpha \land \beta) = 0$ novamente por $(vImp)$. 
                        
                        Com isso, temos $v(\alpha) = 0$ ou $v(\beta) = 0$ por $(vAnd)$, mas isso resulta numa contradição, já que $v(\alpha) = 1$ e $v(\beta) = 1$. 
                        
                        Portanto $v(\alpha \to (\beta \to (\alpha \land \beta))) = 1$.
    
                    \subcasodeprova{} $\phi_{1} = (\alpha \land \beta) \to \alpha$. 
                    
                        Supondo $v((\alpha \land \beta) \to \alpha) = 0$. 
                    
                        Logo $v(\alpha \land \beta) = 1$ e $v(\alpha) = 0$ por $(vImp)$. 
                        
                        Então, $v(\alpha) = 1$ e $v(\beta) = 1$ por $(vAnd)$, o que resulta numa contradição. 
                        
                        Portanto $v((\alpha \land \beta) \to \alpha) = 1$.
    
                    \subcasodeprova{} $\phi_{1} = (\alpha \land \beta) \to \beta$. Como no caso anterior, \textit{mutatis mutandis}.
    
                    \subcasodeprova{} $\phi_{1} = \alpha \to (\alpha \lor \beta)$. 
                    
                        Supondo $v(\alpha \to (\alpha \lor \beta)) = 0$, temos $v(\alpha) = 1$ e $v(\alpha \lor \beta) = 0$, por $(vImp)$. 
                        
                        Então temos $v(\alpha) = 0$ e $v(\beta) = 0$ por $(vOr)$, o que resulta numa contradição. 
                        
                        Portanto $v(\alpha \to (\alpha \lor \beta)) = 1$.
    
                    \subcasodeprova{} $\phi_{1} = \beta \to (\alpha \lor \beta)$. Como no caso anterior, \textit{mutatis mutandis}.
    
                    \subcasodeprova{} $\phi_{1} = (\alpha \to \gamma) \to ((\beta \to \gamma) \to ((\alpha \lor \beta) \to \gamma))$. 
                    
                        Supondo $v((\alpha \to \gamma) \to ((\beta \to \gamma) \to ((\alpha \lor \beta) \to \gamma))) = 0$, temos, por $(vImp)$, $v((\alpha \to \gamma)) = 1$ e $v(((\beta \to \gamma) \to ((\alpha \lor \beta) \to \gamma))) = 0$. 
                        
                        Portanto, novamente por $(vImp)$, temos $v(\alpha) = 0$ ou $v(\gamma) = 1$. Além disso, temos $v(\beta \to \gamma) = 1$ e $v((\alpha \lor \beta) \to \gamma) = 0$. 
                        
                        Então, temos $v(\beta) = 0$ ou $v(\gamma) = 1$. Ademais, $v(\alpha \lor \beta) = 1$ e $v(\gamma) = 0$. 
                        
                        Finalmente, por $(vOr)$, temos $v(\alpha) = 1$ ou $v(\beta) = 1$. O fato de termos $v(\gamma) = 0$ nos permite concluir $v(\alpha) = 0$ e $v(\beta) = 0$, porém isto nos dá $v(\alpha) = 1$ (já que temos $v(\alpha) = 1$ ou $v(\beta) = 1$ e também $v(\beta) = 0$), o que resulta numa contradição, pois já temos $v(\alpha) = 0$. 
                        
                        Portanto, $v((\alpha \to \gamma) \to ((\beta \to \gamma) \to ((\alpha \lor \beta) \to \gamma))) = 1$.
    
                    \subcasodeprova{} $\phi_{1} = (\alpha \to \beta) \lor \alpha$. 
                        
                        Supondo $v((\alpha \to \beta) \lor \alpha) = 0$, então, por $(vOr)$, temos $v(\alpha \to \beta) = 0$ e $v(\alpha) = 0$. 
                        
                        Logo, por $(vImp)$, $v(\alpha) = 1$  e $v(\beta) = 0$, o que resulta numa contradição. 
                        
                        Portanto, $v((\alpha \to \beta) \lor \alpha) = 1$.
    
                    \subcasodeprova{} $\phi_{1} = \alpha \lor \neg \alpha$. 
                    
                        Supondo $v(\alpha \lor \neg \alpha) = 0$, temos por $(vOr)$, $v(\alpha) = 0$ e $v(\neg \alpha) = 0$. 
                        
                        Então, por $(vNeg)$, $v(\alpha) = 1$, o que resulta numa contradição. 
                        
                        Portanto, $v(\alpha \lor \neg \alpha) = 1$.
    
                    \subcasodeprova{} $\phi_{1} = \circ \alpha \to (\alpha \to (\neg \alpha \to \beta))$. 
                    
                        Supondo $v(\circ \alpha \to (\alpha \to (\neg \alpha \to \beta))) = 0$, então, por $(vImp)$, $v(\circ \alpha) = 1$ e $v(\alpha \to (\neg \alpha \to \beta)) = 0$. 
                        
                        Logo, por $(vCon)$, $v(\alpha) = 0$ ou $v(\neg \alpha) = 0$. Ademais, por $(vImp)$, $v(\alpha) = 1$ e $v(\neg \alpha \to \beta) = 0$. 
                        
                        Novamente por $(vImp)$, $v(\neg \alpha) = 1$ e $v(\beta) = 0$. Podemos concluir $v(\alpha) = 0$, já que temos $v(\neg \alpha) = 1$ e também temos $v(\alpha) = 0$ ou $v(\neg \alpha) = 0$. 
                        
                        Isto resulta numa contradição, pois temos $v(\alpha) = 1$ e $v(\alpha) = 0$. 
                        
                        Portanto, $v(\circ \alpha \to (\alpha \to (\neg \alpha \to \beta))) = 1$.
    
                    \subcasodeprova{} $\phi_{1} = \neg \neg \alpha \to \alpha$. 
                        
                        Supondo $v(\neg \neg \alpha \to \alpha) = 0$, então, por $(vImp)$, $v(\neg \neg \alpha) = 1$ e $v(\alpha) = 0$. 
                        
                        Por $(vDNE)$ temos $v(\alpha) = 1$, o que resulta numa contradição. 
                        
                        Portanto, $v(\neg \neg \alpha \to \alpha) = 1$.
    
                    \subcasodeprova{} $\phi_{1} = \alpha \to \neg \neg \alpha$. 
                    
                        Supondo $v(\alpha \to \neg \neg \alpha) = 0$, então, por $(vImp)$, $v(\alpha) = 1$ e $v(\neg \neg \alpha) = 0$. 
                        
                        Por $(vDNE)$ temos $v(\neg \neg \alpha) = 1$, o que resulta numa contradição. 
                        
                        Portanto, $v(\alpha \to \neg \neg \alpha) = 1$.
    
                    \subcasodeprova{} $\phi_{1} = \neg \circ \alpha \to (\alpha \land \neg \alpha)$. 
                    
                        Supondo $v(\neg \circ \alpha \to (\alpha \land \neg \alpha)) = 0$, temos, por $(vImp)$, $v(\neg \circ \alpha) = 1$ e $v(\alpha \land \neg \alpha) = 0$. 
                        
                        Por $(vCi)$ temos $v(\alpha) = 1$ e $v(\neg \alpha) = 1$. Ademais, por $(vAnd)$, temos $v(\alpha) = 0$ ou $v(\neg \alpha) = 0$. 
                        
                        Podemos concluir $v(\alpha) = 0$, já que temos $v(\alpha) = 0$ ou $v(\neg \alpha) = 0$ e também temos $v(\neg \alpha) = 1$. 
                        
                        Isto resulta numa contradição, pois temos $v(\alpha) = 1$ e  $v(\alpha) = 0$. 
                        
                        Portanto, $v(\neg \circ \alpha \to (\alpha \land \neg \alpha)) = 1$.
    
                    \subcasodeprova{} $\phi_{1} = \neg (\alpha \lor \beta) \to (\neg \alpha \land \neg \beta)$. 
                    
                        Supondo $v(\neg (\alpha \lor \beta) \to (\neg \alpha \land \neg \beta)) = 0$, temos, por $(vImp)$, $v(\neg (\alpha \lor \beta)) = 1$ e $v(\neg \alpha \land \neg \beta) = 0$. 
                        
                        Por $(vDM_{\lor})$ temos $v(\neg \alpha) = 1$ e $v(\neg \beta) = 1$. Ademais, por $(vAnd)$, temos $v(\neg \alpha) = 0$ ou $v(\neg \beta) = 0$. 
                        
                        Isto, unido ao fato de termos $v(\neg \beta) = 1$, nos permite concluir $v(\neg \alpha) = 0$, o que resulta numa contradição. 
                        
                        Portanto, $v(\neg (\alpha \lor \beta) \to (\neg \alpha \land \neg \beta)) = 1$.
    
    
                    \subcasodeprova{} $\phi_{1} = (\neg \alpha \land \neg \beta) \to \neg (\alpha \lor \beta)$. Como no caso anterior, \textit{mutatis mutandis}.
    
                    \subcasodeprova{} $\phi_{1} = \neg(\alpha \land \beta) \to (\neg \alpha \lor \neg \beta)$. 
                    
                    Supondo $v(\neg(\alpha \land \beta) \to (\neg \alpha \lor \neg \beta)) = 0$, temos, por $(vImp)$, $v(\neg(\alpha \land \beta)) = 1$ e $v(\neg \alpha \lor \neg \beta) = 0$. 
                    
                    Então, por $(vDM_{\land})$, temos $v(\neg \alpha) = 1$ ou $v(\neg \beta) = 1$. Além disso, por $(vOr)$, temos $v(\neg \alpha) = 0$ e $v(\neg \beta) = 0$. 
                    
                    Isto nos permite concluir $v(\neg \alpha) = 1$, pois temos $v(\neg \alpha) = 1$ ou $v(\neg \beta) = 1$ e também temos $v(\neg \beta) = 0$. 
                    
                    Isto resulta numa contradição, pois temos $v(\neg \alpha) = 1$ e $v(\neg \alpha) = 0$. 
                    
                    Portanto, $v(\neg(\alpha \land \beta) \to (\neg \alpha \lor \neg \beta)) = 1$.
    
                    \subcasodeprova{} $\phi_{1} = (\neg \alpha \lor \neg \beta) \to \neg (\alpha \land \beta)$. Como no caso anterior, \textit{mutatis mutandis}.
    
                    \subcasodeprova{} $\phi_{1} = \neg (\alpha \to \beta) \to(\alpha \land \neg \beta)$. 
                    
                    Supondo $v(\neg (\alpha \to \beta) \to (\alpha \land \neg \beta)) = 0$. Por $(vImp)$ temos $v(\neg (\alpha \to \beta)) = 1$ e $v(\alpha \land \neg \beta) = 0$. 
                    
                    Então, por $(vCip_{\to})$, temos $v(\alpha) = 1$ e $v(\neg \beta) = 1$. Ademais, por $(vAnd)$, temos $v(\alpha) = 0$ ou $v(\neg \beta) = 0$, o que unido ao fato de termos $v(\neg \beta) = 1$, nos permite concluir $v(\alpha) = 0$. 
                    
                    Isto resulta numa contradição, portanto $v(\neg (\alpha \to \beta) \to (\alpha \land \neg \beta)) = 1$.
    
                    \subcasodeprova{} $\phi_{1} = (\alpha \land \neg \beta) \to \neg(\alpha \to \beta)$. Como no caso anterior, \textit{mutatis mutandis}.
                    
                \end{provaporsubcasos}
    
                Com isso, o \textsc{Caso 1} está provado e $\Gamma \conval \phi_{1}$ segue caso $\phi_{1}$ seja um axioma.
    
                \casodeprova{} $\phi_{1} \in \Gamma$. Logo, se $v[\Gamma] \subseteq \{1\}$, temos $v(\phi_{1}) = 1$. Portanto, $\Gamma \conval \phi_{1}$.
    
            \end{provaporcasos}
    
             \noindent \textbf{\textsc{Passo}} Hipótese de indução (HI): Para qualquer sequência da derivação de $\Gamma \conhil \alpha$ de tamanho $k < n$, tem-se $\Gamma \conval \alpha$. 
             
             Portanto, é preciso mostrar que $\Gamma \conval \alpha$ segue caso a sequência de derivação de $\Gamma \conhil \alpha$ tenha tamanho $n$. Então, vamos supor $\Gamma \conhil \phi_{n}$.
             
             Ao analisar a obtenção de $\phi_{n}$ em $\Gamma \conhil \phi_{n}$, existem três casos a se considerar:
             
             \begin{enumerate}
                \item $\phi_{n}$ é um axioma.
                \item $\phi_{n} \in \Gamma$.
                \item $\phi_{n}$ é obtido por \textit{modus ponens} em duas fórmulas $\phi_{j}$ e $\phi_{k}$ com $j, k < n$. 
             \end{enumerate}
             
             Os casos 1 e 2 são análogos aos casos provados na base.
             
             \noindent \textsc{Caso 3.} $\phi_{n}$ é obtido por \textit{modus ponens} em duas fórmulas $\phi_{j}$ e $\phi_{k}$ com $j, k < n$. 
             
             Logo, $\phi_{k} = \phi_{j} \to \phi_{n}$ (ou $\phi{j} = \phi_{k} \to \phi_{n}$, a prova para este caso é análoga). 
             
             Dada nossa suposição de $\Gamma \conhil \phi_{n}$, então $\Gamma \conhil \phi_{j}$ e $\Gamma \conhil \phi_{j} \to \phi_{n}$. 
             
             Pela (HI), temos $\Gamma \conval \phi_{j}$ e $\Gamma \conval \phi_{j} \to \phi_{n}$. 
             
             Então, supondo $v[\Gamma] \subseteq \{1\}$, temos $v(\phi_{j}) = 1$ e $v(\phi_{j} \to \phi_{n}) = 1$. 
             
             Por $(vImp)$ temos $v(\phi_{j}) = 0$ ou $v(\phi_{n}) = 1$. 
             
             Isto, unido ao fato de termos $v(\phi_{j}) = 1$, nos permite concluir $v(\phi_{n}) = 1$. 
             
             Portanto $\Gamma \conval \phi_{n}$.
    
             \noindent Com isso, provamos $\Gamma \conval \alpha$ e a prova está finalizada.
    
        \end{proof}

% -----------------------------------------------------------------
% ELEMENTOS PÓS-TEXTUAIS
% -----------------------------------------------------------------
\postextual{}

% Você pode comentar os elementos que não deseja em seu trabalho;

% Referências bibliográficas
\bibliography{Referencias}	% Elemento Obrigatório

%\include{PosTextuais/Glossario}			% Elemento Opcional

% ----------------------------------------------------------
% Apêndices
% ----------------------------------------------------------

% ---
% Inicia os apêndices
% ---
\begin{apendicesenv}

    \chapter{Prova Formal da Transferência de Completude pela Fusão de Lógicas Modais}
        \label{app:ProvaTransferenciaCompletude}

        \begin{proof}[Prova do Teorema~\ref{teo:TransCompletude}]
            Sendo \(\Gamma \subseteq \Theta\) um conjunto \Mathcali{L}{12}-consistente, devemos provar que \GAMMA é verdadeiro em algum mundo de um 12-modelo baseado
            em um 12-frame para \Mathcali{L}{12}. Como \GAMMA é \Mathcali{L}{12}-consistente, \GAMMA também é \Mathcali{L}{1}-consistente e \Mathcali{L}{2}-consistente. Ademais, como
            \(\Gamma \subseteq \Theta \subseteq \funcao{DC}_{\pi}(\Theta)\), existe um \PImodelo, baseado em um \PIframe para \Mathcali{L}{\pi}, onde \GAMMA é verdadeiro em
            algum mundo (devido à premissa de completude de \Mathcali{L}{1} e \Mathcali{L}{2}). Este \PImodelo será chamado de \textit{modelo inicial}, e será apresentado
            formalmente à frente.

            Inicialmente, devemos definir formalmente os conceitos de \PI-teorias, modelos etiquetados e elementos de modelos.
            \begin{definicao}[\PI-Teorias, \PIMODELOS Etiquetados e \PI-Elementos de Modelos]
                \label{def:Definicao1}
                \phantom{a}
                \begin{enumerate}[label=\textnormal{\ref{def:Definicao1}.\arabic*}]
                    \item Sendo \DDELTA um conjunto de fórmulas, a \textit{\PI-teoria de \DDELTA} é definida como, onde \(\mathcal{L}_{12}\) é uma lógica bimodal:\label{caso:Definicao1-1}
                    \[
                        \funcao{T}_{\pi}(\Delta) = \{\Box^{n}_{\pi} \delta \ | \ \delta \in \funcao{B}(\funcao{S}_{\pi}(\Delta)) \cap \mathcal{L}_{12} \text{ e }
                            \funcao{d}_\pi(\Box^{n}_{\pi} \delta) \leq \funcao{d}_\pi(\Delta) \}
                    \]

                    \item Um \textit{\PImodelo etiquetado} é uma tripla \(\langle \mathcal{M}, \mundobase, \Sigma(\mathcal{M}) \rangle\), onde \(\mathcal{M} = \langle \mathcal{W},
                    \mathcal{R}_{\pi}, \mathcal{V}_{\pi} \rangle\) e:
                    \begin{enumerate}[label=(\roman*)] \label{caso:Definicao1-2}
                        \item \Mathcal{M} é um \PImodelo para \(\mathcal{L}_{\pi}\);
                        \item \(\mundobase \in \mathcal{W}\);
                        \item \(\Sigma(\mathcal{M})\) é um conjunto de elementos (fórmulas interpretadas como átomos), fechado para a operação de sub-elementos sobre
                            elementos; \label{caso:Definicao1-2-3}
                        \item \(\funcao{T}_{\pi}(\Sigma(\mathcal{M}))\) é verdadeiro em \Mundobase em \Mathcal{M}. \label{caso:Definicao1-2-4}
                    \end{enumerate}

                    \item Dado um \PImodelo etiquetado \(\mathcal{E} = \langle \mathcal{M}, \mundobase, \Sigma(\mathcal{M}) \rangle\) onde
                    \(\mathcal{M} = \langle \mathcal{W}, \mathcal{R}_{\pi}, \mathcal{V}_{\pi} \rangle\), para cada mundo \(w \in \mathcal{W}\) \textit{o conjunto de elementos
                    \(\Sigma_{\mathcal{M}}(w)\) de w} é definido indutivamente como:
                        \begin{enumerate}[label=(\roman*)] \label{caso:Definicao1-3}
                            \item \(\Sigma_{\mathcal{M}}(\mundobase) = \Sigma(\mathcal{M})\); \label{caso:Definicao1-3-1}
                            \item Sendo \(w_i, w_j \in \mathcal{W}\), se \(w_i \mathcal{R}_{\pi} w_j\) e \(\Box_\pi \phi \in \Sigma_{\mathcal{M}}(w_i)\),
                                    então \(\funcao{SC}(\phi) \in \Sigma_{\mathcal{M}}(w_j)\). \label{caso:Definicao1-3-2} \qed
                        \end{enumerate}
                \end{enumerate}
            \end{definicao}

            O mundo \Mundobase é chamado de mundo base do modelo \Mathcal{M}, \(\Sigma(\mathcal{M})\) é chamado de conjunto de elementos do modelo \Mathcal{M} e
            \(\Sigma_{\mathcal{M}}(w)\) é o conjunto de elementos de \textit{w} em \Mathcal{M}.
            Sendo \(\mathcal{E} = \langle \mathcal{M}, \mundobase, \Sigma(\mathcal{M}) \rangle\) um modelo etiquetado, escreveremos \(w \in \mathcal{E}\)
            para indicar que \(w \in \mathcal{W}\), sendo \(\mathcal{M} = \langle \mathcal{W}, \mathcal{R}, \mathcal{V} \rangle\).
            % No restante da prova, usaremos o termo ``(\(\pi\)-) modelo'' para referir-se a modelos etiquetados, a não ser quando explicitado ao contrário.

            Com isso, podemos definir formalmente os conceitos de modelo e mundo iniciais, a partir de algum conjunto de fórmulas \GAMMA que deve ser satisfeito.

            \begin{definicao}[Modelo Inicial e Mundo Inicial]
                É chamado de \textit{modelo inicial} o \PImodelo etiquetado:
                \[
                    \modeloinicial = \langle \mathcal{I}, \mundoinicial, \funcao{SC}(\Gamma) \rangle
                \]
                onde o mundo base \Mundoinicial deste modelo é chamado de \textit{mundo inicial}. O modelo \(\mathcal{I}\) é o modelo que satisfaz
                \GAMMA e o mundo \Mundoinicial é um mundo onde \GAMMA é verdadeiro.
            \end{definicao}

            Podemos então provar o seguinte resultado auxiliar:

            \begin{lema}
                \label{teo:Lema1}
                Sendo \(\mathcal{E} = \langle \mathcal{M}, \mundobase, \Sigma(\mathcal{M}) \rangle\) um modelo etiquetado e \(w_i \in \mathcal{E}\), então:
                \begin{enumerate}[label=\textnormal{\ref{teo:Lema1}.\arabic*}]
                    \item \textnormal{Se \(\Sigma_{\mathcal{M}}(w_i) \neq \emptyset\) então \(\funcao{d}_{\pi}(\Sigma_{\mathcal{M}}(w_i)) = (\funcao{d}_{\pi}(\Sigma(\mathcal{M})) - \funcao{dist}(\mundobase, w_i))\);}\label{caso:Lema1-1}
                    \item \textnormal{Sendo \(w_i \neq \mundobase\), \(\Sigma_{\mathcal{M}}(w_i) = \emptyset\) sse \(\funcao{dist}(\mundobase, w_i) > \funcao{d}_{\pi}(\Sigma(\mathcal{M}))\).}\label{caso:Lema1-2} \qed
                \end{enumerate}
            \end{lema}

            \begin{proof}[Prova do Lema~\ref{teo:Lema1}]
                Temos dois casos, um para~\ref{caso:Lema1-1} outro para~\ref{caso:Lema1-2}:
                \begin{description}
                    \item[Caso~\ref{caso:Lema1-1}] Prova-se por indução em \(\funcao{dist}(\mundobase, w_j)\):
                        \begin{description}
                            \item[Base:] Pela definição de \(\funcao{dist}\), temos que \(\mundobase = w_j\), logo, \(\funcao{d}_{\pi}(\Sigma_{\mathcal{M}}(\mundobase)) = \funcao{d}_{\pi}(\Sigma(\mathcal{M}))\),
                            pela definição de \(\Sigma_{\mathcal{M}}\), temos \(\funcao{d}_{\pi}(\Sigma(\mathcal{M})) = \funcao{d}_{\pi}(\Sigma(\mathcal{M}))\).

                            \item[Hipotese:] Assumindo \(\forall w_k, \funcao{dist}(\mundobase, w_k) = k\),
                                temos que \(\funcao{d}_{\pi}(\Sigma_{\mathcal{M}}(w_k)) = \funcao{d}_{\pi}(\Sigma(\mathcal{M})) - \funcao{dist}(\mundobase, w_k)\).

                            \item[Passo:] Pelas definições de \(\Sigma_{\mathcal{M}}\) e \(\funcao{d}_{\pi}\), temos que \(\forall w_a, w_b,\) onde \(w_a \mathcal{R} w_b,
                                \funcao{d}_{\pi}(\Sigma_{\mathcal{M}}(w_a)) = \funcao{d}_{\pi}(\Sigma_{\mathcal{M}}(w_b)) + 1\).
                                Assumindo que \(\funcao{dist}(\mundobase, w_j) = k+1\), sabemos que existe algum \(w_x\) onde \(w_x \mathcal{R} w_j\) e \(\funcao{dist}(\mundobase, w_x) = k\).
                                Logo, pela hipótese podemos concluir que:
                                \[
                                    \funcao{d}_{\pi}(\Sigma(w_x)) = \funcao{d}_{\pi}(\Sigma(\mathcal{M})) - \funcao{dist}(\mundobase, w_x)
                                \]
                                Mais ainda, temos:
                                \begin{itemize}
                                    \item Pela definição de \(\Sigma_{\mathcal{M}}\): \(\Sigma(\mathcal{M}) = \Sigma_{\mathcal{M}}(\mundobase)\)
                                    \item Pelas conclusões anteriores: \(\funcao{d}_{\pi}(\Sigma_{\mathcal{M}}(w_x)) = \funcao{d}_{\pi}(\Sigma_{\mathcal{M}}(w_j)) + 1\)
                                \end{itemize}
                                Portanto temos:
                                \[
                                    \funcao{d}_{\pi}(\Sigma(w_j))+1 = \funcao{d}_{\pi}(\Sigma_{\mathcal{M}}(\mundobase)) - \funcao{dist}(\mundobase, w_x)
                                \]
                                Ou seja:
                                \[
                                    \funcao{d}_{\pi}(\Sigma(w_j)) = \funcao{d}_{\pi}(\Sigma_{\mathcal{M}}(\mundobase)) - (\funcao{dist}(\mundobase, w_x) + 1)
                                \]
                                Logo:
                                \[
                                    \funcao{d}_{\pi}(\Sigma(w_j)) = \funcao{d}_{\pi}(\Sigma_{\mathcal{M}}(\mundobase)) - \funcao{dist}(\mundobase, w_j)
                                \]
                        \end{description}

                    \item[Caso~\ref{caso:Lema1-2}] Consequência imediata do caso anterior e da definição de \(\Sigma_{\mathcal{M}}\). \qedhere
                \end{description}
            \end{proof}
            Este lema demonstra que, dado \(\funcao{d}_{\pi}(\Sigma(\mathcal{M}))\) finito, \(\funcao{d}_{\pi}(\Sigma_{\mathcal{M}}(w_j))\) irá diminuir com o aumento de \(\funcao{dist}(\mundobase, w_j)\) até
            \(\Sigma_{\mathcal{M}}(w_j)\) se tornar vazio, ou seja, o conjunto de elementos de um dado mundo em um modelo depende da distância do mundo até o mundo base do modelo.

            Com isso, podemos formalmente enunciar o conceito de ancoramento de modelos:

            \begin{definicao}[Ancoramento de Modelos]
                \label{def:Definicao2}
                Sendo \(\mathcal{E}_{0} = \langle \mathcal{M}, w_{\mathcal{M}}, \Sigma(\mathcal{M}) \rangle\) e \(\mathcal{E}_{1} = \langle \mathcal{N}, w_{\mathcal{N}}, \Sigma(\mathcal{N}) \rangle\)
                modelos etiquetados, onde \Mathcali{E}{0} tem tipo \PI, é dito que \textit{\Mathcali{E}{1} está ancorado em \Mathcali{E}{0}} se, e somente se:

                \begin{enumerate}[label=\textnormal{\ref{def:Definicao2}.\arabic*}]
                    \item \Mathcali{E}{1} tem tipo \OPI; \label{caso:Definicao2-1}

                    \item Sendo \(\mathcal{M} = \langle \mathcal{W}_{m}, \mathcal{R}_{m}, \mathcal{V}_{m} \rangle\) e
                    \(\mathcal{N} = \langle \mathcal{W}_{n}, \mathcal{R}_{n}, \mathcal{V}_{n} \rangle\) então,
                    \(\mathcal{W}_{m} \cap \mathcal{W}_{n} = \{w_{\mathcal{N}}\}\) e \(\mathcal{E}_{0} = \modeloinicial\) ou \(w_{\mathcal{N}}\)
                    não é o mundo base de \Mathcali{E}{0}; \label{caso:Definicao2-2}

                    \item \(\Sigma(\mathcal{N}) = \funcao{S}_{\pi}(\Sigma_{\mathcal{M}}(w_{\mathcal{N}}))\); \label{caso:Definicao2-3}

                    \item Para todo \(\phi \in \Sigma(\mathcal{N}): \mathcal{M}, w_{\mathcal{N}} \Vdash \phi\) sse \(\mathcal{N}, w_{\mathcal{N}} \Vdash \phi\) \label{caso:Definicao2-4} \qed
                \end{enumerate}
            \end{definicao}

            É fácil de observar que a definição de ancoramento é irreflexiva e antissimétrica, mais ainda, se \Mathcali{E}{1} está ancorado em \Mathcali{E}{0} e \textit{w} é o mundo base de
            \Mathcali{E}{1}, é dito que \textit{\Mathcali{E}{1} está ancorando em \Mathcali{E}{0} no mundo w}. Para cada \PImodelo \Mathcali{E}{0} e mundo \(w \in \mathcal{E}_{0}\),
            o conjunto \(\funcao{S}_{\pi}(\Sigma_{\mathcal{M}}(w))\) é chamado de conjunto de concordância de \textit{w} em \Mathcali{E}{0} pois, para cada \OPImodelo \Mathcali{E}{1} ancorado
            em \textit{w}, \Mathcali{E}{0} e \Mathcali{E}{1} devem concordar com a valoração deste conjunto em \textit{w}.

            Intuitivamente, podemos interpretar a operação de ancoramento de modelos como uma forma de ``alternar o contexto'' durante a valoração de uma fórmula que contenha múltiplas
            modalidades, pois permite que fórmulas com modalidades de um tipo sejam valoradas em um modelos de outro tipo.

            \begin{definicao}[Diagramas]
                \label{def:Diagramas}
                Para cada conjunto \Mathcali{L}{\pi}-consistente \DDELTA, \PImodelo \Mathcal{M} e mundo \textit{w} em \Mathcal{M}, o \textit{diagrama de \DDELTA em \Mathcal{M}
                no mundo w}, denotado por \(\funcao{DG}(\Delta)\) é dado por
                \[
                    \funcao{DG}(\Delta) = \{\delta \ | \ \delta \in \Delta \text{ e } \mathcal{M}, w \Vdash \delta\} \cup \{\neg \delta \ | \ \delta \in \Delta \text{ e } \mathcal{M}, w \nVdash \delta\}
                \]
                O \textit{diagrama de concordância \(\funcao{D}_{\mathcal{M}}(w)\) do mundo w em \Mathcal{M}} é definido como
                \(\funcao{D}_{\mathcal{M}}(w) = \funcao{DG}(\funcao{S}_{\pi}(\Sigma_{\mathcal{M}}(w)))\). \qed
            \end{definicao}

            É fácil observar que a condição~\ref{caso:Definicao2-4} é equivalente à impor a restrição que \(\funcao{D}_{\mathcal{M}}(w_{\mathcal{N}})\) seja verdadeiro
            no mundo \(w_{\mathcal{N}}\) no modelo \Mathcal{N}. Uma importante propriedade de \PImodelos etiquetados é a seguinte:

            \begin{lema}
                \label{teo:Lema2}
                Para todo \PImodelo etiquetado \Mathcal{E} e \(w \in \mathcal{E}\), \(\funcao{D}_{\mathcal{M}}(w)\) é \Mathcali{L}{12}-consistente. \qed
            \end{lema}

            \begin{proof}[Prova do Lema~\ref{teo:Lema2}]
                Inicialmente, devemos assumir que \(\funcao{D}_{\mathcal{M}}(w)\) é \Mathcali{L}{12}-inconsistente,
                para algum \(w \in \mathcal{E}\). Então, existe um conjunto finito e não vazio \(\Delta \subseteq \funcao{D}_{\mathcal{M}}(w)\) onde
                \(\neg \bigwedge \Delta \in \mathcal{L}_{12}\)\footnote{Onde \(\bigwedge \Delta = (\delta_1 \land \dots \land \delta_n)\).}, ou seja,
                existe algum \(\delta_i\) tal que \(\delta_i \in \Delta \text{ e } \neg\delta_i \in \mathcal{L}_{12}\).

                Assumindo que \(\funcao{dist}(w_{\mathcal{M}}, w) = k\). Sabemos que, pela definição de \(\funcao{D}_{\mathcal{M}}\), todo \(\delta \in \Delta\)
                é da forma \PHI ou \(\neg \phi\), sendo \(\phi \in \funcao{S}_{\pi}(\Sigma_{\mathcal{M}}(w))\).
                Portanto \(\phi \in \funcao{S}_{\pi}(\Sigma(\mathcal{M}))\) e \(\Delta \subseteq \funcao{B}(\funcao{S}_{\pi}(\Sigma(\mathcal{M})))\),
                mais ainda \(\funcao{d}_{\pi}(\phi) \leq \funcao{d}_{\pi}(\Sigma(\mathcal{M})) - k\) e, portanto, \(\funcao{d}_{\pi}(\Delta) \leq \funcao{d}_{\pi}(\Sigma(\mathcal{M})) - k\)
                (isso se dá pelo Lema~\ref{caso:Lema1-1}, no caso \(\Sigma_{\mathcal{M}} \neq \emptyset \text{, pois } \Delta \neq \emptyset\)).

                Temos então que \(\Box^{k}_{\pi} \neg \bigwedge \Delta \in \funcao{T}_{\pi}(\Sigma(\mathcal{M}))\) (pela Definição~\ref{caso:Definicao1-1}),
                onde \(\Box^{k}_{\pi} \neg \bigwedge \Delta\) é verdadeiro no mundo \Mundobase em \Mathcal{E}
                (pela Definição~\SubCaso{caso:Definicao1-2}{caso:Definicao1-2-4}). Isso implica que \(\neg \bigwedge \Delta\) é verdadeiro no mundo \textit{w}
                em \Mathcal{E}, contradizendo o fato que \(\bigwedge \Delta\) é também verdadeiro em \Mathcal{E} (pela Definição~\ref{def:Diagramas}).

                Portanto, \(\funcao{D}_{\mathcal{M}}(w)\) não pode ser \Mathcali{L}{12}-inconsistente.
            \end{proof}

            \begin{definicao}[Ancoramento Indireto]
                \label{def:AncoramenteoInidireto}
                Sendo \(\mathbf{E}\) um conjunto de \PImodelos e \OPImodelos etiquetados e sendo \(\mathcal{E}_{0}, \mathcal{E}_{1} \in \mathbf{E}\). A relação de ancoramento indireto,
                dita ``\textit{\Mathcali{E}{1} está indiretamente ancorado em \Mathcali{E}{0}}'', é definida indutivamente como:
                \begin{enumerate}[label=\textnormal{\ref{def:AncoramenteoInidireto}.\arabic*}]
                    \item Se \Mathcali{E}{1} está ancorado em \Mathcali{E}{0}, então \Mathcali{E}{1} está indiretamente ancorado em \Mathcali{E}{0};

                    \item Se, para algum \(\mathcal{E}_{2} \in \mathbf{E}\), onde \Mathcali{E}{2} está ancorado em \Mathcali{E}{0} e \Mathcali{E}{1} está indiretamente ancorado
                        em \Mathcali{E}{2}, então \Mathcali{E}{1} está indiretamente ancorado em \Mathcali{E}{0}. \qed
                \end{enumerate}
            \end{definicao}

            Com essa definição, podemos apresentar um dos conceitos mais importantes para a prova:

            \begin{definicao}[Brotos]
                \label{def:Definicao3}
                Um \textit{broto de \Modeloinicial} é qualquer conjunto \textbf{B} de \PImodelos e \OPImodelos etiquetados tal que:
                \begin{enumerate}[label=\textnormal{\ref{def:Definicao3}.\arabic*}]
                    \item \(\modeloinicial \in \mathbf{B}\) e \Modeloinicial não está ancorado em nada em \textbf{B}; \label{caso:Definicao3-1}

                    \item Para todo \PImodelo etiquetado \(\mathcal{E} \in \mathbf{B}, \text{ se } \mathcal{E} \neq \modeloinicial\) então \(\mathcal{E}\)
                        está indiretamente ancorado em \Modeloinicial; \label{caso:Definicao3-2}

                    \item Para cada \(\mathcal{E}_{0}, \mathcal{E}_{1} \in \mathbf{B}\), onde \(\mathcal{E}_{0} \neq \mathcal{E}_{1}\), \Mathcali{E}{0} e \Mathcali{E}{1} só compartilham
                        um mundo se um está ancorado no outro. \label{caso:Definicao3-3} \qed
                \end{enumerate}
            \end{definicao}

            Um broto pode ser entendido como um conjunto de modelos de tipos alternados, ancorados dois a dois, que inicia em \Modeloinicial e cujo intuito é
            permitir a valoração de fórmulas com múltiplas modalidades distintas. Podemos então demonstrar algumas propriedades sobre brotos.

            \begin{lema}
                \label{teo:Lema3}
                Sendo \textnormal{\textbf{B}} um broto de \Modeloinicial, então:
                \begin{enumerate}[label=\textnormal{\ref{teo:Lema3}.\arabic*}]
                    \item \textnormal{Para todo \(\mathcal{E}_{0} \in \mathbf{B}\) e \(w \in \mathcal{E}_{0}\), no máximo um \(\mathcal{E}_{1} \in \mathbf{B}\) está
                        ancorado em \Mathcal{E} no mundo \textit{w} \label{caso:Lema3-1};}

                    \item \textnormal{Para todo \(\mathcal{E}_{0} \in \mathbf{B}, \text{ se } \mathcal{E}_{0} \neq \modeloinicial\) então, \Mathcali{E}{0} está ancorado em exatamente um
                    \(\mathcal{E}_{1} \in \mathbf{B}\); \label{caso:Lema3-2}}

                    \item \textnormal{Todo \PImodelo etiquetado em \textbf{B} tem conjunto de mundos
                        disjuntos\footnote{Ou seja, apenas modelos etiquetados de tipos diferentes compartilham mundos.};}\label{caso:Lema3-3}

                    \item \textnormal{É chamada de \textit{cadeia de ancoramentos em \Mathcal{E} de comprimento n} a sequência finita \(\langle \mathcal{E}_{i} \ | \ i \leq n, n \geq 1 \rangle\)
                    de modelos etiquetados em \textbf{B} iniciando com \Modeloinicial e terminando com \Mathcal{E} onde, para cada \(1 < i \leq n\), \Mathcali{E}{i} está
                    ancorado em \Mathcali{E}{i-1}. Para todo \(\mathcal{E} \in \mathbf{B}\), existe exatamente uma cadeia de ancoramentos em \Mathcal{E} onde os elementos são
                    disjuntos dois a dois;}\label{caso:Lema3-4}

                    \item \textnormal{Se \(\mathcal{E}_{1} \in \mathbf{B}\) está ancorado em \(\mathcal{E}_{0} \in \mathbf{B}\) em \textit{w} e \(w \neq \mundoinicial\),
                    então \textit{w} não é o mundo base de \Mathcali{E}{0};}\label{caso:Lema3-5}

                    \item \textnormal{Para todo \PImodelo etiquetado \(\mathcal{E}_{0} = \langle \mathcal{M}, \mundobase, \Sigma(\mathcal{M}) \rangle \in \mathbf{B}\) e \(w \in \mathcal{E}_{0}\):}\label{caso:Lema3-6}
                    \begin{enumerate}[label=\textnormal{(\roman*)}]
                        \item \textnormal{Se \Mathcali{E}{0} está ancorado em \(\mathcal{E}_{1} = \langle \mathcal{N}, w_{\mathcal{N}}, \Sigma(\mathcal{N}) \rangle\)
                            então \(\Sigma(\mathcal{M}) \subseteq \Sigma(\mathcal{N})\);} \label{caso:Lema3-6-1}

                        \item \textnormal{\(\Sigma(\mathcal{M}) \subseteq \Sigma(\mathcal{\modeloinicial}) \subseteq \Theta\);} \label{caso:Lema3-6-2}

                        \item \textnormal{\(\funcao{D}_{\mathcal{M}}(w) \subseteq \Theta\);} \label{caso:Lema3-6-3}

                        \item \textnormal{\(\funcao{T}_{\pi}(\Sigma(\mathcal{M})) \subseteq \funcao{DC}_{\pi}(\Theta)\).} \label{caso:Lema3-6-4} \qed
                    \end{enumerate}
                \end{enumerate}
            \end{lema}

            \begin{proof}[Prova do Lema~\ref{teo:Lema3}]
                \phantom{a}
                \begin{description}
                    \item[Caso~\ref{caso:Lema3-1}] Pela Definição~\ref{caso:Definicao2-1}, sabemos que só podemos ancorar um modelo etiquetado de tipo \PI/\OPI em
                    um modelo etiquetado de tipo \OPI/\PI, pela Definição~\ref{caso:Definicao2-2} sabemos que estes dois modelos etiquetados compartilham apenas um mundo e,
                    pela Definição~\ref{caso:Definicao3-3} sabemos que dois modelos etiquetados distinto compartilham mundos apenas quando um está ancorado no outro.
                    Logo, concluímos que, para um mundo \(w \in \mathcal{E}_{0}\), há no máximo um modelo etiquetado \Mathcali{E}{1} em \textbf{B} ancorado neste mundo;

                    \item[Caso~\ref{caso:Lema3-2}] Pela Definição~\ref{caso:Definicao3-2} sabemos que todo \PImodelo etiquetado diferente de \Modeloinicial está indiretamente
                    ancorado em \(\modeloinicial\), esse fato, juntamente com as Definições~\ref{caso:Definicao3-3} e~\ref{caso:Definicao2-1}, nos diz que
                    todo modelo etiquetado em \textbf{B} diferente de \Modeloinicial está ancorado em exatamente um outro modelo etiquetado em \textbf{B};

                    \item[Caso~\ref{caso:Lema3-3}] Pelas Definições~\ref{caso:Definicao3-3} e~\ref{caso:Definicao2-1}, temos que todo par de modelos etiquetados de mesmo tipo
                    tem conjuntos de mundos disjuntos;

                    \item[Caso~\ref{caso:Lema3-4}] Pela definição indutiva de ancoramento indireto (Definição~\ref{def:AncoramenteoInidireto}) e pela
                    Definição~\ref{caso:Definicao3-2}, sabemos que essa cadeia de ancoramentos existe. Pelo Lema~\ref{caso:Lema3-2}, sabemos que todo modelo etiquetado diferente
                    de \Modeloinicial na cadeia tem exatamente um predecessor e, pela Definição~\ref{caso:Definicao3-1}, que nos diz que \Modeloinicial não está ancorado
                    em nenhum modelo etiquetado, sabemos que \Modeloinicial não tem predecessor na cadeia, portanto, pode haver apenas uma cadeia de ancoramentos em \Mathcal{E};

                    \item[Caso~\ref{caso:Lema3-5}] Decorre diretamente da Definição~\ref{caso:Definicao2-2};

                    \item[Caso~\textnormal{\SubCaso{caso:Lema3-6}{caso:Lema3-6-1}}] Decorre da Definição~\ref{caso:Definicao2-3}:
                    \(\Sigma(\mathcal{N}) = \funcao{S}_{\pi}(\Sigma_{\mathcal{M}}(w_{\mathcal{N}}))\) e do fato que \(\Sigma_{\mathcal{M}}(w) \subseteq \Sigma(\mathcal{M})\);

                    \item[Caso~\textnormal{\SubCaso{caso:Lema3-6}{caso:Lema3-6-2}}] Pelos Lemas~\ref{caso:Lema3-4} e~\SubCaso{caso:Lema3-6}{caso:Lema3-6-1} concluímos que
                    \(\Sigma(\mathcal{M}) \subseteq \Sigma(\modeloinicial)\), o que é suficiente para provar o caso, pois \(\Sigma(\modeloinicial) = \funcao{SC}(\Gamma) \subseteq \Theta\);

                    \item[Caso~\textnormal{\SubCaso{caso:Lema3-6}{caso:Lema3-6-3}}] Pelo Lema~\SubCaso{caso:Lema3-6}{caso:Lema3-6-2} e pelos fatos que
                    \(\funcao{D}_{\mathcal{M}}(w) \subseteq \funcao{B}(\Sigma_{\mathcal{M}}(w))\) e \(\Theta\) é fechado para \(\funcao{B}\), pela Definição~\ref{def:EspacoFormula};

                    \item[Caso~\textnormal{\SubCaso{caso:Lema3-6}{caso:Lema3-6-4}}] Pelo Lema~\SubCaso{caso:Lema3-6}{caso:Lema3-6-2} e o fato que \(\Theta\) é fechado
                    para \(\funcao{B}\), podemos concluir que \(\funcao{DC}_{\pi}(\funcao{B}(\Sigma({\mathcal{M}}))) \subseteq \funcao{DC}_{\pi}(\Theta)\) e, pela
                    Definição~\SubCaso{caso:Definicao1-2}{caso:Definicao1-2-4}, que nos diz que \(\funcao{T}_{\pi}(\Sigma(\mathcal{M}))\) é verdadeiro em \Mundobase,
                    temos que \(\funcao{T}_{\pi}(\Sigma(\mathcal{M})) \subseteq \funcao{DC}_{\pi}(\funcao{B}(\Sigma_{\mathcal{M}}))\). \qedhere

                \end{description}
            \end{proof}

            Tendo apresentado estas propriedades, é possível observar que um broto de \Modeloinicial pode ser visto como uma árvore, cuja raiz é \Modeloinicial e
            uma cadeia de ancoramentos para algum modelo etiquetado \Mathcal{E} é um ramo na árvore, partindo da raiz e indo até a folha \Mathcal{E}.

            \begin{definicao}[Função de Seleção de Modelos]
                \label{def:FuncaoSelecaoModelos}
                Uma \textit{função de seleção de modelo f} é uma função que associa, para cada \PI e para cada conjunto \(\mathcal{L}_{\pi}\)-consistente de fórmulas
                \(\Delta \subseteq \funcao{DC}_{\pi}(\Theta)\), um conjunto não vazio \(f_{\pi}(\Delta)\) de pares da forma \(\langle \mathcal{E}, w \rangle\) tal que \Mathcal{E} é
                um modelo etiquetado para \(\mathcal{L}_{\pi}\) que satisfaz \DDELTA em \textit{w}. Para uma dada função de seleção \textit{f}, definimos:
                \begin{itemize}
                    \item \(\mathbb{M}_{f} = \{ \mathcal{E} \ | \ \langle \mathcal{E}, w \rangle \in f_{\pi}(\Delta), \Delta \subseteq \mathsf{LM}_{12} \text{ e \DDELTA é }
                            \mathcal{L}_{\pi} \text{-consistente} \}\)

                    \item \(\mathbb{W}_{f} = \bigcup \{\mathcal{W} \ | \ \mathcal{W} \in \mathcal{M}, \mathcal{M} \in \mathcal{E}, \mathcal{E} \in \mathbb{M}_{f} \}\)

                    \item \(\aleph_{f} = max\{ \aleph_{0}, sup\left(\{|\mathcal{W}| \ | \ \mathcal{M} = \langle \mathcal{W}, \mathcal{R}, \mathcal{V} \rangle \in \mathbb{M}_{f}\}\right) \}\)
                \end{itemize}
                As funções de seleção de modelos analisadas respeitam as seguintes propriedades:
                \begin{enumerate}%[label=\textnormal{\ref{def:FuncaoSelecaoModelos}.\arabic*}]
                    \item \Mathbbi{W}{f} é um conjunto e \(|\mathbb{W}_{f}| > \aleph_{f}\);

                    \item Para cada conjunto \Mathcali{L}{12}-consistente \(\Delta \subseteq \funcao{DC}_{\pi}(\Theta)\), \(f_{\pi}(\Delta)\) é fechado para isomorfismo
                    em (ou seja, entre objetos de) \Mathbbi{W}{f}. \qed
                \end{enumerate}
            \end{definicao}

            Para uma dada função \textit{f}, iremos assumir que \(\langle \modeloinicial, \mundoinicial \rangle \in f_{\pi}(\Gamma \cup \funcao{T}_{\pi}(\Sigma(\modeloinicial)))\)
            e que os mundos de todos os brotos de \Modeloinicial estarão contidos no conjunto \Mathbbi{W}{f}.

            \begin{definicao}[Conjunto de Brotos]
                \label{def:ConjuntoBrotos}
                Um broto de \Modeloinicial é chamado de um \textit{f-broto} de \Modeloinicial se, para cada \PImodelo etiquetado no broto,
                \(\langle \mathcal{E}, \mundobase \rangle \in f_{\pi}(\funcao{DG}(\Sigma(\mathcal{M}) \cup \funcao{T}_{\pi}(\Sigma(\mathcal{M}))))\).

                Para cada \textit{f}, chamaremos de \textit{\(\mathfrak{B}_{f}\) o conjunto de todos os \textit{f}-brotos de \(\modeloinicial\)}. \qed
            \end{definicao}

            É importante ressaltar que \Mathfraki{B}{f} é um conjunto de conjuntos de modelos etiquetados, em específico, é um conjunto de conjuntos que satisfazem as
            Definição~\ref{def:Definicao3} e~\ref{def:ConjuntoBrotos}.

            \begin{lema}
                \label{teo:Lema4}
                Para cada \textit{f}, o conjunto \Mathfraki{B}{f} tem um elemento máximo. \qed
            \end{lema}

            \begin{proof}[Prova do Lema~\ref{teo:Lema4}]
                O Lema de Zorn~\cite{zorn1935remark} nos diz que, sendo \textit{X} um conjunto parcialmente ordenado (por alguma relação de ordenamento <), se todo
                subconjunto totalmente ordenado de \textit{X} tem um limite superior, então \textit{X} tem um elemento máximo (com relação a <).

                Considerando \Mathfraki{B}{f} parcialmente ordenado por \(\subseteq\) e sendo \textit{C} um subconjunto\footnote{Note que \textit{C} é um conjunto de conjuntos.}
                totalmente ordenado de \Mathfraki{B}{f}. O conjunto \(\bigcup C\) é um limite superior de \textit{C} pois \(S \subseteq C\), para todo broto \(S \in C\).
                Mais ainda, \(\bigcup C\) satisfaz todas as condições para ser um \textit{f}-broto (é fácil observar que a operação \(\cup\) preserva as
                propriedades~\ref{caso:Definicao3-1} --~\ref{caso:Definicao3-3} e~\ref{def:ConjuntoBrotos}).

                Portanto, \Mathfraki{B}{f} tem um elemento máximo.
            \end{proof}

            É importante esclarecer o que significa \Mathfraki{B}{f} ter um elemento máximo; como \Mathfraki{B}{f} é um conjunto de brotos e cada broto pode ser entendido
            como uma árvore cuja raiz é \Modeloinicial, o elemento máximo de \Mathfraki{B}{f} é a maior árvore no conjunto, ou seja,
            a maior sequência de modelos ancorados dois a dois partindo do modelo inicial. A importância de \Mathfraki{B}{f} ter um elemento máximo ficará clara a frente.

            O elemento máximo de \Mathfraki{B}{f} respeita a seguinte importante propriedade:

            \begin{lema}
                \label{teo:Lema5}
                Sendo \MathcalI{S}{+} um elemento máximo de \Mathfraki{B}{f}, então, para todo \(\mathcal{E}_{0} \neq \modeloinicial \in \mathcal{S}^{+}\) e mundo não base
                \(w \in \mathcal{E}_{0}\), então existe um \(\mathcal{E}_{1} \in \mathcal{S}^{+}\) ancorado em \Mathcali{E}{0} no mundo \textit{w}. \qed
            \end{lema}

            \begin{proof}[Prova do Lema~\ref{teo:Lema5}]
                Assumindo que \MathcalI{S}{+} é um elemento máximo de \Mathfraki{B}{f} e assumindo que \(\mathcal{E}_{0} \neq \modeloinicial \in \mathcal{S}^{+}\)
                é um \PImodelo etiquetado, \(w \in \mathcal{E}_{0}\) um mundo não base de \Mathcali{E}{0} e que não há qualquer
                \(\mathcal{E}_{1} \in \mathcal{S}^{+}\) ancorado em \Mathcali{E}{0} no mundo \textit{w}.

                Pelo Lema~\ref{teo:Lema2}, sabemos que o diagrama de concordância \(\funcao{D}_{\mathcal{M}}(w)\) de \textit{w} em \Mathcali{E}{0} é
                \Mathcali{L}{12}-consistente. Como \(\funcao{T}_{\mOPI}(\funcao{S}_{\pi}(\Sigma_{\mathcal{M}}(w)))\) é o um conjunto de teoremas de \Mathcali{L}{12},
                sabemos que \(\funcao{D}_{\mathcal{M}}(w) \cup \funcao{T}_{\mOPI}(\funcao{S}_{\pi}(\Sigma_{\mathcal{M}}(w)))\) é
                \Mathcali{L}{12}-consistente e, portanto, \Mathcali{L}{\mOPI}-consistente.

                Como \(\funcao{D}_{\mathcal{M}}(w) \cup \funcao{T}_{\mOPI}(\Sigma(\mathcal{M})) \subseteq \funcao{DC}_{\mOPI}(\Theta)\) (pelo Lema~\ref{caso:Lema3-6}) sabemos que
                \(\funcao{D}_{\mathcal{M}}(w) \cup \funcao{T}_{\mOPI}(\funcao{S}_{\pi}(\Sigma_{\mathcal{M}}(w)))\) é verdadeiro em algum mundo \(w_i\) de um \OPImodelo etiquetado \Mathcali{E}{1}
                tal que \(\langle \mathcal{E}_{1}, w_i \rangle \in f_{\pi}(\funcao{D}_{\mathcal{M}}(w) \cup \funcao{T}_{\mOPI}(\funcao{S}_{\pi}(\Sigma_{\mathcal{M}}(w))))\).
                Portanto, sendo \Mathcal{N} um modelo onde \(\mathcal{N} \in \mathcal{E}_{1}\) e \(\mathcal{W}_{\mathcal{N}} \in \mathcal{N}\), temos que
                \(\mathcal{W}_{\mathcal{N}} \subseteq \mathbb{W}_f\) e \(|\mathcal{W}_{\mathcal{N}}| \leq \aleph_{f}\).

                Como \textit{f} é fechado sob isomorfismo em \(\mathbb{W}_f\), podemos tomar \(w_i\) como sendo \textit{w}, declarar \textit{w} como o mundo
                base de \Mathcali{E}{1} e tomar \(\Sigma(\mathcal{N}) = \funcao{S}_{\pi}(\Sigma_{\mathcal{M}}(w))\)
                (portanto \(\funcao{D}_{\mathcal{M}(w)} = \funcao{DG}(\Sigma(\mathcal{N}))\)).

                Como \(\mathcal{S}^{+} \in \mathfrak{B}_{f}\), todo modelo etiquetado em \MathcalI{S}{+} tem, no máximo, cardinalidade\footnote{A cardinalidade de um modelo etiquetado
                se refere a cardinalidade do conjunto de mundos do modelo associado ao modelo etiquetado.} \(\aleph_{f}\).
                Pelo fato que \MathcalI{S}{+} tem uma estrutura de árvore, sabemos que \MathcalI{S}{+} contém, no máximo, \((\aleph_{f})^{n-1} = \aleph_{f}\)
                modelos etiquetados com cadeias de ancoramento de comprimento \textit{n}. Portanto, \MathcalI{S}{+} é um união contável de conjuntos de modelos etiquetados cuja
                cardinalidade é no máximo \(\aleph_{f}\), ou seja, \(|\mathcal{S}^{+}| \leq \aleph_{f}\).

                Portanto, sendo \(\mathcal{X} = \bigcup \{\mathcal{W} \ | \ \mathcal{W} \in \mathcal{M}, \mathcal{M} \in \mathcal{E}, \mathcal{E} \in \mathcal{S}^{+} \}\), temos
                \(|\mathcal{X}| \leq \aleph_{f} < |\mathbb{W}_{f}|\), adicionalmente \(|\mathcal{W}_{\mathcal{N}}| \leq \aleph_{f} < |\mathbb{W}_{f}|\).
                Como \(f_{\mOPI}(\funcao{DG}(\Sigma(\mathcal{N}) \cup \funcao{T}_{\mOPI}(\Sigma(\mathcal{N}))))\) é fechado sob isomorfismo em \(\mathbb{W}_{f}\),
                podemos assumir que todos os mundos não base de \Mathcali{E}{1} são elementos do conjunto \(\mathbb{W}_{f}-\mathcal{X}\).

                Assim, temos que \Mathcali{E}{1} satisfaz todas as condições para ser um \OPImodelo etiquetado ancorado em \Mathcali{E}{0} no mundo \textit{w}
                e que \Mathcali{E}{1} está indiretamente ancorado em \Modeloinicial (Definição~\ref{def:AncoramenteoInidireto}).
                Os mundos não base de \Mathcali{E}{1} são disjuntos do conjunto \(\mathcal{X}\) e \Mathcali{E}{1} não compartilha o mundo \textit{w}
                com qualquer outro modelo diferente de \Mathcali{E}{0} em \MathcalI{S}{+} (Definição~\ref{caso:Definicao3-1} e hipótese), portanto
                \(\mathcal{S}^{+} \cup \{\mathcal{E}_{1}\}\) é um \textit{f}-broto que contém \MathcalI{S}{+}, ou seja, \MathcalI{S}{+} não é máximo,
                o que é contraditório.
            \end{proof}

            Note que essa prova não afirma que \MathcalI{S}{+} será necessariamente infinito. Considere, por exemplo a cadeia de ancoramentos
            \(\modeloinicial, \dots, \mathcal{A}, \mathcal{B}\) e considere \(\mathcal{W}_{\mathcal{A}(\mathcal{B})}\) o conjunto de mundos do modelo
            associado ao modelo etiquetado \(\mathcal{A}(\mathcal{B})\), sendo \(\mathcal{W}_{\mathcal{A}} = \{w_{\mathcal{A}}, w_{\mathcal{B}}\}\),
            \(\mathcal{W}_{\mathcal{B}} = \{w_{\mathcal{B}}\}\) e \(\modeloinicial, \dots, \mathcal{A}, \mathcal{B} \in \mathcal{S}^{+}\). Portanto, existe um
            modelo etiquetado ancorado em um mundo não base de \Mathcal{A} onde não há nenhum outro modelo ancorado, pois seu único mundo é o mundo base.

            \begin{definicao}[Modelos Máximos]
                \label{def:Definicao4}
                Para cada elemento máximo \MathcalI{S}{+} em \Mathfraki{B}{f}, definimos o seu 12-modelo correspondente \(\mathcal{M}(\mathcal{S}^{+})\),
                sendo \(\mathcal{E} = \langle \mathcal{M}, \mundobase, \Sigma(\mathcal{M}) \rangle \in \mathcal{S}^{+}\) um modelo etiquetado qualquer, como:
                \begin{alignat*}{3}
                    &\mathcal{W}_{\mathcal{M}(\mathcal{S}^{+})}\ &=&\  \bigcup\{\mathcal{W} \ | \ \mathcal{W} \in \mathcal{M}\} \\
                    &\mathcal{R}_{\pi, \mathcal{M}(\mathcal{S}^{+})}\ &=&\  \bigcup\{\mathcal{R}_{\pi} \ | \ \mathcal{R}_{\pi} \in \mathcal{M}
                        \text{ e } \mathcal{M} \text{ é do tipo \PI}\} \\
                    &\mathcal{V}_{\mathcal{M}(\mathcal{S}^{+})}(p)\ &=&\  \bigcup\{\mathcal{V}_{\pi}(p) \ | \ \mathcal{V}_{\pi} \in \mathcal{M}\}
                \end{alignat*}
                Este modelo é chamado de \textit{modelo \Mathfraki{B}{f}-máximo} ou simplesmente \textit{modelo máximo}. \qed
            \end{definicao}

            Com essa definição, temos a seguinte propriedade importante:

            \begin{lema}
                \label{teo:Lema6}
                Sendo \(\mathcal{M}(\mathcal{S}^{+}) = \langle \mathcal{W}, \mathcal{R}_{1}, \mathcal{R}_2, \mathcal{V} \rangle\) um modelo máximo, então:
                \begin{enumerate}[label=\textnormal{\ref{teo:Lema6}.\arabic*}]
                    \item \textnormal{Todo \(w \in \mathcal{W}\) pertence exatamente a um modelo \(\mathcal{M}^{1}(w)\), este que pertence a um 1-modelo etiquetado
                    \(\mathcal{E}_{0} \in \elementomaximo\), e a um modelo \(\mathcal{M}^{2}(w)\), este que pertence a um 2-modelo etiquetado \(\mathcal{E}_{1} \in \elementomaximo\),
                    onde \(\mathcal{E}_{0}/\mathcal{E}_{1}\) está ancorado em \(\mathcal{E}_{1}/\mathcal{E}_{0}\) no mundo \textit{w};} \label{caso:Lema6-1}

                    \item \textnormal{O \PI-frame \(\langle \mathcal{W}, \mathcal{R}_{\pi} \rangle\) é a união disjunta de todos os \PI-frames
                    \(\langle \mathcal{W}, \mathcal{R}_{\pi} \rangle \in \mathcal{M}\) onde \(\mathcal{M} \in \mathcal{E} \text{ e } \mathcal{E} \in \elementomaximo\).} \label{caso:Lema6-2} \qed
                \end{enumerate}
            \end{lema}

            \begin{proof}[Prova do Lema~\ref{teo:Lema6}]
                \phantom{a}
                \begin{description}
                    \item[Caso~\ref{caso:Lema6-1}] Pela definição de \(\mathcal{M}(\elementomaximo)\), sabemos que \textit{w} está em algum \PImodelo \(\mathcal{M}_{\pi}(w)\).
                    Então temos dois casos para considerar:
                    \begin{enumerate}[label=(\roman*)]
                        \item \(w = \mundoinicial\) ou \(w \neq \mundoinicial\) e \textit{w} não é o mundo base de \(\mathcal{M}_{\pi}(w)\).\\
                            Nesse caso, temos que \textit{w} está em algum \OPImodelo \(\mathcal{M}_{\mOPI}(w)\) que está ancorado em \(\mathcal{M}_{\pi}(w)\)
                            no mundo \textit{w}, pelo Lema~\ref{teo:Lema5}.

                        \item \(w \neq \mundoinicial\) e \textit{w} é o mundo base de \(\mathcal{M}_{\pi}(w)\) (portanto \(\mathcal{M}_{\pi}(w) \neq \modeloinicial\)).\\
                            Nesse caso, \textit{w} está em algum \OPImodelo \(\mathcal{M}_{\mOPI}(w)\) tal que \(\mathcal{M}_{\pi}(w)\) está ancorado em \(\mathcal{M}_{\mOPI}(w)\)
                            no mundo \textit{w}, pelo Lema~\ref{caso:Lema3-2}.
                    \end{enumerate}
                    Em ambos os casos, temos que \textit{w} está em pelo menos um \(\mathcal{M}^{1}(w)\) e um \(\mathcal{M}^{2}(w)\), um ancorado no outro em \textit{w}.
                    A unicidade de \(\mathcal{M}^{1}(w)\) e \(\mathcal{M}^{2}(w)\) é consequência direta do Lema~\ref{caso:Lema3-3};

                    \item[Caso~\ref{caso:Lema6-2}] Pelo Lema~\ref{caso:Lema6-1}, sabemos que \(\mathcal{W} = \bigcup\{\mathcal{W} \ | \ \mathcal{M} \in \mathcal{E}, \mathcal{E} \in \mathcal{S}^{+}
                    \text{ e \Mathcal{M} é do tipo \PI}\}\) é valido para \(\pi \in \{1,2\}\), já o Lema~\ref{caso:Lema3-3} nos diz que os conjuntos de mundos de \PImodelos são mutuamente disjuntos.
                    \qedhere
                \end{description}
            \end{proof}

            Os principais resultados relacionados ao modelo \(\mathcal{M}(\elementomaximo)\) são o \textit{Lema de Concordância} e o \textit{Lema do Frame}.
            O Lema da Concordância nos diz que o modelo ``grande'' \(\mathcal{M}(\elementomaximo)\) concorda com todos os modelos ``pequenos'' \(\mathcal{M}_{\pi}(w)\) na valoração
            de todos os componentes booleanos de elementos em \(\Sigma_{\mathcal{M}_{\pi}(w)}(w)\) (lembre-se das Definições~\ref{def:FragmentosMonomodais} e~\ref{def:FuncoesElementos}).
            No que segue, escrevemos \(\mathcal{M}_{\pi}\) ao invés de \(\mathcal{M}_{\pi}(w)\) quando não for necessário indicar \textit{w}.

            \begin{lema}[Lema de Concondância]
                \label{teo:Lema7}
                Sendo \(\mathcal{M} = \langle \mathcal{W}, \mathcal{R}_{1}, \mathcal{R}_{2}, \mathcal{V} \rangle\) um modelo máximo. Então, para cada \(w \in \mathcal{W}\) e
                \(\phi \in \funcao{B}(\Sigma_{\mathcal{M}_{\pi}}(w))\), \PHI será verdadeiro em \textit{w} no \PImodelo
                \(\mathcal{M}_{\pi} = \langle \mathcal{W}_{\pi}, \mathcal{S}_{\pi}, \mathcal{V}_{\pi} \rangle\) se, e somente se, \PHI for verdadeiro em \textit{w} no 12-modelo \Mathcal{M}.
            \end{lema}

            \begin{proof}[Prova do Lema~\ref{teo:Lema7}]
                Provaremos por indução em \PHI:
                \begin{description}
                    \item[\textnormal{Caso Base \(\phi := p, p \in \mathbb{P}\):}] Dividimos essa prova em dois casos:
                    \begin{itemize}
                        \item[(\(\Rightarrow\))] Como \(\mathcal{M}_{\pi}, w \Vdash p\) temos que \(w \in \mathcal{V}_{\pi}(p)\), logo \(w \in \mathcal{V}(p)\)
                            pela Definição~\ref{def:Definicao4};

                        \item[(\(\Leftarrow\))] Como \(\mathcal{M}, w \Vdash p\) sabemos que \(w \in \mathcal{V}(p)\), logo temos que \(w \in \mathcal{V}_{\pi}^{1}(p)\)
                            para algum \(\mathcal{N} = \langle \mathcal{W}_{\pi}^{1}, \mathcal{S}_{\pi}^{1}, \mathcal{V}_{\pi}^{1} \rangle \in \elementomaximo\),
                            pela Definição~\ref{def:Definicao4}, onde \(\mathcal{N}\) deve ser o modelo \(\mathcal{M}^{1}(w)\) ou o modelo \(\mathcal{M}^{2}(w)\) no Lema~\ref{caso:Lema6-1}.
                            O Lema~\ref{caso:Lema6-1} também nos diz que um modelo deve estar ancorado no outro no mundo \textit{w} e, qualquer que seja o caso, temos que
                            \(p \in \Sigma_{\mathcal{M}^{1}}(w)\) e \(p \in \Sigma_{\mathcal{M}^{2}}(w)\), pela Definição~\ref{caso:Definicao2-3}.
                            A Definição~\ref{caso:Definicao2-4} nos diz que \(\mathcal{M}^{1}, w \Vdash p\) sse \(\mathcal{M}^{2}, w \Vdash p\), ou seja,
                            temos \(\mathcal{M}_{\pi}, w \Vdash p\).
                    \end{itemize}

                    \item[\textnormal{Caso \(\phi := \neg \psi\) e \(\phi := \psi \lor \gamma\):}] Decorrem diretamente da hipótese de indução.
                    % Temos a seguinte hipótese de indução:
                    %     \[
                    %         \forall w, \psi, \psi \in \funcao{B}(\Sigma_{\mathcal{M}_{\pi}}(w)) \land \mathcal{M}_{\pi}, w \Vdash \psi \Leftrightarrow \mathcal{M}, w \Vdash \psi
                    %     \]
                    %     Queremos provar \(\mathcal{M}_{\pi}, w \Vdash \neg \psi \Leftrightarrow \mathcal{M}, w \Vdash \neg \psi\), o que é logicamente equivalente
                    %     a hipótese de indução.

                    % \item[\textnormal{Caso \(\phi := \psi \lor \gamma\):}] Temos a seguinte hipótese de indução:
                    %     \begin{equation}
                    %         \forall w, \psi, \psi \in \funcao{B}(\Sigma_{\mathcal{M}_{\pi}}(w)) \land \mathcal{M}_{\pi}, w \Vdash \psi \Leftrightarrow \mathcal{M}, w \Vdash \psi
                    %     \end{equation}
                    %     Dividiremos a prova em dois casos:
                    %     \begin{itemize}
                    %         \item[(\(\Rightarrow\))] Como \(\mathcal{M}_{\pi}, w \Vdash \psi \lor \gamma\), temos dois casos disjuntos:

                    %         \begin{enumerate}[label=(\roman*)]
                    %             \item \(\mathcal{M}_{\pi}, w \Vdash \psi\), nesse caso, temos pela hipótese que \(\mathcal{M}, w \Vdash \psi\)

                    %             \item \(\mathcal{M}_{\pi}, w \Vdash \gamma\), nesse caso, temos pela hipótese que \(\mathcal{M}, w \Vdash \gamma\)
                    %         \end{enumerate}
                    %         Portanto, concluímos que \(\mathcal{M}, w \Vdash \psi \lor \gamma\).

                    %         \item[(\(\Leftarrow\))] Prova análoga ao caso anterior.
                    %     \end{itemize}

                    \item[\textnormal{Caso \(\phi := \Box_{\pi} \psi\):}] Temos a seguinte hipótese de indução:
                    \[
                        \forall w, \psi, \psi \in \funcao{B}(\Sigma_{\mathcal{M}_{\pi}}(w)) \land \mathcal{M}_{\pi}, w \Vdash \psi \Leftrightarrow \mathcal{M}, w \Vdash \psi
                    \]
                    Sabemos que \(\mathcal{M}_{\pi}, w \Vdash \Box_{\pi} \psi\) será válido sse \(\forall v, w \mathcal{S}_{\pi} v \to \mathcal{M}_{\pi}, v \Vdash \psi\) e,
                    pelo Lema~\ref{caso:Lema6-2}, sabemos que isso será válido sse \(\forall v, w \mathcal{R}_{\pi} v \to \mathcal{M}_{\pi}, v \Vdash \psi\)\footnote{Note que
                    \(\mathcal{R}_{\pi}\) se refere às relações \(\mathcal{R}_{1}\) e \(\mathcal{R}_{2}\) do modelo \(\mathcal{M}\).} for válido.
                    Pelo Lema~\ref{caso:Lema6-1}, sabemos que caso \(w \mathcal{R}_{\pi} v\) então \(\mathcal{M}_{\pi} = \mathcal{M}_{\pi}(w) = \mathcal{M}_{\pi}(v)\), ou seja,
                    \textit{w} e \textit{v} estão no mesmo \PImodelo.

                    Ademais, temos que \(\psi \in \funcao{B}(\Sigma_{\mathcal{M}_{\pi}}(v))\) pois \(\funcao{TC}(\psi) \subseteq \Sigma(\mathcal{M}_{\pi}(v))\)
                    (no caso onde \(w \neq v\), isso é devido a Definição~\SubCaso{caso:Definicao1-3}{caso:Definicao1-3-2}, já no caso onde \(w = v\), isso é devido as
                    Definições~\SubCaso{caso:Definicao1-2}{caso:Definicao1-2-3} e~\SubCaso{caso:Definicao1-3}{caso:Definicao1-3-1}).
                    Como \(\psi \in \funcao{B}(\Sigma_{\mathcal{M}_{\pi}}(v))\), podemos aplicar a hipótese de indução em \(\mathcal{M}_{\pi}, v \Vdash \psi\), portanto, temos:
                    \(\forall v, w \mathcal{R}_{\pi} v \to \mathcal{M}, v \Vdash \psi\), que será verdadeiro apenas se \(\mathcal{M}, w \Vdash \Box_{\pi} \psi\), portanto
                    provamos o caso.

                    \item[\textnormal{Caso \(\phi := \Box_{\mOPI} \psi\):}] Temos a seguinte hipótese de indução:
                    \[
                        \forall w, \psi, \psi \in \funcao{B}(\Sigma_{\mathcal{M}_{\pi}}(w)) \land \mathcal{M}_{\pi}, w \Vdash \psi \Leftrightarrow \mathcal{M}, w \Vdash \psi
                    \]
                    Vamos assumir que \(\Box_{\mOPI} \psi \in \Sigma_{\mathcal{M}_{\pi}}(w)\). Temos que ou \(\mathcal{M}_{\pi}\) está ancorado em
                    \(\mathcal{M}_{\mOPI}\) em \textit{w} ou vice versa. Em ambos os casos, o fato que \(\Box_{\mOPI} \psi \in \Sigma_{\mathcal{M}_{\mOPI}}(w)\) decorre da
                    Definição~\ref{caso:Definicao2-3} (no primeiro caso, pelo Lema~\SubCaso{caso:Lema3-6}{caso:Lema3-6-1}, no segundo caso pelo fato que \(\Box_{\mOPI} \psi\) é um \PI-elemento).
                    Pela Definição~\ref{caso:Definicao2-4} temos que \(\mathcal{M}_{\pi}, w \Vdash \Box_{\mOPI} \psi\) sse \(\mathcal{M}_{\mOPI}, w \Vdash \Box_{\mOPI} \psi\),
                    sendo que \(\mathcal{M}_{\mOPI}, w \Vdash \Box_{\mOPI} \psi\) será verdadeiro sse \(\mathcal{M}, w \Vdash \Box_{\mOPI} \psi\), o que pode ser provado de forma análoga ao
                    caso anterior, se substituirmos \(\Box_{\pi}\) por \(\Box_{\mOPI}\). \qedhere
                \end{description}
            \end{proof}

            \begin{definicao}[Condição de Frame]
                \label{def:CondicaoFrame}
                Uma função \textit{f satisfaz a condição de frame} se, para cada conjunto de fórmulas \Mathcali{L}{\pi}-consistente \(\Delta \subseteq \funcao{DC}_{\pi}(\Theta)\),
                os modelos em \(f_{\pi}(\Delta)\) são baseados em frames para \(\mathcal{L}_{\pi}\).\qed
            \end{definicao}

            Uma função \textit{f} que satisfaz a condição de frame existe quando \Mathcali{L}{1} e \Mathcali{L}{2} são completas com relação à \(\funcao{DC}_{\pi}(\Theta)\).
            A existência dessa função se dá pela completude de \Mathcali{L}{1} e \Mathcali{L}{2} - como \(\mathcal{L}_{\pi}\) é completa, temos que todo
            \(\Delta \subseteq \funcao{DC}_{\pi}(\Theta)\) é satisfazível em algum modelo baseado num frame para \(\mathcal{L}_{\pi}\), logo, pela definição de função de seleção de
            modelos, sabemos que todos os modelos em \(f_{\pi}(\Delta)\) são baseados em frames para \(\mathcal{L}_{\pi}\).

            \begin{lema}[Lema do Frame]
                \label{teo:Lema8}
                Sendo \(\mathcal{M} = \langle \mathcal{W}, \mathcal{R}_{1}, \mathcal{R}_{2}, \mathcal{V} \rangle\) um modelo \Mathfraki{B}{f}-máximo onde \textit{f} satisfaz
                a Definição~\ref{def:CondicaoFrame}. Então \(\langle \mathcal{W}, \mathcal{R}_{1}, \mathcal{R}_{2}\rangle\) é um 12-frame para \(\mathcal{L}_{12}\) \qed
            \end{lema}

            \begin{proof}[Prova do Lema~\ref{teo:Lema8}]
                Sabemos, pela definição de união disjunta e pela definição de frames, que conjuntos de frames são fechados para a operação de união
                disjunta\footnote{Por exemplo: \(\mathcal{F}_{1} = \langle \{w_{0}, w_{1}\}, \{\langle w_{0}, w_{1} \rangle, \langle w_{0}, w_{0} \rangle,
                \langle w_{1}, w_{1} \rangle\} \rangle\) e \(\mathcal{F}_{2} = \langle \{w_{2}, w_{3}\}, \{\langle w_{2}, w_{3} \rangle\} \rangle\), sua união disjunta será
                o frame \(\mathcal{F}_{12} = \langle \{w_{{0}_{1}}, w_{{1}_{1}}, w_{{2}_{2}}, w_{{3}_{2}}\}, \{\langle w_{{0}_{1}}, w_{{1}_{1}} \rangle, \langle w_{{0}_{1}}, w_{{0}_{1}} \rangle,
                \langle w_{{1}_{1}}, w_{{1}_{1}} \rangle, \langle w_{{2}_{2}}, w_{{3}_{2}} \rangle\} \rangle\).}.
                Com isso e o Lema~\ref{caso:Lema6-2}, temos que o par \(\langle \mathcal{W}, \mathcal{R}_{1} \rangle\) descreve um frame para a lógica \(\mathcal{L}_{1}\) e o par
                \(\langle \mathcal{W}, \mathcal{R}_{2} \rangle\) descreve um frame para a lógica \(\mathcal{L}_{2}\). Pela definição de 12-frames e pelo Teorema~\ref{teo:TransCorretude},
                concluímos a prova.
            \end{proof}

            Como \(\Gamma \subseteq \funcao{B}(\Sigma_{\modeloinicial}(\mundoinicial))\) é verdadeiro em \Mundoinicial no modelo \(\modeloinicial = \mathcal{M}^{1}(\mundoinicial)\), o
            Lema~\ref{teo:Lema7} nos diz que \GAMMA é verdadeiro em \Mundoinicial no 12-modelo \Mathcal{M}. Como \Mathcali{L}{1} e \Mathcali{L}{2} são completas, podemos assumir
            alguma função de seleção de modelos \textit{f} que satisfaz a Definição~\ref{def:CondicaoFrame}; portanto, o frame que define \(\mathcal{M}\) é um frame para
            \Mathcali{L}{12}, pelo Lema~\ref{teo:Lema8}.

            Portanto, com este método conseguimos construir um modelo que satisfaz todo subconjunto \(\mathcal{L}_{12}-\)consistente do espaço de fórmulas \(\Theta\).
            Assim, concluímos a prova do Teorema~\ref{teo:TransCompletude}.
        \end{proof}


\end{apendicesenv}
% ---			    % Elemento Opcional
%\include{PosTextuais/Anexos}				% Elemento Opcional
%\include{PosTextuais/IndiceRemissivo}		% Elemento Opcional

\end{document}

% -----------------------------------------------------------------
% Fim do Documento
% -----------------------------------------------------------------