\documentclass[
    12pt,					% tamanho da fonte
    openright,				% capítulos começam em pág ímpar (insere página vazia caso preciso)
    oneside,				% para impressão em recto e verso (twoside). Oposto a (oneside)
    a4paper,				% tamanho do papel.
    chapter=TITLE,			% títulos de capítulos convertidos em letras maiúsculas
    section=TITLE,			% títulos de seções convertidos em letras maiúsculas
    sumario=abnt-6027-2012, %
    english,				% idioma adicional para hifenização
    brazil,					% o último idioma é o principal do documento
    fleqn,					% equações alinhadas a esquerda (UDESC/CCT)+
    ]{abntex2}

% ----------------------------------------------------------
% Pacotes básicos
% ----------------------------------------------------------
% \usepackage{Pacotes/abntex2}
\usepackage{amsmath, amssymb, amsfonts, amsthm, mathtools}
\usepackage{mathptmx} 							% Usa a fonte Times New Roman	 (UDESC/CCT)
\usepackage[T1]{fontenc}						% Seleção de códigos de fonte.
\usepackage[utf8]{inputenc}						% Codificação do documento (conversão automática dos acentos)
\usepackage{lastpage}							% Usado pela Ficha catalográfica
\usepackage{indentfirst}						% Indenta o primeiro parágrafo de cada seção.
\usepackage[dvipsnames,table]{xcolor}			% Controle das cores
\usepackage{graphicx}							% Inclusão de gráficos
\usepackage{microtype} 							% para melhorias de justificação
\usepackage[brazilian,hyperpageref]{backref}	% Paginas com as citações na bibl
\usepackage[alf,abnt-emphasize=bf,abnt-full-initials=yes]{abntex2cite}					% Citações padrão ABNT
\usepackage{adjustbox}							% Pacote de ajuste de boxes
\usepackage{subcaption}							% Inclusão de Subfiguras e sublegendas
\usepackage{enumitem}							% Personalização de listas
\usepackage[section]{placeins}					% Manter as figuras delimitadas na respectiva seção com a opção [section]
\usepackage{multirow}							% Multi colunas nas tabelas
\usepackage{array,tabularx} 					% Pacotes de tabelas
\usepackage{booktabs}							% Pacote de tabela profissional
\usepackage{rotating}							% Rotacionar figuras e tabelas
\usepackage{xfrac}								% Fazer frações n/d em linha
\usepackage{bm}									% Negrito em modo matemático
\usepackage{xstring}							% Manipulação de strings
\usepackage{chngcntr}							% Pacte usado para deixar numeração de equações sequencial (UDESC/CCT)
\usepackage{xspace}								% Espaçamento após comandos
\usepackage{bussproofs}							% Provas em estilo de árvore
\usepackage{hyperref}							% Suporte para hiperlinks
\usepackage{listings, lstCoq}					% Highlighting de código Coq
\usepackage{tikz}								% Imagens e Diagramas
\usepackage{ulem}								% Texto strikethrough
\usepackage{calrsfs}

\definecolor{darkgreen}{rgb}{0,0.6,0}
\definecolor{darkorange}{rgb}{0.8,0.4,0.2}
\definecolor{darkred}{rgb}{0.8,0,0}

\usetikzlibrary{arrows.meta,positioning,matrix}

\counterwithout{equation}{chapter}
% fonte: https://latex.org/forum/viewtopic.php?t=15392

% Comando para deixar numeração das equações contínua (1), (2), (3)... ao invés de organizar por capítulos (1.1)(1.2)... (2.1)(2.2)
%\renewcommand{\theequation}{\arabic{equation}}

%\numberwithin{equation}{section}


% Cabeçalho somente com numeração de página 10pt
\makepagestyle{PagNumReduzida}
\makeevenhead{PagNumReduzida}{\ABNTEXfontereduzida\thepage}{}{}
\makeoddhead{PagNumReduzida}{}{}{\ABNTEXfontereduzida\thepage}
%fonte: https://github.com/abntex/abntex2/wiki/HowToCustomizarCabecalhoRodape
%fonte: Manual memoir seção 7.3 pg. 111 pdf http://linorg.usp.br/CTAN/macros/latex/contrib/memoir/memman.pdf

% Personalização das opções das listas
\setlist[itemize]{leftmargin=\parindent}

% Citação online --- MODIFICAR ---
\newcommand{\citeshort}[1]{\citeauthoronline{#1} (\citeyear{#1})}

\newcommand{\me}{Elaborado pelo autor.}

% Configuração do pgfplots
% \pgfplotsset{compat=newest} %compat=1.14
% \pgfplotsset{plot coordinates/math parser=false}
% \newlength\figureheight
% \newlength\figurewidth

% Libraries do TiKz
% \usetikzlibrary{quotes,angles,arrows}
% \usetikzlibrary{through,calc,math}
% \usetikzlibrary{graphs,backgrounds,fit}
% \usetikzlibrary{shapes,positioning,patterns,shadows}
% \usetikzlibrary{decorations.pathreplacing}
% \usetikzlibrary{shapes.geometric}
% \usetikzlibrary{arrows.meta}
% \usetikzlibrary{external}

%\tikzexternalize[]
%\tikzexternalenable
%\tikzexternalize
%\tikzexternaldisable
%\tikzset{external/force remake}
%\tikzexternalize[shell escape=-enable-write18]

% Personalização das legendas
\usepackage[format = plain, %hang
			justification = centering,
			labelsep = endash,
			singlelinecheck = false,
			skip = 6pt,
			listformat = simple]{caption}

% Personalizações de tipo de colunas de tabelas
\newcolumntype{L}[1]{>{\raggedright\let\newline\\\arraybackslash\hspace{0pt}}m{#1}}
\newcolumntype{C}[1]{>{\centering\let\newline\\\arraybackslash\hspace{0pt}}m{#1}}
\newcolumntype{R}[1]{>{\raggedleft\let\newline\\\arraybackslash\hspace{0pt}}m{#1}}

% Personalizações de cores da UDESC
\definecolor{CapaAmareloUDESC}{RGB}{243,186,83}		% Especialização
\definecolor{CapaVerdeUDESC}{RGB}{0,112,52}			% Mestrado
\definecolor{CapaVermelhoUDESC}{RGB}{171,35,21}		% Doutorado
\definecolor{CapaAzulUDESC}{RGB}{38,54,118} 		% Pós-Doutorado

% CONFIGURAÇÕES DE PACOTES
% Configurações do pacote backref
% Usado sem a opção hyperpageref de backref
\renewcommand{\backrefpagesname}{Citado na(s) página(s):~}
% Texto padrão antes do número das páginas
\renewcommand{\backref}{}
% Define os textos da citação
\renewcommand*{\backrefalt}[4]{
	\ifcase #1 %
	Nenhuma citação no texto.%
	\or
	Citado na página #2.%
	\else
	Citado #1 vezes nas páginas #2.%
	\fi}%

% alterando o aspecto da cor azul
%\definecolor{blue}{RGB}{41,5,195}

% informações do PDF
\makeatletter
\hypersetup{
	%pagebackref=true,
	pdftitle={\@title},
	pdfauthor={\@author},
	pdfsubject={\imprimirpreambulo},
	pdfcreator={LaTeX with abnTeX2},
	pdfkeywords={abnt}{latex}{abntex}{abntex2}{trabalho academico},
	colorlinks=true,       		% false: boxed links; true: colored links
	linkcolor=blue,          	% color of internal links
	citecolor=blue,        	% color of links to bibliography
	filecolor=blue,      		% color of file links
	urlcolor=blue,			% color of external links
	bookmarksdepth=4
}
\makeatother


% \makeatletter
% \newcommand{\includetikz}[1]{%
% 	\tikzsetnextfilename{#1}%
% 	\input{#1.tex}%
% }
% \makeatother

% ---
% Possibilita criação de Quadros e Lista de quadros.
% Ver https://github.com/abntex/abntex2/issues/176
%
\newcommand{\quadroname}{Quadro}
\newcommand{\listofquadrosname}{Lista de quadros}

\newfloat[chapter]{quadro}{loq}{\quadroname}
\newlistof{listofquadros}{loq}{\listofquadrosname}
\newlistentry{quadro}{loq}{0}

% configurações para atender às regras da ABNT
\setfloatadjustment{quadro}{\centering}
\counterwithout{quadro}{chapter}
\renewcommand{\cftquadroname}{\quadroname\space}
\renewcommand*{\cftquadroaftersnum}{\hfill--\hfill}

\setfloatlocations{quadro}{hbtp} % Ver https://github.com/abntex/abntex2/issues/176
% ---


% Espaçamento depois do título
\setlength{\afterchapskip}{0.7\baselineskip}
% O tamanho do parágrafo é dado por:
\setlength{\parindent}{1.25cm}
% Controle do espaçamento entre um parágrafo e outro:
\setlength{\parskip}{0.0cm}  % tente também \onelineskip
%\SingleSpacing % Espaçamento simples
\OnehalfSpacing % Espaçamento 1,5 (UDESC/CCT)
%\DoubleSpacing	% Espaçamento duplo

% ---
% Margens - NBR 14724/2011 - 5.1 Formato
% ---
\setlrmarginsandblock{3cm}{2cm}{*}
\setulmarginsandblock{3cm}{2cm}{*}
\checkandfixthelayout[fixed]
% ---


% To use externalize consider
%https://tex.stackexchange.com/questions/182783/tikzexternalize-not-compatible-with-miktex-2-9-abntex2-package
%Lauro Cesar digged into the problem until he came with a solution for me to test. And it Works!
%
%According to this link:
%
%The package calc changed the commands \setcounter and friends to be fragile.
% So you have to make them robust. The example below uses etoolbox with \robustify:
%
\usepackage{etoolbox}
\robustify\setcounter
\robustify\addtocounter
\robustify\setlength
\robustify\addtolength


%% How to silence memoir class warning against the use of caption package?
%% https://tex.stackexchange.com/questions/391993/how-to-silence-memoir-class-warning-against-the-use-of-caption-package
%\usepackage{silence}
%\WarningFilter*{memoir}{You are using the caption package with the memoir class}
%\WarningFilter*{Class memoir Warning}{You are using the caption package with the memoir class}

% --------------------------------------------------------
% INICIO DAS CUSTOMIZACOES PARA A UDESC
% --------------------------------------------------------

% --------------------------------------------------------
% Fontes padroes de part, chapter, section, subsection e subsubsection
% --------------------------------------------------------
% --- Chapter ---
\renewcommand{\ABNTEXchapterfont}{\fontseries{b}} %\bfseries
\renewcommand{\ABNTEXchapterfontsize}{\normalsize}
% --- Part ---
\renewcommand{\ABNTEXpartfont}{\ABNTEXchapterfont}
\renewcommand{\ABNTEXpartfontsize}{\LARGE}
% --- Section ---
\renewcommand{\ABNTEXsectionfont}{\normalfont}
\renewcommand{\ABNTEXsectionfontsize}{\normalsize}
% --- SubSection ---
\renewcommand{\ABNTEXsubsectionfont}{\fontseries{b}} %\bfseries
\renewcommand{\ABNTEXsubsectionfontsize}{\normalsize}
% --- SubSubSection ---
\renewcommand{\ABNTEXsubsubsectionfont}{\itshape}
\renewcommand{\ABNTEXsubsubsectionfontsize}{\normalsize}

\renewcommand{\ABNTEXsubsubsubsectionfont}{\normalfont}
\renewcommand{\ABNTEXsubsubsubsectionfontsize}{\normalsize}
% ---

% --------------------------------------------------------
% Fontes das entradas do sumario
% --------------------------------------------------------

\renewcommand{\cftpartfont}{\ABNTEXpartfont\selectfont}
\renewcommand{\cftpartpagefont}{\normalsize\selectfont}

\renewcommand{\cftchapterfont}{\ABNTEXchapterfont\selectfont}
\renewcommand{\cftchapterpagefont}{\normalsize\selectfont}

\renewcommand{\cftsectionfont}{\ABNTEXsectionfont\selectfont}
\renewcommand{\cftsectionpagefont}{\normalsize\selectfont}

\renewcommand{\cftsubsectionfont}{\ABNTEXsubsectionfont\selectfont}
\renewcommand{\cftsubsectionpagefont}{\normalsize\selectfont}

\renewcommand{\cftsubsubsectionfont}{\normalfont\itshape\selectfont}
\renewcommand{\cftsubsubsectionpagefont}{\normalsize\selectfont}

\renewcommand{\cftparagraphfont}{\normalfont\selectfont}
\renewcommand{\cftparagraphpagefont}{\normalsize\selectfont}

% --------------------------------------------------------
% Usando os pacotes hyperref, uppercase...
% Para deixar a section do toc uppercase precisa de:
% --------------------------------------------------------
\usepackage{textcase}

\makeatletter

\let\oldcontentsline\contentsline
\def\contentsline#1#2{%
	\expandafter\ifx\csname l@#1\endcsname\l@section
	\expandafter\@firstoftwo
	\else
	\expandafter\@secondoftwo
	\fi
	{%
		\oldcontentsline{#1}{\MakeTextUppercase{#2}}%
	}{%
		\oldcontentsline{#1}{#2}%
	}%
}
\makeatother

% --------------------------------------------------------
% Renomenando as entradas de APÊNDICES E ANEXOS
% --------------------------------------------------------

\renewcommand{\apendicesname}{AP\^ENDICES}
\renewcommand{\anexosname}{ANEXOS}


% Manipulação de Strings
%\RequirePackage{xstring}

% Comando para inverter sobrenome e nome
\newcommand{\invertname}[1]{%
	\StrBehind{#1}{{}}, \StrBefore{#1}{{}}%
}%


% --------------------------------------------------------
% Alterando os estilos de Caption e Fonte
% --------------------------------------------------------
\makeatletter
% Define o comando \fonte que respeita as configurações de caption do memoir ou do caption
\renewcommand{\fonte}[2][\fontename]{%
	\M@gettitle{#2}%
	\memlegendinfo{#2}%
	\par
	\begingroup
	\@parboxrestore
	\if@minipage
	\@setminipage
	\fi
	\ABNTEXfontereduzida
	\configureseparator
	\captiondelim{\ABNTEXcaptionfontedelim}
	\@makecaption{#1}{\ignorespaces #2}\par
	\endgroup}


\captionstyle[\raggedright]{\raggedright}

\makeatother

\setlength{\cftbeforechapterskip}{0pt plus 0pt}
\renewcommand*{\insertchapterspace}{}

\newlength{\mylen}	% New length to use with spacing
\setlength{\mylen}{1pt}

\setlength{\cftbeforechapterskip}{\mylen}
\setlength{\cftbeforesectionskip}{\mylen}
\setlength{\cftbeforesubsectionskip}{\mylen}
\setlength{\cftbeforesubsubsectionskip}{\mylen}
\setlength{\cftbeforesubsubsubsectionskip}{\mylen}


% ---
% Ajuste das listas de abreviaturas e siglas ; e símbolos [Personalizada para UDESC com espaçamento 1,5]
% ---

% ---
% Redefinição da Lista de abreviaturas e siglas [Personalizada para UDESC com espaçamento 1,5]
\renewenvironment{siglas}{%
	\pretextualchapter{\listadesiglasname}
	\begin{symbols}
		\setlength{\itemsep}{0pt}	% Ajuste para Espaçamento 1,5 (UDESC/CCT)
	}{%
	\end{symbols}
	\cleardoublepage
}
% ---

% ---
% Redefinição da Lista de símbolos [Personalizada para UDESC com espaçamento 1,5]
\renewenvironment{simbolos}{%
	\pretextualchapter{\listadesimbolosname}
	\begin{symbols}
		\setlength{\itemsep}{0pt}	% Ajuste para Espaçamento 1,5 (UDESC/CCT)
	}{%
	\end{symbols}
	\cleardoublepage
}
% ---

% ---
% FIM DAS CUSTOMIZAÇÕES PARA A  Universidade do Estado de Santa Catarina - UDESC/CCT
% ---	% Inclui pacotes básicos

\graphicspath{{Texto/Figuras/}}
\renewcommand{\orientadorname}{Orientadora:}

% O pacote bm torna a fonte do mathcal feia, a linha abaixo desfaz isso
% (Se tirar o bm caga tudo, então ele fica)
%\DeclareMathAlphabet{\mathcal}{OMS}{cmsy}{m}{n}
\DeclareMathAlphabet{\pazocal}{OMS}{zplm}{m}{n}
\DeclareMathAlphabet{\mathscr}{OMS}{zplm}{m}{n}
% Definindo fonte caligráfica e negrita
\DeclareMathAlphabet\mathbfcal{OMS}{cmsy}{b}{n}

% -----------------------------------------------------------------
% Informações de dados para CAPA e FOLHA DE ROSTO
% -----------------------------------------------------------------
\titulo{Implementação de uma biblioteca da Lógica de Inconsistência Formal LFI1 em Coq}%


\autor{Helena Vargas {}Tannuri}%
\orientador{Karina Girardi {}Roggia}%
\coorientador{Miguel Alfredo {}Nunes}%

\instituicao{Universidade do Estado de Santa Catarina, Centro de Ciências Tecnológicas, Bacharelado em Ciência da Computação}%

\tipotrabalho{Trabalho de Conclusão de Curso}

\preambulo{Trabalho de conclusão de curso submetido à Universidade do Estado de Santa Catarina
como parte dos requisitos para a obtenção do grau de Bacharel em Ciência da Computação}

\local{Joinville}%

\data{\the\year}%

\makeindex

% -----------------------------------------------------------------
% Início do documento
% -----------------------------------------------------------------
\begin{document}

\selectlanguage{brazil}

% Teoremas e Provas

% Estilo padrão, i.e., texto itálico
\newtheorem{teorema}   {Teorema}
\newtheorem{proposicao}{Proposição}
\newtheorem{lema}      {Lema}
\newtheorem{corolario} {Corolário}
\renewcommand{\proofname}{Prova}
\renewcommand\qedsymbol{$\blacksquare$}

% Estilo simples, i.e., texto normal
\theoremstyle{definition}
\newtheorem{definicao}{Definição}
\newtheorem{notacao}{Notação}
\newtheorem{exemplo}{Exemplo}

\theoremstyle{remark}
\newtheorem{observacao}{Observação}

\AtEndEnvironment{teorema}  {\qed}%
\AtEndEnvironment{proposicao}  {\qed}%
% \AtEndEnvironment{lema}     {\qed}%
% \AtEndEnvironment{definicao}{\qed}%
% \AtEndEnvironment{exemplo}  {\qed}%

%% Grego

\renewcommand\phi{\varphi}

% Grego Minúsculo

% Grego Maiúsculo
\newcommand{\GAMMA}{\(\Gamma\)\xspace}
\newcommand{\DDELTA}{\(\Delta\)\xspace}
\newcommand{\SIGMA}{\(\Sigma\)\xspace}
\newcommand{\THETA}{\(\Theta\)\xspace}
\newcommand{\LAMBDA}{\(\Lambda\)\xspace}
\newcommand{\LAMBDAlm}{\(\Lambda_{\mathsf{LM}}\)\xspace}
\newcommand{\Lambdalm}{\Lambda_{\mathsf{LM}}}

% Modalidades e símbolos matemáticos

\newcommand{\meio}{\frac{1}{2}}

\newcommand{\VVDASH}{\(\Vdash\)\xspace}
\newcommand{\VDDASH}{\(\vDash\)\xspace}
\newcommand{\VDASH}{\(\vdash\)\xspace}

\newcommand{\ODOT}{\(\odot\)\xspace}
\newcommand{\OPLUS}{\(\oplus\)\xspace}
\newcommand{\OTIMES}{\(\otimes\)\xspace}
\newcommand{\OMINUS}{\(\ominus\)\xspace}

\newcommand{\tofrom}{\leftrightarrow}

\newcommand{\eqdef}{\mathrel{\overset{\makebox[0pt]{\mbox{\normalfont\tiny\sffamily def}}}{=}}}

% Fontes
\newcommand{\Mathcal}[1]{\(\mathcal{#1}\)\xspace}
\newcommand{\Mathcali}[2]{\(\mathcal{#1}_{#2}\)\xspace}
\newcommand{\MathcalI}[2]{\(\mathcal{#1}^{#2}\)\xspace}
\newcommand{\Mathcalii}[3]{\(\mathcal{#1}{#2}_{#3}\)\xspace}

\newcommand{\Mathfrak}[1]{\(\mathfrak{#1}\)\xspace}
\newcommand{\Mathfraki}[2]{\(\mathfrak{#1}_{#2}\)\xspace}
\newcommand{\MathfrakI}[2]{\(\mathfrak{#1}^{#2}\)\xspace}

\newcommand{\Mathbb}[1]{\(\mathbb{#1}\)\xspace}
\newcommand{\Mathbbi}[2]{\(\mathbb{#1}_{#2}\)\xspace}
\newcommand{\MathbbI}[2]{\(\mathbb{#1}{#2}\)\xspace}

% Comentários
\newcommand{\cortar}[1]{\textcolor{red}{\sout{#1}}}
\newcommand{\ignore}[1]{\textcolor{blue}{\textbf{IGNOREM:} #1}}
\newcommand{\helena}[1]{\textcolor{magenta}{\textbf{HELENA:} #1}}
\newcommand{\migs}[1]{\textcolor{violet}{\textbf{MIGS:} #1}}
\newcommand{\migscortar}[2]{\textcolor{violet}{\textbf{MIGS:} \sout{#1}{#2}}}
\newcommand{\kaqui}[1]{\textcolor{teal}{\textbf{KAQUI:} #1}}

% Outros
\newcommand{\linguagem}[1]{\(\mathsf{LFI1}_{#1}\)\xspace}
\newcommand{\Linguagem}[1]{\mathsf{LFI1}_{#1}\xspace}
\newcommand{\funcao}[1]{\operatorname{#1}\xspace}
\newcommand{\inlinecoq}[1]{\lstinline[columns=fixed,language=coq]{#1}}

\newcommand{\Odot}  {\mathbin{\odot}}
\newcommand{\Oplus} {\mathbin{\oplus}}
\newcommand{\Otimes}{\mathbin{\otimes}}

% Abreviações
\newcommand{\lfium}{\textbf{LFI1}}
\newcommand{\lfi}{\textbf{LFI}}
\newcommand{\lfis}{\textbf{LFI}s}
\newcommand{\Ltac}{\Mathcal{L}\unskip~tac}
\newcommand{\CalcLambda}{Cálculo-\(\lambda\)\xspace}
\newcommand{\CalcsLambda}{Cálculos-\(\lambda\)\xspace}
\newcommand{\SisT}{\(\textbf{KT} \Odot \textbf{K4}\)\xspace}

\newcommand{\CLST}{Cálculo-\(\lambda\) Simplesmente Tipado\xspace}
\newcommand{\TTML}{Teoria de Tipos de Martin-Löf\xspace}
\newcommand{\CCH}{Correspondência de Curry-Howard\xspace}

\newcommand{\PIMODELOS}{\PI-Modelos\xspace}
\newcommand{\PIMODELO} {\PI-Modelo\xspace}
\newcommand{\PImodelos}{\PI-modelos\xspace}
\newcommand{\PImodelo} {\PI-modelo\xspace}

\newcommand{\PIFRAMES}{\PI-Frames\xspace}
\newcommand{\PIFRAME} {\PI-Frame\xspace}
\newcommand{\PIframes}{\PI-frames\xspace}
\newcommand{\PIframe} {\PI-frame\xspace}

\newcommand{\OPIMODELOS}{\OPI-Modelos\xspace}
\newcommand{\OPIMODELO} {\OPI-Modelo\xspace}
\newcommand{\OPImodelos}{\OPI-modelos\xspace}
\newcommand{\OPImodelo} {\OPI-modelo\xspace}

\newcommand{\OPIFRAMES}{\OPI-Frames\xspace}
\newcommand{\OPIFRAME} {\OPI-Frame\xspace}
\newcommand{\OPIframes}{\OPI-frames\xspace}
\newcommand{\OPIframe} {\OPI-frame\xspace}

\newcommand{\Modeloinicial}{\(\mathbfcal{I}\)\xspace}
\newcommand{\modeloinicial}{\mathbfcal{I}\xspace}

\newcommand{\Mundoinicial}{\textbf{\textit{i}}\xspace}
\newcommand{\mundoinicial}{\textbf{\textit{i}}\xspace}

\newcommand{\Mundobase}{\textit{w}\textsubscript{\Mathcal{M}}\xspace}
\newcommand{\mundobase}{w_{\mathcal{M}}\xspace}

\newcommand{\ElementoMaximo}{\MathcalI{S}{+}\xspace}
\newcommand{\elementomaximo}{\mathcal{S}^{+}\xspace}

% Referência de Environments
\newcommand{\SubCaso}[2]{\ref{#1}.\ref{#2}}

% Citações
% MIGUEL: Esses comandos nunca funcionaram, eu deixei pois um dia pretendia arrumar, nunca arrumei lmao
% \newcommand{\cciteshort}[2]{\citeshort{#1} e~\citeshort{#2}}
% \newcommand{\ccciteshort}[3]{\citeshort{#1},~\citeshort{#2} e~\citeshort{#3}} 

\frenchspacing

% -----------------------------------------------------------------
% ELEMENTOS PRÉ-TEXTUAIS
% -----------------------------------------------------------------
\pretextual

% Você pode comentar os elementos que não deseja em seu trabalho;

% A capa pode ser escolhida dentro do arquivo Capa.tex (TCC, Master, Doc, ...)
% ---
% Capa
% ---


% --------------------------------------------------------
% Capa Padrão
% --------------------------------------------------------
\renewcommand{\imprimircapa}{%
	\begin{capa}%
		\center

		{\fontseries{b}\selectfont\MakeTextUppercase{UNIVERSIDADE DO ESTADO DE SANTA CATARINA -- UDESC}}

		{\fontseries{b}\selectfont\MakeTextUppercase{CENTRO DE CIÊNCIAS TECNOLÓGICAS -- CCT  }}

		{\fontseries{b}\selectfont\MakeTextUppercase{BACHARELADO EM CIÊNCIA DA COMPUTAÇÃO -- BCC  }}

		\vfill

		{\fontseries{b}\selectfont\MakeTextUppercase{\normalsize\imprimirautor}}

		\vfill
		\begin{center}
			{\fontseries{b}\selectfont\MakeTextUppercase{\imprimirtitulo}}
		\end{center}
		\vfill

		\vfill

		{\fontseries{b}\selectfont\MakeTextUppercase{\imprimirlocal}}
		\par
		{\fontseries{b}\selectfont \imprimirdata}
		\vspace*{1cm}
	\end{capa}
}

\imprimircapa					% Elemento Obrigatório
% ---
% Folha de rosto
% ---


\makeatletter

\renewcommand{\folhaderostocontent}{
	\begin{center}
		
		{\fontseries{b}\selectfont\MakeTextUppercase{\imprimirautor}}
		
		\vfill
		
		\begin{center}
			{\fontseries{b}\selectfont\MakeTextUppercase{\imprimirtitulo}}
		\end{center}
	
		\vspace*{1.5cm}

		\abntex@ifnotempty{\imprimirpreambulo}{%
			\hspace{.45\textwidth}
			{\begin{minipage}{.5\textwidth}
					\SingleSpacing
					\imprimirpreambulo\par
					\vspace*{4pt}
					{\imprimirorientadorRotulo~\imprimirorientador\par}
					\abntex@ifnotempty{\imprimircoorientador}{%
						{\imprimircoorientadorRotulo~\imprimircoorientador}%
					}%
			\end{minipage}}%
		}%
	
		
		\vfill
		
	{\fontseries{b}\selectfont\MakeTextUppercase{\imprimirlocal}}
	\par
	{\fontseries{b}\selectfont \imprimirdata}
	\vspace*{1cm}
	\end{center}
}


% (o * indica que haverá a ficha bibliográfica)
% ---
\imprimirfolhaderosto*
% ---


			% Elemento Obrigatório
% Caso não utilize a Ficha Catalográfica entre na folha de rosto e retire o * de dentro do arquivo Folha de Rosto
% 
% ---
% Inserir a ficha bibliografica
% ---

% Isto é um exemplo de Ficha Catalográfica, ou ``Dados internacionais de
% catalogação-na-publicação''. Você pode utilizar este modelo como referência. 
% Porém, provavelmente a biblioteca da sua universidade lhe fornecerá um PDF
% com a ficha catalográfica definitiva após a defesa do trabalho. Quando estiver
% com o documento, salve-o como PDF no diretório do seu projeto e substitua todo
% o conteúdo de implementação deste arquivo pelo comando abaixo:



% \begin{fichacatalografica}
%     \includepdf{fig_ficha_catalografica.pdf}
% \end{fichacatalografica}


%	\setlength{\parindent}{0cm}
%	\setlength{\parskip}{0pt}
\begin{fichacatalografica}
	%\sffamily
	%\rmfamily
	\ttfamily \hbadness=10000
	\vspace*{\fill}					% Posição vertical
	\begin{center}					% Minipage Centralizado
	Para gerar a ficha catalográfica de teses e \\ 
	dissertações acessar o link:  \\
	https://www.udesc.br/bu/manuais/ficha
	
	\vspace*{8pt}
	
%	\begin{minipage}[c]{8cm}
%	\centering \sffamily
%	 Ficha catalográfica elaborada pelo(a) autor(a), com auxílio do programa de geração automática da Biblioteca Setorial do CCT/UDESC
%	\end{minipage}
	\fbox{\begin{minipage}[c]{12.5cm}		% Largura
	\flushright
	{\begin{minipage}[c]{10.5cm}		% Largura
	\vspace{1.25cm}
	%\footnotesize
	\setlength{\parindent}{1.5em}
	\noindent \invertname{\imprimirautor} \par
	\imprimirtitulo{ }/{ }\imprimirautor. -- \imprimirlocal, \imprimirdata .\par
	\pageref{LastPage} p. : il. ; 30 cm.\par
	\vspace{1.5em}
	\imprimirorientadorRotulo~\imprimirorientador.\par
	\imprimircoorientadorRotulo~\imprimircoorientador.\par
	\imprimirtipotrabalho~--~\imprimirinstituicao, \imprimirlocal, \imprimirdata.\par
	\vspace{1.5em}
		1. Palavra-chave.
		2. Palavra-chave.
		3. Palavra-chave.
 		4. Palavra-chave.
		5. Palavra-chave.
		I. \invertname{\imprimirorientador}.
		II. \invertname{\imprimircoorientador}.
		III. \imprimirinstituicao.
		IV. Título. %
	\vspace{1.25cm}	%		
	\end{minipage}%
	}% 
	\end{minipage}}%
	
	\vspace*{0.5cm}
	
	\end{center}
\end{fichacatalografica}


%\begin{fichacatalografica}
%	\sffamily
%	\vspace*{\fill}					% Posição vertical
%	\begin{center}					% Minipage Centralizado
%	\fbox{\begin{minipage}[c][8cm]{13.5cm}		% Largura
%	\small
%	\imprimirautor
%	%Sobrenome, Nome do autor
%	
%	\hspace{0.5cm} \imprimirtitulo  / \imprimirautor. --
%	\imprimirlocal, \imprimirdata-
%	
%	\hspace{0.5cm} \pageref{LastPage} p. : il. (algumas color.) ; 30 cm.\\
%	
%	\hspace{0.5cm} \imprimirorientadorRotulo~\imprimirorientador\\
%	
%	\hspace{0.5cm}
%	\parbox[t]{\textwidth}{\imprimirtipotrabalho~--~\imprimirinstituicao,
%	\imprimirdata.}\\
%	
%	\hspace{0.5cm}
%		1. Palavra-chave1.
%		2. Palavra-chave2.
%		3. Palavra-chave3.
% 		4. Palavra-chave4.
%		5. Palavra-chave5.
%		I. Orientador.
%		II. Universidade xxx.
%		III. Faculdade de xxx.
%		IV. Título 			
%	\end{minipage}}
%	\end{center}
%\end{fichacatalografica}
% ---

	% Elemento Obrigatório (Verso da Folha)
% 
% ---
% Inserir errata
% ---
\begin{errata}
Elemento opcional. 

Exemplo:

\vspace{\onelineskip}

SOBRENOME, Prenome do Autor. Título de obra: subtítulo (se houver). Ano de depósito. Tipo do trabalho (grau e curso) - Vinculação acadêmica, local de apresentação/defesa, data.

\begin{table}[htb]
\center
\begin{tabular}{|p{2.4cm}|p{2cm}|p{3cm}|p{3cm}|}
  \hline
   \textbf{Folha} & \textbf{Linha}  & \textbf{Onde se lê}  & \textbf{Leia-se}  \\
    \hline
    1 & 10 & auto-conclavo & autoconclavo\\
   \hline
\end{tabular}
\end{table}

\end{errata}
% ---				% Elemento Opcional

% ---
% Inserir folha de aprovação
% ---

% Isto é um exemplo de Folha de aprovação, elemento obrigatório da NBR
% 14724/2011 (seção 4.2.1.3). Você pode utilizar este modelo até a aprovação
% do trabalho. Após isso, substitua todo o conteúdo deste arquivo por uma
% imagem da página assinada pela banca com o comando abaixo:
%
% \includepdf{folhadeaprovacao_final.pdf}
%
\begin{folhadeaprovacao}



    \begin{center}
        {\fontseries{b}\selectfont\MakeTextUppercase{\normalsize\imprimirautor}}
    \end{center}
    \vfill

    \vfill
    \begin{center}
        {\fontseries{b}\selectfont\MakeTextUppercase{\imprimirtitulo}}
    \end{center}
    \vfill


\abntex@ifnotempty{\imprimirpreambulo}{%
    \hspace{.45\textwidth}
    {\begin{minipage}{.5\textwidth}
            \SingleSpacing
            \imprimirpreambulo\par
            \vspace*{4pt}
            {\imprimirorientadorRotulo~\imprimirorientador\par}
            \abntex@ifnotempty{\imprimircoorientador}{%
                {\imprimircoorientadorRotulo~\imprimircoorientador}%
            }%
    \end{minipage}}%
}%


\vfill

    \begin{center}
        {\fontseries{b}\selectfont BANCA EXAMINADORA: }
        \vspace*{1.75cm}
    \end{center}

    {Orientadora:}

    \begin{center}
        \begin{minipage}{8.75cm}
            \begin{flushleft}
                \rule{8.75cm}{0.1mm}

                Dra. Karina Girardi Roggia \par
                UDESC
            \end{flushleft}
        \end{minipage}
    \end{center}

    \vspace*{\baselineskip}
    {Coorientador:}

    \begin{center}
        \begin{minipage}{8.75cm}
            \begin{flushleft}
                \rule{8.75cm}{0.1mm}

                Miguel Alfredo Nunes \par
                UNICAMP
            \end{flushleft}
        \end{minipage}
    \end{center}

    \vspace*{\baselineskip}
    {Membros:}

    \begin{center}
        \begin{minipage}{8.75cm}
            \begin{flushleft}
                % \vspace*{1.25cm}
                \rule{8.75cm}{0.1mm}

                Dr. Cristiano Damiani Vasconcellos \par
                UDESC

                \vspace*{1cm}
                \rule{8.75cm}{0.1mm}

                Me. Paulo Henrique Torrens \par
                University of Kent
            \end{flushleft}
        \end{minipage}
    \end{center}

    \vspace*{\fill}
    \begin{center}
    {\imprimirlocal, Junho de \imprimirdata}
    \end{center}
    \vspace*{0.25cm}
\end{folhadeaprovacao}
% ---




%\textbf{	{Orientador: \vspace{-16pt} }
%	\assinatura{\textbf{Prof. \imprimirorientador , Dr.} \\ Univ. XXX}
%	{Coorientador: \vspace{-16pt}}
%	\assinatura{\textbf{Prof. \imprimircoorientador , Dr.} \\ Univ. XXX}
%
%	{Membros: \vspace{-16pt} }
%
%	% --- Exemplo de assinaturas em sequência ---
%	\setlength{\ABNTEXsignwidth}{8.5cm}
%
%	\assinatura{\textbf{Prof. Professor, Dr.} \\ Univ. XXX}
%	\assinatura{\textbf{Prof. Professor, Dr.} \\ Univ. XXX}
%	\assinatura{\textbf{Prof. Professor, Dr.} \\ Univ. XXX}
%
%	% --- Exemplo de assinaturas lado a lado ---
%	\setlength{\ABNTEXsignwidth}{7.5cm}
    %
    %    \noindent\hfill\assinatura*{\textbf{Prof. Professor, Dr.} \\ Univ. XXX}%
    %    \hfill%
    %    \assinatura*{\textbf{Prof. Professor, Dr.} \\ Univ. XXX}%
    %    \hfill
    %
    %    \noindent\hfill\assinatura*{\textbf{Prof. Professor, Dr.} \\ Univ. XXX}%
    %    \hfill%
    %    \assinatura*{\textbf{Prof. Professor, Dr.} \\ Univ. XXX}%
    %    \hfill}		% Elemento Obrigatório
% % ---
% Dedicatória
% ---
\begin{dedicatoria}
   \vspace*{\fill}
%   \begin{flushright}
%   \noindent
%	Este trabalho é dedicado às crianças adultas que,\\
%	quando pequenas, sonharam em se tornar cientistas. 
%   \end{flushright}

{%
	\noindent\hspace{.5\textwidth}
	{\begin{minipage}{.5\textwidth}
			\begin{flushleft}
				Aos estudantes da Universidade do Estado de Santa Catarina, pela inspiração de sempre!
			\end{flushleft}
	\end{minipage}}%
\vspace*{3cm}
}%

\end{dedicatoria}
% ---
			% Elemento Opcional
% ---
% Agradecimentos
% ---
\begin{agradecimentos}



\end{agradecimentos}
% ---		% Elemento Opcional
% ---
% Epígrafe
% ---
\begin{epigrafe}
    \vspace*{\fill}
{%
    \noindent\hspace{.5\textwidth}
    {\begin{minipage}{.5\textwidth}
        \textit{``Different conclusions are reached when one fact is viewed from two separate points of view. When that happens, there is no immediate way to judge which point of view is the correct one. There is no way to conclude one’s own conclusion is the correct one. But for that exact reason, it is also premature to decide one’s own conclusion is wrong.''}\\(Senjougahara Hitagi {-} Bakemonogatari, [2009])
    \end{minipage}}%
    \vspace*{3cm}
}%
\end{epigrafe}
% ---				% Elemento Opcional
% ---
% RESUMOS
% ---

% resumo em português
\setlength{\absparsep}{18pt} % ajusta o espaçamento dos parágrafos do resumo
\begin{resumo}
    Na medida em que sistemas de computação modernos escalam, a existência de informações contraditórias torna-se inevitável. As lógicas ortodoxas não são capazes de tratar este tipo de informação sem que o princípio da explosão tome lugar. Com isso, sistemas paraconsistentes {---} sistemas nos quais a explosividade é cuidadosamente separada da contraditoriedade {---} são uma alternativa vantajosa quando comparados às lógicas ortodoxas. Neste contexto, as lógicas de inconsistência formal, sobretudo a \lfium{}, usufruem de propriedades interessantes que as garantem aplicações em diversos campos do conhecimento, como, por exemplo, no desenvolvimento de sistemas de gerenciamento de bancos de dados. Com isso, a prova de metateoremas sobre estas lógicas evidencia características das diferentes abordagens possíveis no estudo destes sistemas. Ademais, assistentes de provas como o Coq proporcionam aos teoremas neles desenvolvidos uma garantia de correção dificilmente encontrada em provas manuais. Este trabalho propõe explanar e definir a lógica de inconsistência formal \lfium{}, bem como desenvolver metateoremas para este sistema no assistente de provas Coq.

 \textbf{Palavras-chave}: Coq, lógica paraconsistente, \lfium{}, lógica de inconsistência formal, lógica trivalorada.
\end{resumo}
				% Elemento Obrigatório
% ---
% Abstract
% ---
% Na medida em que sistemas de computação modernos escalam, a existência de informações
% contraditórias torna-se inevitável. As lógicas ortodoxas não são capazes de tratar este tipo
% de informação sem que o princípio da explosão tome lugar. Com isso, o estabelecimento de
% sistemas paraconsistentes — sistemas nos quais a explosividade é cuidadosamente separada
% da contraditoriedade — é uma alternativa vantajosa quando comparados às lógicas ortodoxas.
% Neste contexto, as lógicas de inconsistência formal, sobretudo a LFI1, usufruem de propriedades
% interessantes que as garantem aplicações em diversos campos do conhecimento, como, por
% exemplo, no desenvolvimento de sistemas de gerenciamento de bancos de dados. Com isso, a
% prova de metateoremas sobre estas lógicas evidencia características das diferentes abordagens
% possíveis no estudo destes sistemas. Ademais, assistentes de provas como o Coq proporcionam
% aos teoremas neles desenvolvidos uma garantia de correção dificilmente encontrada em provas
% manuais. Este trabalho propõe explanar e definir a lógica de inconsistência formal LFI1, bem
% como desenvolver metateoremas para este sistema no assistente de provas Coq.
% resumo em inglês
\begin{resumo}[Abstract]
 \begin{otherlanguage*}{english}
    As modern computer systems scale, the existence of contradictory information becomes inevitable. Most orthodox logics are not able to cope with this kind of information without the principle of explosion taking place. Thus, establishing paraconsistent systems {--} systems in which explosiveness is carefully separated from contradictoriness {--} is an advantageous alternative compared to orthodox logics. From this perspective, the logics of formal inconsistency, specially \lfium{}, enjoy some interesting properties which allow them to be used in many areas, for example in the development of database management systems. In this light, proving metatheorems about these logics highlights characteristics of different approaches when studying these systems. Furthermore, proof assistants, such as Coq, guarantee the theorems proved inside them a degree of certainty about their correctness hardly ever found in manual proofs. The present work explores and defines the logic of formal inconsistency \lfium{}, as well as proves metatheorems for this system inside the Coq proof assistant.

   \textbf{Keywords}: Coq, paraconsistent logic, \lfium{}, logics of formal inconsistency, three-valued logic.
 \end{otherlanguage*}
\end{resumo}

				% Elemento Obrigatório

% ---
% inserir lista de ilustrações
% ---
% \pdfbookmark[0]{\listfigurename}{lof}
% \listoffigures*
% \cleardoublepage
% ---

% ---
% inserir lista de quadros
% ---
% \pdfbookmark[0]{\listofquadrosname}{loq}
% \listofquadros*
% \cleardoublepage
% ---


% ---
% inserir lista de tabelas
% ---
\pdfbookmark[0]{\listtablename}{lot}
\listoftables*
\cleardoublepage
% ---

% ---
% inserir lista de abreviaturas e siglas
% ---
\begin{siglas}
	\item[MP] \textit{Modus Ponens}
	\item[MTD] Metateorema da dedução
	\item[sse] se e somente se
	\item[CIC] \textit{Calculus of Inductive Constructions}
\end{siglas}
% % ---

% % ---
% % inserir lista de símbolos
% % ---
\begin{simbolos}
    \item [$p, q, r\ldots$] Variáveis atômicas.
    \item [$A, B, C, \ldots$] Conjuntos quaisquer.
    \item [$\alpha, \beta, \gamma, \ldots$] Fórmulas quaisquer.
    \item [$\Gamma, \Delta$] Conjuntos de fórmulas.
    \item [$\Sigma, \Theta$] Assinaturas de linguagens.
    \item[$\Vdash$] Relação de consequência qualquer (sintática ou semântica).
    \item[$\vdash$] Relação de consequência sintática.
    \item[$\vDash$] Relação de consequência semântica.
\end{simbolos}


% ---
				% Elemento Opcional
% ---
% inserir o sumario
% ---
\pdfbookmark[0]{\contentsname}{toc}
\tableofcontents*
\cleardoublepage
% ---
				% Elemento Obrigatório

% -----------------------------------------------------------------
% ELEMENTOS TEXTUAIS
% -----------------------------------------------------------------
\textual

\pagestyle{PagNumReduzida}						% Comando para cabeçalho somente com numeração de página 10pt
\aliaspagestyle{chapter}{PagNumReduzida}		% Deixar numeração da primeira página com tamanho igual ao resto da numeração
% ref.: https://groups.google.com/g/abntex2/c/CP7g8ZMgi-c/m/KjfEnn5b9a4J


% ---- Mantenha esta estrutura, assim você deixa o trabalho mais organizado -------

\chapter{Introdução}

\noindent\label{cap:Introducao}
As lógicas paraconsistentes são uma família de lógicas na qual a presença de contradições não implica trivialidade, ou seja, são sistemas lógicos que possuem uma negação que não respeita o Princípio da Explosão\footnote{Definido como $\alpha \rightarrow (\neg \alpha \rightarrow \beta)$.}~\cite{carnielli2007}. Tradicionalmente, em lógicas ortodóxas, qualquer teoria que seja inconsistente {-} e, portanto, não respeite o Princípio da não-contradição\footnote{Definido como $\neg (\alpha \land \neg \alpha)$.} {-} será uma teoria trivial (uma teoria que possui todas as sentenças). Deste modo, as lógicas paraconsistentes surgem como uma ferramenta que permite tratar contradições sem trivializar o sistema lógico~\cite{Carnielli_Coniglio_2016}.

De acordo com~\cite{sep-logic-paraconsistent}, as motivações para o estudo de lógicas paraconsistentes podem ser observadas em diversos campos do conhecimento. Nas ciências naturais, por exemplo, teorias inconsistentes e não-triviais são comuns, como é o caso da teoria do átomo de Bohr, que, segundo~\cite{Brown2015-BROCAP-9}, deve possuir uma inferência paraconsistente. No campo da linguística, inconsistências não-triviais também são possíveis, como a preservação da noção espacial da palavra ``Próximo'' mesmo tratando-se de objetos impossíveis~\cite{McGinnis2013-MCGTUA}. No contexto da ciência da computação é o uso de lógicas de inconsistência formal para a modelagem e o desenvolvimento de bancos de dados evolucionários~\cite{carnielli2000formal}.

As lógicas de inconsistência formal (\textbf{LFI}s), são lógicas paraconsistentes que introduzem os conceitos de consistência e inconsistência como formas de representar o excesso de informações (por exemplo, evidência de $\alpha$ e evidência de $\neg \alpha$ sem evidência conclusiva para $\beta$), para resgatar a capacidade de se obter a trivialidade em alguns casos~\cite{carnielli2007}. Ao explicitamente representar a consistência dentro da sua linguagem, é possivel estudar teorias inconsistentes sem necessariamente assumir que elas são triviais. A ideia por trás das \textbf{LFI}s é que deve-se respeitar as noções da lógica clássica o máximo possível, desviando desta somente na presença de contradições. Isto significa que, na ausência de contradições, o Princípio da Explosão deve ser tomado como válido~\cite{sep-logic-paraconsistent}.



    \section{Objetivo Geral}


    \section{Objetivos Específicos}



    \section{Trabalhos Relacionados}


    \section{Metodologia}
        

    \section{Estrutura do Trabalho}
       
\chapter{Lógicas de Inconsistência Formal}
\label{cap:LFIs}
No estudo de lógicas clássicas uma contradição é considerada inseparável da trivialidade, ou seja, se uma teoria possuir um subconjunto $\{\alpha,\neg \alpha\}$ de fórmulas, pode-se derivar qualquer sentença. Esta propriedade é chamada de \textit{explosividade}. Desta forma, as lógicas clássicas (e certas lógicas não-clássicas, como a lógica intuicionista), expressam sua \textit{explosividade} como representada pela seguinte equação:
\begin{center}
    Contradições = Trivialidade
\end{center}
As \textit{Lógicas de Inconsistência Formal} são lógicas paraconsistentes que se propõem a questionar a noção apresentada anteriormente sem abrir mão completamente da trivialidade. Isto é feito estabelecendo uma nova propriedade, chamada de \textit{explosividade gentil}, que resgata a trivialidade introduzindo o conceito de consistência na sua linguagem~\cite{carnielli2007}. A consistência é expressa na \textit{explosividade gentil} da seguinte forma:
\begin{center}
    Contradições + Consistência = Trivialidade
\end{center}
Definir uma lógica que consiga superar o tabu da \textit{explosividade} e, ao mesmo tempo, representar uma ferramenta legítima capaz de formalizar o raciocínio e separar inferências aceitáveis de inferências equivocadas é um dos objetivos do \textit{paraconsistentista}.\helena{Esse termo é do diabo mas o carnielli e o joão marcos usaram e eu achei até que bonitinho.} As \textit{Lógicas de Inconsistência Formal} cumprem este objetivo de maneira elegante, servindo um propósito importante no estudo de lógicas não-clássicas.\helena{eu boto aquele lerolero de banco de dados aqui tb?}.
Neste capítulo são apresentadas algumas definições necessárias para caracterizar as \textit{Lógicas de Inconsistência Formal}, baseadas em~\citeshort{Carnielli_Coniglio_2016} e em~\citeshort{carnielli2007}. Antes de definir as \lfis{} é preciso apresentar alguns conceitos básicos acerca de sistemas lógicos paraconsistentes. Nas definições que seguem, utiliza-se a seguinte representação: \migs{Isso deveria ir para a lista de símbolos}
\begin{itemize}
    \item Letras minúsculas do alfabeto latim $p, q, r, \ldots$ para representar fórmulas atômicas.
    \item Letras maiúsculas do alfabeto latim $A, B, C, \ldots$ para representar conjuntos quaisquer.
    \item Letra minúsculas do alfabeto grego $\alpha, \beta, \gamma, \ldots$ para representar fórmulas quaisquer.
    \item As letras $\Gamma, \Delta$ para representar teorias (conjuntos de fórmulas).
    \item As letras $\Sigma, \Theta$ para representar a assinatura de uma linguagem.
    \item O operador $\vdash$ para representar uma relação de consequência sintática.
    \item O operador $\vDash$ para representar uma relação de consequência semântica.
    \item O operador $\Vdash$ para representar uma relação de consequência qualquer (sintática ou semântica).
\end{itemize}

Ademais, o presente trabalho segue o mesmo caminho de~\citeshort{Carnielli_Coniglio_2016}, baseando-se na teoria geral de relações de consequências para definir \textit{lógicas tarskianas}. Neste sentido, como a lógica \lfium{} se trata de uma \textit{lógica tarskiana}, o presente trabalho se restringe a trabalhar somente neste escopo. Nas Seções~\ref{sec:paracons} e~\ref{sec:incons}, serão apresentadas definições que se aplicam tanto a relações de consequência semântica quanto a relações de consequência sintática, denotadas genericamente pelo operador $\Vdash$. O leitor é encorajado a tomar conhecimento dos trabalhos de Wójcicki~(\citeyear{Wojcicki1984-WJCLOP,Wojcicki1988-WOJAAT,Wojcicki1988-WOJTOL}) e~\citeshort{analysis_and_synthesis_of_logic} a fim de compreender mais sobre relações e operações de consequência, suas propriedades e as diferenças entre abordagens prova-teóricas (exploradas na Seção~\ref{sec:axiomatizacao}) e abordagens modelo-teóricas (exploradas na Seção~\ref{sec:semantica}).

\section{Paraconsistência}
\label{sec:paracons}
Um sistema lógico que se atreve a romper com o Princípio da Explosão {-} o qual afirma que a partir de uma teoria contraditória, qualquer conclusão segue {-} é dito paraconsistente. As justificativas para questionar tal princípio existem em diversos campos do conhecimento, como na linguística~\cite{McGinnis2013-MCGTUA}, na computação~\cite{carnielli2000formal} e até mesmo nas ciências naturais~\cite{Brown2015-BROCAP-9}. Outra justificativa para o desenvolvimento de sistemas lógicos paraconsistentes é um mero descontentamento com o caráter explosivo das lógicas ortodoxas. Por exemplo, um descontentamento com a conclusão de que ``Um triângulo tem quatro lados.'' a partir da evidência de que ``Choveu na tarde de ontem.'' e da evidência de que ``Não choveu na tarde de ontem.''~\footnote{Esta forma de explosividade é objeto de estudo das lógicas de relevância, que tratam da conexão entre as premissas e a conclusão de uma inferência~\cite{sep-logic-relevance}.}. Nesta seção, a paraconsistência será definida formalmente, partindo de definições básicas sobre lógica e classificando diferentes sistemas de acordo com propriedades acerca de sua \textit{relação de consequência}.

Uma lógica $\mathcal{L}$ será representada como uma dupla $\mathcal{L} = \langle \pazocal{L},\Vdash \rangle$, onde $\pazocal{L}$ é sua linguagem (seu conjunto de fórmulas) e $\Vdash$ é uma relação de consequência de conclusão única, definida como $\Vdash \;\subseteq \wp(\pazocal{L})\times\pazocal{L}$. Em uma consequência do tipo $A \Vdash \alpha$ (lida como ``$\alpha$ é uma consequência de $A$.'') diz-se que o conjunto $A$ é o conjunto de premissas e $\alpha$ é a conclusão. A fim de facilitar a escrita e leitura, uma notação que resume o conjunto de premissas pode ser estabelecida.

\begin{notacao}
    Sejam $\Gamma, \Delta$ teorias e $\phi, \psi$ fórmulas, então $\Gamma, \Delta, \phi \Vdash \psi$ denota $\Gamma \cup \Delta \cup \{\phi\} \Vdash \psi$.
\end{notacao}

\begin{definicao}[Assinatura proposicional]
    \label{def:ass_prop}
    Uma assinatura proposicional $\Theta$ é um conjunto de conectivos lógicos com a informação acerca da aridade de cada um destes.\qed{}
\end{definicao}
Por exemplo, a assinatura proposicional para a lógica proposicional clássica pode ser definida como $\Theta_{LPC} = \{\land^{2}, \lor^{2}, \neg^{1}, \rightarrow^{2}\}$, onde o operador $\land^{2}$ representa uma conjunção, $\lor^{2}$ representa uma disjunção, $\neg^{1}$ representa uma negação e $\rightarrow^{2}$ representa uma implicação. No que segue do texto, as aridades de cada operador serão omitidas.

Uma assinatura proposicional juntamente com um conjunto enumerável de átomos são base para a definição de uma linguagem proposicional, que por sua vez é utilizada para definir uma lógica proposicional. No presente trabalho, a lógica de interesse (a lógica \lfium{}) é uma lógica proposicional.

\begin{definicao}[Lógica proposicional]
    \label{def:proposicional}
    Um sistema lógico $\mathcal{L}$, definido sobre uma linguagem $\pazocal{L}$ é dito proposicional caso $\pazocal{L}$ seja definida a partir de um conjunto enumerável de átomos $\pazocal{P} = \{p_{i} \;| \; i \in \mathbb{N} \}$ e uma assinatura proposicional $\Theta$. A linguagem $\pazocal{L}$ é chamada de linguagem proposicional.\qed{}
\end{definicao}

\begin{definicao}[Substituição]
    \label{def:substituicao}
    Uma substituição $\sigma$ de todas as ocorrências de uma variável $p_{i}$ por uma fórmula $\psi$ em uma fórmula $\phi$, é denotada por $\sigma(\phi) = \phi\{p_{i} \mapsto \psi\}$~\cite{dedo}. A substituição $\phi\{p_{i} \mapsto \psi\}$ é definida indutivamente como (considerando $\triangle$, $\otimes$ conectivos quaisquer de aridade 1 e 2 respectivamente):
    \begin{align*}
         & \text{1.~Se }\phi = p_{i} \text{ então, } \phi\{p_{i} \mapsto \psi\} = \psi;                                                                                             \\
         & \text{2.~Se }\phi = p_{j} \text{ e } j \neq i \text{ então, }\phi\{p_{i} \mapsto \psi\} = \phi;                                                                          \\
         & \text{3.~Se }\phi = \triangle \gamma \text{ então, } \phi\{p_{i} \mapsto \psi\} = \triangle(\gamma\{p_{i} \mapsto \psi\});                                                 \\
         & \text{4.~Se }\phi = \phi_{0} \otimes \phi_{1} \text{ então, } \phi\{p_{i} \mapsto \psi\} = \phi_{0}\{p_{i} \mapsto \psi\} \otimes \phi_{1}\{p_{i} \mapsto \psi\}.
    \end{align*}
    Uma fórmula $\alpha$ é dita \textit{instância de substituição} de uma fórmula $\beta$ caso exista uma substituição $\sigma$ tal que $\alpha = \sigma(\beta)$. Ademais, a aplicação de uma substituição $\sigma$ sobre todos os elementos de uma teoria $\Gamma$ é definida como $\sigma[\Gamma] = \{\sigma(\phi) \; | \; \phi \in \Gamma\}$.\qed{}
\end{definicao}


Tendo definido a noção de substituição para lógicas proposicionais, é possível descrever as \textit{lógicas estruturais} como sendo lógicas nas quais todas as inferências são fechadas para a substituição:

\begin{definicao}[Lógica Estrutural]
    Uma lógica proposicional $\mathcal{L}$ definida sobre uma linguagem proposicional $\pazocal{L}_{\Theta}$ é dita \textit{estrutural} caso respeite a seguinte condição para todo $\Gamma \cup \Delta \cup \{\alpha\} \subseteq \pazocal{L}$:
    \begin{align*}
         & \text{Se } \Gamma \Vdash \alpha \text{ então } \sigma [\Gamma] \Vdash \sigma(\alpha) \text{, para toda substituição } \sigma \text{ de variável por fórmula.}\tag*\qed{}
    \end{align*}
\end{definicao}

\begin{definicao}[Lógica Tarskiana]
    \label{def:tarski}
    Uma lógica $\mathcal{L}$, definida sobre uma linguagem $\pazocal{L}$ e munida com uma relação de consequência $\Vdash$ é dita \textit{Tarskiana} caso satisfaça as seguintes propriedades para todo $\Gamma \cup \Delta \cup \{\alpha\} \subseteq \pazocal{L}$:
    \begin{align}
         & \text{~~(i) Se } \alpha \in \Gamma \text{ então } \Gamma \Vdash \alpha;\tag{reflexividade}                                                                                       \\
         & \text{~(ii) Se } \Delta \Vdash \alpha \text{ e } \Delta \subseteq \Gamma \text{ então } \Gamma \Vdash \alpha;\tag{monotonicidade}                                                \\
         & \text{(iii) Se } \Delta \Vdash \alpha \text{ e } \Gamma \Vdash \delta \text{ para todo } \delta \in \Delta \text{ então } \Gamma \Vdash \alpha.\tag{\textit{cut} para conjuntos}
    \end{align}
    Uma lógica $\mathcal{L}$ é dita \textit{finitária} caso satisfaça o seguinte:
    \begin{align*}
         & \text{~(iv) Se } \Gamma \Vdash \alpha \text{ então existe conjunto finito } \Gamma_{0} \subseteq \Gamma \text{ tal que } \Gamma_{0} \Vdash \alpha.
    \end{align*}
    Por fim, uma lógica proposicional $\mathcal{L}$ é dita \textit{padrão} caso ela seja Tarskiana, finitária e estrutural.\qed{}
\end{definicao}


Com isto, é possível definir formalmente o conceito de \textit{paraconsistência} para lógicas Tarskianas.

\begin{definicao}[Lógica Tarskiana paraconsistente]
    \label{def:tarskiana_paracons}
    Uma lógica Tarskiana $\mathcal{L}$, definida sobre uma linguagem $\pazocal{L}$, é dita \textit{paraconsistente} se ela possuir uma negação\footnote{Esta negação pode ser primitiva (pertencente à assinatura da linguagem) ou definida a partir de outras fórmulas.} $\neg$ tal que existem fórmulas $\alpha, \beta \in \pazocal{L}$ de modo que $\alpha, \neg \alpha \nVdash \beta$.\qed{}
\end{definicao}

Caso a linguagem de $\mathcal{L}$ possua uma implicação $\rightarrow$ que respeite o metateorema da dedução\footnote{Definido como $\Gamma, \alpha \Vdash \beta \Longleftrightarrow  \Gamma\Vdash \alpha \rightarrow \beta$.}, então $\mathcal{L}$ é paraconsistente se e somente se a fórmula $\alpha \rightarrow (\neg \alpha \rightarrow \beta)$ não for válida. Ou seja, o Princípio da Explosão é inválido (em relação a $\neg$), logo $\neg$ é uma negação \textit{não explosiva}.

\section{Inconsistência}
\label{sec:incons}
A motivação para o desenvolvimento das \lfis{} é possuir sistemas lógicos paraconsistentes nos quais é possível resgatar, de maneira \textit{controlada}, o Princípio da Explosão. Ao internalizar o conceito de consistência, as \lfis{} propõem a noção de que uma contradição que é reconhecidamente inconsistente numa dada teoria é inofensiva e é somente fruto do excesso de informação. O resgate \textit{controlado} da \textit{explosividade} é feito definindo um conjunto $\bigcirc(p)$ de fórmulas dependentes somente em uma variável proposicional $p$. Caso uma lógica $\mathcal{L}$ seja explosiva ao unir-se um conjunto $\bigcirc(\alpha)$ {-} definido a partir de $\bigcirc(p)$ {-} com uma contradição $\{\alpha, \neg \alpha\}$, ou seja, se $\bigcirc(\alpha), \alpha, \neg \alpha \Vdash \beta$ para todo $\alpha$ e $\beta$ pertencentes à sua linguagem, e ainda $\bigcirc(\alpha), \alpha \nVdash \beta$ e $\bigcirc(\alpha), \neg \alpha \nVdash \beta$, então dizemos que $\mathcal{L}$ é \textit{gentilmente explosiva}.

\begin{notacao}
    Dado um átomo $p$, define-se $\bigcirc(p)$ como um conjunto não-vazio de fórmulas dependentes somente em $p$. Com base neste conjunto, define-se a notação $\bigcirc(\phi)$ para representar o conjunto obtido pela substituição de todas as ocorrências de $p$ por $\phi$ em todos os elementos de $\bigcirc(p)$, ou seja, para uma fórmula $\phi$ qualquer, $\bigcirc(\phi) = \{\psi\{p \mapsto \phi\} \; | \; \psi \in \bigcirc(p)\}$.
\end{notacao}



\begin{definicao}[Lógica de Inconsistência Formal]
    \label{def:lfi}
    Seja $\mathcal{L} = \langle \pazocal{L}_{\Theta}, \Vdash \rangle$ uma lógica padrão, de forma que sua assinatura proposicional $\Theta$ possua uma negação $\neg$. Seja $\bigcirc(p)$ um conjunto não-vazio de fórmulas dependentes somente na variável proposicional $p$. Então $\mathcal{L}$ será uma \textit{Lógica de Inconsistência Formal} (\lfi{}) (em relação a $\bigcirc(p)$ e $\neg$) caso ela respeite as seguintes condições:
    %(considerando $\bigcirc(\phi) = \{\psi(\phi) \; | \; \psi(p) \in \bigcirc(p)\}$): \helena{O conjunto $\bigcirc(\phi)$ é definido substituindo-se cada ocorrência $p$ por $\phi$ em cada elemento de $\bigcirc(\phi)$.}
    \begin{align*}
         & \text{~~(i) Existem } \gamma, \delta \in \pazocal{L}_{\Theta} \text{ de modo que } \gamma, \neg \gamma \nVdash \delta;               \\
         & \text{~(ii) Existem } \alpha, \beta \in \pazocal{L}_{\Theta} \text{ de modo que:}                                                    \\
         & \qquad \text{(ii.a)} \bigcirc(\alpha), \alpha \nVdash \beta;                                                                         \\
         & \qquad \text{(ii.a)} \bigcirc(\alpha), \neg \alpha \nVdash \beta;                                                                    \\
         & \text{(iii) Para todo } \phi, \psi \in \pazocal{L}_{\Theta} \text{ tem-se } \bigcirc(\phi), \phi, \neg \phi \Vdash \psi. \tag*\qed{}
    \end{align*}
\end{definicao}

A condição (i) diz que toda \lfi{} é \textit{não-explosiva} (em relação a $\neg$) e a condição (iii) diz que toda \lfi{} é \textit{gentilmente explosiva} (em relação a $\bigcirc{p}$ e $\neg$).

Na literatura existem outras definições para as Lógicas de Inconsistência Formal, que relaxam a condição (ii) para obter uma definição mais uniforme. Segundo~\citeshort{Carnielli_Coniglio_2016}, elas definem \lfis{} \textit{fracas} da seguinte forma:


\begin{definicao}[\lfi{} fraca]
    \label{def:lfi_fraca}
    Seja $\mathcal{L} = \langle \pazocal{L}_{\Theta}, \Vdash \rangle$ uma lógica padrão, de forma que sua assinatura proposicional $\Theta$ possua uma negação $\neg$. Seja $\bigcirc(p)$ um conjunto não-vazio de fórmulas dependentes somente na variável proposicional $p$. Então $\mathcal{L}$ será uma \lfi{} \textit{fraca} (em relação a $\bigcirc(p)$ e $\neg$) caso ela respeite as seguintes condições:
    %(considerando $\bigcirc(\phi) = \{\psi(\phi) \; | \; \psi(p) \in \bigcirc(p)\}$): \helena{O conjunto $\bigcirc(\phi)$ é definido substituindo-se cada ocorrência $p$ por $\phi$ em cada elemento de $\bigcirc(\phi)$.}
    \begin{align*}
         & \text{~~(i) Existem } \phi, \psi \in \pazocal{L}_{\Theta} \text{ de modo que } \phi, \neg \phi \nVdash \psi;\\
         & \text{~(ii) Existem } \phi, \psi \in \pazocal{L}_{\Theta} \text{ de modo que } \bigcirc(\phi), \phi \nVdash \psi;\\
         & \text{(iii) Existem } \phi, \psi \in \pazocal{L}_{\Theta} \text{ de modo que } \bigcirc(\phi), \neg \phi \nVdash \psi;\\
         & \text{(iv) Para todo } \phi, \psi \in \pazocal{L}_{\Theta} \text{ tem-se } \bigcirc(\phi), \phi, \neg \phi \Vdash \psi. \tag*\qed{}
    \end{align*}
\end{definicao}

Como é possível observar pelas duas definições acima, toda \lfi{} é uma \lfi{} fraca (já que a condição (ii) da Definição~\ref{def:lfi} satisfaz as condições (ii) e (iii) da Definição~\ref{def:lfi_fraca}), mas o inverso não é necessariamente verdade. Ademais, é possível estabelecer outra definição (também mais uniforme do que a Definição~\ref{def:lfi}) que introduz o conceito de \lfis{} \textit{fortes} como feito abaixo:

\begin{definicao}[\lfi{} forte]
    \label{def:lfi_forte}
    Seja $\mathcal{L} = \langle \pazocal{L}_{\Theta}, \Vdash \rangle$ uma lógica padrão, de forma que sua assinatura proposicional $\Theta$ possua uma negação $\neg$. Seja $\bigcirc(p)$ um conjunto não-vazio de fórmulas dependentes somente na variável proposicional $p$. Então $\mathcal{L}$ será uma \lfi{} \textit{forte} (em relação a $\bigcirc(p)$ e $\neg$) caso ela respeite as seguintes condições:
    %(considerando $\bigcirc(\phi) = \{\psi(\phi) \; | \; \psi(p) \in \bigcirc(p)\}$): \helena{O conjunto $\bigcirc(\phi)$ é definido substituindo-se cada ocorrência $p$ por $\phi$ em cada elemento de $\bigcirc(\phi)$.}
    \begin{align*}
         & \text{~~(i) Existem } \alpha, \beta \in \pazocal{L}_{\Theta} \text{ de modo que:}\\
         & \qquad \text{(i.a) } \alpha, \neg \alpha \nVdash \beta;\\
         & \qquad \text{(i.b) } \bigcirc(\alpha), \alpha \nVdash \beta;\\
         & \qquad \text{(i.c) } \bigcirc(\alpha), \neg \alpha \nVdash \beta;\\
         & \text{~(ii) Para todo } \phi, \psi \in \pazocal{L}_{\Theta} \text{ tem-se } \bigcirc(\phi), \phi, \neg \phi \Vdash \psi. \tag*\qed{}
    \end{align*}
\end{definicao}

É imediato perceber que toda \lfi{} forte é uma \lfi{} (já que a condição (i) da Definição~\ref{def:lfi_forte} satisfaz as condições (i) e (ii) da Definição~\ref{def:lfi}), mas o inverso não é necessariamente verdade. Ademais, no escopo das lógicas proposicionais, é possível estabelecer uma forma mais simples de provar que uma dada lógica proposicional é uma \textbf{LFI} forte, tomando $\alpha$ e $\beta$ como dois átomos $p$ e $q$ quaisquer nas condições (i.a), (i.b) e (i.c) da definição acima.



\chapter{A Lógica de Inconsistência Formal LFI1}\label{cap:LFI1}

Com os avanços da internet no contexto do gerenciamento de bancos de dados, informações passaram a ser coletadas a partir de diferentes fontes que frequentemente se contradizem. Dada a existência das restrições de integridade {--} que impedem contradições {--} a atualização e manutenção de bancos de dados se torna um processo difícil~\cite{carnielli2000formal}. Portanto, uma lógica capaz de lidar com informações inconsistentes sem necessariamente sofrer com a trivialidade é de grande interesse. A lógica de inconsistência formal \lfium{} é capaz de lidar com contradições ao introduzir na sua assinatura o operador $\circ$ para representar a consistência, internalizando este conceito em sua linguagem. Uma informação é dita consistente caso ela e sua negação não sejam simultaneamente verdadeiras, ou seja, dada uma informação $\alpha$, sua consistência $\circ \alpha$ será equivalente a fórmula $\neg (\alpha \land \neg \alpha)$. Com a introdução deste novo operador, é possível lidar com a inconsistência de informações sem que trivialidade ocorra, já que {--} caso uma informação seja conhecidamente \textit{inconsistente}, ou seja, $\neg \circ \alpha$ {--} então ela se trata de uma contradição inofensiva, fruto do excesso de informações numa dada teoria. Com isso, na \lfium{}, o conjunto $\bigcirc(p)$ de fórmulas dependentes somente na variável $p$ (descrito na Definição~\ref{def:lfi}) assume forma $\{\circ p\}$.

No trabalho de~\citeshort{carnielli2000formal}, a lógica \textbf{LFI1*} é definida como uma extensão de primeira ordem da lógica proposicional \lfium{}. A motivação para definir-se uma lógica de inconsistência formal de primeira ordem vem da natureza das informações contidas em bancos de dados, estas que podem ser compreendidas como sentenças de primeira ordem fixas~\cite{Codd}, entretanto, o presente trabalho trata somente da lógica proposicional \lfium{}. Ademais,~\citeshort{carnielli2000formal} tomam o operador de \textit{inconsistência} (denotado por $\bullet$) como primitivo. Isto foi feito pois o foco era explorar a \lfium{} como uma ferramenta para lidar com inconsistências em bancos de dados, portanto tomar a inconsistência como primitiva era de grande interesse. Entretanto, no presente trabalho, será utilizada a definição apresentada em~\citeshort{Carnielli_Coniglio_2016}, onde a linguagem é definida utilizando o operador $\circ$ como primitivo. Isto salienta algumas propriedades interessantes da negação $\neg$, como a presença das leis de De Morgan, axiomatizadas na Seção~\ref{sec:axiomatizacao}.

Este capítulo é dividido da seguinte forma: na Seção~\ref{sec:linguagem}, é apresentada a linguagem da lógica proposicional \lfium{} bem como definições necessárias para desenvolver as provas de metateoremas. A Seção~\ref{sec:axiomatizacao} contém uma breve explicação sobre sistemas de prova sintáticos e um cálculo de Hilbert para a \lfium{} é definido. Na Seção~\ref{sec:semantica}, a semântica da \lfium{} é definida a partir de matrizes lógicas e de uma semântica de valorações não determinística, assim como é provada a equivalência entre essas dois sistemas.

\section{Linguagem}\label{sec:linguagem}
    A lógica proposicional \lfium{} aqui apresentada é definida com base em~\citeshort{Carnielli_Coniglio_2016} sobre a linguagem $\ling{}$, que por sua vez é definida sobre um conjunto enumerável de átomos $\pazocal{P} = \{p_{n} \;|\; n \in \mathbb{N}\}$ e uma assinatura proposicional $\Sigma = \{\land^{2}, \lor^{2}, \to^{2}, \neg^{1}, \circ^{1}\}$. Como de costume, o conectivo $\land$ representa uma conjunção, $\lor$ representa uma disjunção, $\to$ representa uma implicação, $\neg$ representa uma negação e $\circ$ é o conectivo de consistência, definido de forma primitiva. No restante do texto a aridade destes conectivos será omitida. A linguagem $\ling{}$ da \lfium{} é definida da seguinte forma:

    \begin{definicao}[Linguagem da \lfium{}]
        A linguagem $\ling{}$ da \lfium{} é definida indutivamente como o menor conjunto a que respeita as seguintes regras:\label{def:ling}
        \begin{align*}
            & \text{1.~}\pazocal{P} \subseteq \ling{}                                                                                                                        \\
            & \text{2.~Se } \phi \in \ling{}, \text{então } \triangle  \phi \in \ling{}, \text{com } \triangle \in \{\neg, \circ\}                            \\
            & \text{3.~Se } \phi, \psi \in \ling{}, \text{então } \phi \otimes \psi \in \ling{}, \text{com } \otimes \in \{\land, \lor, \to\} \tag*\qed
        \end{align*}
    \end{definicao}

    A precedência dos conectivos é dada de maneira costumeira, com a adição do operador $\circ$ de consistência, seguindo a ordem (da maior precedência para a menor): $\circ$, $\neg$, $\land$, $\lor$, $\to$. Os conectivos binários $\land$ e $\lor$ são associativos à esquerda, ou seja, uma expressão do tipo $\alpha \land \beta \land \gamma$ é lida como $((\alpha \land \beta) \land \gamma)$, e o conectivo $\to$ é associativo à direita, ou seja, uma expressão do tipo $\alpha \to \beta \to \gamma$ é lida como $(\alpha \to (\beta \to \gamma))$.

    A linguagem da \lfium{} pode ser definida de maneira equivalente utilizando-se o operador de inconsistência (representado por $\bullet$), definido como $\bullet \alpha \eqdef \neg \circ \alpha$, como feito por~\citeshort{carnielli2000formal}. 

    % \begin{definicao}[Subfórmulas]
    %     \label{def:subf}
    %     O conjunto Sub$(\phi)$ de subfórmulas de uma fórmula $\phi$ é definido indutivamente da seguinte forma:
    %     \begin{align*}
    %          & \text{1.~Sub}(p_{i}) = \{p_{i}\}, \; p_{i} \in \pazocal{P}                                                                                                            \\
    %          & \text{2.~Sub}(\triangle \phi) = \{\triangle \phi\} \; \cup \;\text{Sub}(\phi), \; \triangle \in \{\neg, \circ\}                                                     \\
    %          & \text{3.~Sub}(\phi \otimes \psi) = \{\phi \otimes \psi\} \; \cup \;\text{Sub}(\phi) \; \cup \;\text{Sub}(\psi), \; \otimes \in \{\land, \lor, \to\} \tag*\qed
    %     \end{align*}
    % \end{definicao}


    Na Definição~\ref{def:complex}, a função $C$ da complexidade de uma fórmula na lógica proposicional clássica foi recursivamente definida. É possível estendê-la para identificar a complexidade de uma fórmula na \lfium{} adicionando-se uma condição para o operador $\circ$:

    \begin{definicao}[Complexidade de uma fórmula na \lfium{}]
        Seja $\phi \in \ling{}$ uma fórmula bem formada, a complexidade $C(\phi)$ é dada adicionando-se a seguinte condição à Definição~\ref{def:complex}:\label{def:complex_lfi1}
        \begin{align*}
            & \text{Se } \phi = \circ \psi \text{, então } C(\phi) = C(\psi) + 2.\tag*\qed{}
        \end{align*}
        
    \end{definicao}

    Note que a complexidade de uma fórmula do tipo $\circ \alpha$ é estritamente maior que a complexidade de $\alpha$ e $\neg \alpha$. Isto se dá pois, como será evidenciado pela semântica de valorações na Definição~\ref{def:valoracoes}, existe uma dependência de $\circ \alpha$ em $\{\alpha, \neg \alpha\}$, como apresentada por~\citeshort{Carnielli_Coniglio_2016}.

\section{Axiomatização}\label{sec:axiomatizacao}

    A teoria das provas é uma das abordagens para o estudo das relações de consequência, onde a validade de uma inferência é atestada caso haja uma \textit{prova} das conclusões a partir das premissas. Uma prova consiste em uma sequência de passos bem definidos aplicados sobre conjuntos (ocasionalmente unitários) de fórmulas, com base nos princípios de um determinado sistema de provas. A teoria das provas é sintática\footnote{Vale notar que a separação \textit{prova {--} sintaxe {--} semântica {--} modelo} não é tão bem definida, algo que é explorado em~\citeshort{Prawitz2005-PRALCA-2}.} por natureza, ou seja, numa inferência $A \vdash B$, é relevante apenas a estrutura das fórmulas presentes em \textit{A} e \textit{B}, não sua interpretação ou valor-verdade. Essa estrutura é manipulada a fim de obter-se uma sequência de passos que {--} além de atestar sua validade {--} serve como argumento para tal~\cite{sep-logical-consequence}. Desta forma, pode-se definir um sistema de provas sintático para servir como relação de consequência para uma determinada lógica. 

    No contexto da \lfium{}, existem dois sistemas de prova sintáticos estabelecidos até o momento: um cálculo de Hilbert, descrito \migscortar{por}{ pelos autores}~\citeshort{carnielli2000formal,Carnielli_Coniglio_2016} e um sistema de \textit{Tableau}, descrito por~\citeshort{tableaulfi}. No presente trabalho, foi escolhido o cálculo de Hilbert para definir a sintaxe da \lfium{}, dada a maior facilidade para desenvolver metateoremas em contraste ao sistema de \textit{Tableau}.

    O cálculo de Hilbert (também conhecido como sistema de Hilbert ou axiomatização de Hilbert) é um sistema composto por um conjunto de fórmulas, chamadas de \textit{axiomas} e um conjunto de \textit{regras de inferência}. Uma regra de inferência é formada por uma lista de fórmulas chamadas de premissas da regra e uma fórmula chamada de conclusão da regra~\cite{Restall1999-RESAIT-4}. Uma prova (também chamada de derivação ou dedução) do tipo $\Gamma \vdash \phi$ consiste em uma sequência finita de fórmulas \(\psi_0, \dots, \psi_n\), onde \(\psi_n = \phi\), e cada  $\psi_i\ (0 \leq i \leq n)$ é um axioma, um elemento do conjunto de premissas $\Gamma$ ou o resultado da aplicação de uma regra de inferência em fórmulas anteriores. Usualmente, o cálculo de Hilbert possui apenas uma regra de inferência, esta sendo o \textit{modus ponens}. Este também é o caso para o cálculo de Hilbert que será utilizado para a \lfium{}.
    
    O cálculo de Hilbert definido a seguir foi apresentado por~\citeshort{Carnielli_Coniglio_2016} como alternativa ao que havia sido definido em trabalhos anteriores~\cite{carnielli2000formal,carnielli2007}. Segundo os autores, esta definição evidencia algumas propriedades interessantes da negação $\neg$ como as leis de De Morgan.


    \begin{definicao}[Cálculo de Hilbert para \lfium{}]\label{def:hilbert_lfi1}
        A lógica \lfium{} é definida a partir da relação de consequência sintática $\conhil{}$ sobre a linguagem $\ling{}$ através do seguinte cálculo de Hilbert:

        \noindent\textbf{Axiomas}:
        \begin{align*}
            & \alpha \to (\beta \to \alpha)                                                     \tag{\textbf{Ax1}}            \label{ax:ax1}\\
            & (\alpha \to (\beta \to \gamma)) \to ((\alpha \to \beta) \to (\alpha \to \gamma )) \tag{\textbf{Ax2}}            \label{ax:ax2}\\
            & \alpha \to (\beta \to (\alpha \land \beta))                                       \tag{\textbf{Ax3}}            \label{ax:ax3}\\
            & (\alpha \land \beta) \to \alpha                                                   \tag{\textbf{Ax4}}            \label{ax:ax4}\\
            & (\alpha \land \beta) \to \beta                                                    \tag{\textbf{Ax5}}            \label{ax:ax5}\\
            & \alpha \to (\alpha \lor \beta)                                                    \tag{\textbf{Ax6}}            \label{ax:ax6}\\
            & \beta \to (\alpha \lor \beta)                                                     \tag{\textbf{Ax7}}            \label{ax:ax7}\\
            & (\alpha \to \gamma) \to ((\beta \to \gamma) \to ((\alpha \lor \beta) \to \gamma)) \tag{\textbf{Ax8}}            \label{ax:ax8}\\
            & (\alpha \to \beta) \lor \alpha                                                    \tag{\textbf{Ax9}}            \label{ax:ax9}\\
            & \alpha \lor \neg \alpha                                                           \tag{\textbf{Ax10}}           \label{ax:ax10}\\
            & \circ \alpha \to (\alpha \to (\neg \alpha \to \beta))                             \tag{\textbf{bc1}}            \label{ax:axbc1}\\
            & \neg \neg \alpha \to \alpha                                                       \tag{\textbf{cf}}             \label{ax:axcf}\\
            & \alpha \to \neg \neg \alpha                                                       \tag{\textbf{ce}}             \label{ax:axce}\\
            & \neg \circ \alpha \to (\alpha \land \neg \alpha)                                  \tag{\textbf{ci}}             \label{ax:axci}\\
            & \neg (\alpha \lor \beta) \to (\neg \alpha \land \neg \beta)                       \tag{\textbf{neg}$\lor_{1}$}  \label{ax:axneglor1}\\
            & (\neg \alpha \land \neg \beta) \to \neg (\alpha \lor \beta)                       \tag{\textbf{neg}$\lor_{2}$}  \label{ax:axneglor2}\\
            & \neg(\alpha \land \beta) \to (\neg \alpha \lor \neg \beta)                        \tag{\textbf{neg}$\land_{1}$} \label{ax:axnegland1}\\
            & (\neg \alpha \lor \neg \beta) \to \neg (\alpha \land \beta)                       \tag{\textbf{neg}$\land_{2}$} \label{ax:axnegland2}\\
            & \neg (\alpha \to \beta) \to(\alpha \land \neg \beta)                              \tag{\textbf{neg}$\to_{1}$}   \label{ax:axnegto1}\\
            & (\alpha \land \neg \beta) \to \neg(\alpha \to \beta)                              \tag{\textbf{neg}$\to_{2}$}   \label{ax:axnegto2}\\
    \end{align*}
        \\
        \noindent\textbf{Regra de inferência:}
        \begin{prooftree}
            \AxiomC{$\alpha, \alpha \to \beta$}
            \RightLabel{MP}
            \UnaryInfC{$\beta$}
        \end{prooftree}
        Desta forma, a lógica \lfium{} é definida como $\lfium{} = \langle \ling, \conhil \rangle.$\qed{}  
    \end{definicao}

    Os axiomas (\textbf{Ax1}) {--} (\textbf{Ax10}) e a regra de inferência \textbf{MP} (\textit{modus ponens}) são importados da lógica proposicional clássica. O axioma (\textbf{bc1}) é chamado de \textit{princípio da explosão gentil}. Os axiomas (\textbf{neg}$\lor_{1}$) {--} (\textbf{neg}$\to_{2}$) expressam as leis de De Morgan em relação a negação paraconsistente $\neg$. Os axiomas (\textbf{ce}) e (\textbf{cf}) expressam, respectivamente, a introdução e a eliminação da dupla negação. O axioma (\textbf{ci}) representa a inconsistência de uma informação.

    Com isso, uma derivação em \lfium{} pode ser definida:
    
    \begin{definicao}[Derivação em \lfium{}]
        Seja $\Gamma \cup \{\phi\} \subseteq \ling{}$ um conjunto de fórmulas, uma derivação de $\phi$ a partir de $\Gamma$ em \lfium{}, denotada como $\Gamma \conhil \phi$, é uma sequência finita de fórmulas \(\phi_0, \dots, \phi_n\) onde, para cada $1 \leq i \leq n$, alguma das seguintes condições é satisfeita:
        \begin{align*}
              \text{(i) } & \phi_{i} \text{ é um axioma;}\\
              \text{(ii) } & \phi_{i} \in \Gamma;\\
              \text{(iii) } & \text{existem } j,k < i \text{, de modo que } \phi_{i} \text{ é o resultado da aplicação de MP em } \phi_{j} \text{ e } \phi_{k}.\tag*\qed{}
        \end{align*}
    \end{definicao}

    Para ilustrar, provaremos um exemplo de derivação no cálculo de Hilbert apresentado:
    
    \begin{exemplo}\label{ex:1}
        A derivação {\normalfont{} $\circ \psi, \alpha \to (\psi \land \neg \psi), \neg \alpha \to (\psi \land \neg \psi) \conhil \phi$} é válida.
    \end{exemplo}

    \begin{proof}[Prova do Exemplo~\ref{ex:1}]
        A seguinte derivação completa a prova:
        \begin{align*}
            1.~& \circ \psi \tag{Premissa} \\
            2.~& \alpha \to (\psi \land \neg \psi) \tag{Premissa} \\
            3.~& \neg \alpha \to (\psi \land \neg \psi) \tag{Premissa} \\
            4.~& \alpha \lor \neg \alpha \tag{Ax10} \\
            5.~& (\alpha \to (\psi \land \neg \psi)) \to ((\neg \alpha \to (\psi \land \neg \psi)) \to ((\alpha \lor \neg \alpha) \to (\psi \land \neg \psi))) \tag{Ax8} \\
            6.~& (\neg \alpha \to (\psi \land \neg \psi)) \to ((\alpha \lor \neg \alpha) \to (\psi \land \neg \psi)) \tag{MP 2, 5}\\
            7.~& (\alpha \lor \neg \alpha) \to (\psi \land \neg \psi) \tag{MP 3, 6}\\
            8.~& \psi \land \neg \psi \tag{MP 4, 7} \\
            9.~& (\psi \land \neg \psi) \to \psi \tag{Ax4} \\
            10.~& (\psi \land \neg \psi) \to \neg \psi \tag{Ax5} \\
            11.~& \psi \tag{MP 8, 9}\\
            12.~& \neg \psi \tag{MP 8, 10}\\
            13.~& \circ \psi \to (\psi \to (\neg \psi \to \phi)) \tag{bc1}\\
            14.~& \psi \to (\neg \psi \to \phi) \tag{MP 1, 13}\\
            15.~& \neg \psi \to \phi \tag{MP 11, 14} \\
            16.~& \phi \tag{MP, 12, 15}
        \end{align*}
    \end{proof}


\section{Semântica}\label{sec:semantica}
    De forma geral, a semântica é o estudo de como um sistema de símbolos (uma linguagem) internaliza informações, ou seja, é o estudo de como interpretar os símbolos de uma linguagem~\cite{brown2005encyclopedia}. Num sistema lógico, \textit{matrizes lógicas} são comumente usadas para estabelecer o comportamento esperado dos conectivos lógicos de sua assinatura. Outra forma de compreender a semântica de um sistema lógico é definir condições que caracterizam funções conhecidas como valorações, que define uma \textit{semântica de valorações}. Nesta seção, a semântica da \lfium{} será definida de duas formas distintas: a partir de uma \textit{matriz lógica} e a partir de uma \textit{semântica de valorações}. As definições e notações para os conceitos de \textit{álgebra} definidos aqui baseiam-se nos trabalhos de~\citeshort{Carnielli_Coniglio_2016},~\citeshort{Sikorski1966-SIKAOF} e~\citeshort{Rasiowa1963-RASTMO}.
    
    \subsection{Semântica Matricial e Bivalorações}
        Uma das formas de definir a semântica de uma lógica proposicional é definir uma \textit{matriz lógica} (também chamada de tabela-verdade) para os conectivos de sua assinatura proposicional. Para isso, é necessário definir o conceito de \textit{álgebra} para assinaturas proposicionais:

        \begin{definicao}[Álgebra para assinaturas proposicionais]\label{def:algebra}
            Uma álgebra para uma assinatura proposicional $\Theta$ é uma dupla $\pazocal{A} = \langle A, O \rangle$, onde $A$ é um conjunto não vazio (chamado de \textit{domínio} da álgebra) e $O$ é uma função de interpretação que associa cada conectivo n-ário $c \in \Theta$ à uma operação $c^{\pazocal{A}}\!\! : \; A^{n} \to A$ em $A$.\qed{}
        \end{definicao}

        Quando não for confuso, o mesmo símbolo será utilizado para representar um conectivo $c$ e sua interpretação $O(c) = c^{\pazocal{A}}$. Além disso o símbolo utilizado para se referir a uma álgebra $\langle A, O \rangle$ será simplesmente o símbolo para seu domínio $A$. Ademais, caso $\Theta$ seja finita, a função $O$ será substituída pela lista de conectivos de $\Theta$. Por exemplo, uma álgebra para a assinatura $\Sigma$ da \lfium{} é escrita como $\pazocal{A} = \langle A,\land, \lor, \to, \neg, \circ \rangle$.

        \begin{observacao}
            Uma linguagem definida sobre uma assinatura proposicional $\Theta = \{c_{1}, \ldots, c_{n}\}$ e um conjunto enumerável de átomos $\pazocal{P} = \{p_{n} \; | \; n \in \mathbb{N} \}$ pode ser compreendida como uma álgebra da forma $\langle \pazocal{L}_{\Theta}, c_{1}\ldots, c_{n} \rangle$, onde $\pazocal{L}_{\Theta}$ é um conjunto de fórmulas bem formadas a partir dos conectivos em $\Theta$ e dos átomos em $\pazocal{P}$. Denotamos esta álgebra simplesmente por $\pazocal{L}_{\Theta}$~\cite{Sikorski1966-SIKAOF,Wojcicki1984-WJCLOP}.
        \end{observacao}

        Duas álgebras $\langle A, o_1, \ldots, o_n \rangle$ e $\langle B, o'_1, \ldots, o'_n \rangle$ definidas sobre uma mesma assinatura proposicional são ditas \textit{similares} caso, para todo $1 \leq j \leq n$, as operações $o_j$ e $o'_j$ tenham mesma aridade. Uma função (mapeamento) entre duas estruturas algébricas similares que preserva sua estrutura é chamada de \textit{homomorfismo}. Ou seja, dadas duas álgebras similares $\langle A, O \rangle$, $\langle B, O' \rangle$ definidas sobre uma assinatura proposicional $\Theta$, um mapeamento $h : A \to B$ será um \textit{homomorfismo} caso, para todo conectivo $c \in \Theta$ de aridade $n$, e para todo $a_{0},\ldots,a_{n} \in A$, tem-se $h(c(a_{0},\ldots, a_{n})) = c(h(a_{0}),\ldots, h(a_{n}))$.

        \begin{definicao}[Matriz Lógica]
            Seja $\Theta$ uma assinatura proposicional. Uma \textit{matriz lógica} $\pazocal{M}$ definida sobre $\Theta$ é uma tripla $\pazocal{M} = \langle A, D, O \rangle$, tal que o par $\langle A, O \rangle$ é uma álgebra para $\Theta$ e $D$ é um subconjunto de $A$ cujos elementos são ditos \textit{designados} (estes são elementos de $A$ considerados como verdadeiros).\qed{}
        \end{definicao}

        Com isso, uma matriz lógica $\pazocal{M} = \langle A, D, O \rangle$ induz uma relação de consequência semântica para uma lógica tarskiana $\mathcal{L}$, definida sobre uma linguagem $\pazocal{L}_{\Theta}$, da seguinte forma: sendo $\Gamma \cup \{\alpha\} \in \pazocal{L}_{\Theta}$, tem-se $\Gamma \vDash_{\pazocal{M}} \alpha$ sse, para todo homomorfismo $h : \pazocal{L}_{\Theta} \to A$, se $h[\Gamma] \subseteq D$, então $h(\alpha) \in D$. Em particular, $\alpha$ é uma tautologia em $\mathcal{L}$ sse $h(\alpha) \in D$ para todo homomorfismo $h$. Perceba que o homomorfismo $h$ nada mais é do que uma função que associa fórmulas da linguagem $\pazocal{L}_{\Theta}$ a valores do domínio $A$ da matriz $\pazocal{M}$. Desta forma, $h$ é dito uma \textit{valoração} sobre $\pazocal{M}$.


       Uma lógica que apresenta três elementos no domínio de sua matriz lógica é dita \textit{trivalorada}. Algumas lógicas trivaloradas introduzem um terceiro valor além da verdade e falsidade para representar uma informação desconhecida, como é o caso da lógica de Kleene~\cite{manyvalued}. A \lfium{}, por sua vez, introduz o valor $\meio{}$ além dos valores clássicos $0$ e $1$ para representar uma informação inconsistente. Ou seja, caso $\alpha$ tenha o valor $\meio{}$, então $\neg \alpha$ também terá o valor $\meio{}$.

        \begin{definicao}[Matriz lógica da \lfium{}]
            A matriz lógica $\pazocal{M}_{\lfium{}} = \langle M, D, O \rangle$ com domínio $M = \{1, \meio{}, 0\}$ e um conjunto de valores designados $D = \{1, \meio{}\}$ é definida da seguinte forma:
            
            \setbool{@fleqn}{false}

            \vspace{\baselineskip} % MIGS: Adicionei esse espaçamento vertical adicional aqui pois o espaçamento no texto estava estranho e, eu acho, errado

            \noindent
            \begin{minipage}{0.3\textwidth}
                % Implication (→)
                \[
                    \begin{array}{c|ccc}
                        \to & 1 & \meio{} & 0 \\
                        \hline
                        1           & 1 & \meio{} & 0 \\
                        \meio{} & 1 & \meio{} & 0 \\
                        0           & 1 & 1           & 1 \\
                    \end{array}
                \]
            \end{minipage}
            \begin{minipage}{0.3\textwidth}
                % Conjunction (∧)
                \[
                    \begin{array}{c|ccc}
                        \land       & 1           & \meio{} & 0 \\
                        \hline
                        1           & 1           & \meio{} & 0 \\
                        \meio{} & \meio{} & \meio{} & 0 \\
                        0           & 0           & 0           & 0 \\
                    \end{array}
                \]
            \end{minipage}
            \begin{minipage}{0.3\textwidth}
                % Disjunction (∨)
                \[
                    \begin{array}{c|ccc}
                        \lor        & 1 & \meio{} & 0           \\
                        \hline
                        1           & 1 & 1           & 1           \\
                        \meio{} & 1 & \meio{} & \meio{} \\
                        0           & 1 & \meio{} & 0           \\
                    \end{array}
                \]
            \end{minipage}

            \vspace{0.5cm}

            \begin{minipage}{0.4\textwidth}
                % Negation (¬)
                \[
                    \begin{array}{c|c}
                                    & \neg        \\
                        \hline
                        1           & 0           \\
                        \meio{} & \meio{} \\
                        0           & 1           \\
                    \end{array}
                \]
            \end{minipage}
            \begin{minipage}{0.3\textwidth}
                \[
                    \begin{array}{c|c}
                                    & \circ   \\
                        \hline
                        1           & 1         \\
                        \meio{} & 0         \\
                        0           & 1         \\
                    \end{array}
                \]
            \end{minipage}

            \vspace{\baselineskip}
            \qed{}
        \end{definicao}

        A partir disso, definimos a relação de consequência semântica $\conmat$ da seguinte forma:

        \begin{definicao}[Relação de consequência semântica $\conmat$]
            Dado um conjunto de fórmulas $\Gamma \cup \{\phi\} \subseteq \ling{}$, tem-se $\Gamma \conmat \phi$ sse, para toda valoração $h : \ling{} \to M$ de $\pazocal{M}_{\lfium{}}$, se $h[\Gamma] \subseteq \{1, \meio{}\}$ então $h(\phi) \in \{1, \meio{}\}$.\qed{}
        \end{definicao}
        Para ilustrar, provaremos um exemplo de inferência na semântica matricial apresentada:

        \begin{exemplo}\label{ex:2}
            A inferência $ \conmat \circ \circ \alpha$ é válida.
        \end{exemplo}

        \begin{proof}[Prova do Exemplo~\ref{ex:2}]
            Vamos mostrar que, para toda valoração $h : \ling{} \to \{1,\meio{},0\}$ de $\pazocal{M}_{\lfium{}}$, $h(\circ \circ \alpha) \in \{1, \meio{}\}$. Para isso, construiremos uma tabela com todas as valorações possíveis para $\circ \circ \alpha$:
            \begin{center}
                \setbool{@fleqn}{false}
                \[
                    \begin{array}{|c|c|c|}
                        \hline
                        \alpha      & \circ \alpha & \circ \circ \alpha   \\
                        \hline
                        1           & 1            &    1\\
                        \meio{    } & 0            &    1\\
                        0           & 1            &    1\\
                        \hline
                    \end{array}
                \]
            \end{center}

            Como é possível observar, a coluna da fórmula $\circ \circ \alpha$ é composta somente por elementos pertencentes ao conjunto $\{1, \meio{}\}$, portanto, para toda valoração $h$ de $\pazocal{M}_{\lfium{}}$, $h(\circ \circ \alpha) \in \{1, \meio{}\}$. Logo $ \conmat \circ \circ \alpha$.
            
        \end{proof}

        
        Outra forma de definir a relação de consequência semântica de uma lógica é descrever uma semântica de valorações, esta que define um conjunto de cláusulas condicionais sobre funções~\cite{DaCosta1977-NEWASA-3}. Quando uma função respeita estas cláusulas, ela é chamada de valoração para a lógica.

        \begin{definicao} [Semântica de valorações para \textbf{$\text{LFI1}$}]\label{def:valoracoes}
            Uma função $v : \ling{} \to \{1, 0\}$ é uma valoração para a lógica \lfium{} caso ela satisfaça as seguintes cláusulas:
            \begin{align*}
                & v(\alpha \land \beta) = 1 \Longleftrightarrow v(\alpha) = 1 \text{ e } v(\beta) = 1\tag{\textbf{$vAnd$}}\\
                & v(\alpha \lor \beta) = 1 \Longleftrightarrow v(\alpha) = 1 \text{ ou } v(\beta) = 1\tag{\textbf{$vOr$}}\\
                & v(\alpha \to \beta) = 1 \Longleftrightarrow v(\alpha) = 0 \text{ ou } v(\beta) = 1\tag{\textbf{$vImp$}}\\
                & v(\neg \alpha) = 0 \Longrightarrow v(\alpha) = 1\tag{\textbf{$vNeg$}}\\
                & v(\circ \alpha) = 1 \Longrightarrow v(\alpha) = 0 \text{ ou } v(\neg \alpha) = 0\tag{\textbf{$vCon$}}\\
                & v(\neg \circ \alpha) = 1 \Longrightarrow v(\alpha) = 1 \text{ e } v(\neg \alpha) = 1\tag{\textbf{$vCi$}}\\
                & v(\neg \neg \alpha) = 1 \Longleftrightarrow v(\alpha) = 1\tag{\textbf{$vDNE$}}\\
                & v(\neg (\alpha \land \beta)) = 1 \Longleftrightarrow v(\neg \alpha) = 1 \text{ ou } v(\neg \beta) = 1\tag{\textbf{$vDM_{\land}$}}\\
                & v(\neg (\alpha \lor \beta)) = 1 \Longleftrightarrow v(\neg \alpha) = 1 \text{ e } v(\neg \beta) = 1\tag{\textbf{$vDM_{\lor}$}}\\
                & v(\neg (\alpha \to \beta)) = 1 \Longleftrightarrow v(\alpha) = 1 \text{ e } v(\neg \beta) = 1\tag{\textbf{$vCip_{\to}$}}
            \end{align*}
            O conjunto de todas as valorações para a lógica \lfium{} será denotado por $V^{\lfium{}}$.\qed{}
        \end{definicao}

        Com isso, definimos a relação de consequência semântica $\conval$ da seguinte forma:

        \begin{definicao}[Relação de consequência semântica $\conval$]
            Dado um conjunto de fórmulas $\Gamma \cup \{\phi\} \subseteq \ling{}$, tem-se $\Gamma \conval \phi$ sse, para todo $v \in V^{\lfium{}}$, se $v[\Gamma] \subseteq \{1\}$ então $v(\phi) = 1$.\qed{}
        \end{definicao}


        Note que, por conta de $(vNeg)$ e $(vCon)$, o valor $v(\triangle \alpha)$ {--} para $\triangle \in \{\neg, \circ\}$ {--} não é determinado pelo valor $v(\alpha)$ da subfórmula $\alpha$. Ou seja, os conectivos $\neg$ e $\circ$ apresentam um comportamento não determinístico em relação a esta semântica de valorações. Por exemplo, caso $v(\alpha)$ seja $1$, pode-se ter $v(\neg \alpha)$ tanto $0$ como $1$ (mas não ambos). Isto pode ser observado na seguinte tabela de possíveis valorações para $\neg \phi$ e $\circ \phi$, dada uma fórmula $\phi$:

        \begin{table}[h]
            \setbool{@fleqn}{false}
            \[
                \begin{array}{|c|c|c|c}
                    \cline{1-3}
                    \phi & \neg\phi & \circ\phi & \\ \hline
                    0 & 1 & 1 & v_1 \\ \hline
                    \multirow{2}{*}{1} & 0 & 1 & v_2 \\ \cline{2-4}
                    & 1 & 0 & v_3 \\ \hline
                \end{array}
                \]
                \caption{Valorações possíveis para $\phi$, $\neg \phi$ e $\circ \phi$, considerando $(vNeg)$, $(vCon)$ e $(vCi)$.}
                \label{tab:negcirc}
        \end{table}

        Uma semântica de valorações que define funções que mapeiam para conjuntos com somente dois elementos é chamada de \textit{bivaloração}. Este tipo de semântica possui características interessantes que a distinguem da semântica matricial. A semântica de matrizes nos permite mostrar a validade de uma inferência com facilidade, contudo, alguns lógicos e filósofos demonstram descontentamento com a existência de múltiplos valores-verdade designados, argumentando que uma distinção deve ser feita entre ``valorações algébricas'' {--} correspondente às matrizes lógicas {--} e sua ``definição genuína'', em termos de bivalorações~\cite{Suszko1975-SUSROL}. Além disso, as bivalorações servem um papel importante na teoria dos modelos, como na comparação de sistemas lógicos e na associação de outros sistemas semânticos a uma lógica já estabelecida~\cite{bivalence}.

        
                
\section{Metateoremas}\label{sec:metateoremas}
    A metalógica é uma área da matemática que estuda sistemas lógicos, desenvolvendo metateoremas~\cite{Jacquette2002-JACACT-7}. Um metateorema é uma prova sobre propriedades de um sistema formal, sobretudo sobre suas relações de consequência, utilizando uma metalinguagem~\cite{Tarski1956-TARLSM, Rasiowa1963-RASTMO, Barile_2024}. Estas propriedades são chamadas de \textit{metapropriedades}. 
    
    Algumas metapropriedades importantes de serem provadas sobre um sistema lógico munido com uma relação de consequência sintática $\vdash$ e uma relação de consequência semântica $\vDash$ (como é o caso da \lfium{}) são a correção, que afirma que tudo que é derivável em $\vdash$ é válido em $\vDash$, e a completude, que afirma que tudo que é válido em $\vDash$ pode ser provado em $\vdash$. Outra metapropriedade relevante é a equivalência entre a semântica matricial e as bivalorações da \lfium{}, apresentados na Seção~\ref{sec:semantica}, já que ambos sistemas possuem suas particularidades e vantagens. Ademais, o metateorema da dedução (definido como $\Gamma, \alpha \vdash \beta \Longleftrightarrow \Gamma \vdash \alpha \to \beta$) é uma propriedade interessante que possibilita a obtenção de outros resultados pertinentes, como será evidenciado ao longo desta seção. Por fim, caracterizaremos a lógica \lfium{} como sendo uma lógica tarskiana, finitária e \lfi{} forte.
    
    Os lemas e teoremas desenvolvidos nesta seção seguem um caminho baseado no que foi apresentado por~\citeshort{Carnielli_Coniglio_2016} e por~\citeshort{DaCosta1977-NEWASA-3}, com modificações para se adequarem ao presente trabalho.


    \begin{lema}\label{lem:matval}
        Seja $v$ uma valoração em $V^{\lfium}$, então existe uma valoração $h$ sobre $\pazocal{M}_{\lfium}$ tal que, para todo $\phi \in \ling{}$, $v(\phi) = 1$ sse $h(\phi) \in \{1, \meio{}\}$.
    \end{lema}

    \begin{proof}[Prova do Lema~\ref{lem:matval}]
        Seja $v$ uma valoração em $V^{\lfium}$. Então, considere a valoração $h$ sobre $\pazocal{M}_{\lfium}$ tal que para toda variável proposicional $p$:
        \begin{center}
            \setbool{@fleqn}{false}
            \begin{equation*}
                h(p) =
                \begin{cases}
                  1\ \, \text{ sse } v(p) = 1 \text{ e } v(\neg p) = 0 \\
                  \meio{} \text{ sse } v(p) = 1 \text{ e } v(\neg p) = 1 \\
                  0\ \ \text{ sse } v(p) = 0
                \end{cases}
              \end{equation*}
        \end{center}

        Então, por indução na complexidade de uma fórmula $\phi \in \ling{}$, será provado o seguinte:
        \begin{center}
            \setbool{@fleqn}{false}
            \begin{equation*}
                h(\phi) =
                \begin{cases}
                  1\ \,  \text{ sse } v(\phi) = 1 \text{ e } v(\neg \phi) = 0 \\
                  \meio{} \text{ sse } v(\phi) = 1 \text{ e } v(\neg \phi) = 1 \\
                  0\ \  \text{ sse } v(\phi) = 0
                \end{cases}
              \end{equation*}
        \end{center}

        \noindent \textbf{\textsc{Base.}} $C(\phi) = 1$.

        Como $C(\phi) = 1$, temos $\phi \in \pazocal{P}$ pela Definição~\ref{def:complex_lfi1}. Com isso, a própria definição de $h$ prova o caso \textbf{\textsc{Base}}.

        
        \noindent \textbf{\textsc{Passo.}} 

        \noindent \textbf{\textsc{Hipótese de indução (HI):}} A propriedade vale para toda fórmula $\phi$ tal que $C(\phi) = k$, com $k < n$.

        Vamos mostrar que a propriedade vale caso $C(\phi) = n$.

        \begin{provaporcasos}
            \casodeprova{} $\phi = \neg \alpha$.
                
                \begin{provaporsubcasos}
                    \subcasodeprova{$h(\phi) = 1$ sse $v(\phi) = 1$ e $v(\neg \phi) = 0$.}
                        
                        ($\Longrightarrow$) Supondo $h(\phi) = h(\neg \alpha) = \neg h(\alpha) = 1$, vamos provar $v(\phi) = 1$ e $v(\neg \phi) = 0$. 
                        
                        Pela matriz de $\neg$ temos $h(\alpha) = 0$. Então, por (HI), $v(\alpha) = 0$. 
                        
                        Por $(vNeg)$ temos $v(\neg \alpha) = v(\phi) = 1$. Enfim, por $(vDNE)$, $v(\neg \phi) = v(\neg \neg \alpha) = v(\alpha) = 0$
                        
                        
                        ($\Longleftarrow$) Supondo $v(\phi) = v(\neg \alpha) = 1$ e $v(\neg \phi) = v(\neg \neg \alpha) = 0$, vamos provar $h(\phi) = h(\neg \alpha) = 1$.
                        
                        Por $(vDNE)$ temos $v(\neg \neg \alpha) = v(\alpha) = 0$.
                        
                        Por (HI) temos $h(\alpha) = 0$. Então, $h(\phi) = h(\neg \alpha) = \neg h(\alpha) = 1$, pela matriz de $\neg$.
                    
                    \subcasodeprova{$h(\phi) = \meio{}$ sse $v(\phi) = 1$ e $v(\neg \phi) = 1$.}
                    
                        ($\Longrightarrow$) Supondo $h(\phi) = h(\neg \alpha) = \meio{}$, vamos provar $v(\phi) = 1$ e $v(\neg \phi) = 1$.
                        
                        Pela matriz de $\neg$ temos $h(\alpha) = \meio{}$. Então, por (HI), $v(\alpha) = 1$ e $v(\neg \alpha) = v(\phi) = 1$.
                        
                        Enfim, por $(vDNE)$ temos $v(\neg \phi) = v(\neg \neg \alpha) = v(\alpha) = 1$.
                        
                        ($\Longleftarrow$) Supondo $v(\phi) = v(\neg \alpha) = 1$ e $v(\neg \phi) = v(\neg \neg \alpha)= 1$, vamos provar $h(\phi) = \meio{}$.

                        Por $(vDNE)$ temos $v(\alpha) = v(\neg \neg \alpha) = 1$.
                        
                        Por (HI), temos $h(\alpha) = \meio{}$. Então, pela matriz de $\neg$, temos $h(\phi) = h(\neg \alpha) = \neg h(\alpha) = \meio{}$.
                    
                    \subcasodeprova{$h(\phi) = 0$ sse $v(\phi) = 0$.}
                        
                        ($\Longrightarrow$) Supondo $h(\phi) = h(\neg \alpha) = \neg h(\alpha) = 0$, vamos provar $v(\phi) = v(\neg \alpha) = 0$.
                        
                        Pela matriz de $\neg$ temos $h(\alpha) = 1$. Então, por (HI), $v(\alpha) = 1$ e $v(\phi) = v(\neg \alpha) = 0$.
                    
                        ($\Longleftarrow$) Supondo $v(\phi) = v(\neg \alpha) = 0$, vamos provar $h(\phi) = h(\neg \alpha) = 0$.
                    
                        Por $(vNeg)$ temos $v(\alpha) = 1$. Logo, por (HI), temos $h(\alpha) = 1$.
                        
                        Pela matriz de $\neg$ temos $h(\phi) = h(\neg \alpha) = \neg h(\alpha) = 0$.
                \end{provaporsubcasos}
                    
            \casodeprova{} $\phi = \circ \alpha$.
                    
                \begin{provaporsubcasos}
                    
                    \subcasodeprova{$h(\phi) = 1$ sse $v(\phi) = 1$ e $v(\neg \phi) = 0$.}

                        ($\Longrightarrow$) Supondo $h(\phi) = h(\circ \alpha) = \circ h(\alpha) = 1$, vamos provar $v(\phi) = 1$ e $v(\neg \phi) = 0$.
                        
                        Pela matriz de $\circ$, temos $h(\alpha) = 1$ ou $h(\alpha) = 0$.

                        \begin{provaporsubsubcasos}
                            \subsubcasodeprova{$h(\alpha) = 1$.} 
                                
                                Por (HI) temos $v(\alpha) = 1$ e $v(\neg \alpha) = 0$.
                            
                                Por $(vCi)$ temos $v(\neg \phi) = v(\neg \circ \alpha) = 0$. Então, por $(vNeg)$, temos $v(\phi) = v(\circ \alpha) = 1$.
                            
                            \subsubcasodeprova{$h(\alpha) = 0$.}
                                
                                Por (HI) temos $v(\alpha) = 0$.
                            
                                Por $(vCi)$, temos $v(\neg \phi) = v(\neg \circ \alpha) = 0$. Então, por $(vNeg)$ temos $v(\phi) = 1$.

                                \newcounter{buffer}
                                \setcounter{buffer}{\theSubSubCasos}
                        \end{provaporsubsubcasos}
                        
                        ($\Longleftarrow$) Supondo $v(\phi) = v(\circ \alpha) = 1$ e $v(\neg \phi) = v(\neg \circ \alpha) = 0$, vamos provar $h(\phi) = h(\circ \alpha) = 1$.
                        
                        Por $(vCon)$ temos $v(\alpha) = 0$ ou $v(\neg \alpha) = 0$.

                        \begin{provaporsubsubcasos}
                            \setcounter{SubSubCasos}{\thebuffer}

                            \subsubcasodeprova{$v(\alpha) = 0$.}

                                Por (HI) temos $h(\alpha) = 0$. Então, pela matriz de $\circ$, temos $h(\phi) = h(\circ \alpha) = \circ h(\alpha) = 1$.
                            
                            \subsubcasodeprova{$v(\neg \alpha) = 0$.} 

                                Por $(vNeg)$ temos $v(\alpha) = 1$. Então, por (HI), temos $h(\alpha) = 1$. 
                                
                                Finalmente, pela matriz de $\circ$, temos $h(\phi) = h(\circ \alpha) = \circ h(\alpha) = 1$.
                        \end{provaporsubsubcasos}
                        
                    \subcasodeprova{$h(\phi) = \meio{}$ sse $v(\phi) = 1$ e $v(\neg \phi) = 1$.}
                        
                        ($\Longrightarrow$) Pela matriz lógica de $\circ$, nunca temos $h(\phi) = h(\circ \alpha) = \meio{}$.
                        
                        ($\Longleftarrow$) Supondo $v(\phi) = 1$ e $v(\neg \phi) = 1$, então, por $(vCon)$ aplicado a $v(\phi) = v(\alpha) = 1$, temos $v(\alpha) = 0$ ou $v(\neg \alpha) = 0$. 
                        
                        Entretanto, por $(vCi)$ aplicado a $v(\neg \phi) = v(\neg \alpha) = 1$, temos $v(\alpha) = 1$ e $v(\neg \alpha) = 1$.

                        Portanto, nunca temos $v(\phi) = 1$ e $v(\neg \phi) = 1$.

                        
                    
                    \subcasodeprova{$h(\phi) = 0$ sse $v(\phi) = 0$.}

                        ($\Longrightarrow$) Supondo $h(\phi) = h(\circ \alpha) = \circ h(\alpha) = 0$, vamos provar $v(\phi) = v(\circ \alpha) = 0$.

                        Pela matriz de $\circ$, temos $h(\alpha) = \meio{}$. Então, por (HI), temos $v(\alpha) = 1$ e $v(\neg \alpha) = 1$. 
                        
                        Finalmente, por $(vCon)$, temos $v(\phi) = v(\circ \alpha) = 0$.
                    
                        ($\Longleftarrow$) Supondo $v(\phi) = v(\circ \alpha) = 0$, vamos provar $h(\phi) = h(\circ \alpha) = 0$.

                        Por $(vDNE)$ temos $v(\neg \neg \circ \alpha) = 0$. Então, por $(vNeg)$, temos $v(\neg \circ \alpha) = 1$.

                        Por $(vCi)$ temos $v(\alpha) = 1$ e $v(\neg \alpha) = 1$. Logo, por (HI), temos $h(\alpha) = \meio{}$.

                        Finalmente, pela matriz de $\circ$, temos $h(\phi) = h(\circ \alpha) = \circ h(\alpha) = 0$.
                    
            \end{provaporsubcasos}
        
            \casodeprova{$\phi = \alpha \land \beta$.}

            \migs{Em alguns dos subcasos dessa prova, você escreveu equações tipo ``\(f(\beta) = x\)'' porém devido ao tamanho da linha, \(f(\beta) = \) e \textit{x} ficaram em linhas diferentes,
            seria bom arrumar isso para ficarem sempre na mesma linha pois essa separação torna a leitura mais difícil (você pode fazer isso encapsulando o bloco matemático com \textbraceleft e \textbraceright).}

            \begin{provaporsubcasos}
                \subcasodeprova{$h(\phi) = h(\alpha \land \beta) = h(\alpha) \land h(\beta) = 1$ sse $v(\phi) = v(\alpha \land \beta) = 1$ e $v(\neg \phi) = v(\neg (\alpha \land \beta)) = 0$.}

                    Teremos $h(\phi) = h(\alpha \land \beta) = h(\alpha) \land h(\beta) = 1$

                    \qquad{}sse, pela matriz de $\land$, $h(\alpha) = 1$ e $h(\beta) = 1$

                    \qquad{}sse, por (HI), $v(\alpha) = 1$, $v(\neg \alpha) = 0$, $v(\beta) = 1$ e $v(\neg \beta) = 0$

                    \qquad{}sse, por $(vAnd)$, $v(\phi) = v(\alpha \land \beta) = 1$ e, por $(vDM_{\land})$, $v(\neg \phi) = v(\neg(\alpha \land \beta)) = 0$.
            
                \subcasodeprova{$h(\phi) = h(\alpha \land \beta) = h(\alpha) \land h(\beta) = \meio{}$ sse $v(\phi) = v(\alpha \land \beta) = 1$ e $v(\neg \phi) = v(\neg (\alpha \land \beta)) = 1$.}

                    Teremos $h(\phi) = h(\alpha \land \beta) = h(\alpha) \land h(\beta) = \meio{}$

                    \qquad{}sse, pela matriz de $\land$, temos $h(\alpha), h(\beta) \in \{1, \meio{}\}$ e temos $h(\alpha) = \meio{}$ ou $h(\beta) = \meio{}$
                    
                    \qquad{}sse, por (HI), temos $v(\alpha) = 1$ e $v(\beta) = 1$ e temos $v(\alpha) = v(\neg \alpha) = 1$ ou $v(\beta) = v(\neg \beta) = 1$

                    \qquad{}sse, pela distributividade da conjunção sobre a disjunção e pela idempotência da conjunção, temos $v(\alpha) = 1$, $v(\neg \alpha) = 1$ e $v(\beta) = 1$ ou temos $v(\alpha) = 1$, $v(\neg \beta) = 1$ e $v(\beta) = 1$

                    \qquad{}sse, pela distributividade da conjunção sobre a disjunção e pela idempotência da conjunção, temos $v(\alpha) = 1$ e $v(\beta) = 1$ e temos $v(\neg \alpha) = 1$ ou $v(\neg \beta) = 1$

                    \qquad{}sse, por $(vAnd)$ e $(vDM_{\land})$, temos $v(\phi) = v(\alpha \land \beta) = 1$ e $v(\neg \phi) = v(\neg (\alpha \land \beta)) = 1$.

                \subcasodeprova{$h(\phi) = h(\alpha \land \beta) = h(\alpha) \land h(\beta) = 0$ sse $v(\phi) = v(\alpha \land \beta) = 0$.}

                    Teremos $h(\phi) = h(\alpha \land \beta) = h(\alpha) \land h(\beta) = 0$

                    \qquad{}sse, pela matriz de $\land$, $h(\alpha) = 0$ ou $h(\beta) = 0$

                    \qquad{}sse, por (HI), $v(\alpha) = 0$ ou $v(\beta) = 0$

                    \qquad{}sse, por $(vAnd)$, $v(\phi) = v(\alpha \land \beta) = 0$.

            \end{provaporsubcasos}

            \casodeprova{$\phi = \alpha \lor \beta$.}

            \begin{provaporsubcasos}
                \subcasodeprova{$h(\phi) = h(\alpha \lor \beta) = h(\alpha) \lor h(\beta) = 1$ sse $v(\phi) = v(\alpha \lor \beta) = 1$ e $v(\neg \phi) = v(\neg (\alpha \lor \beta)) = 0$.}

                    Temos $h(\phi) = h(\alpha \lor \beta) = h(\alpha) \lor h(\beta) = 1$

                    \qquad{}sse, pela matriz de $\lor$, $h(\alpha) = 1$ ou $h(\beta) = 1$

                    \qquad{}sse, por (HI), $v(\alpha) = v(\beta) = 1$ e $v(\neg \alpha) = v(\neg \beta) = 0$

                    \qquad{}sse, por $(vOr)$, $v(\phi) = v(\alpha \lor \beta) = 1$ e, por $(vDM_{\lor})$, $v(\neg \phi) = v(\neg(\alpha \lor \beta)) = 0$.

                \subcasodeprova{$h(\phi) = h(\alpha \lor \beta) = h(\alpha) \lor h(\beta) = \meio{0}$ sse $v(\phi) = v(\alpha \lor \beta) = 1$ e $v(\neg \phi) = v(\neg (\alpha \lor \beta)) = 1$.}

                    Temos $h(\phi) = h(\alpha \lor \beta) = h(\alpha) \lor h(\beta) = \meio{}$

                    \qquad{}sse, pela matriz de $\lor$, temos $h(\alpha) = h(\beta) = \meio{}$ ou temos $h(\alpha) = \meio{}$ e $h(\beta) = 0$ ou temos $h(\alpha) = 0$ e $h(\beta) = \meio{}$.

                    \qquad{}sse, por (HI), temos $v(\alpha) = 1$, $v(\neg \alpha) = 1$, $v(\beta) = 1$ e $v(\neg \beta) = 1$ ou temos $v(\alpha) = 1$, $v(\neg \alpha) = 1$ e $v(\beta) = 0$ ou temos $v(\alpha) = 0$, $v(\beta) = 1$ e $v(\neg \beta) = 1$

                    \qquad{}sse, por $(vNeg)$, temos $v(\alpha) = 1$, $v(\neg \alpha) = 1$, $v(\beta) = 1$ e $v(\neg \beta) = 1$ ou temos $v(\alpha) = 1$, $v(\neg \alpha) = 1$, $v(\beta) = 0$ e $v(\neg \beta) = 1$ ou temos $v(\alpha) = 0$, $v(\neg \alpha) = 1$, $v(\beta) = 1$, $v(\neg \beta) = 1$

                    \qquad{}sse, pela distributividade da disjunção sobre a conjunção, temos $v(\neg \alpha) = 1$ e $v(\neg \beta) = 1$ e temos $v(\alpha) = 1$ e $v(\beta) = 1$ ou $v(\alpha) = 1$ e $v(\beta) = 0$ ou $v(\alpha) = 0$ e $v(\beta) = 1$.

                    \qquad{}sse, por $(vOr)$, temos $v(\neg \alpha) = 1$ e $v(\neg \beta) = 1$ e temos $v(\alpha \lor \beta) = 1$ ou $v(\alpha \lor \beta) = 1$ ou $v(\alpha \lor \beta) = 1$
                    
                    \qquad{}sse, por $(vDM_{\lor})$ e pela idempotência da disjunção, temos $v(\phi) = v(\alpha \lor \beta) = 1$ e $v(\neg \phi) = v(\neg(\alpha \lor \beta)) = 1$.

                \subcasodeprova{$h(\phi) = h(\alpha \lor \beta) = h(\alpha) \lor h(\beta) = 0$ sse $v(\phi) = v(\alpha \lor \beta) = 0$.}

                    Temos $h(\phi) = h(\alpha \lor \beta) = h(\alpha) \lor h(\beta) = 0$

                    \qquad{}sse, pela matriz de $\lor$, temos $h(\alpha) = 0$ e $h(\beta) = 0$

                    \qquad{}sse, por (HI), temos $v(\alpha) = 0$ e $v(\beta) = 0$

                    \qquad{}sse, por $(vOr)$, temos $v(\phi) = v(\alpha \lor \beta) = 0$.

            \end{provaporsubcasos}

            \casodeprova{$\phi = \alpha \to \beta$.}

            \begin{provaporsubcasos}
                \subcasodeprova{$h(\phi) = h(\alpha \to \beta) = h(\alpha) \to h(\beta) = 1$ sse $v(\phi) = v(\alpha \to \beta) = 1$ e $v(\neg \phi) = v(\neg (\alpha \to \beta)) = 0$.}

                    Temos $h(\phi) = h(\alpha \to \beta) = h(\alpha) \to h(\beta) = 1$

                    \qquad{}sse, pela matriz de $\to$, $h(\alpha) = 0$ ou $h(\beta) = 1$

                    \qquad{}sse, por (HI), temos $v(\alpha) = 0$ ou temos $v(\beta) = 1$ e $v(\neg \beta) = 0$

                    \qquad{}sse, por $(vImp)$ e $(vCip)$, temos $v(\alpha \to \beta) = 1$ e $v(\neg (\alpha \to \beta)) = 0$ ou temos $v(\alpha \to \beta) = 1$ e $v(\neg (\alpha \to \beta)) = 0$

                    \qquad{}sse, pela idempotência da disjunção, temos $v(\phi) = v(\alpha \to \beta) = 1$ e $v(\neg \phi) = v(\neg (\alpha \to \beta)) = 0$

                \subcasodeprova{$h(\phi) = h(\alpha \to \beta) = h(\alpha) \to h(\beta) = \meio{}$ sse $v(\phi) = v(\alpha \to \beta) = 1$ e $v(\neg \phi) = v(\neg (\alpha \to \beta)) = 1$.}

                    Temos $h(\phi) = h(\alpha \to \beta) = h(\alpha) \to h(\beta) = \meio{}$

                    \qquad{}sse, pela matriz de $\to$, temos $h(\alpha) = 1$ e $h(\beta) = \meio{}$ ou temos $h(\alpha) = \meio{}$ e $h(\beta) = \meio{}$

                    \qquad{}sse, por (HI), temos $v(\alpha) = v(\beta) = v(\neg \beta) = 1$ e $v(\neg \alpha) = 0$ ou temos $v(\alpha) = v(\neg \alpha) = v(\beta) = v(\neg \beta) = 1$

                    \qquad{}sse, por $(vImp)$ e $(vCip)$, temos $v(\alpha \to \beta) = v(\neg (\alpha \to \beta)) = 1$ e $v(\alpha \to \beta) = v(\neg (\alpha \to \beta)) = 1$

                    \qquad{}sse, pela idempotência da disjunção, temos $v(\phi) = v(\alpha \to \beta) = 1$ e $v(\neg \phi) = v(\neg (\alpha \to \beta)) = 1$.

                \subcasodeprova{$h(\phi) = h(\alpha \to \beta) = h(\alpha) \to h(\beta) = 0$ sse $v(\phi) = v(\alpha \to \beta) = 0$.}

                    Temos $h(\phi) = h(\alpha \to \beta) = h(\alpha) \to h(\beta) = 0$

                    \qquad{}sse, pela matriz de $\to$, temos $h(\alpha) = 1$ e $h(\beta) = 0$ ou temos $h(\alpha) = \meio{}$ e $h(\beta) = 0$

                    \qquad{}sse, por (HI), temos $v(\alpha) = 1$ e $v(\neg \alpha) = v(\beta) = 0$ ou temos $v(\alpha) = v(\neg \alpha) = 1$ e $v(\beta) = 0$

                    \qquad{}sse, por $(vImp)$, temos $v(\alpha \to \beta) = 0$ ou $v(\alpha \to \beta) = 0$

                    \qquad{}sse, pela idempotência da disjunção, temos $v(\alpha \to \beta) = 0$.
            \end{provaporsubcasos}
        \end{provaporcasos}
    \end{proof}

    \begin{corolario}\label{cor:matval}
        Para todo conjunto de fórmulas $\Gamma \cup \{\phi\} \subseteq \ling{}$:

        \centering
        $\Gamma \conmat \phi \Longrightarrow \Gamma \conval \phi$. 
    \end{corolario}

    \begin{proof}[Prova do Corolário~\ref{cor:matval}]
        Seja $\Gamma \cup \{\phi\} \subseteq \ling{}$ um conjunto de fórmulas e seja $v \in V^{\lfium{}}$ uma valoração. Vamos supor $\Gamma \conmat \phi$.

        Pelo Lema~\ref{lem:matval}, então existe uma valoração $h$ tal que, para todo $\psi \in \ling{}$, $v(\psi) = 1$ sse $h(\psi) \subseteq \{1, \meio{}\}$. Supondo $v[\Gamma] = 1$, então temos $h[\Gamma] \subseteq \ummeio{}$. Pela nossa suposição de $\Gamma \conmat \phi$, temos $h(\phi) \in \{1, \meio{}\}$. Novamente, pelo Lema~\ref{lem:matval}, temos $v(\phi) = 1$. Portanto, segue $\Gamma \conval \phi$.

    \end{proof}

    O metateorema da dedução é uma propriedade conveniente de ser provada, já que nos fornece corolários importantes a fim de desenvolver provas de outros metateoremas.
    
    \begin{teorema}[Metateorema da dedução para $\conhil$]\label{teo:deducao}
        Para todo conjunto de fórmulas $\Gamma \cup \{\alpha, \beta \} \subseteq \ling{}$:

        \centering
        {\normalfont{} $\Gamma, \alpha \conhil \beta \Longleftrightarrow \Gamma \conhil \alpha \to \beta$.}
    \end{teorema}

    O seguinte lema torna a prova do teorema mais imediata:
    \begin{lema}\label{lem:id}
        A derivação $\Gamma \conhil \alpha \to \alpha$ é válida para todo $\Gamma \cup \{\alpha\} \subseteq \ling{}$.
    \end{lema}
    
    \begin{proof}[Prova do Lema~\ref{lem:id}]
        A seguinte sequência de derivação demonstra o lema:
        
        \begin{align*}
            & \text{1. } (\alpha \to ((\alpha \to \alpha) \to \alpha)) \to ((\alpha \to (\alpha \to \alpha)) \to (\alpha \to \alpha))\tag{Ax2}\\
            & \text{2. } \alpha \to ((\alpha \to \alpha) \to \alpha)\tag{Ax1}\\
            & \text{3. } \alpha \to (\alpha \to \alpha)\tag{Ax1}\\
            & \text{4. } (\alpha \to (\alpha \to \alpha)) \to (\alpha \to \alpha)\tag{MP 1,2}\\
            & \text{5. } \alpha \to \alpha\tag{MP 3,4}
        \end{align*}
    \end{proof}

    \begin{proof}[Prova do Teorema~\ref{teo:deducao}] A prova será dividida em duas partes:\\
        ($\Longleftarrow$) Supondo $\Gamma \conhil \alpha \to \beta$, então existe uma sequência de derivação $\phi_{1} \ldots \phi_{n}$ onde $\phi_{n} = \alpha \to \beta$. 
        
        A seguinte sequência de derivação completa a prova de $\Gamma, \alpha \conhil \beta$:
        \begin{align*}
            \text{1}&.~ \; \ldots\\
            & \vdots \; ~\ddots\\
            \text{$n$}&.~ \alpha \to \beta\tag{Suposição}\\
            \text{$n + 1$}&.~ \alpha\tag{Premissa}\\
            \text{$n + 2$}&.~ \beta\tag{MP $n, n + 1$}
        \end{align*}

        \noindent  ($\Longrightarrow$) Supondo $\Gamma, \alpha \conhil \beta$, então existe uma sequência de derivação $\phi_{1} \ldots \phi_{n}$ onde $\phi_{n} = \beta$ a partir do conjunto de premissas $\Gamma \cup \{\alpha\}$. A prova de $\Gamma \conhil \alpha \to \beta$ é feita pela indução no tamanho $n$ da sequência de derivação:\\

        \noindent \textbf{\textsc{Base.}} $n = 1$.
        A sequência contem somente uma fórmula $\phi_{1} = \beta$. Portanto, existem duas possibilidades:
        \begin{enumerate}
            \item $\phi_{1}$ é um axioma.
            \item $\phi_{1} \in \Gamma \cup \{\alpha\}$.
        \end{enumerate}

        \begin{provaporcasos}
            \casodeprova{} $\phi_{1}$ é um axioma. 
            
                A seguinte derivação mostra $\Gamma \conhil \alpha \to \phi_{1}$:
                \begin{align*}
                    & \text{1. } \phi_{1} \tag{Axioma}\\
                    & \text{2. } \phi_{1} \to (\alpha \to \phi_{1}) \tag{Ax1}\\
                    & \text{3. } \alpha \to \phi_{1} \tag{MP 1,2}
                \end{align*}

                \casodeprova{} $\phi_{1} \in \Gamma \cup \{\alpha\}$. 
                
                Existem dois casos a serem considerados:

                \begin{enumerate}
                    \item[2.1] $\phi_{1} = \alpha$
                    \item[2.2] $\phi_{1} \in \Gamma$
                \end{enumerate}

                \begin{provaporsubcasos}
                    \subcasodeprova{} $\phi_{1} = \alpha$. 
                    
                        É necessário mostrar $\Gamma \conhil \alpha \to \alpha$, o que foi provado pelo Lema~\ref{lem:id}.

                    \subcasodeprova{} $\phi_{1} \in \Gamma$.
                    
                        Então $\Gamma \conhil \alpha \to \phi_{1}$ é provado pela seguinte sequência de derivações:
                        \begin{align*}
                            & \text{1. } \phi_{1} \tag{Premissa}\\
                            & \text{2. } \phi_{1} \to (\alpha \to \phi_{1}) \tag{Ax1}\\
                            & \text{3. } \alpha \to \phi_{1} \tag{MP 1, 2}
                        \end{align*}
                \end{provaporsubcasos}
                
            \end{provaporcasos}
            
            Portanto, $\Gamma \conhil \alpha \to \phi_{1}$ segue para o caso \textbf{\textsc{Base.}}


        \noindent \textbf{\textsc{Passo.}}

        \noindent \textbf{\textsc{Hipótese de indução (HI):}} Para qualquer sequência de derivação $\Gamma, \alpha \conhil \beta$ de tamanho $i$, com $i < n$, tem-se $\Gamma \conhil \alpha \to \beta$. 

        É preciso mostrar que $\Gamma \conhil \alpha \to \beta$ segue caso a dedução $\Gamma, \alpha \conhil \beta$ seja de tamanho $n$. Então, vamos supor $\Gamma, \alpha \conhil \phi_{n}$ e mostrar $\Gamma \conhil \alpha \to \phi_{n}$.
        
        Ao analisar a obtenção de $\phi_{n}$ na sequência de derivação de $\Gamma, \alpha\conhil \phi_{n}$, existem três casos a se considerar:
        \begin{enumerate}
            \item $\phi_{n}$ é um axioma.
            \item $\phi_{n} \in \Gamma \cup \{\alpha\}$.
            \item $\phi_{n}$ é obtido por \textit{modus ponens} em duas fórmulas $\phi_{j}$ e $\phi_{k}$ com $j, k < n$.
        \end{enumerate}
        
         Os casos 1 e 2 são análogos aos casos provados na base.

         \noindent \textsc{Caso 3.} $\phi_{n}$ é obtido por \textit{modus ponens} em duas fórmulas $\phi_{j}$ e $\phi_{k}$ com $j, k < n$. 
         
         Então, $\phi_{k} = \phi_{j} \to \phi_{n}$ (ou $\phi_{j} = \phi_{k} \to \phi_{n}$, a prova para este caso é análoga). 
         
         Dada a nossa suposição de $\Gamma, \alpha \conhil \phi_{n}$, temos $\Gamma, \alpha \conhil \phi_{j}$ e $\Gamma, \alpha \conhil \phi_{j} \to \phi_{n}$ sequências de dedução anteriores à aplicação da regra do \textit{modus ponens} na linha $n$. 
         
         Então, pela (HI), temos $\Gamma \conhil \alpha \to \phi_{j}$ e $\Gamma \conhil \alpha \to (\phi_{j} \to \phi_{n})$. 
         
         A seguinte sequência de derivação mostra $\Gamma \conhil \alpha \to \phi_{n}$:
         \begin{align*}
             \text{1}&.~ \; \ldots\\
             &\vdots \; ~\ddots\\
             \text{$j$}&.~ \alpha \to \phi_{j} \tag{HI sobre $\phi_{j}$}\\
             &\vdots \; ~\ddots\\
             \text{$j + k$}&.~ \alpha \to (\phi_{j} \to \phi_{n}) \tag{HI sobre $\phi_{k}$}\\
             \text{$j + k + 1$}&.~ (\alpha \to (\phi_{j} \to \phi_{n})) \to ((\alpha \to \phi_{j}) \to (\alpha \to \phi_{n})) \tag{Ax2}\\
             \text{$j + k + 2$}&.~ (\alpha \to \phi_{j}) \to (\alpha \to \phi_{n}) \tag{MP $j + k$\text{,}$j + k + 1$}\\
              \text{$j + k + 3$}&.~ \alpha \to \phi_{n} \tag{MP $j$\text{,}$j + k + 2$}
         \end{align*}
         Portanto, temos $\Gamma \conhil \alpha \to \beta$ e a prova está finalizada.
         
    \end{proof}


    Com a prova do metateorema da dedução para o cálculo de Hilbert da \lfium{}, os seguintes corolários são imediatos:

    \begin{corolario}\label{cor:prova_por_casos}
        Para todo conjunto de fórmulas $\Gamma \cup \{\alpha, \beta, \phi\} \subseteq \ling{}$:

        \centering
        {\normalfont{}Se $\Gamma, \alpha \conhil \phi \text{ e }\Gamma, \beta \conhil \phi \text{ então } \Gamma, \alpha \lor \beta \conhil \phi$.}
    \end{corolario}

    \begin{proof}[Prova do Corolário~\ref{cor:prova_por_casos}]
        Seja $\Gamma \cup \{\alpha, \beta, \phi\} \subseteq \ling{}$ um conjunto de fórmulas qualquer.
        
        Suponha $\Gamma, \alpha \conhil \phi \text{ e }\Gamma, \beta \conhil \phi$. 
        Então, por MTD (Teorema~\ref{teo:deducao}), temos $\Gamma \conhil \alpha \to \phi \text{ e }\Gamma \conhil \beta \to \phi$.

        A seguinte sequência de derivação mostra $\Gamma, \alpha \lor \beta \conhil \phi$:
        \begin{align*}
            1. ~& \alpha \to \phi \tag{MTD aplicado à suposição} \\
            2. ~& \beta \to \phi \tag{MTD aplicado à suposição} \\
            3. ~& \alpha \lor \beta \tag{Premissa} \\
            4. ~& (\alpha \to \phi) \to ((\beta \to \phi) \to ((\alpha \lor \beta) \to \phi)) \tag{Ax8} \\
            5. ~& (\beta \to \phi) \to ((\alpha \lor \beta) \to \phi) \tag{MP 1, 4}\\
            6. ~& (\alpha \lor \beta) \to \phi \tag{MP 2, 5} \\
            7. ~& \phi \tag{MP 3, 6}
        \end{align*}
    \end{proof}


    \begin{corolario}\label{cor:caso_neg}
        Para todo conjunto de fórmulas $\Gamma \cup \{\alpha, \beta, \phi\} \subseteq \ling{}$:

        \centering
        {\normalfont{}Se $\Gamma, \alpha \conhil \phi \text{ e }\Gamma, \neg\alpha \conhil \phi \text{ então } \Gamma \conhil \phi$.}
    \end{corolario}

    \begin{proof}[Prova do Corolário~\ref{cor:caso_neg}]
        Seja $\Gamma \cup \{\alpha, \beta, \phi\} \subseteq \ling{}$ um conjunto de fórmulas qualquer.

        Suponha $\Gamma, \alpha \conhil \phi \text{ e }\Gamma, \neg\alpha \conhil \phi$. Pelo Corolário~\ref{cor:prova_por_casos}, temos $\Gamma, \alpha \lor \neg \alpha \conhil \phi$. Por MTD (Teorema~\ref{teo:deducao}), temos $\Gamma \conhil (\alpha \lor \neg \alpha) \to \phi$.

        A seguinte sequência de derivação mostra $\Gamma\conhil \phi$:
        \begin{align*}
            1. ~& (\alpha \lor \neg \alpha) \to \phi\tag{MTD aplicado à suposição} \\
            2. ~& \alpha \lor \neg \alpha \tag{Ax10} \\
            3. ~& \phi \tag{MP 1,2}
        \end{align*}
    \end{proof}

    A ideia da prova da correção para a \lfium{} é, como de costume, provar que todo axioma do cálculo de Hilbert $\conhil$ é válido na semântica matricial $\conmat$ e mostrar que a regra de inferência \textit{modus ponens} preserva sua validade. No desenvolvimento da prova, as igualdades do tipo $h(\triangle \phi) = \triangle h(\phi)$ e $h(\phi \otimes \psi) = h(\phi) \otimes h(\psi)$, onde $\triangle, \otimes$ são conectivos quaisquer, ficam implícitas.


    \begin{teorema}[Correção em relação a semântica matricial]\label{teo:correcao_mat}
        A lógica {\normalfont\lfium{}} é correta em relação a sua semântica matricial, ou seja, para todo conjunto de fórmulas $\Gamma \cup \{\alpha\} \subseteq \ling{}$:

        \centering
        {\normalfont{} $\Gamma \conhil \alpha \Longrightarrow \Gamma \conmat \alpha$.}
    \end{teorema}


    \begin{proof}[Prova do Teorema~\ref{teo:correcao_mat}]
        Seja $\Gamma \cup \{\alpha\} \subseteq{} \ling{}$ um conjunto de fórmulas.
        Supondo $\Gamma \conhil \alpha$, existe uma sequência de derivação $\phi_{1} \ldots \phi_{n}$ onde $\phi_{n} = \alpha$. A prova de $\Gamma \conmat \alpha$ é obtida por indução no tamanho $n$ da sequência de derivação:\\

        \noindent \textbf{\textsc{Base.}} $n = 1$. A sequência contém somente uma fórmula $\phi_{1} = \alpha$. Portanto, existem duas possibilidades:
        \begin{enumerate}
            \item $\phi_{1}$ é um axioma.
            \item $\phi_{1} \in \Gamma$.
        \end{enumerate}

        \begin{provaporcasos}
            
            \casodeprova{} $\phi_{1}$ é um axioma. Então vamos mostrar que para toda valoração $h : \ling{} \to M$ sobre $\mat{}$, se $h[\Gamma] \subseteq \ummeio{}$, então $h(\phi_{1}) \in \ummeio{}$. Como $\phi_{1}$ é um axioma, basta analisar todos os casos possíveis:

            \helena{usar longtable mas sera que funciona mudar isso?}
            \migs{Não vejo pq não funcionaria}
            \begin{provaporsubcasos}
                \subcasodeprova{} $\phi_{1} = \alpha \to (\beta \to \alpha)$
                    \begin{center}
                        \setbool{@fleqn}{false}
                        \[
                            \begin{array}{|c|c|c|}
                                \hline
                                \alpha      & \beta & \alpha \to (\beta \to \alpha)  \\
                                \hline
                                1            & 1            &    1\\
                                1            & \meio{}  &    1\\
                                1            & 0            &    1\\
                                \meio{}  & 1            &    \meio{}\\
                                \meio{}  & \meio{}  &    \meio{}\\
                                \meio{}  & 0            &    1\\
                                0            & 1            &    1\\
                                0            & \meio{}  &    1\\
                                0            & 0            &    1\\
                                \hline
                            \end{array}
                        \]
                    \end{center}

                \subcasodeprova{} $\phi_{1} = (\alpha \to (\beta \to \gamma)) \to ((\alpha \to \beta) \to (\alpha \to \gamma ))$.
                
                    \begin{center}
                        \setbool{@fleqn}{false}
                        \[
                            \begin{array}{|c|c|c|c|}
                                \hline
                                \alpha      & \beta & \gamma & (\alpha \to (\beta \to \gamma)) \to ((\alpha \to \beta) \to (\alpha \to \gamma)) \\
                                \hline
                                1           & 1           & 1           & 1 \\
                                1           & 1           & \meio{} & \meio{} \\
                                1           & 1           & 0           & 1 \\
                                1           & \meio{} & 1           & 1 \\
                                1           & \meio{} & \meio{} & \meio{} \\
                                1           & \meio{} & 0           & \meio{} \\
                                1           & 0           & 1           & 1 \\
                                1           & 0           & \meio{} & 1 \\
                                1           & 0           & 0           & 1 \\
                                \meio{} & 1           & 1           & 1 \\
                                \meio{} & 1           & \meio{} & \meio{} \\
                                \meio{} & 1           & 0           & \meio{} \\
                                \meio{} & \meio{} & 1           & 1 \\
                                \meio{} & \meio{} & \meio{} & \meio{} \\
                                \meio{} & \meio{} & 0           & \meio{} \\
                                \meio{} & 0           & 1           & 1 \\
                                \meio{} & 0           & \meio{} & \meio{} \\
                                \meio{} & 0           & 0           & \meio{} \\
                                0           & 1           & 1           & 1 \\
                                0           & 1           & \meio{} & 1 \\
                                0           & 1           & 0           & 1 \\
                                0           & \meio{} & 1           & 1 \\
                                0           & \meio{} & \meio{} & 1 \\
                                0           & \meio{} & 0           & 1 \\
                                0           & 0           & 1           & 1 \\
                                0           & 0           & \meio{} & 1 \\
                                0           & 0           & 0           & 1 \\
                                \hline
                            \end{array}
                        \]
                    \end{center}
                

                \subcasodeprova{} $\phi_{1} = \alpha \to (\beta \to (\alpha \land \beta))$. 

                \begin{center}
                    \setbool{@fleqn}{false}
                    \[
                        \begin{array}{|c|c|c|}
                            \hline
                            \alpha      & \beta & \alpha \to (\beta \to (\alpha \land \beta)) \\
                            \hline
                            1&1&               1\\ 
                            1&\meio{}&\meio{}\\
                            1&0&1\\
                            \meio{}&1&\meio{}\\
                            \meio{}&\meio{}&\meio{}\\
                            \meio{}&0&1\\
                            0&1&1\\
                            0&\meio{}&1\\
                            0&0&1\\
                            \hline
                        \end{array}
                    \]
                \end{center}

                   

                \subcasodeprova{} $\phi_{1} = (\alpha \land \beta) \to \alpha$. 


                \begin{center}
                    \setbool{@fleqn}{false}
                    \[
                        \begin{array}{|c|c|c|}
                            \hline
                            \alpha      & \beta & (\alpha \land \beta) \to \alpha \\
                            \hline
                            1 & 1 &                 1\\
                            1 & \meio{} &1\\
                            1 & 0 &1\\
                            \meio{} & 1 &\meio{}\\
                            \meio{} & \meio{} &\meio{}\\
                            \meio{} & 0 &1\\
                            0 & 1 &1\\
                            0 & \meio{} &1\\
                            0 & 0 &1\\
                            \hline
                        \end{array}
                    \]
                \end{center}
                

                \subcasodeprova{} $\phi_{1} = (\alpha \land \beta) \to \beta$.

                \begin{center}
                    \setbool{@fleqn}{false}
                    \[
                        \begin{array}{|c|c|c|}
                            \hline
                            \alpha      & \beta & (\alpha \land \beta) \to \beta \\
                            \hline
                            1 & 1 & 1\\
                            1 & \meio{} & \meio{}\\
                            1 & 0 & 1\\
                            \meio{} & 1 & 1\\
                            \meio{} & \meio{} & \meio{}\\
                            \meio{} & 0 & 1\\
                            0 & 1 & 1\\
                            0 & \meio{} & 1\\
                            0 & 0 & 1\\
                            \hline
                        \end{array}
                    \]
                \end{center}

                \subcasodeprova{} $\phi_{1} = \alpha \to (\alpha \lor \beta)$. 

                \begin{center}
                    \setbool{@fleqn}{false}
                    \[
                        \begin{array}{|c|c|c|}
                            \hline
                            \alpha      & \beta & \alpha \to (\alpha \lor \beta) \\
                            \hline
                            1 & 1 &1\\
                            1 & \meio{} &1\\
                            1 & 0 &1\\
                            \meio{} & 1 &1\\
                            \meio{} & \meio{} &\meio{}\\
                            \meio{} & 0 &\meio{}\\
                            0 & 1 &1\\
                            0 & \meio{} &1\\
                            0 & 0 &1\\
                            \hline
                        \end{array}
                    \]
                \end{center}
                
                \subcasodeprova{} $\phi_{1} = \beta \to (\alpha \lor \beta)$.

                \begin{center}
                    \setbool{@fleqn}{false}
                    \[
                        \begin{array}{|c|c|c|}
                            \hline
                            \alpha      & \beta & \beta \to (\alpha \lor \beta) \\
                            \hline
                            1 & 1 &1\\
                            1 & \meio{} &1\\
                            1 & 0 &1\\
                            \meio{} & 1 &1\\
                            \meio{} & \meio{} &\meio{}\\
                            \meio{} & 0 &1\\
                            0 & 1 &1\\
                            0 & \meio{} &\meio{}\\
                            0 & 0 &1\\
                            \hline
                        \end{array}
                    \]
                \end{center}

                \subcasodeprova{} $\phi_{1} = (\alpha \to \gamma) \to ((\beta \to \gamma) \to ((\alpha \lor \beta) \to \gamma))$. 

                \begin{center}
                    \setbool{@fleqn}{false}
                    \[
                        \begin{array}{|c|c|c|c|}
                            \hline
                            \alpha      & \beta & \gamma & (\alpha \to \gamma) \to ((\beta \to \gamma) \to ((\alpha \lor \beta) \to \gamma)) \\
                            \hline
                            1 & 1 & 1 &1\\
                            1 & 1 & \meio{} &\meio{}\\
                            1 & 1 & 0 &1\\
                            1 & \meio{} & 1 &1\\
                            1 & \meio{} & \meio{} &\meio{}\\
                            1 & \meio{} & 0 &1\\
                            1 & 0 & 1 &1\\
                            1 & 0 & \meio{} &\meio{}\\
                            1 & 0 & 0 &1\\
                            \meio{} & 1 & 1 &1\\
                            \meio{} & 1 & \meio{} &\meio{}\\
                            \meio{} & 1 & 0 &1\\
                            \meio{} & \meio{} & 1 &1\\
                            \meio{} & \meio{} & \meio{} &\meio{}\\
                            \meio{} & \meio{} & 0 &\meio{}\\
                            \meio{} & 0 & 1 &1\\
                            \meio{} & 0 & \meio{} &\meio{}\\
                            \meio{} & 0 & 0 &\meio{}\\
                            0 & 1 & 1 &1\\
                            0 & 1 & \meio{} &\meio{}\\
                            0 & 1 & 0 &1\\
                            0 & \meio{} & 1 &1\\
                            0 & \meio{} & \meio{} &\meio{}\\
                            0 & \meio{} & 0 &\meio{}\\
                            0 & 0 & 1 &1\\
                            0 & 0 & \meio{} &1\\
                            0 & 0 & 0 &1\\
                            \hline
                        \end{array}
                    \]
                \end{center}

                
                    
                \subcasodeprova{} $\phi_{1} = (\alpha \to \beta) \lor \alpha$. 

                \begin{center}
                    \setbool{@fleqn}{false}
                    \[
                        \begin{array}{|c|c|c|}
                            \hline
                            \alpha      & \beta & (\alpha \to \beta) \lor \alpha \\
                            \hline
                            1 & 1 & 1\\
                            1 & \meio{} & 1\\
                            1 & 0 & 1\\
                            \meio{} & 1 & 1\\
                            \meio{} & \meio{} & \meio{}\\
                            \meio{} & 0 & \meio{}\\
                            0 & 1 & 1\\
                            0 & \meio{} & 1\\
                            0 & 0 & 1\\
                            \hline
                        \end{array}
                    \]
                \end{center}
                    

                \subcasodeprova{} $\phi_{1} = \alpha \lor \neg \alpha$. 

                \begin{center}
                    \setbool{@fleqn}{false}
                    \[
                        \begin{array}{|c|c|c|}
                            \hline
                            \alpha      & \neg \alpha & \alpha \lor \neg \alpha \\
                            \hline
                            1 & 0 & 1\\
                            \meio{} & \meio{} & \meio{}\\
                            0 & 1 & 1\\
                            \hline
                        \end{array}
                    \]
                \end{center}
                
                 

                \subcasodeprova{} $\phi_{1} = \circ \alpha \to (\alpha \to (\neg \alpha \to \beta))$. 


                \begin{center}
                    \setbool{@fleqn}{false}
                    \[
                        \begin{array}{|c|c|c|c|c|}
                            \hline
                            \alpha      & \neg \alpha &\circ \alpha & \beta & \circ \alpha \to (\alpha \to (\neg \alpha \to \beta)) \\
                            \hline
                            1 & 0 & 1 & 1 & 1 \\
                            1 & 0 & 1 & \meio{} & 1 \\
                            1 & 0 & 1 & 0 & 1 \\
                            \meio{} & \meio{} & 0 & 1 & 1 \\
                            \meio{} & \meio{} & 0 & \meio{} & 1 \\
                            \meio{} & \meio{} & 0 & 0 & 1 \\
                            0 & 1 & 1 & 1 & 1 \\
                            0 & 1 & 1 & \meio{} & 1 \\
                            0 & 1 & 1 & 0 & 1 \\
                            \hline
                        \end{array}
                    \]
                \end{center}
                
                  
                \subcasodeprova{} $\phi_{1} = \neg \neg \alpha \to \alpha$. 

                \begin{center}
                    \setbool{@fleqn}{false}
                    \[
                        \begin{array}{|c|c|c|}
                            \hline
                            \alpha      & \neg \neg \alpha & \neg \neg \alpha \to \alpha \\
                            \hline
                            1 & 1 & 1\\
                            \meio{} & \meio{} & \meio{}\\
                            0 & 0 & 1\\
                            \hline
                        \end{array}
                    \]
                \end{center}
                    
                   
                \subcasodeprova{} $\phi_{1} = \alpha \to \neg \neg \alpha$. 
                
                \begin{center}
                    \setbool{@fleqn}{false}
                    \[
                        \begin{array}{|c|c|c|}
                            \hline
                            \alpha      & \neg \neg \alpha &  \alpha \to \neg \neg \alpha\\
                            \hline
                            1 & 1 & 1\\
                            \meio{} & \meio{} & \meio{}\\
                            0 & 0 & 1\\
                            \hline
                        \end{array}
                    \]
                \end{center}
                
                   
                \subcasodeprova{} $\phi_{1} = \neg \circ \alpha \to (\alpha \land \neg \alpha)$. 

                \begin{center}
                    \setbool{@fleqn}{false}
                    \[
                        \begin{array}{|c|c|c|c|}
                            \hline
                            \alpha      & \neg \alpha & \neg \circ \alpha & \neg \circ \alpha \to (\alpha \land \neg \alpha)\\
                            \hline
                            1 & 0 & 0 &1\\
                            \meio{} & \meio{} & 1&\meio{}\\
                            0 & 1 & 0&1\\
                            \hline
                        \end{array}
                    \]
                \end{center}
               

                \subcasodeprova{} $\phi_{1} = \neg (\alpha \lor \beta) \to (\neg \alpha \land \neg \beta)$. 

                \begin{center}
                    \setbool{@fleqn}{false}
                    \[
                        \begin{array}{|c|c|c|}
                            \hline
                            \alpha      & \beta & \neg (\alpha \lor \beta) \to (\neg \alpha \land \neg \beta) \\
                            \hline
                            1 & 1 & 1\\
                            1 & \meio{} & 1\\
                            1 & 0 & 1\\
                            \meio{} & 1 & 1\\
                            \meio{} & \meio{} &\meio{}\\ 
                            \meio{} & 0 & \meio{}\\
                            0 & 1 & 1\\
                            0 & \meio{} & \meio{}\\
                            0 & 0 & 1\\
                            \hline
                        \end{array}
                    \]
                \end{center}
                
                   

                \subcasodeprova{} $\phi_{1} = (\neg \alpha \land \neg \beta) \to \neg (\alpha \lor \beta)$.

                \begin{center}
                    \setbool{@fleqn}{false}
                    \[
                        \begin{array}{|c|c|c|}
                            \hline
                            \alpha      & \beta & (\neg \alpha \land \neg \beta) \to \neg (\alpha \lor \beta) \\
                            \hline
                            1 & 1 & 1\\
                            1 & \meio{} & 1\\
                            1 & 0 & 1\\
                            \meio{} & 1 & 1\\
                            \meio{} & \meio{} &\meio{}\\ 
                            \meio{} & 0 & \meio{}\\
                            0 & 1 & 1\\
                            0 & \meio{} & \meio{}\\
                            0 & 0 & 1\\
                            \hline
                        \end{array}
                    \]
                \end{center}

                \subcasodeprova{} $\phi_{1} = \neg(\alpha \land \beta) \to (\neg \alpha \lor \neg \beta)$. 

   
                \begin{center}
                    \setbool{@fleqn}{false}
                    \[
                        \begin{array}{|c|c|c|}
                            \hline
                            \alpha      & \beta & \neg(\alpha \land \beta) \to (\neg \alpha \lor \neg \beta) \\
                            \hline
                            1 & 1 & 1\\
                            1 & \meio{} & \meio{}\\
                            1 & 0 & 1\\
                            \meio{} & 1 & \meio{}\\
                            \meio{} & \meio{} &\meio{}\\ 
                            \meio{} & 0 & 1\\
                            0 & 1 & 1\\
                            0 & \meio{} & 1\\
                            0 & 0 & 1\\
                            \hline
                        \end{array}
                    \]
                \end{center}
                
               

                \subcasodeprova{} $\phi_{1} = (\neg \alpha \lor \neg \beta) \to \neg (\alpha \land \beta)$. 
   
                \begin{center}
                    \setbool{@fleqn}{false}
                    \[
                        \begin{array}{|c|c|c|}
                            \hline
                            \alpha      & \beta & (\neg \alpha \lor \neg \beta) \to \neg (\alpha \land \beta) \\
                            \hline
                            1 & 1 & 1\\
                            1 & \meio{} & \meio{}\\
                            1 & 0 & 1\\
                            \meio{} & 1 & \meio{}\\
                            \meio{} & \meio{} &\meio{}\\ 
                            \meio{} & 0 & 1\\
                            0 & 1 & 1\\
                            0 & \meio{} & 1\\
                            0 & 0 & 1\\
                            \hline
                        \end{array}
                    \]
                \end{center}

                \subcasodeprova{} $\phi_{1} = \neg (\alpha \to \beta) \to(\alpha \land \neg \beta)$. 

                \begin{center}
                    \setbool{@fleqn}{false}
                    \[
                        \begin{array}{|c|c|c|}
                            \hline
                            \alpha      & \beta & \neg (\alpha \to \beta) \to(\alpha \land \neg \beta) \\
                            \hline
                            1 & 1 & 1\\
                            1 & \meio{} & \meio{}\\
                            1 & 0 & 1\\
                            \meio{} & 1 & 1\\
                            \meio{} & \meio{} &\meio{}\\ 
                            \meio{} & 0 & \meio{}\\
                            0 & 1 & 1\\
                            0 & \meio{} & 1\\
                            0 & 0 & 1\\
                            \hline
                        \end{array}
                    \]
                \end{center}
               

                \subcasodeprova{} $\phi_{1} = (\alpha \land \neg \beta) \to \neg(\alpha \to \beta)$.

                \begin{center}
                    \setbool{@fleqn}{false}
                    \[
                        \begin{array}{|c|c|c|}
                            \hline
                            \alpha      & \beta & (\alpha \land \neg \beta) \to \neg(\alpha \to \beta) \\
                            \hline
                            1 & 1 & 1\\
                            1 & \meio{} & \meio{}\\
                            1 & 0 & 1\\
                            \meio{} & 1 & 1\\
                            \meio{} & \meio{} &\meio{}\\ 
                            \meio{} & 0 & \meio{}\\
                            0 & 1 & 1\\
                            0 & \meio{} & 1\\
                            0 & 0 & 1\\
                            \hline
                        \end{array}
                    \]
                \end{center}
                
            \end{provaporsubcasos}

            Com isso, o \textsc{Caso 1} está provado e $\Gamma \conmat \phi_{1}$ segue caso $\phi_{1}$ seja um axioma.

            \casodeprova{} $\phi_{1} \in \Gamma$. Logo, dada uma valoração $h : \ling{} \to M$ de $\mat{}$ se $h[\Gamma] \subseteq \{1, \meio{}\}$, temos $h(\phi_{1}) \in \{1, \meio{}\}$. Portanto, $\Gamma \conmat \phi_{1}$.

        \end{provaporcasos}

         \noindent \textbf{\textsc{Passo.}}
         
         \noindent \textbf{\textsc{Hipótese de indução (HI):}} Para qualquer sequência da derivação de $\Gamma \conhil \alpha$ de tamanho $k < n$, tem-se $\Gamma \conmat \alpha$. 
         
         Portanto, é preciso mostrar que $\Gamma \conmat \alpha$ segue caso a sequência de derivação de $\Gamma \conhil \alpha$ tenha tamanho $n$. Então, vamos supor $\Gamma \conhil \phi_{n}$.
         
         Ao analisar a obtenção de $\phi_{n}$ em $\Gamma \conhil \phi_{n}$, existem três casos a se considerar:
         
         \begin{enumerate}
            \item $\phi_{n}$ é um axioma.
            \item $\phi_{n} \in \Gamma$.
            \item $\phi_{n}$ é obtido por \textit{modus ponens} em duas fórmulas $\phi_{j}$ e $\phi_{k}$ com $j, k < n$. 
         \end{enumerate}
         
         Os casos 1 e 2 são análogos aos casos provados na base.
         
         \noindent \textsc{Caso 3.} $\phi_{n}$ é obtido por \textit{modus ponens} em duas fórmulas $\phi_{j}$ e $\phi_{k}$ com $j, k < n$. 
         
         Logo, $\phi_{k} = \phi_{j} \to \phi_{n}$ (ou $\phi_{j} = \phi_{k} \to \phi_{n}$, a prova para este caso é análoga). 
         
         Dada nossa suposição de $\Gamma \conhil \phi_{n}$, então $\Gamma \conhil \phi_{j}$ e $\Gamma \conhil \phi_{j} \to \phi_{n}$. 
         
         Pela (HI), temos $\Gamma \conmat \phi_{j}$ e $\Gamma \conmat \phi_{j} \to \phi_{n}$.

         Seja $h$ uma valoração qualquer sobre $\pazocal{M}_{\lfium{}}$. Então, vamos supor $h[\Gamma] \subseteq \{1, \meio{}\}$. 
         
         Temos $h(\phi_{j}) \in \{1, \meio{}\}$ e $h(\phi_{j} \to \phi_{n}) \in \{1,\meio{}\}$. 
         
         Pela matriz de $\to$, temos $h(\phi_{j}) = 0$ ou $h(\phi_{n}) \in \{1,\meio{}\}$. 
         
         Isto, unido ao fato de termos $h(\phi_{j}) \in \{1,\meio{}\}$, nos permite concluir $h(\phi_{n}) \in \{1,\meio{}\}$. 
         
         Portanto $\Gamma \conmat \phi_{n}$.

         \noindent Com isso, provamos $\Gamma \conmat \alpha$ e a prova está finalizada.

    \end{proof}

    Com a correção em relação a semântica matricial, o seguinte resultado é imediato:


    \begin{corolario}[Correção em relação a semântica de valorações]\label{cor:correcao_val}
        A lógica {\normalfont\lfium{}} é correta em relação a sua semântica de valorações, ou seja, para todo conjunto de fórmulas $\Gamma \cup \{\alpha\} \subseteq \ling{}$:

        \centering
        {\normalfont{} $\Gamma \conhil \alpha \Longrightarrow \Gamma \conval \alpha$.}
    \end{corolario}

    \begin{proof}[Prova do Corolário~\ref{cor:correcao_val}]
        Seja $\Gamma \cup \{\alpha\} \subseteq \ling{}$ um conjunto de fórmulas. Então, supondo $\Gamma \conhil \alpha$, temos, pelo Teorema~\ref{teo:correcao_mat}, $\Gamma \conmat \alpha$. Finalmente, pelo Corolário~\ref{cor:matval}, temos $\Gamma \conval \alpha$.
    \end{proof}

    Com o resultado anterior, a prova de que a lógica \lfium{} se trata de uma \lfi{} forte como apresentado na Definição~\ref{def:lfi_forte_prop} é imediata.

    \begin{corolario}\label{cor:lfi_forte}
        Seja $p$ uma variável proposicional qualquer e $\bigcirc(p) = \{\circ p\}$ um conjunto de fórmulas dependente somente de $p$. A lógica \lfium{} é uma \lfi{} forte em relação a $\neg$ e $\bigcirc(p)$.
    \end{corolario}

    \begin{proof}[Prova do Corolário~\ref{cor:lfi_forte}]
        Sejam $a, b \in \pazocal{P}$ variáveis proposicionais distintas e sejam $\phi, \psi \in \ling{}$ fórmulas quaisquer.

        \begin{adjustwidth}{1cm}{}

            \noindent\textbf{Prova da condição (i):} A prova das condições será feita com base nas valorações $v_1, v_2$ e $v_3$ presentes na Tabela~\ref{tab:negcirc}.
    
            \begin{adjustwidth}{1cm}{}

                \textbf{(i.a)} Vamos provar $a, \neg a \nconhil b$. Tomando a valoração $v_3$, com $v_3(b) = 0$, então temos $a, \neg a \nconval b$. Pela contraposta do Corolário~\ref{cor:correcao_val}, temos  $a, \neg a \nconhil b$.

                \noindent\textbf{(i.b)} Vamos provar $\circ a, a \nconhil b$. Tomando a valoração $v_2$, com $v_2(b) = 0$, então temos $a, \neg a \nconval b$. Pela contraposta do Corolário~\ref{cor:correcao_val}, temos $\circ a, a \nconhil b$.

                \noindent\textbf{(i.c)} Vamos provar $\circ a, \neg a \nconhil b$. Tomando a valoração $v_1$, com $v_1(b) = 0$, então temos $\circ a, \neg a \nconval b$. Pela contraposta do Corolário~\ref{cor:correcao_val}, temos $\circ a, \neg a \nconhil b$.
            \end{adjustwidth}
    
            \noindent\textbf{Prova da condição (ii):} A sequência de derivação abaixo mostra $\phi, \neg \phi, \circ \phi \conhil \psi$:
            \begin{align*}
                1.~ & \phi \tag{Premissa}       \\
                2.~ & \neg \phi \tag{Premissa}  \\
                3.~ & \circ \phi \tag{Premissa} \\
                4.~ & \circ \phi \to (\phi \to (\neg \phi \to \psi)) \tag{bc1} \\
                5.~ & \phi \to (\neg \phi \to \psi) \tag{MP 3, 4} \\
                6.~ & \neg \phi \to \psi \tag{MP 1, 5} \\
                7.~ & \psi \tag{MP 2, 6}
            \end{align*}

        \end{adjustwidth}


        Portanto, a \lfium{} se trata de uma \lfi{} forte.
    \end{proof}

    A prova da completude para a lógica \lfium{} depende das seguintes definições e lemas auxiliares para ser desenvolvida:

    \begin{proposicao}\label{prop:tarski}        
        A lógica $\lfium{} = \langle \ling, \conhil \rangle$ é tarskiana (Definição~\ref{def:tarski}).
    \end{proposicao}

    \begin{proof}[Prova da Proposição~\ref{prop:tarski}]
        Queremos mostrar que a \lfium{} satisfaz as seguintes condições para todo $\Gamma \cup \Delta \cup \{\phi\} \subseteq \ling{}$:
        \begin{itemize}
            \item [(i)] Se $\phi \in \Gamma$, então $\Gamma \conhil \phi$.\hfill(reflexividade)
            \item [(ii)] Se $\Delta \conhil \phi$ e $\Delta \subseteq \Gamma$, então $\Gamma \conhil \phi$.\hfill(monotonicidade)
            \item [(iii)] Se $\Delta \conhil \phi$ e $\Gamma \conhil \delta$ para todo $\delta \in \Delta$, então $\Gamma \conhil \phi$.\hfill(corte)
        \end{itemize}
        \begin{adjustwidth}{1cm}{}
            \textbf{Prova do item (i)}. Vamos supor $\phi \in \Gamma$. Então, por definição de derivação, temos que $\Gamma \conhil \phi$.
            % \begin{align*}
            %     1.~& \phi\tag{Premissa}
            % \end{align*}

            \noindent{}\textbf{Prova do item (ii)}. Vamos supor $\Delta \conhil \phi$ e $\Delta \subseteq \Gamma$. Então, existe uma prova (sequência de fórmulas) de $\phi$ a partir do conjunto de fórmulas $\Delta$. Como, pela suposição, temos que essa sequência está inteiramente contida em $\Gamma$, concluímos $\Gamma \conhil \phi$.
            
            \noindent{}\textbf{Prova do item (iii)}. Vamos supor $\Delta \conhil \phi$ e $\Gamma \conhil \delta$ para todo $\delta \in \Delta$. Então, existe uma prova $\phi_1, \ldots, \phi_n$ de $\phi_n = \phi$ a partir de $\Delta$. Vamos fazer uma indução no tamanho $n$ desta prova.
            
            \begin{adjustwidth}{1cm}{}
                
                \textbf{\textsc{Base.}} $n = 1$.
                Analisando a obtenção de $\phi$, existem dois casos:
                \begin{provaporcasos}
                \casodeprova{$\phi_1$ é um axioma}. 
                
                Neste caso, basta aplicar o mesmo axioma para mostrar $\Gamma \conhil \phi_1$.
                
                \casodeprova{$\phi_1 \in \Delta$}. 
                
                Neste caso, $\Gamma \conhil \phi_1$ por suposição.
            \end{provaporcasos}

            \noindent\textbf{\textsc{Passo.}} 
            
            \noindent \textbf{\textsc{Hipótese de indução (HI):}} A propriedade segue para todo $i < n$.

            Analisando a obtenção de $\phi_n$, existem três casos:
            \begin{provaporcasos}
                \casodeprova{$\phi_n$ é um axioma}. Análogo ao caso da base.
                
                \casodeprova{$\phi_n \in \Delta$}. Análogo ao caso da base.
                
                \casodeprova{$\phi_n$ é resultado de MP em duas fórmulas $\phi_j, \phi_k$ com $j, k < n$}.
                
                Então, $\phi_j = \phi_k \to \phi_n$ (ou $\phi_k = \phi_j \to \phi_n$, a prova para este caso é análoga). Além disso, temos $\Delta \conhil \phi_k$ e $\Delta \conhil \phi_k \to \phi_n$. Logo, por (HI), temos $\Gamma \conhil \phi_k$ e $\Gamma \conhil \phi_k \to \phi_n$. Então, basta aplicar MP para obter $\phi_n$. Portanto, $\Gamma \conhil \phi_n$.
            \end{provaporcasos}
        \end{adjustwidth}
    \end{adjustwidth}

        Então, a \lfium{} é uma lógica tarskiana. \qedhere
        
    \end{proof}

    
    \begin{proposicao}\label{prop:finit}        
        A lógica $\lfium{} = \langle \ling, \conhil \rangle$ é finitária (Definição ~\ref{def:padrao}).
    \end{proposicao}

    \begin{proof}[Prova da Proposição~\ref{prop:finit}]
        Seja $\Gamma \cup \{\phi\} \subseteq \ling{}$ um conjunto qualquer de fórmulas com $\Gamma \conhil \phi$. Então, existe uma sequência finita de derivação $\phi_1, \ldots, \phi_n$, a partir de $\Gamma$, onde $\phi_n = \phi$. 
        
        Vamos definir o conjunto $\Gamma_0$ como sendo o conjunto formado somente pelas premissas desta derivação (logo, $\Gamma_0$ é finito). Como todo elemento de $\Gamma_0$ é uma premissa de $\Gamma$, é evidente que $\Gamma_0 \subseteq \Gamma$. Vamos provar $\Gamma_0 \conhil \phi$ por indução no tamanho $n$ da sequência de derivação.


        \begin{adjustwidth}{1cm}{}

            \textbf{\textsc{Base.}} $n = 1$.
            Analisando a obtenção de $\phi_1$ em $\Gamma \conhil \phi_1$, existem dois casos:
            \begin{provaporcasos}
                \casodeprova{$\phi_1$ é um axioma}. 
                
                    Neste caso, basta aplicar o mesmo axioma para mostrar $\Gamma_0 \conhil \phi_1$.
    
                \casodeprova{$\phi_1 \in \Gamma$}. 
                
                    Neste caso, $\Gamma_0 \conhil \phi_1$ pela construção de $\Gamma_0$.
            \end{provaporcasos}

            \noindent\textbf{\textsc{Passo.}} 
            
            \noindent \textbf{\textsc{Hipótese de indução (HI):}} A propriedade segue para todo $i < n$.

            Analisando a obtenção de $\phi_n$ em $\Gamma \conhil \phi_n$, existem três casos:
            \begin{provaporcasos}
                \casodeprova{$\phi_n$ é um axioma}. Análogo ao caso da base.
    
                \casodeprova{$\phi_n \in \Delta$}. Análogo ao caso da base.

                \casodeprova{$\phi_n$ é resultado de MP em duas fórmulas $\phi_j, \phi_k$ com $j, k < n$}.
                
                Então, $\phi_j = \phi_k \to \phi_n$ (ou $\phi_k = \phi_j \to \phi_n$, a prova para este caso é análoga). Além disso, temos $\Gamma \conhil \phi_k$ e $\Gamma \conhil \phi_k \to \phi_n$. Logo, por (HI), temos $\Gamma_0 \conhil \phi_k$ e $\Gamma_0 \conhil \phi_k \to \phi_n$. Então, basta aplicar MP para obter $\phi_n$. Portanto, $\Gamma_0 \conhil \phi_n$.
            \end{provaporcasos}
        \end{adjustwidth}
        Então, a \lfium{} é finitária. \qedhere

    \end{proof}



    \begin{definicao}[Conjunto de fórmulas maximal não-trivial]\label{def:nao-trivial_maximal}
        Seja $\mathcal{L}$ uma lógica tarskiana definida sobre uma linguagem $\pazocal{L}$ e sejam $\Gamma, \{\phi\}$ conjuntos de fórmulas de modo que $\Gamma \cup \{\phi\} \subseteq \pazocal{L}$. O conjunto $\Gamma$ é dito maximal não-trivial em relação a $\phi$ em $\mathcal{L}$ se $\Gamma \nVdash_{\mathcal{L}} \phi$ mas $\Gamma, \psi \Vdash_{\mathcal{L}} \phi$ para qualquer $\psi \notin \Gamma$.\qed{}
    \end{definicao}

    \begin{definicao}[Teoria fechada]\label{def:fechada}

        Seja $\mathcal{L}$ uma lógica tarskiana. Um conjunto de fórmulas $\Gamma$ é dito fechado em $\mathcal{L}$ (ou dito uma \textit{teoria fechada} em $\mathcal{L}$) se, para toda fórmula $\phi$, tem-se $\Gamma \Vdash_{\mathcal{L}} \phi$ sse $\phi \in \Gamma$.\qed{}
    \end{definicao}

    \begin{lema}\label{lem:nao_trivial_maximal_fechado}
        Todo conjunto de fórmulas maximal não-trivial em relação a $\phi$ em $\mathcal{L}$ é fechado em $\mathcal{L}$.
    \end{lema}

    \begin{proof}[Prova do Lema~\ref{lem:nao_trivial_maximal_fechado}]
        Seja $\Gamma$ uma conjunto de fórmulas maximal não-trivial em relação a $\phi$ em $\mathcal{L}$. 
        
        \noindent($\Longrightarrow$) Se $\psi \in \Gamma$, então $\Gamma \Vdash_{\mathcal{L}} \psi$, já que $\mathcal{L}$ é tarskiana. 
        
        \noindent($\Longleftarrow$) Se $\Gamma \Vdash_{\mathcal{L}} \psi$, então supondo $\psi \notin \Gamma$, temos $\Gamma, \psi \Vdash_{\mathcal{L}} \phi$, pela Definição~\ref{def:nao-trivial_maximal}. Pela propriedade do corte, segue que $\Gamma \Vdash_{\mathcal{L}} \phi$, o que contradiz o fato de $\Gamma$ ser maximal não-trivial em relação a $\phi$ em $\mathcal{L}$. Portanto, $\psi \in \Gamma$.
        
        Então $\Gamma$ é fechado em $\mathcal{L}$.
    \end{proof}





    Provaremos agora o lema de Lindenbaum, proposto originalmente por Adolf Lindenbaum (de acordo com~\citeshort{Tarski1956-TARLSM}) e adaptado por~\citeshort{los_lindenbaum,Wojcicki1984-WJCLOP}. A versão apresentada por~\citeshort{Carnielli_Coniglio_2016} utiliza o princípio da boa ordem e aplica uma recursão transfinita para que o lema siga verdadeiro mesmo quando se tratando de sistemas definidos sobre linguagens não enumeráveis. Para o propósito do presente trabalho, isso não se mostra necessário, tendo em vista que a linguagem da \lfium{} (Definição~\ref{def:ling}) é enumerável.

    \begin{lema}[Lindenbaum-{\L}oś para \lfium{}]\label{lem:lindenbaum}
        Dada a lógica $\lfium{} = \langle \ling, \conhil \rangle$ então temos, para qualquer conjunto de fórmulas $\Gamma \cup \{\phi\} \subseteq \ling$, se $\Gamma \nconhil \phi$ então existe um conjunto de fórmulas $\Delta$, com $\Gamma \subseteq \Delta \subseteq \ling$, tal que $\Delta$ é um conjunto maximal não-trivial em relação a $\phi$ em \lfium{}.
    \end{lema}

    \begin{proof}[Prova do Lema~\ref{lem:lindenbaum}]\setbool{@fleqn}{false}
        Dada $\lfium{} = \langle \ling, \conhil \rangle$ e seja $\Gamma \cup \{\phi\} \subseteq \ling$ um conjunto de fórmulas com $\Gamma \nconhil \phi$. Arranje \migs{Como?} \helena{TAMBÉM NÃO SEI ;-;} \migs{Como não sabe mulher, foi tu quem fez a prova!?} as fórmulas de $\ling$ numa sequência $\pazocal{C} = \phi_1, \phi_2, \ldots, \phi_i, \ldots$ e defina $\Gamma_i$ recursivamente da seguinte forma:
        \begin{align*}
                \text{(i)}\quad&\Gamma_0 = \Gamma\\
                \text{(ii)}\quad&\Gamma_i =
                \begin{cases}
                    \Gamma_{i - 1} &\text{sse}~ \Gamma_{i - 1}, \phi_i \conhil \phi\\
                    \Gamma_{i - 1} \cup \{\phi_i\} &\text{sse}~ \Gamma_{i - 1}, \phi_i \nconhil \phi
                \end{cases}
        \end{align*}

        Então, defina um conjunto de fórmulas $\Delta = \bigcup_{i=0}^{\infty}\Gamma_i$. Dividiremos a prova em algumas partes:
        
        \begin{adjustwidth}{1cm}{}
            \noindent(1) $\Gamma \subseteq \Delta \subseteq \ling$.

            \begin{adjustwidth}{1cm}{}
                Pela construção de $\Gamma_i$, temos $\Gamma_0 = \Gamma \subseteq \Gamma_i$ e $\Gamma_i \subseteq \ling$ para todo $i \in \mathbb{N}$. Como $\Delta$ é formado a partir da união de todos os conjuntos $\Gamma_i$, temos $\Gamma_0 \subseteq \Delta$ e $\Delta \subseteq \ling$.
            \end{adjustwidth}

            \noindent(2) $\Gamma_k \subseteq \Gamma_i$, onde $0 \leq k \leq i$.

            \begin{adjustwidth}{1cm}{}
                Dados dois números $i, k \in \mathbb{N}$ com $k \leq i$, o conjunto $\Gamma_i$ é construído pela união de todos os seus predecessores, portanto, $\Gamma_k \subseteq \Gamma_i$.
                
            \end{adjustwidth}

            \noindent(3) $\Gamma_i \nconhil \phi$ para todo $i \geq 0$.

            Provaremos por indução em $i$:

            \begin{adjustwidth}{1cm}{}
                \noindent\textbf{\textsc{Base.}} $i = 0$.
                
                \begin{adjustwidth}{1cm}{}
                    
                    Temos $\Gamma_0 \nconhil \phi$ pela nossa hipótese de $\Gamma \nconhil \phi$.
                    
                \end{adjustwidth}
                
                \noindent\textbf{\textsc{Passo.}} 
                
                \noindent\textbf{Hipótese de indução (HI):} Para qualquer $k < i$, temos $\Gamma_k \nconhil \phi$.
                
                    Como caso particular da Hipótese de Indução, temos que $\Gamma_{i-1} \nconhil \phi$. Portanto, pela construção de $\Gamma_i$, temos $\Gamma_i \nconhil \phi$.
                    
            \end{adjustwidth}

            
            \noindent(4) Para todo $i \geq 1$, se $\phi_i \in \Delta$, então $\phi_i \in \Gamma_i$.

            \begin{adjustwidth}{1cm}{}


                Vamos supor $\phi_i \in \Delta$ e $\phi_i \not \in \Gamma_i$. Temos, por (3), $\Gamma_{i - 1} \nconhil \phi$. Pela definição de $\Gamma_i$, temos $\Gamma_{i - 1}, \phi_i \conhil \phi$. Portanto, por (2) e pela monotonicidade da \lfium{} (Proposição~\ref{prop:tarski}), temos $\Gamma_k, \phi_i \conhil \phi$, para qualquer $k \geq i$. Então, por (3), $\phi_i \not \in \Gamma_k$ para qualquer $k > i$. Portanto, $\phi_i \not \in \Delta$, o que contradiz a nossa suposição inicial. Portanto, $\phi_i \in \Gamma_i$.

            \end{adjustwidth}

            \noindent(5) Para todo $\Delta' \subseteq \Delta$ finito, existe um $\Gamma_i$ na sequência com $\Delta' \subseteq \Gamma_i$.

            \begin{adjustwidth}{1cm}{}

                Seja $\Delta' \subseteq \Delta$ um subconjunto finito. Como $\Delta$ é formado a partir da sequência $\pazocal{C} = \phi_1, \phi_2, \ldots, \phi_i, \ldots$, então existe um $\phi_{max} \in \Delta'$ tal que $max \geq i$ para qualquer $\phi_i \in \Delta'$. Logo, $\phi_{max} \in \Delta$ e, por (4), temos $\phi_{max} \in \Gamma_{max}$. Para qualquer outro elemento $\phi_k \in \Delta'$, com $max \geq k$, temos, por (4), $\phi_k \in \Gamma_k$, e, por (2), $\Gamma_k \subseteq \Gamma_{max}$. Portanto, $\Delta' \subseteq \Gamma_{max}$.

            \end{adjustwidth}


            \noindent(6) $\Delta \nconhil \phi$.

            \begin{adjustwidth}{1cm}{}
                Vamos supor $\Delta \conhil \phi$. Portanto, como a \lfium{} é finitária (Proposição~\ref{prop:finit}), existe um $\Delta' \subseteq \Delta$ finito com $\Delta' \conhil \phi$. Logo, por (3), existe um $\Gamma_i$ na sequência tal que $\Delta' \subseteq \Gamma_i$. Então, como a \lfium{} é tarskiana (Proposição~\ref{prop:tarski}), temos $\Gamma_i \conhil \phi$, o que é uma contradição (por (2)). Portanto, $\Delta \nconhil \phi$.
            \end{adjustwidth}

            \noindent(7) $\Delta$ é maximal não-trivial em relação a $\phi$ em \lfium{}.
            
            \begin{adjustwidth}{1cm}{}
                Seja $\psi$ uma fórmula qualquer com $\psi \not \in \Delta$. Então $\psi = \phi_i$ para algum $i \in \mathbb{N}$. Supondo $\phi_i \in \Gamma_i$, chegamos numa contradição (já que, por construção de $\Delta$, teríamos $\phi_i \in \Delta$), portanto, temos $\phi_i \not \in \Gamma_i$. Logo, pela construção de $\Gamma_i$, temos $\Gamma_{i - 1}, \phi_i \conhil \phi$. Então, pela monotonicidade da \lfium{}, temos $\Delta, \phi_i \conhil \phi$. Portanto, $\Delta$ é maximal não-trivial em relação a $\phi$ em \lfium{}.

            \end{adjustwidth}


        \end{adjustwidth}
            
            
        \end{proof}
        
    \begin{lema}\label{lem:valoracao}
        Seja $\Gamma \cup \{\phi\} \subseteq \ling{}$ um conjunto qualquer de fórmulas com $\Gamma$ sendo um conjunto maximal não-trivial em relação a $\phi$ em \lfium{}. A função $v \; : \; \ling \to \{1, 0\}$ definida, para todo $\psi \in \ling$, como:
        \begin{center}
            \setbool{@fleqn}{false}
                $v(\psi) = 1 \Longleftrightarrow \psi \in \Gamma$
        \end{center}

        é uma valoração para \lfium{}.
    \end{lema}

    \begin{proof}[Prova do Lema~\ref{lem:valoracao}]
        Queremos provar que $v$ satisfaz todas as cláusulas da Definição~\ref{def:valoracoes}. \migs{Sugestão: Nomear cada caso como ``Cláusula X'' ao invés só do nome da cláusula.}

        \begin{provaporcasos}
            
            \casodeprova{$(vAnd)$}

                \begin{adjustwidth}{1cm}{}
                    \noindent ($\Longrightarrow$) Vamos supor $v(\alpha \land \beta) = 1$. Logo $(\alpha \land \beta) \in \Gamma$. 
                    
                    \noindent Supondo $\alpha \not \in \Gamma$, temos $\Gamma, \alpha \conhil \phi$. Pelo axioma \textbf{(Ax4)} e MP, temos $\Gamma \conhil \alpha$. Logo, pela monotonicidade da \lfium{} (Proposição~\ref{def:tarski}), temos $\Gamma \conhil \phi$, o que é uma contradição. Portanto $\alpha \in \Gamma$.
        
                    \noindent Supondo $\beta \not \in \Gamma$, temos $\Gamma, \beta \conhil \phi$. Pelo axioma \textbf{(Ax5)} e MP, temos $\Gamma \conhil \beta$. Logo, pela monotonicidade da \lfium{} (Proposição~\ref{def:tarski}), temos $\Gamma \conhil \phi$, o que é uma contradição. Portanto $\beta \in \Gamma$.
        
                    \noindent Logo, temos $v(\alpha) = 1$ e $v(\beta) = 1$.

                \end{adjustwidth}

                \begin{adjustwidth}{1cm}{}
                    \noindent ($\Longleftarrow$) Vamos supor $v(\alpha) = 1$ e $v(\beta) = 1$. Logo, $\alpha \in \Gamma$ e $\beta \in \Gamma$. Pelo axioma \textbf{(Ax3)} e MP, temos $\Gamma \conhil \alpha \land \beta$. Pelo Lema~\ref{lem:nao_trivial_maximal_fechado}, temos $\alpha \land \beta \in \Gamma$.

                    Logo, temos $v(\alpha \land \beta) = 1$

                \end{adjustwidth}

            \casodeprova{$(vOr)$}

                \begin{adjustwidth}{1cm}{}
                    \noindent ($\Longrightarrow$) Vamos supor $v(\alpha \lor \beta) = 1$. Logo, $\alpha \lor \beta \in \Gamma$.

                    \noindent Supondo que \cortar{não é o caso que} $\alpha \notin \Gamma$ \migscortar{nem}{ e} $\beta \notin \Gamma$. Temos $\Gamma, \alpha \conhil \phi$ e $\Gamma, \beta \conhil \phi$. Logo, pelo Corolário~\ref{cor:prova_por_casos}, temos $\Gamma, \alpha \lor \beta \conhil \phi$. Logo, pela monotonicidade da \lfium{} (Proposição~\ref{def:tarski}), temos $\Gamma \conhil \phi$, o que é uma contradição. Portanto $\alpha \in \Gamma$ ou $\beta \in \Gamma$.
                    
                    \noindent Logo, temos $v(\alpha) = 1$ ou $v(\beta) = 1$.

                    \noindent ($\Longleftarrow$) Vamos supor $v(\alpha) = 1$ ou $v(\beta) = 1$. Logo, temos $\alpha \in \Gamma$ ou $\beta \in \Gamma$.

                    \noindent Caso tenhamos $\alpha \in \Gamma$, então, pelo axioma \textbf{(Ax6)}, temos $\Gamma \conhil \alpha \to (\alpha \lor \beta)$. Pelo Lema~\ref{lem:nao_trivial_maximal_fechado}, temos $\Gamma \conhil \alpha$. Então, por MP, temos $\Gamma \conhil \alpha \lor \beta$. Finalmente, pelo Lema~\ref{lem:nao_trivial_maximal_fechado}, temos $\alpha \lor \beta \in \Gamma$ e, consequentemente, $v(\alpha \lor \beta) = 1$.

                    \noindent Caso tenhamos $\beta \in \Gamma$, a prova é análoga.

                    \noindent Portanto, temos $v(\alpha \lor \beta) = 1$.

                \end{adjustwidth}

                \casodeprova{$(vImp)$}

                    \begin{adjustwidth}{1cm}{}
                        
                        \noindent ($\Longrightarrow$) Vamos supor $v(\alpha \to \beta) = 1$. Logo, $\alpha \to \beta \in \Gamma$.

                        \noindent Supondo $v(\alpha) = 1$ e $v(\beta) = 0$, temos $\alpha \in \Gamma$ e $\beta \not \in \Gamma$. Então, pelo Lema~\ref{lem:nao_trivial_maximal_fechado}, temos $\Gamma \conhil \alpha$, $\Gamma \nconhil \beta$ e $\Gamma \conhil \alpha \to \beta$. Logo, por MP, $\Gamma \conhil \beta$, o que é uma contradição. Portanto, $v(\alpha) = 0$ ou $v(\beta) = 1$.

                        \noindent ($\Longleftarrow$) Vamos supor $v(\alpha) = 0$ ou $v(\beta) = 1$.

                        \noindent Caso $v(\alpha) = 0$, então temos $\alpha \not \in \Gamma$. Supondo $\alpha \to \beta \not \in \Gamma$, temos, pelo Lema~\ref{lem:nao_trivial_maximal_fechado}, $\Gamma, \alpha \conhil \phi$ e $\Gamma, \alpha \to \beta \conhil \phi$. Logo, pelo Corolário~\ref{cor:prova_por_casos}, temos $\Gamma, (\alpha \to \beta) \lor \alpha \conhil \phi$. Portanto, pelo axioma \textbf{(Ax9)} e pela monotonicidade da \lfium{}, temos $\Gamma \conhil \alpha$, o que é uma contradição. Portanto, $\alpha \to \beta \in \Gamma$ e, então, $v(\alpha \to \beta) = 1$.

                        \noindent Caso $v(\beta) = 1$, então temos $\beta \in \Gamma$. Pelo axioma \textbf{(Ax1)}, temos $\Gamma \conhil \beta \to (\alpha \to \beta)$. Portanto, por MP, temos $\Gamma \conhil \alpha \to \beta$. Logo $\alpha \to \beta \in \Gamma$ e, então, $v(\alpha \to \beta) = 1$.

                        \noindent Logo, $v(\alpha \to \beta) = 1$.

                    \end{adjustwidth}

                \casodeprova{$(vNeg)$}
                    
                    \begin{adjustwidth}{1cm}{}
                        \noindent Supondo $v(\neg \alpha) = 0$, temos $\neg \alpha \not \in \Gamma$. Então, \migscortar{supondo}{ caso} $v(\alpha) = 0$, temos $\alpha \not \in \Gamma$. 
                        
                        \noindent Portanto, $\Gamma, \alpha \conhil \phi$ e $\Gamma \neg \alpha \conhil \phi$. Pelo Corolário~\ref{cor:caso_neg}, temos $\Gamma \conhil \phi$, o que é uma contradição. Logo, $v(\alpha) = 1$.

                    \end{adjustwidth}

                \casodeprova{$(vCon)$}

                    \begin{adjustwidth}{1cm}{}
                        \noindent Supondo $v(\circ \alpha) = 1$, temos $\circ \alpha \in \Gamma$. Logo, pelo Lema~\ref{lem:nao_trivial_maximal_fechado}, temos $\Gamma \conhil \circ \alpha$.

                        \noindent Vamos supor $v(\alpha) = 1$ e $v(\neg \alpha) = 1$. Portanto, temos $\alpha \in \Gamma$ e $\neg \alpha \in \Gamma$ e, pelo Lema~\ref{lem:nao_trivial_maximal_fechado}, $\Gamma \conhil \alpha$ e $\Gamma \conhil \neg \alpha$.

                        \noindent Pelo axioma \textbf{(bc1)}, temos $\Gamma \conhil \circ \alpha \to (\alpha \to (\neg \alpha \to \phi))$. Portanto, aplicando MP três vezes, temos $\Gamma \conhil \phi$, o que é uma contradição. Logo, $v(\alpha) = 0$ ou $v(\neg \alpha) = 0$
                    


                    \end{adjustwidth}
                
                \casodeprova{$(vCi)$}

                    \begin{adjustwidth}{1cm}{}
                        
                        \noindent Supondo $v(\neg \circ \alpha) = 1$, temos $\neg \circ \alpha \in \Gamma$. Então, pelo Lema~\ref{lem:nao_trivial_maximal_fechado}, temos $\Gamma \conhil \neg \circ \alpha$.

                        Pelo axioma \textbf{(ci)}, temos $\Gamma \conhil \neg \circ \alpha \to (\alpha \land \neg \alpha)$. Portanto, por MP, temos $\Gamma \conhil \alpha \land \neg \alpha$. 
                        
                        Pelo axioma \textbf{(Ax4)}, temos $\Gamma \conhil (\alpha \land \neg \alpha) \to \alpha$. Aplicando MP, temos $\Gamma \conhil \alpha$.
                        
                        Pelo axioma \textbf{(Ax5)}, temos $\Gamma \conhil (\alpha \land \neg \alpha) \to \neg \alpha$. Aplicando MP, temos $\Gamma \conhil \neg \alpha$.

                        Pelo Lema~\ref{lem:nao_trivial_maximal_fechado}, temos $\alpha \in \Gamma$ e $\neg \alpha \in \Gamma$. Logo, $v(\alpha) = 1$ e $v(\neg \alpha) = 1$.

                    \end{adjustwidth}

        \end{provaporcasos}


    \end{proof}
    

    \begin{teorema}[Completude]\label{teo:completude}
        A lógica {\normalfont\lfium{}} é completa em relação a sua semântica de valorações, ou seja, para todo conjunto de fórmulas $\Gamma \cup \{\alpha\} \subseteq \ling{}$:

        \centering
        {\normalfont{} $\Gamma \conval \alpha \Longrightarrow \Gamma \conhil \alpha$.}
    \end{teorema}

    \begin{proof}[Prova do Teorema~\ref{teo:completude}]
 

    \end{proof}





% -----------------------------------------------------------------
% ELEMENTOS PÓS-TEXTUAIS
% -----------------------------------------------------------------
\postextual

% Você pode comentar os elementos que não deseja em seu trabalho;

% Referências bibliográficas
\bibliography{Referencias}	% Elemento Obrigatório

%% ----------------------------------------------------------
% Glossário
% ----------------------------------------------------------

%Consulte o manual da classe abntex2 para orientações sobre o glossário.

%\glossary




% ----------------------------------------------------------
% Glossário (Formatado Manualmente)
% ----------------------------------------------------------

\chapter*{GLOSSÁRIO}
\addcontentsline{toc}{chapter}{GLOSSÁRIO}

{ \setlength{\parindent}{0pt} % ambiente sem indentação

\textbf{Ardósia}: Rocha metamórfica sílico-argilosa formada pela transformação da argila sob pressão e temperatura, endurecida em finas lamelas.

\textbf{Arenito}: rocha sedimentária de origem detrítica formada de grãos agregados por um cimento natural silicoso, calcário ou ferruginoso que comunica ao conjunto em geral qualidades de dureza e compactação.

\textbf{Feldspato}: grupo de silicatos de sódio, potássio, cálcio ou outros elementos que compreende dois subgrupos, os feldspatos alcalinos e os plagioclásios.






} % fim ambiente sem indentação


			% Elemento Opcional

% ----------------------------------------------------------
% Apêndices
% ----------------------------------------------------------

% ---
% Inicia os apêndices
% ---
\begin{apendicesenv}



\end{apendicesenv}
% ---			    % Elemento Opcional
%
% ----------------------------------------------------------
% Anexos
% ----------------------------------------------------------
%
% ---
% Inicia os anexos
% ---
\begin{anexosenv}

% Imprime uma página indicando o início dos anexos
%\partanexos

% ---
\chapter{TÍTULO}
% ---



\end{anexosenv}
				% Elemento Opcional
%
%%---------------------------------------------------------------------
%% INDICE REMISSIVO
%%---------------------------------------------------------------------

%\phantompart
%\printindex

%---------------------------------------------------------------------

%%---------------------------------------------------------------------
%% INDICE REMISSIVO (Formatado Manualmente)
%%---------------------------------------------------------------------

\chapter*{ÍNDICE}
\addcontentsline{toc}{chapter}{ÍNDICE}

{ \setlength{\parindent}{0pt}  % ambiente sem indentação
	
Andesito, 22, 50, 73

Argila, 52, 75, 121

Basalto, 25, 230, 235

	
	
	
	
} % fim ambiente sem indentação


		% Elemento Opcional

\end{document}

% -----------------------------------------------------------------
% Fim do Documento
% -----------------------------------------------------------------
