% MIGUEL: Criei esse arquivo para melhor separar os comandos customizados que estavam largados pelo main
%           Seria uma boa ideia documentar o que cada um faz, mas isso é um problema para o futuro

% Teoremas e Provas

% Estilo padrão, i.e., texto itálico
\newtheorem{teorema}   {Teorema}
\newtheorem{proposicao}{Proposição}
\newtheorem{lema}      {Lema}
\newtheorem{corolario} {Corolário}
\newtheorem{exemplo}{Exemplo}
\renewcommand{\proofname}{Prova}
\renewcommand\qedsymbol{$\blacksquare$}

% Estilo simples, i.e., texto normal
\theoremstyle{definition}
\newtheorem{definicao}{Definição}
\newtheorem{notacao}{Notação}

\theoremstyle{remark}
\newtheorem{observacao}{Observação}

% \AtEndEnvironment{teorema}  {\qed}%
% \AtEndEnvironment{proposicao}  {\qed}%
% \AtEndEnvironment{lema}     {\qed}%
% \AtEndEnvironment{definicao}{\qed}%
% \AtEndEnvironment{exemplo}  {\qed}%

%% Grego

\renewcommand\phi{\varphi}

% Grego Minúsculo

% Grego Maiúsculo
\newcommand{\GAMMA}{\(\Gamma\)\xspace}
\newcommand{\DDELTA}{\(\Delta\)\xspace}
\newcommand{\SIGMA}{\(\Sigma\)\xspace}
\newcommand{\THETA}{\(\Theta\)\xspace}
\newcommand{\LAMBDA}{\(\Lambda\)\xspace}
\newcommand{\LAMBDAlm}{\(\Lambda_{\mathsf{LM}}\)\xspace}
\newcommand{\Lambdalm}{\Lambda_{\mathsf{LM}}}

% Modalidades e símbolos matemáticos

\newcommand{\meio}{\frac{1}{2}}

\newcommand{\VVDASH}{\(\vdash\)\xspace}
\newcommand{\VDDASH}{\(\vDash\)\xspace}
\newcommand{\VDASH}{\(\vdash\)\xspace}

\newcommand{\ODOT}{\(\odot\)\xspace}
\newcommand{\OPLUS}{\(\oplus\)\xspace}
\newcommand{\OTIMES}{\(\otimes\)\xspace}
\newcommand{\OMINUS}{\(\ominus\)\xspace}

\newcommand{\tofrom}{\leftrightarrow}
\newcommand{\conmat}{\vDash_{\pazocal{M}_{\lfium{}}}}
\newcommand{\conval}{\vDash_{\lfium{}}}
\newcommand{\conhil}{\vdash_{\lfium{}}}

\newcommand{\nconmat}{\nvDash_{\pazocal{M}_{\lfium{}}}}
\newcommand{\nconval}{\nvDash_{\lfium{}}}
\newcommand{\nconhil}{\nvdash_{\lfium{}}}

\newcommand{\eqdef}{\mathrel{\overset{\makebox[0pt]{\mbox{\normalfont\tiny\sffamily def}}}{=}}}

% Fontes
\newcommand{\Mathcal}[1]{\(\mathcal{#1}\)\xspace}
\newcommand{\Mathcali}[2]{\(\mathcal{#1}_{#2}\)\xspace}
\newcommand{\MathcalI}[2]{\(\mathcal{#1}^{#2}\)\xspace}
\newcommand{\Mathcalii}[3]{\(\mathcal{#1}{#2}_{#3}\)\xspace}

\newcommand{\Mathfrak}[1]{\(\mathfrak{#1}\)\xspace}
\newcommand{\Mathfraki}[2]{\(\mathfrak{#1}_{#2}\)\xspace}
\newcommand{\MathfrakI}[2]{\(\mathfrak{#1}^{#2}\)\xspace}

\newcommand{\Mathbb}[1]{\(\mathbb{#1}\)\xspace}
\newcommand{\Mathbbi}[2]{\(\mathbb{#1}_{#2}\)\xspace}
\newcommand{\MathbbI}[2]{\(\mathbb{#1}{#2}\)\xspace}

% Comentários
\newcommand{\cortar}[1]{\textcolor{red}{\sout{#1}}}
\newcommand{\ignore}[1]{\textcolor{blue}{\textbf{IGNOREM:} #1}}
\newcommand{\helena}[1]{\textcolor{magenta}{\textbf{HELENA:} #1}}
\newcommand{\migs}[1]{\textcolor{violet}{\textbf{MIGS:} #1}}
\newcommand{\migscortar}[2]{\textcolor{violet}{\textbf{MIGS:} \sout{#1}{#2}}}
\newcommand{\kaqui}[1]{\textcolor{teal}{\textbf{KAQUI:} #1}}

% Outros
\newcommand{\linguagem}[1]{\(\mathsf{LFI1}_{#1}\)\xspace}
\newcommand{\Linguagem}[1]{\mathsf{LFI1}_{#1}\xspace}
\newcommand{\funcao}[1]{\operatorname{#1}\xspace}
\newcommand{\inlinecoq}[1]{\lstinline[columns=fixed,language=coq]{#1}}

\newcommand{\Odot}  {\mathbin{\odot}}
\newcommand{\Oplus} {\mathbin{\oplus}}
\newcommand{\Otimes}{\mathbin{\otimes}}

% Abreviações
\newcommand{\lfium}{{\normalfont\textbf{LFI1}}}
\newcommand{\lfi}{{\normalfont\textbf{LFI}}}
\newcommand{\lfis}{{\normalfont\textbf{LFI}s}}
\newcommand{\Ltac}{\Mathcal{L}\unskip~tac}
\newcommand{\CalcLambda}{Cálculo-\(\lambda\)\xspace}
\newcommand{\CalcsLambda}{Cálculos-\(\lambda\)\xspace}
\newcommand{\SisT}{\(\textbf{KT} \Odot \textbf{K4}\)\xspace}

\newcommand{\CLST}{Cálculo-\(\lambda\) Simplesmente Tipado\xspace}
\newcommand{\TTML}{Teoria de Tipos de Martin-Löf\xspace}
\newcommand{\CCH}{Correspondência de Curry-Howard\xspace}

\newcommand{\PIMODELOS}{\PI-Modelos\xspace}
\newcommand{\PIMODELO} {\PI-Modelo\xspace}
\newcommand{\PImodelos}{\PI-modelos\xspace}
\newcommand{\PImodelo} {\PI-modelo\xspace}

\newcommand{\PIFRAMES}{\PI-Frames\xspace}
\newcommand{\PIFRAME} {\PI-Frame\xspace}
\newcommand{\PIframes}{\PI-frames\xspace}
\newcommand{\PIframe} {\PI-frame\xspace}

\newcommand{\OPIMODELOS}{\OPI-Modelos\xspace}
\newcommand{\OPIMODELO} {\OPI-Modelo\xspace}
\newcommand{\OPImodelos}{\OPI-modelos\xspace}
\newcommand{\OPImodelo} {\OPI-modelo\xspace}

\newcommand{\OPIFRAMES}{\OPI-Frames\xspace}
\newcommand{\OPIFRAME} {\OPI-Frame\xspace}
\newcommand{\OPIframes}{\OPI-frames\xspace}
\newcommand{\OPIframe} {\OPI-frame\xspace}

\newcommand{\Modeloinicial}{\(\mathbfcal{I}\)\xspace}
\newcommand{\modeloinicial}{\mathbfcal{I}\xspace}

\newcommand{\Mundoinicial}{\textbf{\textit{i}}\xspace}
\newcommand{\mundoinicial}{\textbf{\textit{i}}\xspace}

\newcommand{\Mundobase}{\textit{w}\textsubscript{\Mathcal{M}}\xspace}
\newcommand{\mundobase}{w_{\mathcal{M}}\xspace}

\newcommand{\ElementoMaximo}{\MathcalI{S}{+}\xspace}
\newcommand{\elementomaximo}{\mathcal{S}^{+}\xspace}

% Referência de Environments
\newcommand{\SubCaso}[2]{\ref{#1}.\ref{#2}}

% Ambiente customizado para provas com casos e subcasos

\newcounter{Casos}    % Contador que enumera os casos da prova
\newcounter{SubCasos} % Contador que enumera o subcaso da prova
\newenvironment{provaporcasos} % Ambiente customizado para fazer provas por casos
    {
        % O que é feito quando este ambiente é aberto    
        \setcounter{Casos}{0} % resetando o counter por via das dúvidas
        \begin{description}[font=\mdseries\scshape] % abrindo o ambiente de description com a formatação bonita
    }   
    {
        % O que é feito quando esse ambiente é fechado
        \setcounter{Casos}{0} % resetando o counter por via das dúvidas
        \end{description} % fechando o ambiente
    }

% Comando para printar o caso com a numeração correta
\newcommand{\casodeprova}{\refstepcounter{Casos}\item[Caso \theCasos{}.]}

% Ambiente customizado para fazer provas por casos que contenham subcasos
% Só deve ser usado dentro de um ambiente provaporcasos
\newenvironment{provaporsubcasos}
    {
        % O que é feito quando este ambiente é aberto    
        \setcounter{SubCasos}{0} % resetando o counter por via das dúvidas
        \begin{description}[font=\mdseries\scshape] % abrindo o ambiente de description com a formatação bonita
    }   
    {
        % O que é feito quando esse ambiente é fechado
        \setcounter{SubCasos}{0} % resetando o counter por via das dúvidas
        \end{description} % fechando o ambiente
    }

% Comando para printar o subcaso com a numeração correta
\newcommand{\subcasodeprova}{\refstepcounter{SubCasos}\item[Subcaso \theCasos{}.\theSubCasos{}.]}

% Para testar, mantenha comentado se não estiver ativamente usando
\usepackage{lipsum}